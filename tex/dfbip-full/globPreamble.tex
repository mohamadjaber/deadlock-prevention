
The supercycle formation condition
(Proposition~\ref{prop:supercycle-formation}) tells us that, when a
supercycle $\SC$ is created, some component $\B_i$ that participates
in the interaction $\act$ whose execution created $\SC$, must be a
node of a strongly connected component $\CC$ of $\SC$, and moreover
$\CC$ is itself a supercycle in its own right. In a sense, $CC$ is the
``essential'' part of $\SC$.

Hence, for a BIP system $(\B, Q_0)$, our fundamental condition for the
prevention of supercycles is that for every reachable transition
$s \goesto[\act] t$ resulting from execution of $\act$, every
component $\B_i$ of $\act$ must exhibit a supercycle-violation
(Definition~\ref{def:supercycle.violation}) in state $t$ (the state
resulting from the execution of $a$). For a given BIP system
$(\B, Q_0)$ and interaction $\act$, we denote that condition
$\GAO(\B, Q_0, \act)$, and define it formally below.  This condition
is, in a sense, the ``most general'' condition for supercycle-freedom.

If $\GAO(\B, Q_0, \act)$ holds, and global state $s$ is
supercycle-free, and $s \goesto[\act] t$, then it follows (as we
establish below) that global state $t$ is also supercycle-free.  So,
by requiring (1) that all initial states are supercycle-free, and (2)
that $\GAO(\B, Q_0, \act)$ holds for all interactions
$\act \in \gamma$, we obtain, by straightforward induction on length
of executions, that every reachable state is supercycle-free.

It also follows that any condition which implies $\GAO(\B, Q_0, \act)$ is also sufficient to guarantee  supercycle-freedom, and
hence deadlock-freedom. We exploit this in two ways:
\bn

\item To provide ``local variants'' of $\GAO$ and $\GLin$,  which can often be
evaluated in small subsystems of $(\B, Q_0)$, thereby avoiding state-explosion. The local conditions imply the
corresponding global ones, \ie they are sufficient but not necessary for deadlock-freedom.

\item To provide a ``linear'' condition, $\GLin$, that is easier to evaluate than $\GAO$, since it requires only the
evaluation of lengths of wait-for-paths, \ie it does not have the ``alternating'' character of $\GAO$. 

\en

Since we will present several conditions for supercycle-freedom, we now present an abstract definition of the essential
properties that all such conditions must have. First, we define a condition for the prevention of supercycles as a 
``behavioral restriction condition''. Moreover, since implication of $\GAO(\act)$ can be on a ``per interaction'' basis,
with different conditions being applied to different interactions, we have:



 
\begin{figure}[ht]
\setcounter{lctr}{0}
\begin{tabbing}\label{alg:check-scViol}
mm\= mm\= mm\= mm\= mm\=\kill
$\lviol{v}{d}{t}{a}{\l}$\\
\lio{\IFC{\mbox{$v$ is an interior node of \dsk{a}{\l}}}}
   \litc{\RETURNE{\ilviol{v}{d}{t}{a}{\l}}}{use function for interior nodes}
\lioc{\ELSE}{$v$ is a border node}
   \litc{\RETURNE{\blviol{v}{d}{t}{a}{\l}}}{use function for border nodes}
\lion{\FI}
\end{tabbing}
\vspace{-3ex}

\setcounter{lctr}{0}
\begin{tabbing}\label{alg:check-scViol}
mm\= mm\= mm\= mm\= mm\=\kill
$\ilviol{v}{d}{t}{a}{\l}$\\
//Precondition: $v$ is a not border node of $\dsk{a}{\l}$\\
\lio{\IFC{d = 1}}
   \litc{\IFC{(\neg \ex u : u \ar v \not\in \wfg{\B}{t})}\ \RETURNE{\ttt}\ \FI}{$v$ has no incoming wait-for-edges}
   \litc{\IFC{(\neg \ex u : v \ar u \not\in \wfg{\B}{t})}\ \RETURNE{\ttt}\ \FI}{$v$ has no outgoing wait-for-edges}
   \lit{\RETURNE{\fff}}
\lio{\FI}

\lio{\IFC{\mbox{$v$ is a component $\B_i$}}}
   \lit{\IFC{(\ex \act : \B_i \ar \act \in \wfg{\B}{t} \land \lviol{\act}{d-1}{t}{a}{\l})} \ \RETURNE{\ttt}\ \FI}
   \lit{\IFC{(\fa \act : \act \ar \B_i \in \wfg{\B}{t} \imp \lviol{\act}{d-1}{t}{a}{\l})} \ \RETURNE{\ttt}\ \FI}
   \lit{\RETURNE{\fff}}
\lio{\FI}


\lio{\IFC{\mbox{$v$ is an interaction $\act$}}}
   \lit{\IFC{(\fa \B_i : \act \ar \B_i \in \wfg{\B}{t} \imp \lviol{\B_i}{d-1}{t}{a}{\l})} \ \RETURNE{\ttt}\ \FI}
   \lit{\IFC{(\fa \B_i : \B_i \ar \act \in \wfg{\B}{t} \imp \lviol{\B_i}{d-1}{t}{a}{\l})} \ \RETURNE{\ttt}\ \FI}
   \lit{\RETURNE{\fff}}
\lion{\FI}
\end{tabbing}
\vspace{-5ex}
\caption{Formal definition of $\ilviol{v}{d}{t}{a}{\l}$}

\setcounter{lctr}{0}
\begin{tabbing}\label{alg:check-scViol}
mm\= mm\= mm\= mm\= mm\=\kill
$\blviol{v}{d}{t}{a}{\l}$\\
//Precondition: $v$ is a border node of $\dsk{a}{\l}$\\
\lio{\IFC{\mbox{$v$ is an interaction $\act$}}\ \RETURNE{\fff} \ \FI}

\lio{\IFC{\mbox{$v$ is a component $\B_i$ and $d > 1$}}}
   \lit{\IFC{(\ex \act : \B_i \ar \act \in \wfg{\B}{t} \land \lviol{\act}{d-1}{t}{a}{\l})} \ \RETURNE{\ttt}}
   \lit{\ELSE\ \RETURNE{\fff}}
   \lit{\FI}
\lion{\FI}
%% \lio{\IFC{d = 1}}
%%    \litc{\IFC{(\neg \ex u : u \ar v \not\in \wfg{\B}{t})}\ \RETURNE{\ttt}\ \FI}{$v$ has no incoming wait-for-edges}
%%    \litc{\IFC{(\neg \ex u : v \ar u \not\in \wfg{\B}{t})}\ \RETURNE{\ttt}\ \FI}{$v$ has no outgoing wait-for-edges}
%%    \lit{\RETURNE{\fff}}
%% \lio{\FI}
\end{tabbing}
\vspace{-5ex}
\caption{Formal definition of $\lviol{v}{d}{t}{a}{\l}$}
\label{fig:scViolateLoc}
\end{figure}





\clearpage




We now seek a local condition, which we evaluate in $\dsk{a}{\l}$, and which implies \GAO.
To achieve a conservative approximation, we make the ``pessimistic'' assumption that the violation status of border
nodes of $\dsk{a}{\l}$ cannot be known, since it depends on nodes outside of $\dsk{a}{\l}$.
Now, if an internal node $v$ of $\dsk{a}{\l}$ can be marked with a level $d$ $SC$-violation, by applying 
Definition~\ref{def:supercycle-violation} only within $\dsk{a}{\l}$, and with the border nodes makrked non-violating,
then it is also the case, as we show below, that $v$ has a level $d$ $SC$-violation overall.




We define the predicate $\lviol{v}{d}{t}{a}{\l}$ to hold iff node $v$ in $\wfg{B}{t}$ has a level-$d$ supercycle-violation
\emph{that can be confirmed within $\dsk{a}{\l}$}.


\bd[Local supercycle violation, $\lviol{v}{d}{t}{a}{\l}$]
\label{def:supercycle.violation.local}
We define  $\lviol{v}{d}{t}{a}{\l}$ by induction on $d$, as follows.

\noindent
\ul{Base case, $d=1$.} $\lviol{v}{1}{t}{a}{\l} = \true$  iff $v$ is an interior node of $\dsk{a}{\l}$ that has either
no incoming wait-for-edges or $v$ has no outgoing wait-for-edges. A border node cannot be verified to have a level-1
supercycle-violation, since it may wait-for-edge to/from nodes outside of $\dsk{a}{\l}$.

\noindent
\ul{Inductive step, $d > 1$.} $\lviol{v}{1}{t}{a}{\l} = \true$ iff any of the following cases hold. Otherwise 
$\lviol{v}{1}{t}{a}{\l} = \false$.

% A node $v$ in $\wfg{B}{t}$ has a level-$d$ supercycle-violation iff 
%$v$ has a level-$d$ out-violation or $v$ has a level-$d$ in-violation.
%We now define these by cases, depending on whether $v$ is a component or an interaction. 

\be

\item \ul{$v$ is a component $B$ and a border node of $\dsk{a}{\l}$}
   \be

   \item there exists interaction $\act$ such that $B_i \ar \act \in \wfg{\dsk{a}{\l}}{t}$ and $\lviol{\act}{d-1}{t}{a}{\l} = \true$,
         \ie $\B_i$ enables an interaction $\act$ which has a level-$(d-1)$ supercycle-violation in $\dsk{a}{\l}$. 

   \ee
Since this conditions requires only some interaction $\act$, rather than all, it can be verified within $\dsk{a}{\l}$ if
a suitable interaction $\act$ within $\dsk{a}{\l}$ exists.



\item \ul{$v$ is a component $B$ and an internal node of $\dsk{a}{\l}$, and either }
   \be

   \item there exists interaction $\act$ such that $B_i \ar \act \in \wfg{\dsk{a}{\l}}{t}$ and $\lviol{\act}{d-1}{t}{a}{\l} = \true$,
         \ie $\B_i$ enables an interaction $\act$ which has a level-$(d-1)$ supercycle-violation in $\dsk{a}{\l}$, or

   \item for all interactions $\act$ such that $\act \ar B_i \in \wfg{\dsk{a}{\l}}{t}$, $\lviol{\act}{d-1}{t}{a}{\l} = \true$,
         \ie all interactions $\act$ that wait for $\B_i$ have a level-$(d-1)$ supercycle-violation in $\dsk{a}{\l}$.

   \ee


\item \ul{$v$ is an interaction $\act$ and an internal node of $\dsk{a}{\l}$, and either}
   \be

   \item for all components $\B_i$ such that $\act \ar \B_i \in \wfg{\dsk{a}{\l}}{t}$, $\lviol{\B_i}{d-1}{t}{a}{\l} = \true$,
         \ie all components $\B_i$ that $\act$ waits for have a level-$(d-1)$ supercycle-violation in $\dsk{a}{\l}$,

   \item for all components $\B_i$ such that $\B_i \ar \act \in \wfg{\dsk{a}{\l}}{t}$, $\lviol{\B_i}{d-1}{t}{a}{\l} = \true$,
         \ie all  components $\B_i$ that enable $\act$ have a level-$(d-1)$ supercycle-violation in $\dsk{a}{\l}$.

   \ee

\ee
Note that if $v$ is an interaction and a border node, then $\lviol{v}{d}{t}{a}{\l}$ is false,for all $d$.   This is
because $v$ may have both incoming and outgoing wiat-for-edges from.to nodes outside of $\dsk{a}{\l}$, and these edges
may cause $v$ to be in a strongly connected supercycle.

Figure~\ref{fig:scViolateLoc} gives a formal definition of $\lviol{v}{d}{t}{a}{\l}$. Border and interior nodes are
handled by separate functions $\blviol{v}{d}{t}{a}{\l}, \ilviol{v}{d}{t}{a}{\l}$, respectively, and the base case $d=1$
is split across these. Note that $\blviol{v}{d}{t}{a}{\l}$ always returns $\false$ for $d=1$.

\ed




\bd[$\LAO(a, \l)$] \label{def:lao}
Let $s_a \goesto[a] t_a$ be a reachable transition of $\dsk{a}{\l}$.
Then, in $t_a$, the following holds. 
For every component $\B_i$ of $\cmps{a}$:  
$\B_i$ has a supercycle-violation that can be confirmed within $\dsk{a}{\l}$.
Formally,\\
\ind  $\fa \B_i \in \cmps{a}, \ex d \ge 1: \lviol{\B_i}{d}{t}{a}{\l}$.
\ed

We showed previously that $\GAO$ impleis deadlock-freedom, and so it remains to establish that $\LAO$ implies $\GAO$. 


\bp
$\fa d : \lviol{v}{d}{t}{a}{\l} \imp \viol{v}{d}{t}$
\ep
\prf{
Proof is by induction on $d$. 

\noindent
\ul{Base case, $d=1$.} Assume $\lviol{v}{1}{t}{a}{\l}$ for some node $v$. Then, by 
Definition~\ref{def:supercycle.violation.local}, $v$ cannot be a border node of $\dsk{a}{\l}$.
Hence $v$ is an interior node of $\dsk{a}{\l}$, and either has no incoming wait-for-edges or no outgoing 
wait-for-edges. Therefore, in $\wfg{\B}{t}$, it is still the case that $v$ either has no incoming wait-for-edges or no outgoing 
wait-for-edges. Hence $\viol{v}{1}{t}$ holds.


\noindent
\ul{Inductive step, $d > 1$.}
Assume $\lviol{v}{d}{t}{a}{\l}$ for some node $v$ and some $d > 1$. 
We proceed by cases on Definition~\ref{def:supercycle.violation.local}.

\be

\item \ul{$v$ is a component $B$ and a border node of $\dsk{a}{\l}$}

Hence there exists interaction $\act$ such that $B_i \ar \act \in \wfg{\dsk{a}{\l}}{t}$ and $\lviol{\act}{d-1}{t}{a}{\l}
= \true$. By the induction hypothesis applied to $\lviol{\act}{d-1}{t}{a}{\l}$, we have $\viol{\act}{d-1}{t}$. Hence 
Since $\wfg{\dsk{a}{\l}}{t} \sub \wfg{\B}{t}$ by construction, we have 
$B_i \ar \act \in \wfg{\B}{t}$ and $\viol{\act}{d-1}{t}$.
Hence by Definition~\ref{def:supercycle-violation}, Clause~\ref{def:supercycle.violation.component.out}, 
we have $\viol{v}{d}{t}$.


\item \ul{$v$ is a component $B$ and an internal node of $\dsk{a}{\l}$, and either }
   \be

   \item there exists interaction $\act$ such that $B_i \ar \act \in \wfg{\dsk{a}{\l}}{t}$ and $\lviol{\act}{d-1}{t}{a}{\l} = \true$,
         \ie $\B_i$ enables an interaction $\act$ which has a level-$(d-1)$ supercycle-violation in $\dsk{a}{\l}$, or

   \item for all interactions $\act$ such that $\act \ar B_i \in \wfg{\dsk{a}{\l}}{t}$, $\lviol{\act}{d-1}{t}{a}{\l} = \true$,
         \ie all interactions $\act$ that wait for $\B_i$ have a level-$(d-1)$ supercycle-violation in $\dsk{a}{\l}$.

   \ee


\item \ul{$v$ is an interaction $\act$ and an internal node of $\dsk{a}{\l}$, and either}
   \be

   \item for all components $\B_i$ such that $\act \ar \B_i \in \wfg{\dsk{a}{\l}}{t}$, $\lviol{\B_i}{d-1}{t}{a}{\l} = \true$,
         \ie all components $\B_i$ that $\act$ waits for have a level-$(d-1)$ supercycle-violation in $\dsk{a}{\l}$,

   \item for all components $\B_i$ such that $\B_i \ar \act \in \wfg{\dsk{a}{\l}}{t}$, $\lviol{\B_i}{d-1}{t}{a}{\l} = \true$,
         \ie all  components $\B_i$ that enable $\act$ have a level-$(d-1)$ supercycle-violation in $\dsk{a}{\l}$.

   \ee

\ee
Note that if $v$ is an interaction and a border node, then $\lviol{v}{d}{t}{a}{\l}$ is false,for all $d$.   This is
because $v$ may have both incoming and outgoing wiat-for-edges from.to nodes outside of $\dsk{a}{\l}$, and these edges
may cause $v$ to be in a strongly connected supercycle.

Figure~\ref{fig:scViolateLoc} gives a formal definition of $\lviol{v}{d}{t}{a}{\l}$. Border and interior nodes are
handled by separate functions $\blviol{v}{d}{t}{a}{\l}, \ilviol{v}{d}{t}{a}{\l}$, respectively, and the base case $d=1$
is split across these. Note that $\blviol{v}{d}{t}{a}{\l}$ always returns $\false$ for $d=1$.

}



\bd[$\LAO(a, \l)$] \label{def:lao}
Let $s_a \goesto[a] t_a$ be a reachable transition of $\dsk{a}{\l}$.
Then, in $t_a$, the following holds. 
For every component $\B_i$ of $\cmps{a}$:  
$\B_i$ has a supercycle-violation that can be confirmed within $\dsk{a}{\l}$.
Formally,\\
\ind  $\fa \B_i \in \cmps{a}, \ex d \ge 1: \lviol{\B_i}{d}{t}{a}{\l}$.
\ed

We showed previously that $\GAO$ impleis deadlock-freedom, and so it remains to establish that $\LAO$ implies $\GAO$. 


\bp
$\fa d : \lviol{v}{d}{t}{a}{\l} \imp \viol{v}{d}{t}$
\ep
\prf{
Proof is by induction on $d$. 

\noindent
\ul{Base case, $d=1$.} Assume $\lviol{v}{1}{t}{a}{\l}$ for some node $v$. Then, by 
Definition~\ref{def:supercycle.violation.local}, $v$ cannot be a border node of $\dsk{a}{\l}$.
Hence $v$ is an interior node of $\dsk{a}{\l}$, and either has no incoming wait-for-edges or no outgoing 
wait-for-edges. Therefore, in $\wfg{\B}{t}$, it is still the case that $v$ either has no incoming wait-for-edges or no outgoing 
wait-for-edges. Hence $\viol{v}{1}{t}$ holds.


\noindent
\ul{Inductive step, $d > 1$.}
Assume $\lviol{v}{d}{t}{a}{\l}$ for some node $v$ and some $d > 1$. 
We proceed by cases on Definition~\ref{def:supercycle.violation.local}.

\be

\item \ul{$v$ is a component $B$ and a border node of $\dsk{a}{\l}$}

Hence there exists interaction $\act$ such that $B_i \ar \act \in \wfg{\dsk{a}{\l}}{t}$ and $\lviol{\act}{d-1}{t}{a}{\l}
= \true$. By the induction hypothesis applied to $\lviol{\act}{d-1}{t}{a}{\l}$, we have $\viol{\act}{d-1}{t}$. Hence 
Since $\wfg{\dsk{a}{\l}}{t} \sub \wfg{\B}{t}$ by construction, we have 
$B_i \ar \act \in \wfg{\B}{t}$ and $\viol{\act}{d-1}{t}$.
Hence by Definition~\ref{def:supercycle-violation}, Clause~\ref{def:supercycle.violation.component.out}, 
we have $\viol{v}{d}{t}$.


\item \ul{$v$ is a component $B$ and an internal node of $\dsk{a}{\l}$}
We have two subcases.
   \be

   \item there exists interaction $\act$ such that $B_i \ar \act \in \wfg{\dsk{a}{\l}}{t}$ and $\lviol{\act}{d-1}{t}{a}{\l} = \true$.

Hence there exists interaction $\act$ such that $B_i \ar \act \in \wfg{\dsk{a}{\l}}{t}$ and $\lviol{\act}{d-1}{t}{a}{\l}
= \true$. By the induction hypothesis applied to $\lviol{\act}{d-1}{t}{a}{\l}$, we have $\viol{\act}{d-1}{t}$. Hence 
Since $\wfg{\dsk{a}{\l}}{t} \sub \wfg{\B}{t}$ by construction, we have 
$B_i \ar \act \in \wfg{\B}{t}$ and $\viol{\act}{d-1}{t}$.
Hence by Definition~\ref{def:supercycle-violation}, Clause~\ref{def:supercycle.violation.component.out}, 
we have $\viol{v}{d}{t}$.


   \item for all interactions $\act$ such that $\act \ar B_i \in \wfg{\dsk{a}{\l}}{t}$, $\lviol{\act}{d-1}{t}{a}{\l} = \true$.

Hence for all interactions $\act$ such that $\act \ar \B_i \in \wfg{\dsk{a}{\l}}{t}$, we have $\lviol{\act}{d-1}{t}{a}{\l}$.
By the induction hypothesis applied to $\lviol{\act}{d-1}{t}{a}{\l}$, we have 
for all interactions $\act$ such that $\act \ar \B_i \in \wfg{\dsk{a}{\l}}{t}: \viol{\act}{d-1}{t}$.

Now $\wfg{\dsk{a}{\l}}{t} \sub \wfg{\B}{t}$ by construction. Since $\B_i$ is an internal node of $\dsk{a}{\l}$, 
all edges $\act \ar \B_i \in \wfg{\B}{t}$ are also edges in $\dsk{a}{\l}$. Hence, from the above, we have
for all $\act \ar B_i \in \wfg{\B}{t} : \viol{\act}{d-1}{t}$.
Hence by Definition~\ref{def:supercycle-violation}, Clause~\ref{def:supercycle.violation.component.in}, 
we have $\viol{v}{d}{t}$.

   \ee


\item \ul{$v$ is an interaction $\act$ and an internal node of $\dsk{a}{\l}$.}

We have two subcases.
   \be

   \item for all components $\B_i$ such that $\act \ar \B_i \in \wfg{\dsk{a}{\l}}{t}$, $\lviol{\B_i}{d-1}{t}{a}{\l} = \true$.

By the induction hypothesis applied to $\lviol{\B_i}{d-1}{t}{a}{\l}$, we have $\viol{\B_i}{d-1}{t}$. 
Now $\wfg{\dsk{a}{\l}}{t} \sub \wfg{\B}{t}$ by construction.
Since $\act$ is an internal node of $\dsk{a}{\l}$,
all edges $\act \ar \B_i \in \wfg{\B}{t}$ are also edges in $\dsk{a}{\l}$.
Hence, for all components $\B_i$ such that $\act \ar \B_i \in \wfg{\B}{t}: \viol{\B_i}{d-1}{t}$.
By Definition~\ref{def:supercycle-violation}, Clause~\ref{def:supercycle.violation.interaction.out}, 
$\viol{\B_i}{d-1}{t}$.

   \item for all components $\B_i$ such that $\B_i \ar \act \in \wfg{\dsk{a}{\l}}{t}$, $\lviol{\B_i}{d-1}{t}{a}{\l} = \true$.

By the induction hypothesis applied to $\lviol{\B_i}{d-1}{t}{a}{\l}$, we have $\viol{\B_i}{d-1}{t}$. 
Now $\wfg{\dsk{a}{\l}}{t} \sub \wfg{\B}{t}$ by construction.
Since $\act$ is an internal node of $\dsk{a}{\l}$,
all edges $\B_i \ar \act \in \wfg{\B}{t}$ are also edges in $\dsk{a}{\l}$.
Hence, for all components $\B_i$ such that $\B_i \ar \act \in \wfg{\B}{t}: \viol{\B_i}{d-1}{t}$.
By Definition~\ref{def:supercycle-violation}, Clause~\ref{def:supercycle.violation.interaction.in}, 
$\viol{\B_i}{d-1}{t}$.

   \ee

\ee
}




\bl \label{lemma:loc.ANDOR.implies.glob.AND-OR}
$\LAO(a, \l)$ implies $\GAO$
\el



\bt[Deadlock-freedom using $\LAO(a, \l)$]
\label{theorem:local.ANDOR.deadlock-free}
Assume that 
\bn
\item for all $s_0 \in Q_0$, $\wfg{B}{s_0}$ is supercycle-free, and
\item for all interactions $a$ of $B$ ($a \in \gamma$), $\LAO(a,\l)$ holds for some $\l \ge 0$.
\en
Then for every reachable state $u$ of $(B, Q_0)$:  $\wfg{B}{u}$ is supercycle-free.
\et
%
\prf{
Let $a$ be an arbitrary interaction in $\gamma$.
By $(\ex \l: \LAO(a,\l))$ and Lemma~\ref{lemma:loc.ANDOR.implies.glob.AND-OR}, we have $\GAO(a)$.
By Theorem~\ref{theorem:global.ANDOR.deadlock-free},
for every reachable state $u$ of $(B, Q_0)$:  $\wfg{B}{u}$ is supercycle-free.
}


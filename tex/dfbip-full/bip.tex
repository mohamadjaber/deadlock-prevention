%\section{BIP - Behavior Interaction Priority}

BIP is a component framework for constructing
systems by superposing three layers of modeling: Behavior,
Interaction, and Priority.
%
A technical treatment of priority is beyond the scope of this paper. Adding
priorities never introduces a deadlock, since priority enforces a choice between
possible transitions from a state, and deadlock-freedom means that there is at
least one transition from every (reachable) state.  Hence if a BIP system
without priorities is deadlock-free, then the same system with priorities
added will also be deadlock-free.

\bd[Atomic Component]
An  {\em atomic component} $\B_i$ is a labeled transition system represented by a triple
$(Q_i, P_i, \goesto_i)$ where $Q_i$ is a set of {\em states}, $P_i$ is a set of {\em communication ports}, and
$\goesto_i\, \subseteq Q_i \times P_i \times Q_i$ is a set of {\em possible transitions}, each labeled by some port.
\ed

For states $s_i, t_i \in Q_i$ and port $p_i \in P_i$, write $s_i  \goesto[p_i]_i t_i$, iff
$(s_i,p_i,t_i) \in\,\goesto_i$. When $p_i$ is irrelevant, write 
$s_i \goesto_i t_i$. Similarly, $s_i \goesto[p_i]_i$ means that there
exists $t_i \in Q_i$ such that
$s_i \goesto[p_i]_i t_i$. In this case, $p_i$ is \emph{enabled} in state $s_i$.
Ports are used for communication between different components, as
discussed below.

In practice, we describe the transition system using some syntax, \eg
involving variables.  We abstract away from issues of syntactic
description since we are only interested in enablement of ports and
actions. We assume that enablement of a port depends only on the local
state of a component. In particular, it cannot depend on the state of
other components. This is a restriction on BIP, and we defer to
subsequent work how to lift this restriction.  So, we assume the
existence of a predicate $\gd{p_i}{i}$ that holds 
in state $s_i$ of component $\B_i$ iff port $p_i$ is enabled in $s_i$, 
\ie $s_i(\gd{p_i}{i}) = \true \mbox{ iff } s_i \goesto[p_i]_i$.

Figure~\ref{fig:components} shows atomic components for a philospher $P$
and a fork $F$ in dining philosophers.
%
A philosopher $P$ that is hungry (in state $h$) can eat by
executing $\get$ and moving to state $e$ (eating). From $e$, $P$
releases its forks by executing $\release$ and moving back to $h$.
%
Adding the thinking state does not change the deadlock behaviour of the system, since the
thinking to hungry transition is internal to $P$, and so we omit it.
%
A fork $F$ is taken by either: (1) the left philosopher 
(transition $\get_l$) and so moves to state $u_l$ (used by left
philosopher), or (2) the right philosopher  (transition
$\get_r$) and so moves to state $u_r$ (used by right
philosopher). From state $u_r$ (resp. $u_l$), $F$ is released by the
right philosopher (resp. left philosopher) and so moves back to
state $f$ (free).



\bd[Interaction] For a given system built from a set of $n$ atomic components 
$\{\B_i = (Q_i, P_i, \goesto_i)\}_{i=1}^n$, we require that their respective sets of ports are
pairwise disjoint,
\ie for all $i, j$ such that $i, j \in \oneton \land i \ne j$, we have $P_i \ints P_j = \emptyset$.
An {\em interaction} is a set of ports not containing two or more ports from the same component.
That is, for an interaction $\act$ we have 
$a \subseteq P \land (\fa i \in \set{1..n}: |\act \ints P_i| \le 1)$, where 
$P = \bigcup_{i=1}^n P_i$ is the set of all ports in the
system. When we write $a = \{p_i\}_{i \in I}$, we assume that
$p_i \in P_i$ for all $i \in I$, where $I \subseteq \set{1..n}$.
\ed

\noindent
Execution of an interaction $\act$ involves all the components which have
ports in $\act$.  



\begin{figure}[t]
  \begin{center}
    \mbox{
      \subfigure[Philosopher $P$ and fork $F$ atomic components.]{\label{fig:components}\scalebox{0.4}{\input{figs/philAndFork.pdf_t}}} \quad
      \subfigure[Dining philosophers composite component with four philosophers.]{\label{fig:diningbip}\scalebox{0.35}{\input{figs/diningbip.pdf_t}}}
      }
    \caption{Dining philosophers.}
    \label{fig:diningSpectrum}
  \end{center}
\end{figure}








\bd[Composite Component]\label{def.bip.composition} A {\em composite
  component} (or simply {\em component}) 
 $\B \df \gamma(\B_1,\dots,\B_n)$
is defined by a composition
operator parameterized by a set of interactions $\gamma \subseteq
2^P$.  $\B$ has a transition system
$(Q,\gamma, \goestog)$, where %$Q = \bigotimes_{i=1}^n Q_i$ and
$Q = Q_1 \times \cdots \times Q_n$ and
$\goestog \sub Q \times \gamma \times Q$ is the least set of transitions satisfying the rule
%
\begin{mathpar}
\inferrule
{
    \act = \{p_i\}_{i\in I}\in \gamma\\
    \forall i\in I: s_i \goesto[p_i]_i t_i\\
    \forall i\not\in I: s_i = t_i
}
{
    \tpl{s_1,\dots,s_n} \goestog[\act] \tpl{t_1,\dots,t_n}
}
\end{mathpar}
\ed
%
This inference rule says that a composite component $\B = \gamma(\B_1,\dots,\B_n)$ can
execute an interaction $\act \in \gamma$, iff for each port $p_i \in \act$, the
corresponding atomic component $\B_i$ can execute a transition labeled with
$p_i$; the states of components that do not participate in the interaction stay
unchanged. 
%
Given an interaction $\act = \{p_i\}_{i \in I}$, we denote by $\cmps{\act}$ the
set of atomic components participating in $\act$, formally:
$\cmps{\act} = \{\B_i \st p_i \in \act\}$. 
%Given a component $\B_i$, we denote by $A_i$ the
%interactions that $\B_i$ participates in, formally: $A_i = \set{a \st P_i \ints a \ne \emptyset}$.
Figure \ref{fig:diningbip} shows a composite component consisting of
four philosophers and the four forks between them. Each philosopher
and its two neighboring forks share two interactions: 
$\Get = \set{\get, \use_l, \use_r}$ in which the philosopher obtains the forks, and 
$\Rel = \set{\release, \free_l, \free_r}$ in which the philosopher releases the forks.



\bd[Interaction enablement]\label{def.bip.enablement} An atomic
component $\B_i = (Q_i,P_i,\goesto_i)$ enables a port $p_i \in P_i$ in state $s_i$ iff $s_i \goesto[p_i]_i$.
$\B_i$ enables interaction $\act$ in 
state $s_i$ iff $s_i \goesto[p_i]_i$, where $p_i = P_i \ints a$ is the port of $\B_i$ involved in $\act$.
That is, $\B_i$ enables $\act$ in state $s_i$ iff $\B_i$ enables port $\act \ints P_i$ in state $s_i$. 

Let $\gd{p_i}{i}$ denote the enablement condition for port $p_i$ in component $\B_i$, that is, $\gd{a}{p_i}$ holds iff
the current state of $\B_i$ is an $s_i$ such that $s_i \goesto[p_i]_i$.
Let $\gd{a}{i}$ denote the enablement condition for interaction $\act$ in
component $\B_i$, that is,  $\gd{a}{i} = \gd{p_i}{i}$ where $p_i = a \ints P_i$.  
%$\gd{a}{i}$ holds iff the current state of $\B_i$ is an $s_i$ such that $s_i \goesto[p_i]_i$.

Let $\B = \gamma(\B_1,\dots,\B_n)$ be a composite component, and let $s =
\tpl{s_1,\ldots,s_n}$ be a state of $\B$.  Then $\B$ enables $\act$ in $s$
iff every $\B_i \in \cmps{\act}$ enables $\act$ in $s_i$.  
\ed
%
The definition of  interaction enablement is a consequence of 
Definition~\ref{def.bip.composition}.
Interaction $a$ being enabled in state $s$ means that executing
$a$ is one of the possible transitions that can be taken from $s$.

To avoid pathological cases of deadlock due solely to a single component refusing to enable any interaction at all, 
we assume that every component always enables at least one interaction.

\bd[Local Enablement Assumption] \label{def.bip.local-enablement}
For every component  $\B_i = (Q_i,P_i,\goesto_i)$, the followingholds. In every $s_i \in Q_i$, $\B_i$ enables some
interaction $\act$.
\ed




\bd[BIP System]\label{def.bip.system} Let $\B = \gamma(\B_1,\dots,\B_n)$ be a composite component with transition system $(Q,\gamma,
\goestog)$, and let $Q_0 \sub Q$ be a set of initial states. Then $(\B,Q_0)$ is a BIP system.  \ed
%We usually refer to $B$ as a BIP system, understanding that $Q_0$ is implicit.

\noindent
Figure \ref{fig:diningbip} gives a BIP-system with philosophers initially in state $h$ (hungry) and forks initially in
state $f$ (free).

To avoid tedious repetition, we fix, for the rest of the paper, an arbitrary BIP-system $(\B, Q_0)$, with
$\B \df \gamma(\B_1,\dots,\B_n)$, and transition system $(Q,\gamma, \goestog)$.





\bd[Execution]\label{def.bip.execution} Let $(\B, Q_0)$ be a BIP system
with transition system $(Q,\gamma, \goestog)$.  
Let $\rho = s_0 a_1 s_1 \ldots s_{i-1} a_i s_i \ldots$ be an alternating sequence of
states of $\B$ and interactions of $\B$. Then $\rho$ is an execution of
$(\B, Q_0)$ iff (1) $s_0 \in Q_0$, and (2) $\fa i > 0: s_{i-1} \goesto[a_i] s_i$.  \ed



\bd[Reachable state, transition]\label{def.bip.reachable}
A state or transition that occurs in some execution is called \emph{reachable}.

\ed


\bd[State Projection]\label{def.bip.state.projection} Let $(\B, Q_0)$
be a BIP system where $\B = \gamma(\B_1,\dots,\B_n)$ and let $s =
\tpl{s_1,\dots,s_n}$ be a state of $(B, Q_0)$. Let 
$\set{\B_{j_1},\dots,\B_{j_k}} \sub \set{\B_1,\dots,\B_n}$. Then $s \pj
\set{\B_{j_1},\dots,\B_{j_k}} \df \tpl{s_{j_1},\dots,s_{j_k}}$. For a
single $\B_i$, we write $s \pj \B_i = s_i$.
%
We extend state projection to sets of states element-wise.
\ed


\bd[Subcomponent]\label{def.bip.subcomponent}
Let $\B \df \gamma(\B_1,\dots,\B_n)$ be a composite component, and let 
$\set{\B_{j_1},\dots,\B_{j_k}}$ be a subset of $\set{\B_{1},\dots,\B_{n}}$.
Let $P' = P_{j_1} \un \cdots \un P_{j_k}$, \ie the union of the ports of $\set{\B_{j_1},\dots,\B_{j_k}}$.
Then the subcomponent $\B'$ of $\B$ based on $\set{\B_{j_1},\dots,\B_{j_k}}$ 
is as follows:
\bn
\item $\gamma' \df \set{ \act \ints P' \st \act \in \gamma \land \act \ints P' \ne \emptyset}$
\item $\B' \df \gamma'(\B_{j_1},\dots,\B_{j_k})$ 
\en
\ed
That is, $\gamma'$ consists of those interactions in $\gamma$ that have at least one participant in 
$\set{\B_{j_1},\dots,\B_{j_k}}$, and restricted to the participants in $\set{\B_{j_1},\dots,\B_{j_k}}$,
\ie participants not in $\set{\B_{j_1},\dots,\B_{j_k}}$ are removed.

We write $s \pj \B'$ to indicate state projection onto $\B'$, and define 
$s \pj \B' \df  s \pj \set{\B_{j_1}, \dots, \B_{j_k}}$, where $\B_{j_1},\dots,\B_{j_k}$ are the atomic
components in $\B'$.


%MENTION THAT THE INTERACTION $a$ SHOULD ALSO BE PROJECTED. WHEN WE WRITE $a$ IN A TRANSITION $s_{k-1} \goestog[a] s_k$
%OF THE POROJECTED SUBSYSTEM, WE REALLY MEAN THE PROJECTION OF $a$. SINCE $a$ IS JUST USED AS A LABEL, THERE IS NO HARM
%DONE, BUT WE KEEP IN MIND THAT THE PROJECTION OF $a$ ONTO THE SUBSYSTEM IS MEANT




\bd[Subsystem]\label{def.bip.subsystem}
Let $(\B, Q_0)$ be a BIP system where $\B = \gamma(\B_1,\dots,\B_n)$, and let 
$\set{\B_{j_1},\dots,\B_{j_k}}$ be a subset of $\set{\B_{1},\dots,\B_{n}}$.
Then the subsystem $(\B', Q'_0)$ of  $(\B, Q_0)$ based on $\set{\B_{j_1},\dots,\B_{j_k}}$ 
is as follows:
\bn
\item $\B'$ is the subcomponent of $\B$ based on $\set{\B_{j_1},\dots,\B_{j_k}}$ 
\item $Q'_0 = Q_0 \pj \set{\B_{j_1},\dots,\B_{j_k}}$
\en
\ed

We write $s \pj \B'$ as an abbreviation for $s \pj \set{\B_{j_1},\dots,\B_{j_k}}$.



\bd[Execution Projection]\label{def.bip.execution.projection}
Let $(\B, Q_0)$ be a BIP system where $\B = \gamma(\B_1,\dots,\B_n)$, and let $(\B', Q'_0)$, with $\B'
= \gamma'(\B_{j_1},\dots,\B_{j_k})$ be the subsystem of $(\B, Q_0)$ based on $\set{\B_{j_1},\dots,\B_{j_k}}$.
Let $P' = P_{j_1} \un \cdots \un P_{j_k}$,\ie $P'$ is the set of ports of $(\B', Q'_0)$.
%
Let $\rho = s_0 \act_1 s_1 \ldots s_{i-1} \act_i s_i \ldots$ be an execution of $(\B, Q_0)$.  Then, $\rho \pj (\B', Q'_0)$, the projection
of $\rho$ onto $(B', Q'_0)$, is the sequence resulting from:
\bn
\item \label{def.clause.bip.execution.projection.state} 
replacing each $s_i$ by $s_i \pj \set{\B_{j_1},\dots,\B_{j_k}}$, \ie replacing each state by its
projection onto $\set{\B_{j_1},\dots,\B_{j_k}}$

\item \label{def.clause.bip.execution.projection.action} removing all $\act_i s_i$ where $\act_i \not \in \gamma'$ 

\item \label{def.clause.bip.execution.projection.port} replacing each $\act_i$ by $\act_i \ints P'$, \ie replacing each
interaction by its projection onto the port set $P'$ 

\en
\ed



\bp[Execution Projection]\label{prop.bip.execution.projection}
Let $(\B, Q_0)$ be a BIP system where $\B = \gamma(\B_1,\dots,\B_n)$, and let
$(\B', Q'_0)$, with $B' = \gamma'(\B_{j_1},\dots,\B_{j_k})$ be 
the subsystem of $(B, Q_0)$ based on $\set{\B_{j_1},\dots,\B_{j_k}}$.
Let $\rho = s_0 \act_1 s_1 \ldots s_{i-1} \act_i s_i \ldots$ be an execution of $(\B, Q_0)$. 
Then, $\rho \pj (\B', Q'_0)$ is an execution of $(\B', Q'_0)$.
\ep
%\begin{proof}
\prf{
Let $\rho = s_0 \act_1 s_1 \ldots s_{i-1} \act_i s_i \ldots$, be an execution of $(\B, Q_0)$,
and let $s'_0 = s_0 \pj \set{\B_{j_1},\dots,\B_{j_k}}$.

Then, by Definition~\ref{def.bip.execution.projection}, $s'_0 \in Q'_0$ and
$\rho \pj (B', Q'_0) = s'_0 b_1 s'_1 b_2 s'_2 \ldots$ for some $b_1 s'_1 b_2 s'_2 \ldots$,
where $s'_j \in Q' = Q \pj \set{\B_{j_1},\dots,\B_{j_k}}$ for $j \ge 1$.

Consider an arbitrary step $(s'_{j-1}, b_j, s'_j)$ of $\rho \pj (\B', Q'_0)$.
Since $b_j s'_j$ was not removed in Clause~\ref{def.clause.bip.execution.projection.action} of
Definition~\ref{def.bip.execution.projection}, we have 

\noindent
(1) $s'_j = s_k \pj \set{\B_{j_1},\dots,\B_{j_k}}$ for some $k > 0$ and such that $\act_k \in \gamma'$\\
(2) $b_j = \act_k$, and\\
(3) $s'_{j-1} = s_\l \pj \set{\B_{j_1},\dots,\B_{j_k}}$ for the smallest $\l$ such that\\
   \ind $\l < k$ and
        $\fa m : {\l+1 \leq m < k} : \act_{m} \not\in \gamma'$

From (3) and Definitions~\ref{def.bip.composition} and \ref{def.bip.execution.projection}
    $s_\l \pj \set{\B_{j_1},\dots,\B_{j_k}} = s_{k-1} \pj \set{\B_{j_1},\dots,\B_{j_k}}$.
Hence $s'_{j-1} = s_{k-1} \pj \set{\B_{j_1},\dots,\B_{j_k}}$.
%
From $s_{k-1} \goestog[\act_k] s_k$,    $\act_k \in \gamma'$, and
Definition~\ref{def.bip.composition}, we have 
        $s_{k-1} \pj \set{\B_{j_1},\dots,\B_{j_k}} \goestog[\act_k] s_k \pj \set{\B_{j_1},\dots,\B_{j_k}}$.
Hence 
        $s'_{j-1} \goestog[b_j] s'_j$
        from $s'_{j-1} = s_{k-1} \pj \set{\B_{j_1},\dots,\B_{j_k}}$ established above and (1), (2).
%

Since $(s'_{j-1}, b_j, s'_j)$ was arbitrarily chosen, we conclude that
every step of $\rho \pj (\B', Q'_0)$  is a step of $(\B', Q'_0)$. Since the first state of
$\rho \pj (\B', Q'_0)$  is $s_0$, and $s_0 \in Q'_0$, we have established
that $\rho \pj (\B', Q'_0)$ is an execution of $(\B', Q'_0)$.
}%\end{proof}






\bco \label{cor.bip.reachability}
\label{cor:bip.reachability}
Let $(\B', Q'_0)$ be a subsystem of $(\B, Q_0)$, and let $P'$ be the port set of $(\B', Q'_0)$.
Let $s$ be a reachable state of $(\B, Q_0)$. Then $s \pj \B'$ is a reachable state of  $(\B',Q'_0)$. Let $s \la{\act} t$ be
a reachable transition of $(\B, Q_0)$, and let $\act$ be an interaction of
 $(\B', Q'_0)$. Then  $s \pj \B' \la{\act \ints P'} t \pj \B'$ is a reachable transition of $(\B', Q'_0)$.
\eco
\prf{
Immediate corollary of Proposition~\ref{prop.bip.execution.projection}.
}







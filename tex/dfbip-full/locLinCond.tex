We now formulate a local version of $\GLin$.
Observe that if
$\widepth{\B}{\B_i}{t} < \omega \lor \wodepth{\B}{\B_i}{t} < \omega$,
then there is some finite $\l$ such that 
$\widepth{\B}{\B_i}{t} = \l \lor \wodepth{\B}{\B_i}{t} = \l$.


% OMIT THIS AND LEAVE TO ANOTHER PAPER
% %containing $\B_i$ and all nodes(components and interactions) have distance from $\B_i$ of up to $\l+1$.
% Then $\widepth{\B}{\B_i}{t} = \l \lor \wodepth{\B}{\B_i}{t} = \l$ can be verified in 
% the wait-for-graph of 
% $\ssg{i}{\l+1}$, since we verify either that all wait-for-paths ending in $\B_i$
% have length $\le \l$, or that 
% all wait-for-paths starting in $\B_i$ 
% have length $\le \l$.
% These conditions can be checked in $\ssg{i}{\l+1}$, since $\ssg{i}{\l+1}$
% contains every node in a
% wait-for-path of length $\l+1$ or less and which starts or ends in $\B_i$.
% %
% Since $\ssg{i}{\l+1} \sub \ssg{\act}{\l+2}$ for $\B_i \in \cmps{\act}$, we use 
% $\ssg{\act}{\l+2}$ instead of the set of subsystems 
% $\set{\ssg{i}{\l+1} : \B_i \in \cmps{\act}}$. 
% %% DISCUSS tradeoff: one larger $\ssg{a}{\l+2}$ instead of several smaller 
% %$\ssg{i}{\l+1}?
% %We leave analysis of the tradeoff between using one larger system
% %($\ssg{a}{\l+2}$) versus several smaller ones ($\ssg{i}{\l+1}$) to another paper.




\bd[$\LLin(\B, Q_0, \act, \l)$] \label{def:ldfc-k}
%Let $(B, Q_0)$ be a BIP system, with $B =\gamma(\B_1,\dots,\B_n)$.
%Let $a$ be an interaction of $(B, Q_0)$, \ie $a \in \gamma$.
Let $\l > 0$ and $s_\act \goesto[\act] t_\act$ be an arbitrary reachable transition of $\dsk{\act}{\l}$.
Then, in $t_\act$, the following holds. 
For every component $\B_i$ of $\cmps{\act}$:  
either $\B_i$ has in-depth less than $2\l - 1$, or out-depth less than $2\l - 1$, in $\wfg{\dsk{\act}{\l}}{t_\act}$. 
Formally,\\
\ind  $\fa \B_i \in \cmps{\act}: 
\widepth{\dsk{\act}{\l}}{\B_i}{t_\act} < 2\l-1 \lor \wodepth{\dsk{\act}{\l}}{\B_i}{t_\act} < 2\l-1$.
\ed


To infer deadlock-freedom in $(\B, Q_0)$ by checking $\LLin(\B, Q_0, a, \l)$,
% which is a condition on the behavior of the subsystem $\ds{a}$,
 we show that wait-for behavior in $\B$ ``projects down''
to any subcomponent $\B'$, and that wait-for behavior in $\B'$ ``projects up'' to $\B$.

% Let $e$ be a wait-for-edge with both endpoints in $B'$. Then if $e$ is
% present in the wait-for-graph of $B$ for state $s$, then $e$ is also present in the wait-for-graph
% $B'$ for $s \pj B'$ ($e$ projects down), and vice-versa ($e$ projects up).

%Thus $\wfg{B'}{s \pj B'}$ is a ``projection'' of $\wfg{\B}{s}$ onto $B'$.


\vspace{0.5ex}

Since wait-for-edges project up and down, it follows that wait-for-paths
project up and down, provided that the subsystem contains the entire wait-for-path.

\bp[In-projection, Out-projection] \label{prop:in-out-projection}
%Let $(B, Q_0)$ be a BIP system, 
Let $\l > 0$, let $\B_i$ be an atomic component of $\B$, and let 
$(\B', Q'_0)$ be a subsystem of $(\B, Q_0)$ which is based on a superset of $\ssg{a}{2\l}$.
Let $s$ be a state of $(B, Q_0)$, and $s' = s \pj B'$. Then
(1) $\widepth{\B}{\B_i}{s} < 2\l-1$ iff $\widepth{\B'}{\B_i}{s'} < 2\l-1$, and
(2) $\wodepth{\B}{\B_i}{s} < 2\l-1$ iff $\wodepth{\B'}{\B_i}{s'} < 2\l-1$.
\ep
\prfs{Follows from Defintion~\ref{def:depth},
Proposition~\ref{prop:edge-projection}, and the observation that $\wfg{\B'}{s'}$
contains all wait-for-paths of length $\le \l$ that start or end in $\B_i$.}
%
\prf{
We establish clause (1). The proof of clause (2) is analogous, except we replace paths ending in
$\B_i$ by paths starting from $\B_i$.
The proof of clause (1) is by double implication.



%%%%%%%%%%%%%%%%%%%%%%%%
\ul{$\widepth{\B}{\B_i}{s} < 2\l-1$ implies $\widepth{\B'}{\B_i}{s'} < 2\l-1$}:
%
Assume $\widepth{\B}{\B_i}{s} < 2\l-1$.
Let $\pi$ be an arbitrary wait-for path in $\wfg{\B'}{s'}$ that ends in $\B_i$. 
%
Since $(\B', Q'_0)$ is a subsystem of $(\B, Q_0)$, by Definition~\ref{def:static:wait-for-graph} and $s' = s \pj B'$,
$\wfg{\B'}{s'}$ is a subgraph of $\wfg{\B}{s}$.
%
Hence $\pi$ is a wait-for-path in $\wfg{\B}{s}$.
By $\widepth{\B}{\B_i}{s} < 2\l-1$, we have $|\pi| < 2\l-1$. 
Hence $\widepth{\B'}{\B_i}{s'} < 2\l-1$ since $\pi$ was arbitrarily chosen.




%%%%%%%%%%%%%%%%%%%%%%%%%%%%%%%%%%%%%%%%
\ul{$\widepth{\B'}{\B_i}{s'} < 2\l-1$ implies $\widepth{\B}{\B_i}{s} < 2\l-1$}:
%
Assume $\widepth{\B}{\B_i}{s} \ge 2\l-1$. Then there exists a wait-for path $\pi$ in $\wfg{\B}{s}$ such that 
$|\pi| \ge 2\l-1$. Let $\rho$ be the prefix of $\pi$ with length $2\l-1$. 
%
Since $(\B', Q'_0)$ is based on a superset of $\ssg{a}{2\l}$, and since the distance from $\B_i$ to the border of 
$\ssg{a}{2\l}$ is $2\l-1$, we conclude that $\rho$ is a wait-for path
that is wholly contained in $\wfg{\B'}{s'}$. Hence we have $\widepth{\B'}{\B_i}{s'} \ge 2\l-1$.
%
We have thus established 
$\widepth{\B}{\B_i}{s} \ge 2\l-1$ implies $\widepth{\B'}{\B_i}{s'} \ge 2\l-1$.
The contrapositive is the desired result.
}

%%%%%%%%%%%%%%%%%%%%%%%%%%%%%%%%%%%%%%%%%%%%%%%%%%%%%%



\vspace{0.5ex}

\noindent
We now show that $\LLin(\B, Q_0, \act, \l)$ implies $\GLin(\B, Q_0, \act)$, which in turn implies deadlock-freedom.  

\bl
\label{lemma:loc-implies-glob}
\label{lemma:locLinear-implies-globlinear}
%Let $(B, Q_0)$ be a BIP system, with $B = \gamma(\B_1,\dots,\B_n)$, and 
Let $\act$ be an interaction of $\B$, \ie $\act \in \gamma$.
If $\LLin(\B, Q_0, \act, \l)$ holds for some finite $\l > 0$, then $\GLin(\B, Q_0, \act)$ holds.
\el
%
%% \prfs{
%% Let $s \la{\act} t$ be a reachable transition of $(B, Q_0)$ and let 
%% $s_\act = s \pj \dsk{\act}{\l}$, $t_\act = t \pj \dsk{\act}{\l}$.
%% Then $s_\act \la{\act} t_\act$ is a reachable transition of $\dsk{\act}{\l}$ by 
%% Corollary~\ref{cor.bip.reachability}.
%% By $\LLin(\B, Q_0, \act, \l)$,
%% $\widepth{\dsk{\act}{\l}}{\B_i}{t_\act} \le \l \lor \wodepth{\dsk{\act}{\l}}{\B_i}{t_\act} \le \l$.
%% Hence by Proposition~\ref{prop:in-out-projection},
%% $\widepth{\B}{\B_i}{t} \le \l \lor \wodepth{\B}{\B_i}{t} \le \l$.
%% So
%% $\widepth{\B}{\B_i}{t} < \omega \lor \wodepth{\B}{\B_i}{t} < \omega$.
%% Hence $\GLin(\B, Q_0, \act)$ holds.}
%
\prf{
Let $s \la{\act} t$ be a reachable transition of $(\B, Q_0)$ and let 
$\B_i \in \cmps{\act}$, $s_\act = s \pj \dsk{\act}{\l}$, $t_\act = t \pj \dsk{\act}{\l}$.
Then $s_\act \la{\act} t_\act$ is a reachable transition of $\dsk{\act}{\l}$ by Corollary~\ref{cor.bip.reachability}.
By $\LLin(\B, Q_0, \act, \l)$, 
$\widepth{\dsk{\act}{\l}}{\B_i}{t_\act} < 2\l-1 \lor \wodepth{\dsk{\act}{\l}}{\B_i}{t_\act} < 2\l-1$.
Hence by Proposition~\ref{prop:in-out-projection},
$\widepth{\B}{\B_i}{t} < 2\l-1 \lor \wodepth{\B}{\B_i}{t} < 2\l-1$.
So
$\widepth{\B}{\B_i}{t} < \omega \lor \wodepth{\B}{\B_i}{t} < \omega$.
Hence $\GLin(\B, Q_0, \act)$.
}


\bt \label{thm:LLin.SC-free-preserving}
$\LLin$ is supercycle-freedom preserving
\et
\prf{
Follows immediately from Lemma~\ref{lemma:locLinear-implies-globlinear} and Theorem~\ref{thm:GLin.SC-free-preserving}.
}




%% \bt[Deadlock-freedom via \LLin] 
%% \label{theorem:local:deadlock-free}
%% \label{theorem:local.linear.deadlock-free}
%% %Let $(B, Q_0)$ be a BIP system, with $B = \gamma(\B_1,\dots,\B_n)$.
%% Assume that
%% \bn
%% \item \label{theorem:local.linear.deadlock-free.initial} 
%%       for all $s_0 \in Q_0$, $\wfg{\B}{s_0}$ is supercycle-free, and
%% \item \label{theorem:local.linear.deadlock-free.localLinear}
%%       for all interactions $a$ of $B$ ($a \in \gamma$), $\LDFC(a,\l)$ holds for some $\l \ge 0$.
%% \en 
%% Then for every reachable state $u$ of $(B, Q_0)$:  $\wfg{\B}{u}$ is supercycle-free.
%% \et
%% %
%% \prfs{
%% Immediate from Lemma~\ref{lemma:loc-implies-glob} and
%% Theorem~\ref{theorem:global:deadlock-free}. }
%% %
%% \prf{
%% Let $a$ be an arbitrary interaction in $\gamma$.
%% By $(\ex \l: \LLin(a,\l))$ and Lemma~\ref{lemma:locLinear-implies-globlinear}, we have $\GLin(a)$.
%% By Theorem~\ref{theorem:global.linear.deadlock-free},
%% for every reachable state $u$ of $(\B, Q_0)$:  $\wfg{\B}{u}$ is supercycle-free.
%% }




%% \paragraph{Complexity of evaluating $\LDFC(a, \l)$.}
%% Using explicit state enumeration, $\LDFC(a, \l)$ can be evaluated in
%% time $O(\SUM_{a \in \gamma} |\dsk{a}{\l}|)$, where $|\dsk{a}{\l}|$ denotes the
%% size of the transition system of $\dsk{a}{\l}$.








%%%%%%%%%%%%%%%%%%%%%%%%%%%%%%%%%%%%%%%%%%%%%%%%%%%%%%%%%
\subsection{Deadlock freedom using local and global restrictions}


\bt[Deadlock-freedom via \LAO, \LLin] 
\label{theorem:local.deadlock-free}
Assume that
\bn
\item \label{theorem:local.deadlock-free.initial}
      for all $s_0 \in Q_0$, $\wfg{\B}{s_0}$ is supercycle-free, and
\item \label{theorem:local.deadlock-free.scfPres}
      for all interactions $\act$ of $\B$ (\ie $\act \in \gamma$), one of
      the following holds:
      \bn
      \item $\GAO(\B,Q_0,\act)$
      \item $\GLin(\B,Q_0,\act)$
      \item $\ex \l > 0: \LAO(\B,Q_0,\act,\l)$ 
      \item $\ex \l > 0: \LLin(\B,Q_0,\act,\l)$ 
      \en
\en
Then for every reachable state $u$ of $(\B, Q_0)$:  $\wfg{\B}{u}$ is supercycle-free, and so 
$(\B, Q_0)$ is free of local deadlock.
\et
\prf{Immediate from
Theorems~\ref{thm:GAO.SC-free-preserving}, \ref{thm:GLin.SC-free-preserving}, \ref{thm:LAO.SC-free-preserving}, \ref{thm:LLin.SC-free-preserving}
and Corollary~\ref{cor:SC-free-preserving.deadlock-free}.}


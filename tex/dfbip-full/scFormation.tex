





\subsection{The supercycle formation condition}

\noindent
To proceed, we show that wait-for-edges not involving some interaction
$\act$ and its participants $\B_i \in \cmps{\act}$ are unaffected by the execution
of $\act$.  Say that edge $e$ in a wait-for-graph
is \emph{$\B_i$-incident} iff $\B_i$ is one of the endpoints of $e$.


\bp[Wait-for-edge preservation] \label{prop:wait-for-edge-preservation}
Let $s \la{\act} t$ be a transition of composite component $B = \gamma(\B_1,\dots,\B_n)$, and let $e$ be a wait-for edge
that is not $\B_i$-incident, for every $\B_i \in \cmps{\act}$. Then $e \in
\wfg{\B}{s}$ iff $e \in \wfg{\B}{t}$. 
\ep
%
\prfs{
Components not involved in the execution of $\act$ do not change state
along $s \la{\act} t$. Hence the endpoint of $e$ that is a
component has the same state in $s$ as in $t$. The proposition then
follows from Definition~\ref{def:static:wait-for-graph}. 
}
%
\prf{
Fix $e$ to be an arbitrary wait-for-edge that is not
$\B_i$-incident. $e$ is either $\B_j \ar b$ or $b \ar \B_j$, for some
component $\B_j$ of $B$ that is not in $\cmps{\act}$, and an interaction $b$
(different from $\act$) that $\B_j$ participates in.
%
Now $s \pj \B_j = t \pj \B_j$, since $s \la{\act} t$ and $\B_j \not\in \cmps{\act}$. Hence $s(\gd{b}{j}) =
t(\gd{b}{j})$. It follows from Definition~\ref{def:static:wait-for-graph} that 
$e \in \wfg{\B}{s}$ iff $e \in \wfg{\B}{t}$.
}



\bp[Supercycle formation condition] \label{prop:supercycle-formation}
Assume that $s \goesto[\act] t$ is a transition of $(\B, Q_0)$, $\wfg{\B}{s}$ is supercycle-free, and that $\wfg{\B}{t}$
contains a supercycle.  Then, in $\wfg{\B}{t}$, there exists a $\CC$ such that
\bn
\item $\CC$ is a subgraph of $\wfg{\B}{t}$
\item $\CC$ is strongly connected
\item $\CC$ is a supercycle
\item  there is no wait-for-edge from a node in $\CC$ to a node outside of $\CC$.
\item there exists a component $\B_i \in \cmps{\act}$ such that $\B_i$ is in $\CC$
\en
\ep
%
\prf{
By assumption, there is a supercycle $\SC$ that is a subgraph of $\wfg{\B}{t}$.
By Proposition~\ref{prop:supercycle:contains-mssc}, $\SC$ contains a
subgraph $\CC$ that is strongly connected, is itself a supercycle, and
such that there is no wait-for-edge from a node in $\CC$ to a node outside of $\CC$.
This establishes Clauses~1--4.

Now suppose $\B_i \not\in \CC$ for every $\B_i \in \cmps{\act}$. Then, no edge in $\CC$ is $\B_i$-incident. 
Hence, by Proposition~\ref{prop:wait-for-edge-preservation}, every edge in $\CC$ is an edge in 
$\wfg{\B}{s}$. Hence $\CC$ is a subgraph of $\wfg{\B}{s}$. 
%
Now in $\wfg{\B}{s}$, let $e$ by an arbitrary wait-for-edge from a node in $\CC$ to a node outside of $\CC$. $e$
cannot be present in $\wfg{\B}{t}$, by Clause~4. Hence $e$ is $\B_i$-incident, for some $\B_i \in \cmps{\act}$, by
Proposition~\ref{prop:wait-for-edge-preservation}.  Since $\B_i \not\in \CC$, $e$ must end in $\B_i$.
Let $v$ by an arbitrary node in $\CC$. If $v$ is a component, all its wait-for-edges must end in an interaction that is
in $\CC$, by the above, since an edge starting in a component must end in an interaction. 
If $v$ is an interaction, it must also have a wait-for-edge $e'$ to some component $\B_j \in \CC$, since $CC$ is
a supercycle in $\wfg{\B}{t}$, and $CC$ contains no  $\B_i \in \cmps{\act}$.
Hence $v$ has enough successors in $\CC$ to satisfy the supercycle definition (Def.~\ref{def:supercycle}), 
%
Hence $\CC$ by itself is a supercycle in $\wfg{\B}{s}$,
which contradicts the assumption that 
$\wfg{\B}{s}$ is supercycle-free. Hence, $\B_i \in \CC$ for some $\B_i \in \cmps{\act}$, and so Clause~5 is established.
}





\subsection{Violations of the supercycle formation condition}

We wish to check whether supercycles can be formed or not, \ie whether the supercycle formation
condition is satisfied or not. In principle, we could check directly
whether $\wfg{\B}{t}$ contains a supercycle, for each reachable state
$t$. However, this approach is subject to state-explosion, and so is usually unlikely to be viable in practice.
Instead, we formulate global conditions for supercycle-freedom, and then ``project'' these
conditions onto small subsystems, to obtain local versions of these conditions that are efficiently checkable.

For transition $s \goesto[\act] t$, we determine for every component $\B_i \in \cmps{\act}$ whether
it is possible for $\B_i$ to be a node in a supercycle in $\wfg{\B}{t}$. 
 If not, then we say that $\B_i$ satisfies the \emph{supercycle
  violation} condition, and we formalize this condition below.

There are two ways for $\B_i$ to not be a node in a strongly-connected supercycle:
\bn
\item \textit{no supercycle membership}: $\B_i$ is not a node of any supercycle, \ie $\neg \scyc{\B}{s}{\B_i}$.

\item \textit{no strong-connectedness}: $\B_i$ is a node in a supercycle, but not a node in a \emph{strongly-connected} supercycle. 

\en
%To avoid state-explosion, we do not check the transition $s \goesto[\act] t$ itself, but rather its
%projection onto a small subsystem, which we call the





\subsection{Supercycle membership and supercycle violation}

We use the term \emph{supercycle-violation} to indicate that a node is not in any supercycle. 

Definition~\ref{def:supercycle} implies that the supercycle-membership status of a node $v$ depends
solely on its outgoing wait-for edges, and the outgoing wait-for edges of the nodes that $v$ waits
for, etc. so that $v$'s supercycle-membership is a function of the subgraph of the wait-for graph
that is reachable from $v$.  Hence, we only look at outgoing wait-for edges in computing
supercycle-violation, which is just the negation of supercycle-membership.

A node that is not in any supercycle may nevertheless be a node of a wait-for cycle, since a cycle
of wait-for-edges does not necesarily imply a supercycle. Hence, to compute the supercycle violation
condition properly, we introduce a notion of the \emph{level} of a violation. A node with no
outgoing wait-for edges has a level-1 violation. A node whose violation is based on outgoing edges
to neighbors whose violation level is at most $d-1$, has itself a level-$d$ violation.  Hence we
formalize the notion of \emph{level-$d$ supercycle violation} as the predicate $\viol{v}{d}{t}$.


\bd[Supercycle violation, $\scViol{v}{d}{t}$]
\label{def:supercycle-violation}
\label{def:supercycle.violation}
We define the predicate $\viol{v}{d}{t}$ by induction on $d$, as follows. We indicate the
justification for each clause of the definition.

\noindent
\ul{Base case, $d=1$.} $\viol{v}{1}{t} = \true$ iff $v$ has no outgoing wait-for-edges, otherwise $\viol{v}{1}{t} = \false$.

Justification: If $v$ has no outgoing wait-for-edges, then it cannot be in a supercycle.  Note that $v$ must be an
interaction in this case, since a component must have at least one outgoing wait-for edge at all times.

\noindent
\ul{Inductive step, $d > 1$.}  $\viol{v}{d}{t} = \true$ iff any of the following cases hold. Otherwise 
$\viol{v}{d}{t} = \false$.

\bn

%\item $v$ is a component $\B_i$ and at least one of these two clauses holds:
\item $v$ is a component $\B_i$ and %at least one of these two clauses holds:
   \bn

   \item \label{def:supercycle.violation.component.out}
         There exists interaction $\act$ such that $\B_i \ar \act \in \wfg{\B}{t}$ and $\viol{\act}{d-1}{t} = \true$,
         \ie $\B_i$ enables an interaction $\act$ which has a level-$(d-1)$ supercycle-violation.
    Justification is Proposition~\ref{prop:sc-membership}, Clause~\ref{prop:sc-membership:comp-out}.

   % \item \label{def:supercycle.violation.component.in}
   %       There exists an interaction $\act$ such that $\act \ar \B_i \in \wfg{\B}{t}$, $\viol{\act}{d-1}{t} = \true$,
   %       \ie some interaction $\act$ that waits for $\B_i$ has a level-$(d-1)$ supercycle-violation. 
   %  Justification is Proposition~\ref{prop:sc-membership}, Clause~\ref{prop:sc-membership:comp-in}.

 \en

\item $v$ is an interaction $\act$ and
   \bn

   \item \label{def:supercycle.violation.interaction.out}
         For all components $\B_i$ such that $\act \ar \B_i \in \wfg{\B}{t}$, $\viol{\B_i}{d-1}{t} = \true$,
         \ie all components $\B_i$ that $\act$ waits for have a level-$(d-1)$ supercycle-violation.
    Justification is Proposition~\ref{prop:sc-membership}, Clause~\ref{prop:sc-membership:act-out}.

%    \item \label{def:supercycle.violation.interaction.in}
%          For all components $\B_i$ such that $\B_i \ar \act \in \wfg{\B}{t}$, $\viol{\B_i}{d-1}{t} = \true$,
%          \ie all  components $\B_i$ that enable $\act$ have a level-$(d-1)$ supercycle-violation.
% The justfication here is that $\act$ cannot be a node of a \emph{strongly-connected} supercycle.

   \en

\en

Figure~\ref{fig:scViolate} gives a formal definition of $\viol{v}{d}{t}$.
\ed


\begin{figure}[ht]

\setcounter{lctr}{0}
\begin{tabbing}\label{alg:check-scViol}
mm\= mm\= mm\= mm\= mm\=\kill
$\viol{v}{d}{t}$\\
\lio{\IFC{d = 1}}
   \litc{\IFC{\neg \ex u : v \ar u \in \wfg{\B}{t}}\ \RETURNE{\ttt}}{\cmnt $v$ has no outgoing wait-for-edges}
   \litc{\ELSE\ \RETURNE{\fff}}{\cmnt $v$ has some outgoing wait-for-edge}
   \lit{\FI}
\lio{\FI}

\lio{\cmnt \mbox{now } d > 1}

\lio{\IFC{\mbox{$v$ is a component $\B_i$}}}
   \litc{\IFC{\ex \act : \B_i \ar \act \in \wfg{\B}{t} :
       \viol{\act}{d-1}{t}} \ \RETURNE{\ttt}}{\cmnt component violations}
   \lit{\ELSE\ \RETURNE{\fff}}
   \lit{\FI}
\lio{\FI}

\lio{\IFC{\mbox{$v$ is an interaction $\act$}}}
   \litc{\IFC{\fa \B_i : \act \ar \B_i \in \wfg{\B}{t} : \viol{\B_i}{d-1}{t}} \ \RETURNE{\ttt}}{\cmnt interaction violations}
   \lit{\ELSE\ \RETURNE{\fff}}
   \lit{\FI}
\lio{\FI}
\end{tabbing}
\vspace{-4ex}
\caption{Formal definition of $\viol{v}{d}{t}$}
\label{fig:scViolate}
\end{figure}


In the sequel, we say sc-violation rather than ``supercycle violation.''  The crucial result is
that, if $v$ has a level-$d$ sc-violation, for some $d \ge 1$, then $v$ cannot be a node of a
supercycle.


\bp \label{prop:supercycle-violation} \label{prop:scViol-implies-notInSC}
If $(\ex d \ge 1: \viol{v}{d}{t})$ then $v$ is not a node of a supercycle in $\wfg{\B}{t}$.
\ep
\prf{
Proof is by induction in $d$. 

\noindent
\textit{Base case, $d=1$}. $v$ has no outgoing edges. Hence  $v$ cannot be in a supercycle.

\noindent
\textit{Induction step, $d >1$}. Assume that $v$ has a level $d$ $SC$-violation. We have two cases. 

\topcase{1}{$v$ is a component $\B_i$}   %\scase{1.1}{$v$ has an out-violation}
Hence there exists an interaction $\act$ such that $\B_i \ar \act \in \wfg{\B}{t}$ and $\act$ has a level-$(d-1)$
$SC$-violation. By the induction hypothesis, $\act$ cannot occur in a supercycle in $\wfg{\B}{t}$.
By Proposition~\ref{prop:sc-membership}, Clause~\ref{prop:sc-membership:comp-out}, 
$\B_i$ cannot be in a supercycle. 

% \scase{1.2}{$v$ has an in-violation}
% Hence for some interaction $\act$ such that $\act \ar \B_i \in \wfg{\B}{t}$, $\act$ has a level-$(d-1)$
% $SC$-violation. By the induction hypothesis, $\act$ cannot occur in a supercycle in $\wfg{\B}{t}$.
% By Proposition~\ref{prop:sc-membership}, Clause~\ref{prop:sc-membership:comp-in}, 
% $\B_i$ cannot be in a supercycle. 


\topcase{2}{$v$ is an interaction $\act$}    %\scase{2.1}{$v$ has an out-violation}
Hence for all components $\B_i$ such that $\act \ar \B_i \in \wfg{\B}{t}$, $\B_i$ has a level-$(d-1)$
$SC$-violation. By the induction hypothesis, $\B_i$ cannot occur in a strongly connected supercycle in $\wfg{\B}{t}$.
By Proposition~\ref{prop:sc-membership}, Clause~\ref{prop:sc-membership:act-out}, 
$\act$ cannot be in a supercycle. 

% \scase{2.2}{$v$ has an in-violation}
% Hence for all components $\B_i$ such that $\B_i \ar \act \in \wfg{\B}{t}$, $\B_i$ has a level-$(d-1)$
% $SC$-violation. By the induction hypothesis, $\B_i$ cannot occur in a strongly connected supercycle in $\wfg{\B}{t}$.
% Hence by definition~(\ref{def:supercycle}), $\act$ cannot be in a strongly connected supercycle, since $\B_i$ has no
% incoming edges that can be in a strongly connected  supercycle.
}


%In the other direction, we have a slightly weaker result: if $v$ has no level-$d$ scsc-violation,
%for all $d \ge 1$, then $v$ is a node of a supercycle.


\bp \label{prop:notInSC-implies-scViol}
If $(\fas d \ge 1: \neg \viol{v}{d}{t})$ then $v$ is a node of a supercycle in $\wfg{\B}{t}$.
\ep
%
\prf{
%
Let $V$ be the set of nodes in $\wfg{\B}{t}$ with a supercycle-violation, \ie
$V = \set{w \stt w \in \wfg{\B}{t} \land (\exs d: \viol{w}{d}{t})}$.  Let $\bV$ be the remaining nodes, \ie all nodes
in $\wfg{\B}{t}$ that do not have a supercycle-violation, so 
$\bV = \set{w \stt w \in \wfg{\B}{t} \land (\fas d \ge 1: \neg \viol{v}{d}{t})}$.

if $\bV$ is empty then the proposition holds vacuously and we are done. So assume that $\bV$ is non-empty and let 
$v$ be an arbitrary node in $\bV$.

\topcase{1}{$v$ is a component $\B_i$}  
Suppose that there is a wait-for-edge from $v$ to some interaction $\act$ that is in $V$.
Then, by Definition~\ref{def:supercycle.violation}, $v$ has a supercycle violation, which contradicts the choice of $v$
as a member of $\bV$. Hence all wait-for-edges starting in $v$ must end in a node in $\bV$.

\topcase{2}{$v$ is an interaction $\act$}  
Suppose that every wait-for-edge from $v$ to some component  $\B_i$ that is in $V$.
Then, by Definition~\ref{def:supercycle.violation}, $v$ has a supercycle violation, which contradicts the choice of $v$
as a member of $\bV$. Hence some wait-for-edge starting in $v$ must end in a node in $\bV$.


Hence we have that $\bV$ satisfies all three clauses of Definition~\ref{def:supercycle}: it is nonempty, each component
in $\bV$ has all its enabled interactions also in $\bV$, and each interaction in $\bV$ waits for a component in $\bV$. 
Hence $\bV$ as a whole is a supercycle. Since the nodes of $\bV$ are, by definition of $\bV$, exactly the nodes $v$ such that 
$(\fas d \ge 1: \neg \viol{v}{d}{t})$, we have that any such node is a node of a supercycle in $\wfg{\B}{t}$. Hence the
Proposition is established.
}

%Note that the above implies that there are no wait for edges from component in $U$ to an interaction outside $U$???


\bp \label{prop:scViol-iff-notInSC}
$v$ is not a node of a supercycle in $\wfg{\B}{t}$ iff $(\ex d \ge 1: \viol{v}{d}{t})$.
\ep
\prf{Immediate from Propositions~\ref{prop:scViol-implies-notInSC} and
  \ref{prop:notInSC-implies-scViol}.}




\subsection{Strong connectedness condition}
% commentary: strong connectedess violation cannot be defined "locally" in a good way, like supercycle violation
% can, since it depends on reachability, which is a global property. For a supercycle, once the set
% of ndoes is fixed, the required wait-for realtions are all local properties. HEnce, violation is
% also a local property. For strong connectedness, the best local property is "no incoming" or "no
% outging", which is just the linear condition of the FORTE 2013 paper. Hence a local condition for
% strong conectedness violation is no improvement over FORTE 2013.

Given that $\B_i$ is a node in a supercycle, we wish to determine whether or not it is a node in a
\emph{strongly-connected} supercycle. We do this by removing all nodes with supercycle-violations,
and then finding the maximal strongly connected components of the resulting wait-for subgraph.


\bd[Strong connectedness violation, $\connViol{v}{t}$]
\label{def:sConn.violation}
 Let $v$ be a node of $\wfg{\B}{t}$.   Then $\connViol{v}{t}$ holds iff there does not exist a strongly connected
 supercycle $SSC$ such that $v \in SSC$ and $SSC \sub \wfg{B}{t}$.
% Let $v$ be a node that is in a supercycle of $\wfg{\B}{t}$.  Let $W$ be the result of removing from $\wfg{\B}{t}$ every
% node $u$ such that $(\ex d \ge 1: \viol{u}{d}{t})$. Let $V$ be the maximal strongly connected component of $W$ that
% contains $v$. Then $\connViol{v}{t}$ holds iff $V$ (by itself) is not a supercycle.
% For technical convenience, we also define $\connViol{v}{t}$ to be false when $(\ex d \ge 1: \scViol{v}{d}{t})$, \ie when
% $v$ is not in a supercycle.
% Hence $\connViol{v}{t}$ is always well-defined.
\ed




\subsection{Supercycle formation violation condition}

\bd[Formation violation, $\formViol{v}{t}$]
\label{def:formation.violation}
Let $v$ be a node of $\wfg{\B}{t}$.
Then $\formViol{v}{t}  \df (\exs d \ge 1: \scViol{v}{d}{t}) \lor \connViol{v}{t}$.
\ed
%
Let $s \goesto[\act] t$ be a reachable transition. If, for every $\B_i \in \cmps{\act}$, 
$\formViol{v}{t}$ holds, then $s \goesto[\act] t$ does not introduce a supercycle, \ie if $s$ is
supercycle-free, then so is $t$. We establish this in the sequel.

We remark that, as shown above $(\exs d \ge 1: \scViol{v}{d}{t})$
implies that $v$ cannot be in a supercycle. Hence, $v$ cannot be in a strongly-connected supercycle. 
Hence $(\exs d \ge 1: \scViol{v}{d}{t})$ implies $\connViol{v}{t}$. It is however convenient to state the formation
violation condition in this manner, since, in the sequel, we formulate a ``local'' version for each of  
$(\exs d \ge 1: \scViol{v}{d}{t})$ and $\connViol{v}{t}$, and the implication does not necessarily hold for the local
versions. The advantage of the local versions is that they are usually efficiently computable, as we show in the sequel.

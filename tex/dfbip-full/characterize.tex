%\section{Deadlock-freedom}


\bd[Global Deadlock-freedom]
\label{def:static:deadlock-free}
\label{def:global.deadlock-free} 
\label{defn:static:deadlock-free}
\label{defn:global.deadlock-free} 
A BIP-system $(\B, Q_0)$ is \emph{free of global deadlock} iff,
in every reachable state $s$ of $(B, Q_0)$, some interaction $\act$ is enabled.
Formally, $\fa s \in \rstates{\B,Q_0}, \ex \act : s \enb{\act}{\B}$.
%% A BIP-system $(\B, Q_0)$ is \emph{global deadlock prone} iff there exists a reachable state
%% $s$ of $(\B, Q_0)$ in which no interaction $a$ is enabled.
%% A BIP-system $(\B, Q_0)$ is \emph{free of global deadlock} iff it is not global deadlock prone.
\ed


\bd[Local Deadlock-freedom]
\label{def:local.deadlock-free} 
\label{defn:local.deadlock-free} 
A BIP-system $(\B, Q_0)$ is \emph{free of local deadlock} iff, 
for every subsystem $(\B', Q'_0)$ of  $(\B, Q_0)$, and every reachable state $s$ of $(\B, Q_0)$,
$(\B', Q'_0)$ has some interaction enabled in state $s \pj \B'$.
Formally:\\
\ind for every subsystem $(\B', Q'_0)$ of  $(\B, Q_0)$:\\
\ind \ind $\fa s \in \rstates{\B,Q_0},  \ex \act : s \pj \B' \enb{\act}{\B'}$.
%A BIP-system $(\B, Q_0)$ is \emph{local deadlock prone}  iff there exists a subsystem $(\B', Q'_0)$ of  $(\B, Q_0)$
%and a reachable state $s$ of $(\B, Q_0)$ such that $(\B', Q'_0)$ has no interaction enabled in state $s \pj \B'$.
\ed


Proposition~\ref{prop:static:supercycle-is-sufficient} states that the
existence of a supercycle implies a local deadlock: all components in
the supercycle are blocked forever.

Proposition~\ref{prop:static:supercycle-is-necessary} states that the
existence of a supercycle is necessary for a local deadlock to occur:
if a set of components, \emph{considered in isolation}, are blocked,
then there exists a supercycle consisting of exactly those components,
together with the interactions that each component enables.




\subsection{Wait-for graphs}
\label{secn:wait-for-graphs}

The wait-for-graph for a state $s$ is a directed bipartite and-or
graph which contains as nodes the atomic components $\B_1, \ldots,
\B_n$, and all the interactions $\gamma$.  Edges in the wait-for-graph
are from a $\B_i$ to all the interactions that $\B_i$ enables (in $s$),
and from an interaction $a$ to all the components that participate in
$a$ and which do not enable it (in $s$).


\bd[Wait-for-graph  $\wfg{\B}{s}$]
\label{def:static:wait-for-graph} 
\label{defn:static:wait-for-graph} 
Let $\B = \gamma(\B_1,\dots,\B_n)$ be a BIP composite component, and let
$s = \tpl{s_1,\ldots,s_n}$ be an arbitrary state of $\B$.
The {\em wait-for-graph} $\wfg{\B}{s}$ of $s$ is a directed bipartite and-or graph, where
\begin{nlst1}

\item \label{def:static:wait-for-graph:nodes} the nodes of $\wfg{\B}{s}$ are as follows:
%
   \begin{nlst2}
   \item the and-nodes are the atomic components $\B_i$, $i \in \oneton$,
   \item the or-nodes are the interactions $\act \in \gamma$,
% such that for some  $i \in \oneton$: $B_i \in C_a$ and $s_i \enb{a}{B_i}$  OMIT this
   \end{nlst2}

\item \label{def:static:wait-for-graph:edges-aut-action} 
   there is an edge in $\wfg{\B}{s}$ from $\B_i$ to every node 
   $\act$ such that $\B_i \in \cmps{\act}$ and $s_i(\gd{\act}{i}) = \true$, \ie from $\B_i$ to every interaction
   which $\B_i$ enables in $s_i$,

\item  \label{def:static:wait-for-graph:edges-action-aut}
   there is an edge in $\wfg{\B}{s}$ from $a$ to every 
   $\B_i$ such that $\B_i \in \cmps{\act}$ and $s_i(\gd{\act}{i}) = \false$, \ie from $\act$ to every component
   $\B_i$ which participates in $\act$ but does not enable it, in state $s_i$.
                  
\end{nlst1}
\ed

A component $\B_i$ is an and-node since all of its successor actions (or-nodes) must be disabled for
$\B_i$ to be incapable of executing.  An interaction $\act$ is an or-node since it is disabled if
any of its participant components do not enable it.  An edge (path) in a wait-for-graph is called a
wait-for-edge (wait-for-path).  Write $\act \ar \B_i$ ($\B_i \ar \act$ respectively) for a
wait-for-edge from $\act$ to $\B_i$ ($\B_i$ to $\act$ respectively).  We abuse notation by writing
$e \in \wfg{\B}{s}$ to indicate that $e$ (either $\act \ar \B_i$ or $\B_i \ar \act$) is an edge in
$\wfg{\B}{s}$.  Also $\B_i \ar \act \ar \B'_i \in \wfg{\B}{s}$ for
$B_i \ar \act \in \wfg{\B}{s} \land \act \ar B'_i \in \wfg{\B}{s}$, \ie for a wait-for-path of
length 2, and similarly for longer wait-for-paths.

% done in Local Enablement Assumption above
%We assume, in the sequel, that every component $B_i$ always enables at least one interaction, \ie 
%in all reachable states $s$,  $B_i \ar a$ for at least $a$. This prevents pathological situations where a component
%causes a deadlock by disabling every interaction that it participates in.

Consider the dining philosophers system given in Figure~\ref{fig:diningSpectrum}.
Figure~\ref{fig:wfg0} shows its wait-for-graph in its sole initial state.  Figure~\ref{fig:wfg1}
shows the wait-for-graph after execution of $\get_0$.  Edges from components to interactions are
shown solid, and edges from interactions to components are shown dashed.

\begin{figure*}[ht]
  \begin{center}
      \subfigure[Wait-for-graph in initial state.]{\label{fig:wfg0}\scalebox{0.4}{\input{figs/wfgDiningInitial.pdf_t}}} \quad
      \subfigure[Wait-for-graph after execution of $\get_0$.]{\label{fig:wfg1}\scalebox{0.4}{\input{figs/wfgDining1.pdf_t}}} 
      \caption{Example wait-for-graphs for dining philosophers system of Figure~\ref{fig:diningSpectrum}.}
       \label{fig:wfg}
  \end{center}
\end{figure*}


A key principle of the dynamics of the change of wait-for graphs is that 
wait-for edges not involving some interaction
$\act$ and its participants $\B_i \in \cmps{\act}$ are unaffected by the execution
of $\act$.  Say that edge $e$ in a wait-for-graph
is \emph{$\B_i$-incident} iff $\B_i$ is one of the endpoints of $e$.


\bp[Wait-for edge preservation] \label{prop:wait-for-edge-preservation}
Let $s \la{\act} t$ be a transition of composite component $B =
\gamma(\B_1,\dots,\B_n)$, and let $e$ be a wait-for edge in $\wfg{\B}{s}$
that is not $\B_i$-incident, for every $\B_i \in \cmps{\act}$. Then $e \in
\wfg{\B}{s}$ iff $e \in \wfg{\B}{t}$. 
\ep
%
\prfs{
Components not involved in the execution of $\act$ do not change state
along $s \la{\act} t$. Hence the endpoint of $e$ that is a
component has the same state in $s$ as in $t$. The proposition then
follows from Definition~\ref{def:static:wait-for-graph}. 
}
%
\prf{
Fix $e$ to be an arbitrary wait-for-edge that is not
$\B_i$-incident. $e$ is either $\B_j \ar b$ or $b \ar \B_j$, for some
component $\B_j$ of $B$ that is not in $\cmps{\act}$, and an interaction $b$
(different from $\act$) that $\B_j$ participates in.
%
Now $s \pj \B_j = t \pj \B_j$, since $s \la{\act} t$ and $\B_j \not\in \cmps{\act}$. Hence $s(\gd{b}{j}) =
t(\gd{b}{j})$. It follows from Definition~\ref{def:static:wait-for-graph} that 
$e \in \wfg{\B}{s}$ iff $e \in \wfg{\B}{t}$.
}



%%%%%%%%%%%%%%%%%%%%%%%%%%%%%%%%%%%%%%%%%%%%%%%%%%%%%%%%%%%%%%%%%%%%%%%%%%%%%%%%%%%%%%%%%%%%%%%%%%%%%
\subsection{Supercycles and deadlock-freedom}

We characterize a deadlock as the existence in the wait-for-graph of a
graph-theoretic construct that we call a {\em supercycle}.

\bd[Supercycle]
\label{def:supercycle} 
\label{defn:supercycle} 
Let $\B = \gamma(\B_1,\dots,\B_n)$ be a composite component and $s$ be a state of $\B$.
A subgraph $\SC$ of $\wfg{\B}{s}$ is a supercycle in $\wfg{\B}{s}$ if and only if all of the following hold:
\begin{nlst1}
   \item \label{def:supercycle.nonempty} \label{clause:supercycle.nonempty} 
$\SC$ is nonempty, \ie contains at least one node,

   \item \label{def:supercycle.component-blocked} \label{clause:supercycle.component-blocked} 
if $\B_i$ is a node in $\SC$, then for all interactions $\act$ such that
there is an edge in $\wfg{\B}{s}$ from $\B_i$ to $\act$:
      \begin{nlst2}
      \item $\act$ is a node in $\SC$, and 
      \item there is an edge in $\SC$ from $\B_i$ to $\act$,
      \end{nlst2}
that is, $\B_i \ar \act \in \wfg{\B}{s}$ implies $\B_i \ar \act \in \SC$,

   \item \label{def:supercycle.action-blocked}  \label{clause:supercycle.action-blocked}  
if $\act$ is a node in $\SC$, then there exists a $\B_j$ such that:
      \begin{nlst2}
      \item $\B_j$  is a node in $\SC$, and
      \item there is an edge from $\act$ to $\B_j$ in $\wfg{\B}{s}$, and
      \item there is an edge from $\act$ to $\B_j$ in $\SC$,
      \end{nlst2}
that is, $\act \in \SC$ implies $\ex \B_j : \act \ar \B_j \in \wfg{\B}{s} \land \act \ar \B_j \in \SC$,

\end{nlst1}
\ed
where $\act \in SC$ means that $\act$ is a node in $\SC$, etc. 
Also, write $SC \sub \wfg{\B}{s}$ when $\SC$ is a subgraph of $\wfg{\B}{s}$.


\bd[Supercycle-free]
\label{def:supercycle-free}
\label{defn:supercycle-free}
$\wfg{\B}{s}$ is \emph{supercycle-free} iff 
there does not exist a supercycle $\SC$ in $\wfg{\B}{s}$. 
In this case, say that state $s$ is supercycle-free.
Formally, we define the predicate 
    $\scf{B}{s} \df \neg \ex \SC: \SC \sub \wfg{\B}{s} \ \mbox{and $\SC$ is a supercycle}$. 
\ed

\begin{figure*}[ht]
  \begin{center}
   \scalebox{0.4}{\input{figs/wfgDiningCycle.pdf_t}}
   \caption{Example supercycle for dining philosophers system of Figure~\ref{fig:diningSpectrum}.}
   \label{fig:sc}
  \end{center}
\end{figure*}


Figure~\ref{fig:sc} shows an example supercycle (with boldened edges) for the dining philosophers
system of Figure~\ref{fig:diningSpectrum}.
$P_0$ waits for (enables) a single interaction, $\Get_0$. 
$\Get_0$ waits for (is disabled by) fork $F_0$, which waits for interaction $\Rel_0$.
$\Rel_0$ in turn waits for $P_0$. However, this supercycle occurs in a state where $P_0$ is in $h$
and $F_0$ is in $u_l$. This state is not reachable from the initial state. 



Figure~\ref{fig:cycleOK} shows an ``abstract'' example, where there is
a cycle of wait-for-edges, without there being a supercycle. This shows
that a cycle does not necessarily imply a supercycle, and hence
deadlock. Adding a wait-for-edge from $d$ to $\B_1$ would create a
supercycle in Figure~\ref{fig:cycleOK}.
%
\begin{figure}[ht]
\begin{center}
\scalebox{0.5}{\input{figs/cycleOK.pdf_t}}
\caption{Example where a wait-for cycle does not imply deadlock}
\label{fig:cycleOK}
\end{center}
\end{figure}






The existence of a supercycle is sufficient and necessary for the occurrence of
a deadlock, and so checking for supercycles gives a sound and complete check for
deadlocks.  
%
Proposition~\ref{prop:static:supercycle-is-sufficient} states that the
existence of a supercycle implies a local deadlock: all components in
the supercycle are blocked forever.

\bp
\label{prop:static:supercycle-is-sufficient}
Let $s$ be a state of $\B$.
If $\SC \sub \wfg{\B}{s}$ is a supercycle, then all components $\B_i$ in $\SC$ cannot execute a transition in any state reachable
from $s$, including $s$ itself.
\ep
%\begin{proof}
\prfs{Every interaction $\act$ that $\B_i$ enables is not enabled by some
participant. By Defintion~\ref{def.bip.enablement}, $\act$ cannot be
executed. Hence $\B_i$ cannot execute any transition.
}
%
\prf{
Let $\B_i$ be an arbitrary component in $SC$. By Definition~\ref{def:supercycle}, every interaction that $\B_i$ enables
has a wait-for-edge to some other component $\B_j$ in $\SC$ and so cannot be executed in state $s$. Hence in any
transition from $s$ to another global state $t$, all of the components $\B_i$ in $\SC$ remain in the same local state. 
Hence $\SC \sub \wfg{\B}{t}$, \ie the same supercycle $\SC$ remains in global state $t$. Repeating this argument from state $t$
and onwards leads us to conclude that $\SC \sub \wfg{\B}{u}$ for any state $u$ reachable from $s$.
}%\end{proof}

Proposition~\ref{prop:static:supercycle-is-necessary} states that the
existence of a supercycle is necessary for a local deadlock to occur:
if a set of components, \emph{considered in isolation}, are blocked,
then there exists a supercycle consisting of exactly those components,
together with the interactions that each component enables.
%
\bp
\label{prop:static:supercycle-is-necessary} 
\label{prop:supercycle-is-necessary}
Let $\B'$ be a subcomponent of $\B$, and
let $s$ be an arbitrary state of $\B$ such that $\B'$,
when considered in isolation, has no enabled interaction in state $s \pj \B'$.
%
Then, $\wfg{\B}{s}$ contains a supercycle.
\ep
\prfs{Every atomic component $\B_i$ in $\B'$ is individually deadlock free, by assumption, 
and so there is at least one interaction $a_i$ which
$\B_i$ enables. Now $a_i$ is not enabled in $\B'$, by the antecedent of the proposition. Hence $a_i$ has some
outgoing wait-for-edge in $\wfg{\B}{s}$. The subgraph of $\wfg{\B}{s}$ induced by all
the $\B_i$ and all their (locally) enabled interactions is therefore a supercycle.
}
%
\prf{Let $\B_i$ be an arbitrary atomic component in $\B'$, and let $\act_i$ be any interaction that $\B_i$ enables. Since $\B'$ has no
enabled interaction, it follows that $\act_i$ is not enabled in $\B'$, and
therefore has a wait-for-edge to some atomic component $\B_j$ in
$\B'$. Let $SC$ be the subgraph of $\wfg{\B}{s}$ induced by:
\bn
\item the atomic components of $\B'$, 
\item the interactions $\act$ that each atomic component $\B_i$ enables,
     and the edges $\B_i \ar \act$, and
\item  the edges $\act \ar \B_j$ from each interaction to some atomic
     component $\B_j$ in $\B'$ that does not enable $\B_j$.
\en
$\SC$ satisfies Definition~\ref {def:supercycle} and so is a supercycle.
}

We consider subcomponent $\B'$ in isolation to avoid other phenomena that prevent interactions from executing, \eg
conspiracies \cite{AFG93}.  
Now the converse of Proposition~\ref{prop:supercycle-is-necessary} is that absence of
supercycles in $\wfg{\B}{s}$ means there is no locally deadlocked subsystem. 



\bco[Supercycle-free implies free of local deadlock]
\label{cor:static:dead-free}
%Let $(B,Q_0)$ be a BIP system, with $B = \gamma(\B_1,\dots,\B_n)$. 
If, for every reachable state $s$ of $(\B,Q_0)$, $\wfg{\B}{s}$ is supercycle-free, then
$(\B,Q_0)$ is free of local deadlock.
\eco
%
\prf{
We establish the contrapositive.
Suppose that $(\B, Q_0)$ is not free of local deadlock. Then there exists a subsystem $(\B', Q'_0)$ of $(\B, Q_0)$, and
a reachable state $s$ of $(\B', Q'_0)$, such that $\B'$ enables no interaction in state  $s \pj \B'$.
By Proposition~\ref{prop:supercycle-is-necessary}, $\wfg{\B}{s}$ contains a supercycle.
}

In the sequel, we say ``deadlock-free'' to mean ``free of local deadlock''.



We wish to check whether supercycles can be formed or not.
% \ie whether the supercycle formation condition is satisfied or not.
In principle, we could check directly whether $\wfg{\B}{t}$ contains a supercycle, for each
reachable state $t$. However, this approach is subject to state-explosion, and so is usually
unlikely to be viable in practice.  Instead, we formulate global conditions for supercycle-freedom,
and then ``project'' these conditions onto small subsystems, to obtain local versions of these
conditions that are (1) efficiently checkable, and (2) imply the global versions.
To formulate these condition, we need to characterize the static (structural) and dynamic
(formation) properties of supercycles.



%%%%%%%%%%%%%%%%%%%%%%%%%%%%%%%%%%%%%%%%%%%%%
\subsection{Structural properties of supercycles}
\label{secn:supercycle-structural}

We present some structural properties of supercycles, which
are central to our deadlock-freedom conditions.


\bd[Path, path length] \label{def:path} \label{defn:path}
Let $G$ be a directed graph and $v$ a vertex in $G$. A path $\pi$ in $G$ is a \emph{finite} sequence
$v_0, v_1, \ldots,v_n$ such that $(v_i, v_{i+1})$ is an edge in $G$ for all $i \in \rng{0}{n-1}$.
Write $\pth{G}{\pi}$ iff $\pi$ is a path in $G$.
Define $\first{\pi} = v_0$ and $\last{\pi} = v_n$. 
%
Let $|\pi|$ denote the length of $\pi$, which we define as follows:
\be
\item if $\pi$ is simple, \ie all $v_i$, $0 \le i \le n$, are distinct, then $|\pi| = n$, \ie the
number of edges in $\pi$
\item if $\pi$ contains a cycle, \ie there exist $v_i, v_j$ such that $i \ne j$ and $v_i = v_j$, then
$|\pi| = \omega$ ($\omega$ for ``infinity'').
\ee

\ed

\bd[In-depth, Out-depth] \label{def:depth} \label{defn:depth} 
Let $G$ be a directed graph and $v$ a vertex in $G$. Define the in-depth of $v$ in $G$, notated as
$\idepth{G}{v}$, as follows:
\be
\item if there exists a path $\pi$ in $G$ that contains a cycle and ends in $v$, \ie $|\pi| = \omega
  \land \last{\pi} = v$, then $\idepth{G}{v} = \omega$,
%THIS DEFINITION OF INFINITE IN-DEGREEE IS STRANGE, SINCE YOU HAVE ``INFINITE' PATHS THAT NEVERTHELESS END IN A NODE!

\item otherwise, let $\pi$ be a longest (simple) path ending in $v$. Then $\idepth{G}{v} = |\pi|$.
\ee
Formally, $\idepth{G}{v} = (\MAX\ \pi : \pth{G}{\pi} \land \last{\pi} = v : |\pi|)$.

Likewise define the out-depth of $v$ in $G$, notated as
$\odepth{G}{v}$, as follows:
\be
\item if there exists a path $\pi$ in $G$ that contains a cycle and starts in $v$, \ie $|\pi| = \omega
  \land \first{\pi} = v$, then $\odepth{G}{v} = \omega$,

\item otherwise, let $\pi$ be a longest (simple) path starting in $v$. Then $\odepth{G}{v} = |\pi|$.
\ee
Formally, $\odepth{G}{v} = (\MAX\ \pi : \pth{G}{\pi} \land \first{\pi} = v : |\pi|)$.
\ed

\noindent
We use $\widepth{\B}{v}{s}$ for $\idepth{\wfg{\B}{s}}{v}$, and also
$\wodepth{\B}{v}{s}$ for $\odepth{\wfg{\B}{s}}{v}$.

% REDUNDANT 
% \bp \label{prop:supercycle:finite-indepth-not-essential} Let $\SC$ be
% a supercycle and let $\SC'$ result from removing all nodes in $\SC$ with
% finite in-depth. Then $\SC'$ is also a supercycle.
% \ep
% %
% \prf{
% Proof is by induction on the in-depth $i$ of nodes in $\SC$.

% Base case: in-depth = 0. Let $\SC'$ be the result of removing all nodes of in-depth 0 from $SC$. Since no node waits
% for a node of in-depth 0, it follows that a node of in-depth 0 does not contribute to the satisfaction of
% Definition~\ref{def:supercycle}, and hance $\SC'$ is still a supercycle.

% Inductive step: in-depth = $i$: let $\SC"$ be the result of removing all nodes of in-depth $i-1$ from $SC$.
% By the induction hypothesis, $\SC"$ is a supercycle. Now let $\SC'$ be the result of 
% removing, from $\SC"$, all nodes that have in-depth $i$ in $SC$.
% Now all nodes of in-depth $i$ in $\SC$ have an in-depth of 0 in
% $\SC"$. Hence removing all such nodes from $\SC"$ results, by the same argument as for the base case, 
% in a supercyle $\SC'$.
% }



\bp \label{prop:supercycle:no-finite-outdegree}
\label{prop:supercycle:no-finite-outdepth} A supercycle $SC$
contains no nodes with finite out-depth.
\ep
%
\prfs{By contradiction. Let $v$ be a node in $\SC$ with finite out-depth.
Hence all outgoing paths from $v$ end in a
sink node. By assumption, all atomic components are individually
deadlock-free, \ie they always enable at least one interaction. Hence
these sink nodes are all interactions, and therefore they violate
clause~\ref{clause:supercycle:action-blocked} in Definition~\ref{def:supercycle}.
}
%
\prf{By contradiction. Let $v$ be a node in $\SC$ with finite out-depth.
Hence by Definition~\ref{def:depth}  all outgoing paths from $v$ are
simple (and finite), and end in a
sink node $w$, so $w$ has no outgoing wait-for-edges.
By assumption, all atomic components are individually
deadlock-free, \ie they always enable at least one interaction. So if
$w$ is an atomic component $\B_i$, we have a wait-for-edge $\B_i \ar \act$
for some interaction $\act$, contradicting the fact that $w$ is a sink node.
Hence $w$ is some interaction $\act$.
Since $\act$ has no outgoing edges, it violates
clause~\ref{def:supercycle.action-blocked} in
Definition~\ref{def:supercycle}, contradicting the assumption that
$\SC$ is a supercycle.
}


% The next two results concern the structure of supercycle. The first
% shows that a supercycle contains at least one strongly connected
% component. The second shows that removing nodes with only simple paths
% leading into them leaves a resulting graph that is also a supercycle,
% \ie that such nodes are not ``essential'' elements of a
% supercycle. This idea is central to our deadlock-freedom condition.

\bp \label{prop:supercycle:contains-twoNodes}
Every supercycle $\SC$ contains at least two nodes.
\ep
\prf{
By Definition~\ref{def:supercycle}, $\SC$ is nonempty, and so contains at least one node $v$.
If $v$ is an interaction $\act$, then by Definition~\ref{def:supercycle}, $\SC$ also contains some component $\B_i$ such that 
$\act \ar \B_i$. 
If $v$ is a  component $\B_i$, then, by assumption, $\B_i$ enables at least one interaction $\act$, and by 
Definition~\ref{def:supercycle}, every interaction that $\B_i$ enables must be in $\SC$.
Hence in both cases, $\SC$ contains at least two nodes.
}


\bp \label{prop:supercycle:contains-cycle}
Every supercycle $\SC$ contains at least one cycle.
\ep
%
\prfs{Suppose not. Then $\SC$ is an acyclic supercycle. Hence every node in $\SC$
 has finite out-depth, which contradicts
 Proposition~\ref{prop:supercycle:no-finite-outdegree}. }
%
\prf{By contradiction. Suppose that $\SC$ is a supercycle and is also
  acyclic. Then every path in $\SC$ is simple, and therefore finite.
  Hence every node in $\SC$ has finite out-depth. By
  Proposition~\ref{prop:supercycle:no-finite-outdegree}, $\SC$ cannot
  be a supercycle.}



\bp \label{prop:supercycle:essential-subgraph-of} Let $\B =
\gamma(\B_1,\dots,\B_n)$ be a composite component and $s$ a state of
$\B$.  Let $SC$ be a supercycle in $\wfg{\B}{s}$, and let $SC'$ be the
graph obtained from $SC$ by removing all vertices of finite in-depth
and their incident edges. Then $SC'$ is also a supercycle in
$\wfg{\B}{s}$.  \ep
%
\prfs{
By Proposition~\ref{prop:supercycle:contains-cycle}, $\SC'$ is nonempty.
Thus $SC'$ satisfies clause (1) of Definition~\ref{def:supercycle}.
%
Let $v$ be an arbitrary vertex of $\SC'$. Hence $v$ has infinite in-depth, and
therefore so do all of $v$'s sucessors in $\SC$. Hence all of these successors
are in $\SC'$. 
Hence every vertex $v$ in $SC'$ has successors
in $\SC'$ that satisfy clauses (2) and (3) of Definition~\ref{def:supercycle}.
}
%
\prf{
A vertex with finite in-depth cannot lie on a cycle in $\SC$.  Hence
by Proposition~\ref{prop:supercycle:contains-cycle}, $\SC' \neq
\emptyset$. Thus $\SC'$ satisfies clause (1) of the supercycle
definition~(\ref{def:supercycle}).
%
Let $v$ be an arbitrary vertex of $\SC'$.  Thus $v \in \SC$ and $\idepth{\SC}{v} = \omega$ by definition of $SC'$. Let
$w$ be an arbitrary successor of $v$ in $\SC$. $\idepth{\SC}{w} = \omega$ by Definition~\ref{def:depth}. Hence
$w \in \SC'$, by definition of $SC'$.  Furthermore, $w$ is a successor of $v$ in $\SC'$, since $SC'$ consists of
\emph{all} nodes of $SC$ with infinite in-depth. Hence the successors of $v$ in $\SC'$ are
the same as the successors of $v$ in $\SC$
%
%Thus every vertex $v$ of $\SC'$ is also a vertex of $\SC$, and 
%
Now since $\SC$ is a supercycle, every vertex $v$ in $\SC$ has enough successors in $\SC$ to satisfy clauses (2) and (3)
of the supercycle definition~(\ref{def:supercycle}). It follows that every vertex $v$ in $\SC'$ has enough successors in
$\SC'$ to satisfy clauses (2) and (3) of the supercycle definition~(\ref{def:supercycle}).  }


\bp \label{prop:supercycle:contains-mssc}
Every supercycle $\SC$ contains a maximal strongly connected component $\CC$
such that (1) $\CC$ is itself a supercycle, and (2) there is no wait-for-edge from a node in $\CC$ to a node outside of $\CC$.
\ep
%
\prf{
$\SC$ is a directed graph, and so consider the decomposition of $\SC$
into its maximal strongly connected components (MSCC). Let $\overline{\SC}$ be the graph resulting
from replacing each MSCC by a single node. By its construction,  $\overline{\SC}$ is acyclic, and so contains at least one
node $x$ with no outgoing edges. Let $\CC$ be the MSCC corresponding to $x$. It follows that $\CC$ is nonempty, and
hence $\CC$ satisfies clause (1) of the supercycle definition~(\ref{def:supercycle}).
It also follows from the construction of $\CC$ that no node in $\CC$ has a wait-for-edge going to a node outside of
$\CC$, and so Clause (2) of the Proposition is established.

Let $v$ be an arbitrary node in $\CC$. Since $\CC \sub \SC$, $v$ is a node of $\SC$. Let $w$ be an arbitrary successor of
$v$ in $\SC$. Since no node in $\CC$ has an edge going to a node outside of $\CC$, it follows that $w$ is a node of $\CC$.
Hence $v$ has the same successors in $\CC$ as in $\SC$. 
Now since $\SC$ is a supercycle, every vertex $v$ in $\SC$ has enough successors in $\SC$ to satisfy clauses (2) and (3)
of the supercycle definition~(\ref{def:supercycle}). It follows that every vertex $v$ in $\CC$ has enough successors in
$\CC$ to satisfy clauses (2) and (3) of the supercycle definition~(\ref{def:supercycle}).  

Hence, by Definition~\ref{def:supercycle}, $\CC$ is itself a supercycle, and so Clause (1) of the Proposition is established.
}

Note also that by Proposition~\ref{prop:supercycle:contains-twoNodes}, $\CC$ contains at least two nodes. Hence $\CC$ is
not a trivial strongly connected component.



\bp \label{prop:supercycle:union}
Let $\SC, \SC'$ be supercycles in $\wfg{B}{s}$. Then $\SC \un \SC'$ is
a supercycle in $\wfg{B}{s}$.
\ep
%
\prf{Striaghtforward, since each node in 
$\SC \un \SC'$ has enough successors that it waits for to satisfy 
 \defn{supercycle}.
}



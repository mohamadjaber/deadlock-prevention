
We now seek a local condition, which we evaluate in $\dsk{\act}{\l}$, and which implies \GAO.
To achieve a conservative approximation, we make the ``pessimistic'' assumption that the violation status of border
nodes of $\dsk{\act}{\l}$ cannot be known, since it depends on nodes outside of $\dsk{\act}{\l}$.
Now, if an internal node $v$ of $\dsk{\act}{\l}$ can be marked with a level $d$ sc-violation, by applying 
Definition~\ref{def:supercycle-violation} only within
$\dsk{\act}{\l}$, and with the border nodes marked as non-violating,
then it is also the case, as we show below, that $v$ has a level $d$ sc-violation overall.

We define the predicate $\lviol{v}{d}{t}{\act}{\l}$ to hold iff node $v$ in $\wfg{B}{t}$ has a level-$d$ supercycle-violation
\emph{that can be confirmed within $\dsk{\act}{\l}$}.


\subsubsection{Local supercycle violation condition}

\bd[Local supercycle violation, $\locScViol{v}{d}{t_\act}{\act}{\l}$]
\label{def:supercycle.violation.local}
We define $\lviol{v}{d}{t_\act}{\act}{\l}$ by induction on $d$, as follows.

\noindent
\ul{Base case, $d=1$.} $\lviol{v}{1}{t_\act}{\act}{\l} = \true$  iff
$v$ is an interaction $\actp$ and 
$\actp$ is an interior node of $\dsk{\act}{\l}$ that has no outgoing wait-for edges.


\noindent
\ul{Inductive step, $d > 1$.} $\lviol{v}{1}{t_\act}{\act}{\l} = \true$ iff any of the following cases hold. Otherwise 
$\lviol{v}{1}{t_\act}{\act}{\l} = \false$.

\be

\item \ul{$v$ is a component $\B_i$} such that there exists interaction $\actp$ such that $B_i \ar \actp \in \wfg{\dsk{\act}{\l}}{t_\act}$ and $\lviol{\actp}{d-1}{t_\act}{\act}{\l} = \true$,
         \ie $\B_i$ enables an interaction $\actp$ which has a level-$(d-1)$ supercycle-violation in $\dsk{\act}{\l}$. 
It does not matter whether $\B_i$ is border or interior.


\item \ul{$v$ is an interaction $\actp$ and an internal node of $\dsk{\act}{\l}$} such that
   for all components $\B_i$ such that $\actp \ar \B_i \in \wfg{\dsk{\act}{\l}}{t_\act}$, $\lviol{\B_i}{d-1}{t_\act}{\act}{\l} = \true$,
         \ie all components $\B_i$ that $\actp$ waits for have a level-$(d-1)$ supercycle-violation in $\dsk{\act}{\l}$,

\ee
\ed
%
Note that if $v$ is an interaction $\actp$ and a border node, then
$\lviol{\actp}{d}{t_\act}{\act}{\l}$ is false, for all $d$.  This is because $\actp$ has some
component that is outside $\dsk{\act}{\l}$, and so this component cannot be checked.  A component
cannot have a level-1 sc-violation (regardless of whether it is border or interior) since it must
have at least one outgoing wait-for edge at all times.
%
Figure~\ref{fig:scViolateLoc} gives a formal definition of $\lviol{v}{d}{t_\act}{\act}{\l}$.
% Border and interior nodes are handled by separate functions $\blviol{v}{d}{t_\act}{\act}{\l},
%\ilviol{v}{d}{t_\act}{\act}{\l}$, respectively, and the base case $d=1$ is split across these. Note
%that $\blviol{v}{d}{t_\act}{\act}{\l}$ always returns $\false$ for $d=1$.  \ed



 
\begin{figure}[ht]
\setcounter{lctr}{0}
\begin{tabbing}
mm\= mm\= mm\= mm\= mm\=\kill
$\lviol{v}{d}{t_\act}{\act}{\l}$\\
\cmnt\ Precondition: $v$ is a node of $\dsk{\act}{\l}$ and $d \ge 1$\\

\lio{\IFC{d = 1}}
       \lit{\IFC{\mbox{$v$ is an interior interaction $\actp$ and }
              \neg (\ex \B_i : \actp \ar \B_i \in \wfg{\dsk{\act}{\l}}{t_\act})}}
                    \lihc{\RETURNE{\ttt}}{\cmnt no outgoing wait-for-edges}
       \lit{\ELSE\ \RETURNE{\fff}}
       \lit{\FI}

\cmnt\ here $d > 1$\\

\lio{\IFC{\mbox{$v$ is an interior interaction $\actp$ and } 
                 (\fa \B_i : \actp \ar \B_i \in \wfg{\dsk{\act}{\l}}{t_\act} : \lviol{\B_i}{d-1}{t_\act}{\act}{\l})}}
        \lit{\RETURNE{\ttt}}

\lio{\ELSFC{\mbox{$v$ is a component $\B_i$ and }
            (\ex \actp : \B_i \ar \actp \in \wfg{\dsk{\act}{\l}}{t_\act} : \lviol{\actp}{d-1}{t_\act}{\act}{\l})}}
      \lit{\RETURNE{\ttt}}


\lio{\ELSE\ \RETURNE{\fff}}
\end{tabbing}
\caption{Formal definition of $\lviol{v}{d}{t_\act}{\act}{\l}$.}
\label{fig:scViolateLoc}
%\label{alg:check-scViol}
\end{figure}



\bp
\label{prop:locScViol-implies-scViol}
 \label{prop:lviol-implies-viol}
Let $t$ be an arbitrary reachable state of BIP-system $(\B, Q_0)$.
For all interactions $\act \in \gamma$, and $\l \ge 2$, let $t_\act = t \pj \dsk{\act}{\l}$.
Then\\
\ind $\fa d \ge 1: \locScViol{v}{d}{t_\act}{\act}{\l} \imp \scViol{v}{d}{t}$.
\ep
\prf{
Proof is by induction on $d$. 

\noindent
\ul{Base case, $d=1$.} Assume $\lviol{v}{1}{t_\act}{\act}{\l}$ for some node $v$. Then, by 
Figure~\ref{fig:scViolateLoc}, 
$v$ is an interior node and an interaction $\actp$ of $\dsk{\act}{\l}$, andhas no outgoing 
wait-for-edges. Therefore, in $\wfg{\B}{t}$, it is still the case that $v$ has no outgoing 
wait-for-edges. Hence $\viol{v}{1}{t}$ holds.


\noindent
\ul{Inductive step, $d > 1$.}
Assume $\lviol{v}{d}{t_\act}{\act}{\l}$ for some node $v$ and some $d > 1$. 
We proceed by cases on Figure~\ref{fig:scViolateLoc}.

\be

\item \ul{$v$ is $v$ is an interior interaction $\actp$ and 
$(\fa \B_i : \actp \ar \B_i \in \wfg{\dsk{\act}{\l}}{t_\act} : \lviol{\B_i}{d-1}{t_\act}{\act}{\l})$}.

By the induction hypothesis applied to $\lviol{\B_i}{d-1}{t_\act}{\act}{\l}$, we have $\viol{\B_i}{d-1}{t}$. 
Since $\wfg{\dsk{\act}{\l}}{t_\act} \sub \wfg{\B}{t}$ by construction, we have 
$\actp \ar B_i \in \wfg{\B}{t}$ and $\viol{\B_i}{d-1}{t}$.
Hence by Definition~\ref{def:supercycle-violation}, Clause~\ref{def:supercycle.violation.component.out}, 
we have $\viol{v}{d}{t}$.


\item \ul{$v$ is a component $\B_i$ and
   $(\ex \actp : \B_i \ar \actp \in \wfg{\dsk{\act}{\l}}{t_\act} : \lviol{\actp}{d-1}{t_\act}{\act}{\l})$.}

similar to the previous case.

\ee
}










\subsubsection{Local strong connectedness condition}

\bd[Local strong connectedness violation, $\locConnViol{v}{t_\act}{\act}{\l}$]
\label{def:sConn.violation.loc}

Let $L$ be the nodes of $\wfg{\dsk{a}{\l}}{t_\act}$ that have no local
supercycle violation, \ie $L = \set{v \stt v \in \dsk{a}{\l} \land \neg (\ex d: \locScViol{v}{d}{t_\act}{\act}{\l}) }$.
Let $v$ be an arbitrary node in $L$. 
Let $WL = \wfg{\dsk{a}{\l}}{t_\act} \pj L$, \ie $WL$ is the subgraph of $\wfg{\dsk{a}{\l}}{t_\act}$ consisting of the
nodes in $L$, and the edges between those nodes that are also edges in $\wfg{\dsk{a}{\l}}{t_\act}$.

Then, $\locConnViol{v}{t_\act}{\act}{\l}$ holds iff:
\bn

\item \label{def:sConn.violation.loc:scc}
there does not exist a strongly connected supercycle $SSC$ such that $v \in SSC$ and $SSC \sub WL$, and

%\item \label{def:sConn.violation.loc:wait-for-out} every wait-for path $\pi$ from $v$ to a border node
%  of $\dsk{a}{\l}$ contains at least one node with a local supercycle violation

\item \label{def:sConn.violation.border}
either
    \bn

    \item \label{def:sConn.violation.loc:wait-for-out} every wait-for path $\pi$ from $v$ to a border node
      of $\dsk{a}{\l}$ contains at least one node with a local supercycle violation

     or

    \item \label{def:sConn.violation.loc:wait-for-in} every wait-for path $\pi$ from a border node
      of $\dsk{a}{\l}$ to $v$ contains at least one node with a local supercycle violation

    \en

\en
\ed



\bp
\label{prop:locConnViol-implies-ConnViol}
 \label{prop:locConnViol-implies-connViol}
Let $t$ be an arbitrary reachable state of BIP-system $(\B, Q_0)$.
For all interactions $\act \in \gamma$, and $\l \ge 2$, let $t_\act = t \pj \dsk{\act}{\l}$.
Then\\
\ind $\locConnViol{v}{t_\act}{\act}{\l} \imp \connViol{v}{t}$.
\ep
%
\bpr
By contradiction. Assume there exists a node $v$ in $\dsk{a}{\l}$ such that $\locConnViol{v}{t_\act}{\act}{\l} \land \neg \connViol{v}{t}$.
By $\neg \connViol{v}{t}$ and Definition~\ref{def:sConn.violation}, there exists a strongly connected
supercycle $SSC$ such that $v \in SSC$ and $SSC \sub \wfg{B}{t}$. Then, there are two cases:
%
\bn
\item $SSC \sub \wfg{\dsk{a}{\l}}{t_\act}$: let $x$ be any node in $SSC$. Since $x$ is a node in a supercycle, we have by
  Proposition~\ref{prop:scViol-implies-notInSC}, that $\neg (\ex d \ge 1: \scViol{x}{d}{t})$. Hence 
  $(\fa d \ge 1: \neg \scViol{x}{d}{t})$. Hence by Proposition~\ref{prop:locScViol-implies-scViol}, 
  we have $(\fa d \ge 1: \neg \locScViol{x}{d}{t_\act}{\act}{\l})$. Let $L, WL$ be as given in Definition~\ref{def:sConn.violation.loc}.
  Then $x \in L$, and since $x$ is an arbitrary node of $SSC$, we have $SSC \sub WL$. 
  Thus Clause~\ref{def:sConn.violation.loc:scc} of Definition~\ref{def:sConn.violation.loc} is violated.

\item $SSC \not\sub \wfg{\dsk{a}{\l}}{t_\act}$: then there exists a node $x \in SSC -
  \dsk{a}{\l}$. Since $v \in SSC$, there must exist a wait-for path $\pi$
  from $v$ to $x$ and a wait-for path $\pi'$ from $x$ to
  $v$. Since $v \in \dsk{a}{\l}$ and $x \not\in \dsk{a}{\l}$, it
  follows that both $\pi$, $\pi'$  cross a border node of
  $\dsk{a}{\l}$. Furthermore, since $\pi$, $\pi'$ are part of $SSC$, every node
  along $\pi$, $\pi'$ is in a supercycle, and so cannot have a supercycle violation.
  By Proposition~\ref{prop:locScViol-implies-scViol}, the nodes on
  $\pi$, $\pi'$  cannot have a local supercycle violation.
  Hence Clauses~\ref{def:sConn.violation.loc:wait-for-out} and
  \ref{def:sConn.violation.loc:wait-for-in} of Definition~\ref{def:sConn.violation.loc} are violated,
  since they require that at least one node along $\pi$, $\pi'$ respectively, have a local supercycle violation.
  
\en
In both cases,  Definition~\ref{def:sConn.violation.loc} is violated. 
But  Definition~\ref{def:sConn.violation.loc} must hold, since we have $\locConnViol{v}{t_\act}{\act}{\l}$. 
Hence the desired contradiction.
\epr


% Let $v$ be a node that is in a supercycle of $\wfg{\dsk{a}{\l}}{t_\act}$, and $t_\act$ a state of $\dsk{\act}{\l}$.
% Let $W$ be the result of removing from $\wfg{\dsk{a}{\l}}{t}$ every node $u$ such that 
% $(\ex d \ge 1: \locScViol{u}{d}{t_\act}{\act}{\l})$. Let $V$ be the maximal strongly connected component of $W$ that
% contains $v$. Then $\locConnViol{v}{t_\act}{\act}{\l}$ holds iff $V$ (by itself) is not a supercycle.
% For technical convenience, we also define $\locConnViol{v}{t_\act}{\act}{\l}$ to be false when $(\ex d \ge 1:
% \locScViol{v}{d}{t_\act}{\act}{\l})$, \ie when $v$ is not in a supercycle.
% Hence $\locConnViol{v}{t_\act}{\act}{\l}$ is always well-defined.




\subsubsection{Local formation violation condition}

\bd[Local Formation violation, $\locFormViol{v}{t_\act}{\act}{\l}$]
\label{def:locFormation.violation}
Let $v$ be a node of $\dsk{\act}{\l}$.
Then $\locFormViol{v}{t_\act}{\act}{\l}  \df  (\exs d \ge 1: \locScViol{v}{d}{t_\act}{\act}{\l}) \lor \locConnViol{v}{t_\act}{\act}{\l}$.
\ed
%Let $s \goesto[\act] t$ be a reachable transition. If, for every $\B_i \in \cmps{\act}$, 
%$\formViol{v}{t}$ holds, then $s \goesto[\act] t$ does not introduce a supercycle, \ie if $s$ is
%supercycle-free, then so is $t$. We establish this in the sequel.
%



\bp \label{prop:locFromViol-implies-formViol}
\label{prop:locformviol-implies-formviol}
Let $t$ be an arbitrary reachable state of BIP-system $(\B, Q_0)$.
For all interactions $\act \in \gamma$, and $\l \ge 2$, let $t_\act = t \pj \dsk{\act}{\l}$.
Then\\
%\ind $\fa d \ge 1: $\locFormViol{v}{t_\act}{\act}{\l}\locFormViol{v}{d}{t_\act}{\act}{\l} \imp \formViol{v}{d}{t}$.
\ind $ \locFormViol{v}{t_\act}{\act}{\l} \imp \formViol{v}{t}$.
\ep
%
\bpr
Assume that $\locFormViol{v}{t_\act}{\act}{\l}$ holds. Then, by Definition~\ref{def:formation.violation}, 
$(\exs d \ge 1: \locScViol{v}{d}{t_\act}{\act}{\l}) \lor \locConnViol{v}{t_\act}{\act}{\l}$.
We proceed by cases:
\bn
\item $(\exs d \ge 1: \locScViol{v}{d}{t_\act}{\act}{\l})$: hence $(\exs d \ge 1: \scViol{v}{d}{t})$ by Proposition~\ref{prop:locScViol-implies-scViol}.
\item $\locConnViol{v}{t_\act}{\act}{\l}$: hence $\connViol{v}{t}$ by Proposition~\ref{prop:locConnViol-implies-connViol}.
\en
By Definition~\ref{def:formation.violation},  $\formViol{v}{t}  \df (\exs d \ge 1: \scViol{u}{d}{t}) \lor \connViol{v}{t}$.
Hence we coonclude that $\formViol{v}{t}$ holds.
\epr




\subsubsection{Local AND-OR Condition}

\bd[$\LAO(\B, Q_0, \act, \l)$] \label{def:lao}
Let $\l \ge 2$, and let $s_\act \goesto[\act] t_\act$ be an arbitrary reachable transition of $\dsk{\act}{\l}$.
Then, in $t_\act$, the following holds. 
For every component $\B_i$ of $\cmps{\act}$:  
$\B_i$ has a supercycle formation violation that can be confirmed within $\dsk{\act}{\l}$.
Formally,\\
\ind  $\fa \B_i \in \cmps{\act}, \ex d \ge 1: \locFormViol{\B_i}{t_\act}{\act}{\l}$.
\ed

We showed previously that $\GAO$ implies deadlock-freedom, and so it remains to establish that $\LAO$ implies $\GAO$. 




\bl \label{lemma:loc.ANDOR.implies.glob.AND-OR}
Let $\act \in \gamma$ be an interaction of BIP-system $(\B, Q_0)$. Then\\
\ind $(\ex \l \ge 2: \LAO(\B, Q_0, \act, \l))$ implies $\GAO(\B, Q_0, \act)$
\el
%
%\prf{Immediate from Proposition~\ref{prop:locformviol-implies-formviol} and Definitions~\ref{def:global.ANDOR-cond}, \ref{def:lao}.}

\bpr
Assume $\LAO(\act, \l)$ for some $\l \ge 2$. 
%
Let $s \goesto[\act] t$ be an arbitrary reachable transition of BIP-system $(\B, Q_0)$, and let 
$s_\act \goesto[\act] t_\act$ be the projection of $s \goesto[\act] t$ onto $\dsk{\act}{\l}$.
By Corollary~\ref{cor:bip.reachability}, $s_\act \goesto[\act] t_\act$ is a reachable transition of $\dsk{\act}{\l}$.

\noindent
By Definition~\ref{def:lao}, we have for some $\l \ge 1$:\\
\ind for every reachable transition $s_\act \goesto[\act] t_\act$ of $\dsk{\act}{\l}$:\\
\ind \ind $\fa \B_i \in \cmps{\act} : \locFormViol{\B_i}{t_\act}{\act}{\l}$.
%\ind \ind  $\fa \B_i \in \cmps{\act}, \exs d \ge 1: \lviol{\B_i}{d}{t_\act}{\act}{\l}$. 

\noindent
From this and Proposition~\ref{prop:locFromViol-implies-formViol},\\
\ind for every reachable transition $s \goesto[\act] t$ of  $(\B, Q_0)$:\\ 
\ind \ind $\fa \B_i \in \cmps{\act} : \formViol{B_i}{t}$

\noindent
Hence, by Definition~\ref{def:global.ANDOR-cond}, $\GAO(\B, Q_0, \act)$ holds.
\epr



\bt \label{thm:LAO.SC-free-preserving}
$\LAO$ is supercycle-freedom preserving
\et
\prf{
Follows immediately from Lemma~\ref{lemma:loc.ANDOR.implies.glob.AND-OR} and Theorem~\ref{thm:GAO.SC-free-preserving}.
}



%%%%%%%%%%%%%%%%%%%%%%%%%%%%%%%%%%%%%%%%%%%%%%%%%%%%%%%%%%%%%%%%%%%%
\begin{figure}[ht]

\begin{tabular}{|l|l|}
\hline
$\scViol{v}{d}{t}$  & $v$ confirmed at depth $d$ to not be in supercycle\\ 
              %supercycle violation condition
$\locScViol{v}{d}{t_\act}{\act}{\l}$ & $v$ locally determined to not be in a supercycle\\
              % local supercycle violation condition:

$\connViol{v}{t}$ & $v$ not in a strongly connected supercycle \\
              %strongly connected supercycle violation: 

$\locConnViol{v}{t_\act}{\act}{\l}$ & $v$ locally determined to not be in a strongly connected supercycle \\
               %local strongly connected supercycle violation 

$\formViol{v}{t}$ & $v$ does not contribute to a supercycle\\
               %supercycle formation violation: 

$\locFormViol{v}{t_\act}{\act}{\l}$ & $v$ locally determined to not contribute to a supercycle\\
                %local supercycle formation violation condition

\hline
\end{tabular}

\caption{Summary of predicates}
\label{fig:summaryPredicates}
\end{figure}



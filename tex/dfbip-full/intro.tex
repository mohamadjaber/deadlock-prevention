
Deadlock freedom is a crucial property of concurrent and
distributed systems. With increasing system complexity,
the challenge of assuring deadlock freedom and other correctness
properties becomes even greater.
In contrast to the alternatives of (1) deadlock detection and recovery,
and (2) deadlock avoidance, we advocate deadlock prevention:
design the system so that deadlocks do not occur.
%during the normal functioning of the system.

Deciding deadlock freedom
of finite-state concurrent programs is PSPACE-complete in general
\cite[chapter 19]{papadimitriou1994computational}. To achieve
tractability, we can either make our deadlock freedom check
incomplete (sufficient but not necessary), or we 
can restrict the systems that we
check to special cases.  We choose the first option: a system
meeting our condition is free of both local and global
deadlocks, while a
system which fails to meet our condition may or may not be
deadlock free.

We generalize previous works \cite{Att99a,AC05,AE98} by removing
the requirement that interaction between processes be expressed pairwise, 
and also by applying to BIP~\cite{bip06}, a framework from which efficient
distributed code can be generated. In contrast, the model of concurrency
in \cite{Att99a,AC05,AE98} requires shared memory
read-modify-write operations with a large grain of atomicity.
%
The full paper, including proofs for all theorems, is
available on-line, as is our implementation of the method.
% is available at \url{http://www.cs.aub.edu.lb/pa07/files/pubs.html}.


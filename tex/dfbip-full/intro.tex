
% Since deciding deadlock-freedom for finite-state concurrent systems is PSPACE-complete, our
% criterion gives up completeness in return for tractability of evaluation. Our criterion can be
% evaluated by model-checking subsystems of the overall large system. The size of these subsystems
% depends only on the local topology of direct interaction between components, and \emph{not} on the
% number of components in the overall system.



Deadlock freedom is a crucial property of concurrent and distributed systems. With increasing system
complexity, the challenge of assuring deadlock freedom and other correctness properties becomes even
greater.  In contrast to the alternatives of (1) deadlock detection and recovery, and (2) deadlock
avoidance, we advocate deadlock prevention: design the system so that deadlocks do not occur.
%during the normal functioning of the system.

Deciding deadlock freedom of finite-state concurrent programs is PSPACE-complete, in general
\cite[chapter 19]{papadimitriou1994computational}. To achieve tractability, we present a criterion
for deadlock-freedom that can be evaluated by model-checking a set of subsystems of the overall
large system. If the subsystems are small, the criterion can be checked quickly, and is sound (if
true, it implies deadlock-freedom) but not complete (if false, then it yields no information about
deadlock). If the subsystems are larger, then our criterion becomes more ``accurate'': roughly
speaking, there is less possibility for the criterion to evaluate to false when the system is
actually deadlock-free. In the limit, when the set of subsystems includes the entire system itself,
our criterionbecomes complete, so that evaluation to false implies that the system is actually
deadlock-prone. Hence, our criterion only fails to resolve the question of deadlock-freedom 
when it exhuasts available computational resources becuase it has not yet evaluated to true, and the
subssystems being checked have become too large, because of state-explosion.

Our method thus combines the possibility of fast response together with theoretical completeness.
All deadlock-freedom checks presents in the literature to date are, to our knowledge, incomplete in
principle, and so remain incomplete even if unlimited computational resources are available.
Hence there criteria could fail to resolve deadlock freedom for theoretical reasons, as well as for 
lack of computational resources.
%
The reason for this incompleteness is that existing criteria all characterize deadlock by the
occurrence of a wait-for cycle, \eg as stated in Antonio \etal \cite{AGR16}, discussion of related
work:
\begin{quote}
All these methods were designed, to some extent, around the principle that under reasonable
assumptions about the system, any deadlock state would contain a proper cycle of ungranted requests.
\end{quote}
In a model of concurrency which includes choice of actions
(\eg BIP, CSP, I/O automata, CCS, etc) a wait-for cycle is an \emph{incomplete} characterization of
deadlock, since a process can be in a wait-for cycle, but not deadlocked, due to having a choice of
interaction with some process not in the wait-for cycle (see \fig{cycleOK} below).

Our method, in contrast, characterizes deadlock by the occurrence of a \emph{supercycle}
\cite{AE98,AC05}, which, very roughly, is the AND-OR analogue of a wait-for cycle. We show that
supercycles are a sound and complete characterization of deadlock: a system is deadlock-prone iff a
supercycle can arise in some reachable state.
%
We then present our criterion, which prevents the occurrence of supercycles in reachable states of
the system. We first present a ``global'' version of our criterion, which is both sound and complete
\wrt absence of supercycles, and then a ``local'' version, which is sound \wrt absence of
supercycles, and can be evaluated over small subsystems.

We present experimental results for dining philosophers and for a multi token-based resource
allocation system, which generalizes the token based Milner scheduler~\cite{milner}.  These show
that our method compares favorably with existing approaches.




% can either make our deadlock freedom check incomplete (sufficient but not necessary), or we can
% restrict the systems that we check to special cases.  We choose the first option: a system meeting
% our condition is free of both local and global deadlocks, while a system which fails to meet our
% condition may or may not be deadlock free.

% We generalize previous works \cite{Att99a,AC05,AE98} by removing the requirement that interaction
% between processes be expressed pairwise, and also by applying to BIP~\cite{bip06}, a framework from
% which efficient distributed code can be generated. In contrast, the model of concurrency in
% \cite{Att99a,AC05,AE98} requires shared memory read-modify-write operations with a large grain of
% atomicity.




\begin{abstract}

  We present a criterion for checking deadlock freedom of finite state systems expressed in BIP: a
  component-based framework for the construction of complex distributed systems.  Our criterion 
  is evaluated by model-checking a set of subsystems of the overall large system. If satisfied in small
  subsystems, it implies deadlock-freedom of the overall system (\ie is sound for
  deadlock-freedom). If not satisfied, then we increase the size of the subsystems and re-evaluate,
  as this improves the accuracy of the check.  In the limit, the subsystem being checked becomes the
  entire system, and then our criterion is also complete for deadlock-freedom. 
%
  Our criterion can thus only fail to decide the deadlock-freedom of a
  system because of computational limitations: state-space explosion
  sets in when the subsystems being checked become too large, and
  cannot be model-checked in practice.
%
  Our method thus combines the possibility of fast response together with theoretical completeness.
  This is in contrast to other criteria for deadlock-freedom, which are incomplete in
  principle, and so remain incomplete even if unlimited computational resources are available. 

  We present experimental results for dining philosophers and for a multi token-based resource
  allocation system, which generalizes Milner's token based scheduler~\cite{milner}.  These show
  that our method compares favorably with existing approaches.

%  For example, in verifying deadlock freedom of dining philosphers, our method shows linear increase
%  in computation time with the number of philosophers, whereas other methods (even those that use
%  abstraction) show super-linear increase, due to state-explosion.

\end{abstract}




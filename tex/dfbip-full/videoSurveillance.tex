


The video surveillance case study consists of a video capture device component
feeding video frames into several buffer components. 
Several edge detection components, each with its separate algorithm and configuration, 
extract edge marks from the frames of the buffers. 
An edge fusion component aggregates the results of the edge detection components 
into one edge mark result. 
Several segmentation components, each with its separate algorithm and configuration,
segment the frames from the buffers with the help of the edge marks resulting from 
the edge fusion component.
A segmentation fusion component aggregates the results of the segmentation components
into a final segmentation result. 
The edge marks and the segmentation results from the fusion edge and segmentation components, 
respectively, 
are fed to an object identification component. 
The object identification component reports whether specific objects exist in the video or not. 


%We assume the system was designed for military usages that need high 
%resolution analysis, and then it is used for home surveillance purposes
%where lower resolutions are suitable. 
%If only home surveillance resolution
%is desired, just two detectors and one segmentor
%are needed. 




%\usepackage{pgf}
%\usepackage{tikz}
%\usetikzlibrary{arrows,automata}

\begin{figure}
\begin{tikzpicture}[->,>=stealth',shorten >=1pt,auto,node distance=2.8cm,
                    semithick]
  \tikzstyle{every state}=[fill=red,draw=none,text=white]

  \node[initial,state] (A)                    {$s_0$};
  \node[state]         (B) [above right of=A] {$s_{01}$};
  \node[state]         (C) [below right of=A] {$s_{02}$};
  \node[state]         (D) [below right of=B] {$s_1$};
  \node[state]         (E) [below of=C]       {$s_2$};

  \path (A) edge              node {edge,true} (B);
  
  \path (B) edge              node {edge,true} (D);
  
  \path (D) edge  [bend left] node {label,true} (E);
  
  \path (E) edge  [bend left] node {reset0,true} (A);
        
  \path (A)  edge             node {segment,true} (C);
  \path (C)  edge [bend left] node {segment,true} (D);
\end{tikzpicture}
\caption{State diagram of the object identification 
  component}
\end{figure}


We now present a condition for deadlock-freedom of infinite-state systems,
which can be checked using \eg an SMT solver.
$\LDFC(\BD_a)$ is implied by the conjunction of the following Hoare triples:
%
     $$\{ I \land (\land B \in C_a : ready_a(B)) \}\ a\ \{ (\land B_i \in \BD_a : noIn(B_i) \lor noOut(B_i)) \}$$ 
%
  where $I$ is any invariant, i.e., any predicate that characterizes a superset of the reachable
  states. $a'$ is an interaction whose participant set is contained in $\BD_a$, and $en(a')$ means that
  $a'$ is enabled. $noIn(B_i)$ means that $B_i$ has no incoming wait-for edges. $noOut(B_i)$ means
  that no interaction that $B_i$ readies has outgoing wait-for edges.

  Assuming that the state changes effected by interactions can be described by simple code,
  e.g., conditionals and assignments, but not loops, these triples can be mechanically reduced to
  first order formulae, e.g., using weakest preconditions \cite{Dij75}. The validity of these
  formulae can then be checked using SMT solvers.


\documentclass[11pt]{article}
\pagestyle{empty}

\usepackage{url}

\setlength{\topmargin}{0.5in}
\setlength{\headheight}{0in}
\setlength{\headsep}{0.1in}
\setlength{\footskip}{0.3in}


\setlength{\oddsidemargin}{0in}
\setlength{\textheight}{8.0in}
\setlength{\textwidth}{6.5in}

\setlength{\parindent}{0em}
\setlength{\parskip}{1.5ex}


%%%%%%%%%%%%%%%%%%%%%%%%%%%%%%%%%%%%%%%%%%%
\begin{document}



\begin{center}
\Large{Differences between the conference version and the submission.}
\end{center}

\begin{itemize}


\item The conference version gives a restricted ``linear'' criterion for local and global deadlock freedom which is \emph{not} complete for local and
  global deadlock-freedom.  The
  submission gives an ``alternating'' AND-OR criterion, which is complete for local and global deadlock freedom. The linear criterion can fail in
  cases where the alternating criterion succeeds in verifying deadlock freedom. The linear condition is actually a special case of the AND-OR condition.

\item The submission provides new results concerning the graph-theoretic properties of the waiting patterns that constitute a local or global deadlock.

\item Submission provides experiments that are new and different from the conference version, and which, among other results, give an example where
  the linear criterion fails while the AND-OR criterion succeeds. Experiments also deal with more challenging examples than in the conference version,
  including a generalization of Milner's token-based scheduler.

\end{itemize}


\end{document}





\subsection{Supercycle Membership} 


\bd[Supercycle membership, $\scyc{\B}{s}{v}$]
\label{defn:supercycle.membership}
Let $v$ be a node of $\wfg{\B}{s}$. Then
$\scyc{\B}{s}{v}$ holds iff there exists a supercycle $SC \sub
\wfg{\B}{s}$ such that $v \in SC$. 
%Define $\scyc{\B}{s}{\B_i}$, $\scyc{\B}{s}{\act}$ to mean that $\B_i$,
%$\act$, respectively, are nodes of a supercycle of $\wfg{\B}{s}$.
\ed

If a component or interaction is not a node of a supercycle, then we say that it has a
\emph{SC-violation}, \ie a supercycle-violation.
%

Define
$\preds{\B}{s}{v} = \set{w \stt w \ar v \in \wfg{\B}{s}}$ and 
$\succs{\B}{s}{v} = \set{w \stt v \ar w \in \wfg{\B}{s}}$.
The definition of a supercycle (\defn{supercycle}) 
imposes certain constraints on supercycle membership of a node \wrt its predecessors and successors
in the wait-for-graph, as follows:

\bp[Supercycle-membership constraints]
\label{prop:sc-membership}
Let $\act, \B_i$ be nodes of $\wfg{\B}{s}$. Then
\bn

\item \label{clause:sc-membership:comp-out}
$\scyc{\B}{s}{\B_i} \ifof (\fa \act \in \succs{\B}{s}{\B_i} : \scyc{\B}{s}{\act})$.

\item \label{clause:sc-membership:comp-in}
$\scyc{\B}{s}{\B_i} \imp (\fa \act \in \preds{\B}{s}{\B_i} : \scyc{\B}{s}{\act})$.

\item \label{clause:sc-membership:act-out}
$\scyc{\B}{s}{\act} \ifof (\ex \B_i \in \succs{\B}{s}{\act} : \scyc{\B}{s}{\B_i})$.

\item \label{clause:sc-membership:act-in}
$\scyc{\B}{s}{\act} \folf (\ex \B_i \in \preds{\B}{s}{\act} : \scyc{\B}{s}{\B_i})$.

\en
\ep
%
\bpr
We deal with each clause in turn.


%%%%%%%%%%%%%%%%%%%%%
\textit{Proof of \clause{sc-membership:comp-out}}.
%
Assume $\scyc{\B}{s}{\B_i}$, and let $\SC \sub \wfg{B}{s}$ be the supercycle containing $\B_i$.  Let
$\actp \in \succs{\B}{s}{\B_i}$.  By \defn{supercycle}, \clause{supercycle.component-blocked},
$\actp \in \SC$.  Hence $(\fa \act \in \succs{\B}{s}{\B_i} : \scyc{\B}{s}{\act})$.
We conclude
$\scyc{\B}{s}{\B_i} \imp (\fa \act \in \succs{\B}{s}{\B_i} : \scyc{\B}{s}{\act})$.
%
Now assume $(\fa \act \in \succs{\B}{s}{\B_i} : \scyc{\B}{s}{\act})$, and let 
$\SC$ be the union of all the supercycles containing all the $\act \in \succs{\B}{s}{\B_i}$. 
By \prop{supercycle:union}, $\SC \sub \wfg{B}{s}$ is a supercycle.
Let $\SC'$ be $\SC$ with $\B_i \ar \act$ added, for all 
$\act \in \succs{\B}{s}{\B_i}$.
Then $\SC'$ is a supercycle by 
\defn{supercycle}, and also $\SC' \sub \wfg{B}{s}$. Hence $\scyc{\B}{s}{\act}$.
We conclude 
$\scyc{\B}{s}{\B_i} \folf (\fa \act \in \succs{\B}{s}{\B_i} : \scyc{\B}{s}{\act})$.




%%%%%%%%%%%%%%%%%%%%%%%%%%
\textit{Proof of \clause{sc-membership:comp-in}}.
%
Assume $\scyc{\B}{s}{\B_i}$, so that $\SC \sub \wfg{B}{s}$ is the supercycle containing $\B_i$.
Let  $\act \in \preds{\B}{s}{\B_i}$, and let $\SC'$ be $\SC$ with 
$\act \ar \B_i$ added. Hence $\SC'$ is a supercycle 
by Definition~\ref{def:supercycle},
  Clause~\ref{def:supercycle.action-blocked}.
Since $\act$ was chosen arbitrarily, we conclude 
$(\fa \act \in \preds{\B}{s}{\B_i} : \scyc{\B}{s}{\act})$.



%%%%%%%%%%%%%%
\textit{Proof of \clause{sc-membership:act-out}}.
%
Assume $\scyc{\B}{s}{\act}$, and let $\SC \sub \wfg{B}{s}$ be the supercycle containing $\act$.  By
\defn{supercycle}, \clause{supercycle.action-blocked}, there exists a
$\B_i \in \succs{\B}{s}{\act}$ such that $\B_i \in \SC$.  Hence $\scyc{\B}{s}{\B_i}$.
We conclude
$\scyc{\B}{s}{\act} \imp (\ex \B_i \in \succs{\B}{s}{\act} : \scyc{\B}{s}{\B_i})$.
%
Now assume $(\ex \B_i \in \succs{\B}{s}{\act} : \scyc{\B}{s}{\B_i})$, and let 
$\SC \sub \wfg{B}{s}$ be the supercycle containing some $\B_i \in \succs{\B}{s}{\act}$. 
Let $\SC'$ be $\SC$ with $\act \ar \B_i$ added. Then $\SC'$ is a supercycle by 
\defn{supercycle}, and also $\SC' \sub \wfg{B}{s}$. Hence $\scyc{\B}{s}{\act}$.
We conclude 
$\scyc{\B}{s}{\act} \folf (\ex \B_i \in \succs{\B}{s}{\act} : \scyc{\B}{s}{\B_i})$.


%%%%%%%%%%%%%%%%%%%%%%%%
\textit{Proof of \clause{sc-membership:act-in}}.  
%
Assume $\neg \scyc{\B}{s}{\act}$, so that $\act$ is not in any supercycle of $\wfg{B}{s}$.
Let $\B_i \in \preds{\B}{s}{\act}$. 
By \defn{supercycle}, \clause{supercycle.component-blocked}, 
$\B_i$ cannot be in any supercycle of $\wfg{B}{s}$, since all $\actp \in \succs{\B}{s}{\B_i}$ must
also be in the supercycle. Hence $\neg \scyc{\B}{s}{\B_i}$.
Since $\B_i$ was chosen arbitrarily, we conclude
$\neg \scyc{\B}{s}{\act} \imp  (\fa \B_i \in \preds{\B}{s}{\act} : \neg \scyc{\B}{s}{\B_i})$, the
contrapositive of \clause{sc-membership:act-in}.  
\epr

Note that \clause{sc-membership:comp-in} cannot be strengthened to an equivalence: if all
the interactions that wait for a component $\B_i$ are in a supercycle, then $\B_i$ itself may or may
not be in a supercycle, depending on whether $\B_i$ is waiting for some $\actp$ that is not in a
supercycle.
%
Likewise, \clause{sc-membership:act-in} cannot be strengthened to an equivalence: if $\act$
is in a supercycle, then any component $\B_i$ that waits for $\act$ may or may not be in a 
supercycle, depending on whether $\B_i$ is waiting for some $\actp$ that is not in a supercycle. 




While \prop{sc-membership} gives relationships between supercycle membership of a node and both its
successors and predecessors, nevertheless \defn{supercycle} implies that the ``causality'' of
supercycle-membership of a node $v$ is from the successors of $v$ to $v$, \ie membership of $v$ in a
supercycle is caused only by membership of $v$'s successors in a supercycle. Repeating this step, we
infer that $v$'s supercycle-membership is caused by the subgraph of the wait-for graph that is
reachable from $v$.

Hence, we follow outgoing wait-for edges in computing supercycle-membership. Actually, it turns out
to be easier to compute the negation of supercycle membership, which we call \intro{supercycle
  violation}. This is because supercycle-violation has a base case: when a node has no outgoing
wait-for edges. We need a base case, and an inductive definition, because 
a node that is not in any supercycle may nevertheless be a node of a wait-for cycle, since a cycle
of wait-for-edges does not necessarily imply a supercycle. Hence, to compute supercycle violation
properly, we introduce a notion of the \emph{level} of a violation. A node with no
outgoing wait-for edges has a level-1 violation. A node whose violation is based on outgoing edges
to neighbors whose violation level is at most $d-1$, has itself a level-$d$ violation. We
formalize the notion of \emph{level-$d$ supercycle violation} as the predicate $\viol{v}{d}{t}$,
defined by induction on $d$.


\bd[Supercycle violation, $\scViol{v}{d}{t}$]
\label{def:supercycle-violation}
\label{def:supercycle.violation}
Let $t$ be a state of $(\B, Q_0)$, $v$ be a node of $\wfg{\B}{t}$, and $d$ an integer $\ge 1$.
We define the predicate $\viol{v}{d}{t}$ by induction on $d$, as follows. We indicate the
justification for each clause of the definition.

\noindent
\ul{Base case, $d=1$.} $\viol{v}{1}{t}$ iff $v$ is an interaction $\act$ and it has no outgoing wait-for-edges, otherwise $\neg \viol{v}{1}{t}$.
%
Justification: if $v$ has no outgoing wait-for-edges, then it cannot be in a supercycle.  Note that $v$ must be an
interaction in this case, since a component must have at least one outgoing wait-for edge at all times.

\noindent
\ul{Inductive step, $d > 1$.}  $\viol{v}{d}{t}$ iff any of the following cases hold. Otherwise $\neg \viol{v}{d}{t}$.
%
\bn

\item  \label{def:supercycle.violation.component.out}
$v$ is a component $\B_i$ and there exists interaction $\act$ such that $\B_i \ar \act \in \wfg{\B}{t}$ and $(\ex d' : 1 \le d' < d: \viol{\act}{d'}{t})$.
That is, $\B_i$ enables an interaction $\act$ which has a level-$d'$ supercycle-violation, for some $d' < d$.
Justification is \prop{sc-membership}, \clause{sc-membership:comp-out}.


\item \label{def:supercycle.violation.interaction.out}
$v$ is an interaction $\act$ and for all components $\B_i$ such that $\act \ar \B_i \in \wfg{\B}{t}$, we have $(\ex d' : 1 \le d' < d: \viol{\B_i}{d'}{t})$.
That is, each component $\B_i$ that $\act$ waits for has a level-$d'$ supercycle-violation, for some $d' < d$.
Justification is \prop{sc-membership}, \clause{sc-membership:act-out}.

\en
\ed
%
Figure~\ref{fig:scViolate} gives a formal, recursive definition of $\viol{v}{d}{t}$. 
The notation $v = \B_i$ means that $v$ is some component $\B_i$. Likewise, 
$v = \act$ means that $v$ is some interaction $\act$.
Line 0 corresponds to the base case, line 1 corresponds to item 1 of the inductive case, and line 2 corresponds to item 2 of the inductive case.
Line 3 handles all cases that do not return true.


\begin{figure}[t]
\setcounter{lctr}{-1}
\horline
\begin{tabbing}\label{alg:check-scViol}
aa\= aa\= aa\= \kill
$\viol{v}{d}{t}$\\
\lioc{\IFC{d = 1 \land v = \act \land \neg (\ex \B_i : \act \ar \B_i \in \wfg{\B}{t})}\ \RETURNE{\ttt}\ \FI}{\cmnt base case for \ttt\ result}
\lio{\IFC{v = \B_i \land (\ex \act : \B_i \ar \act \in \wfg{\B}{t} : (\ex d' : 1 \le d' < d : \viol{\act}{d'}{t}))} \ \RETURNE{\ttt}\ \FI}
\lio{\IFC{v = \act \land (\fa \B_i : \act \ar \B_i \in \wfg{\B}{t} : (\ex d' : 1 \le d' < d : \viol{\B_i}{d'}{t}))} \  \RETURNE{\ttt}\ \FI}
\lioc{\RETURNE{\fff}}{\cmnt no case for \ttt\ result, so result is \fff}
\end{tabbing}
\vspace{-3ex}
\horline
\caption{Formal definition of $\viol{v}{d}{t}$}
\label{fig:scViolate}
\end{figure}


In the sequel, we say sc-violation rather than ``supercycle violation.''  The crucial result is
that, if $v$ has a level-$d$ sc-violation, for some $d \ge 1$, then $v$ cannot be a node of a
supercycle.


\bp[Soundness of supercycle violation \wrt supercycle non-membership]
 \label{prop:supercycle-violation} \label{prop:scViol-implies-notInSC}
If $(\ex d \ge 1: \viol{v}{d}{t})$ then  $\neg \scyc{\B}{t}{v}$,
\ie supercycle violation implies supercycle non-membership.
%$v$ is not a node of a supercycle in $\wfg{\B}{t}$.
\ep
\prf{
Proof is by induction in $d$. 

\noindent
\textit{Base case, $d=1$}. $v$ has no outgoing edges. Hence  $v$ cannot be in a supercycle.

\noindent
\textit{Induction step, $d >1$}. Assume that $v$ has a level $d$ $SC$-violation. We have two cases. 

\topcase{1}{$v$ is a component $\B_i$}   %\scase{1.1}{$v$ has an out-violation}
Hence there exists an interaction $\act$ such that $\B_i \ar \act \in \wfg{\B}{t}$ and $\act$ has a level-$(d-1)$
$SC$-violation. By the induction hypothesis, $\neg \scyc{\B}{t}{\act}$.
By \prop{sc-membership}, \clause{sc-membership:comp-out}, 
$\neg \scyc{\B}{t}{\B_i}$.

\topcase{2}{$v$ is an interaction $\act$}    %\scase{2.1}{$v$ has an out-violation}
Hence for all components $\B_i$ such that $\act \ar \B_i \in \wfg{\B}{t}$, $\B_i$ has a level-$(d-1)$
$SC$-violation. By the induction hypothesis, 
$(\fa \B_i : \act \ar \B_i \in \wfg{\B}{t}: \neg \scyc{\B}{t}{\B_i})$.
By \prop{sc-membership}, \clause{sc-membership:act-out}, 
$\neg \scyc{\B}{t}{\act}$.
}


%In the other direction, we have a slightly weaker result: if $v$ has no level-$d$ scsc-violation,
%for all $d \ge 1$, then $v$ is a node of a supercycle.


\bp[Completeness of supercycle violation \wrt supercycle non-membership]
 \label{prop:notInSC-implies-scViol}
If  $\neg \scyc{\B}{t}{v}$ then $(\ex d \ge 1: \viol{v}{d}{t})$,
\ie supercycle non-membership implies supercycle violation.
\ep
%
\prf{
%
We establish the contrapositive $(\fas d \ge 1: \neg \viol{v}{d}{t})$ then $\scyc{\B}{t}{v}$.
Let $V$ be the set of nodes in $\wfg{\B}{t}$ with a supercycle-violation, \ie
$V = \set{w \stt w \in \wfg{\B}{t} \land (\exs d: \viol{w}{d}{t})}$.  Let $\bV$ be the remaining nodes, \ie all nodes
in $\wfg{\B}{t}$ that do not have a supercycle-violation, so 
$\bV = \set{w \stt w \in \wfg{\B}{t} \land (\fas d \ge 1: \neg \viol{v}{d}{t})}$.

If $\bV$ is empty then the proposition holds vacuously and we are done. So assume that $\bV$ is non-empty and let 
$v$ be an arbitrary node in $\bV$.

\topcase{1}{$v$ is a component $\B_i$}  
Suppose that there is a wait-for-edge from $v$ to some interaction $\act$ that is in $V$.
Then, by Definition~\ref{def:supercycle.violation}, $v$ has a supercycle violation, which contradicts the choice of $v$
as a member of $\bV$. Hence all wait-for-edges starting in $v$ must end in a node in $\bV$.

\topcase{2}{$v$ is an interaction $\act$}  
Suppose that every wait-for-edge from $v$ to some component  $\B_i$ that is in $V$.
Then, by Definition~\ref{def:supercycle.violation}, $v$ has a supercycle violation, which contradicts the choice of $v$
as a member of $\bV$. Hence some wait-for-edge starting in $v$ must end in a node in $\bV$.


Hence we have that $\bV$ satisfies all three clauses of Definition~\ref{def:supercycle}: it is
nonempty, each component in $\bV$ has all its enabled interactions also in $\bV$, and each
interaction in $\bV$ waits for a component in $\bV$.  Hence $\bV$ as a whole is a supercycle. Since
the nodes of $\bV$ are, by definition of $\bV$, exactly the nodes $v$ such that
$(\fas d \ge 1: \neg \viol{v}{d}{t})$, we have that any such node $v$ is a node of a supercycle in
$\wfg{\B}{t}$, \ie $\scyc{\B}{t}{v}$. Hence the Proposition is established.  }

%Note that the above implies that there are no wait for edges from component in $U$ to an interaction outside $U$???


\bp \label{prop:scViol-iff-notInSC}
$\neg \scyc{\B}{t}{v}$ iff $(\ex d \ge 1: \viol{v}{d}{t})$.
\ep
\prf{Immediate from Propositions~\ref{prop:scViol-implies-notInSC} and
  \ref{prop:notInSC-implies-scViol}.}





%%%%%%%%%%%%%%%%%%%%%%%%%%
\subsection{The supercycle formation condition}

We use the structural properties of supercycles (\secn{supercycle-structural}) and the 
dynamics of wait-for graphs (\prop{wait-for-edge-preservation}) to define a condition that 
must hold whenever a supercycle is created. Negating this condition then implies the absence of
supercycles. 


\bp[Supercycle formation condition] \label{prop:supercycle-formation}
Assume that $s \goesto[\act] t$ is a transition of $(\B, Q_0)$, $\wfg{\B}{s}$ is supercycle-free, and that $\wfg{\B}{t}$
contains a supercycle.  Then, in $\wfg{\B}{t}$, there exists a $\CC$ such that
\bn
\item $\CC$ is a subgraph of $\wfg{\B}{t}$
\item $\CC$ is strongly connected
\item $\CC$ is a supercycle
\item  in $\wfg{\B}{t}$, there is no wait-for edge from a node in $\CC$ to a node outside of $\CC$
\item there exists a component $\B_i \in \cmps{\act}$ such that $\B_i$ is in $\CC$
\en
\ep
%
\bpr
By assumption, there is a supercycle $\SC$ that is a subgraph of $\wfg{\B}{t}$.
By Proposition~\ref{prop:supercycle:contains-mssc}, $\SC$ contains a
subgraph $\CC$ that is strongly connected, is itself a supercycle, and
such that there is no wait-for-edge from a node in $\CC$ to a node outside of $\CC$.
This establishes Clauses~1--4.

Now suppose $\B_i \not\in \CC$ for every $\B_i \in \cmps{\act}$. Then, no edge in $\CC$ is
$\B_i$-incident.  Hence, by Proposition~\ref{prop:wait-for-edge-preservation}, every edge in $\CC$
is an edge in $\wfg{\B}{s}$. Hence $\CC$ is a subgraph of $\wfg{\B}{s}$.
%
Now let $v$ be an arbitrary node in $\CC$.
%
Suppose $v$ is a component $\B_j$.  By assumption, $\B_j \not\in \cmps{\act}$, and so
$s \pj \B_j = t \pj \B_j$ by Definition~\ref{def.bip.composition}. Hence $\B_j$ enables the same set
of interactions in state $s$ as in state $t$. Also, in $\wfg{\B}{t}$, all of $\B_j$'s wait-for edges
must end in an interaction that is in $\CC$, since $\CC$ is a supercycle in $\wfg{\B}{t}$. Hence the
same holds in $\wfg{\B}{s}$.
%
If $v$ is an interaction, it must also have a wait-for-edge $e'$ to some component $\B_j \in \CC$,
since $\CC$ is a supercycle in $\wfg{\B}{t}$. Hence this also holds in $\wfg{\B}{s}$.
%
Hence $v$ has enough successors in $\CC$ to satisfy the supercycle definition (\defn{supercycle}).
%
We conclude that $\CC$ by itself is a supercycle in $\wfg{\B}{s}$, which contradicts the assumption
that $\wfg{\B}{s}$ is supercycle-free. Hence, $\B_i \in \CC$ for some $\B_i \in \cmps{\act}$, and so
Clause~5 is established.  
\epr



%\subsection{Violations of the supercycle formation condition}

\subsection{General supercycle violation condition}

We use \prop{supercycle-formation} to formulate a condition that prevents the formation of
supercycles. 
For transition $s \goesto[\act] t$, we determine for every component $\B_i \in \cmps{\act}$ whether
it is possible for $\B_i$ to be a node in a strongly-connected supercycle $\CC$ in $\wfg{\B}{t}$. 
%If not, then we say that $\B_i$ satisfies the \emph{supercycle violation} condition, and we
%formalize this condition below.
There are two ways for $\B_i$ to not be a node in a strongly-connected supercycle:
\bn
\item \textit{no supercycle membership}: $\B_i$ is not a node of any supercycle, \ie $\neg \scyc{\B}{s}{\B_i}$.

\item \textit{no strong-connectedness}: $\B_i$ is a node in a supercycle, but not a node in a \emph{strongly-connected} supercycle. 

\en

%\subsection{Strong connectedness condition}
% commentary: strong connectedness violation cannot be defined "locally" in a good way, like supercycle violation
% can, since it depends on reachability, which is a global property. For a supercycle, once the set
% of nodes is fixed, the required wait-for relations are all local properties. Hence, violation is
% also a local property. For strong connectedness, the best local property is "no incoming" or "no
% outgoing", which is just the linear condition of the FORTE 2013 paper. Hence a local condition for
% strong connectedness violation is no improvement over FORTE 2013.

%Given that $\B_i$ is a node in a supercycle, we wish to determine whether or not it is a node in a
%\emph{strongly-connected} supercycle. We do this by removing all nodes with supercycle-violations,
%and then finding the maximal strongly connected components of the resulting wait-for subgraph.

We formalize the second condition as follows.

\bd[Strong connectedness violation, $\connViol{v}{t}$]
\label{def:sConn.violation}
 Let $v$ be a node of $\wfg{\B}{t}$.   Then $\connViol{v}{t}$ holds iff there does not exist a 
strongly connected
 supercycle $SSC$ such that $v \in SSC$ and $SSC \sub \wfg{B}{t}$.
\ed






The general supercycle violation condition is then a disjunction of the supercycle violation condition
and the strong connectedness violation conditions.


\bd[General supercycle violation, $\genViol{v}{t}$]
\label{def:formation.violation} 
\label{defn:formation.violation} 
Let $v$ be a node of $\wfg{\B}{t}$.
Then $\genViol{v}{t}  \df (\exs d \ge 1: \scViol{v}{d}{t}) \lor \connViol{v}{t}$.
\ed
%
Let $s \goesto[\act] t$ be a reachable transition. If, for every $\B_i \in \cmps{\act}$,
$\formViol{v}{t}$ holds, then, as we show below, $s \goesto[\act] t$ does not introduce a
supercycle, \ie if $s$ is supercycle-free, then so is $t$.  However, evaluating this condition over
all global transitions is subject to state explosion, and so we formulate below a ``local'' version
of the general condition, which can be evaluated in ``small subsystems'', and so we often avoid
state-explosion. Hence the advantage of the local versions is that they are usually efficiently
computable, as we show in the sequel.  We also formulate a ``linear'' condition (both global and
local), which is simpler (but ``more incomplete'') than the general condition, and so is easier to
evaluate.

We remark that, as shown above $(\exs d \ge 1: \scViol{v}{d}{t})$ implies that $v$ cannot be in a
supercycle. Hence, $v$ cannot be in a strongly-connected supercycle.  Hence
$(\exs d \ge 1: \scViol{v}{d}{t})$ implies $\connViol{v}{t}$. It is however convenient to state the
formation violation condition in this manner, since we will formulate a local version
for each of $(\exs d \ge 1: \scViol{v}{d}{t})$ and $\connViol{v}{t}$, and the implication does not
necessarily hold for the local versions. 

We therefore now have four deadlock-freedom conditions: global general, local general, 
global linear, and local linear. We therefore define an abstract version of 
the deadlock-freedom condition first.


%%%%%%%%%%%%%%%%%%%%%%%%%%%%%%%%%%%%%%%%%%%%%
\subsection{Abstract supercycle freedom conditions}

Since we will present several conditions for supercycle-freedom, we now present an abstract
definition of the essential properties that all such conditions must have.  The key idea is that
execution of an interaction $\act$ does not create a supercycle, and so any condition which implies 
this for $\act$ is sufficient. if a different condition implies the same for another interaction
$\actp$, this presents no problem \wrt establishing deadlock-freedom. Hence, it is sufficient to
have one such condition for each interaction in  $(\B,Q_0)$. Since each condition restricts the
behavior of interaction execution, we call it a ``behavioral restriction condition''.

%Moreover, since implication of $\GAO(\act)$ can be on a ``per interaction'' basis, with different
%conditions being applied to different interactions, we have:


\bd[Behavioral restriction condition]
A \introdf{behavioral restriction condition} $\BC$ is a predicate
$\BC: (\B, Q_0, \act) \to \set{\ttt, \fff}$.
\ed
%
$\BC$ is a predicate on the effects of a particular interaction $\act$ within a given system
$(\B, Q_0)$.

\bd[Supercycle-freedom preserving] \label{def:SC-free-preserving}
A behavioral restriction condition $\BC$ is \introdf{supercycle-freedom preserving} iff, for every system 
$(\B, Q_0)$ and $\act \in \gamma$ such that $\BC(\B, Q_0, \act) = \ttt$, the following holds:\\[2ex]
%
%\ind for all interactions $\act$ of $\B$ (\ie $a \in \gamma$),\\
\ind \ind for every reachable transition $s \goesto[\act] t$ of $(\B, Q_0)$\\
\ind \ind \ind if $s$ is supercycle-free, then $t$ is supercycle-free.
\ed


\bt[Deadlock-freedom via supercycle-freedom preserving restriction]
\label{theorem:SC-free-preserving.deadlock-free}
\label{thm:SC-free-preserving.deadlock-free}
Assume that
\bn
\item \label{theorem:SC-free-preserving.initial}
      for all $s_0 \in Q_0$, $\wfg{\B}{s_0}$ is supercycle-free, and
\item \label{theorem:SC-free-preserving.reachable-transitions}
   there exists a supercycle-freedom preserving restriction $\BC$ such that,
   for all $\act \in \gamma$: $\BC(\B, Q_0, \act) = \ttt$ 
\en
Then for every reachable state $u$ of $(\B, Q_0)$:  $\wfg{\B}{u}$ is supercycle-free.
\et
%
\bpr Let $u$ be an arbitrary reachable state. The proof is by induction on the length of the finite
execution $\al$ that ends in $u$.  Assumption~\ref{theorem:SC-free-preserving.initial} provides the
base case, for $\al$ having length 0, and so $u \in Q_0$.  For the induction step, we establish: for
every reachable transition $s \la{\act} t$, $\wfg{\B}{s}$ is supercycle-free implies that
$\wfg{\B}{t}$ is supercycle-free. This is immediate from
Assumption~\ref{theorem:SC-free-preserving.reachable-transitions}, and
Definition~\ref{def:SC-free-preserving}.  \epr

Since the above proof does not make any use of the requirement that there is a single restriction
$\BC$ for all interactions, we immediately have:


\bco[Deadlock-freedom via several supercycle-freedom preserving restrictions]

\label{cor:SC-free-preserving.deadlock-free}
Assume that
\bn
\item \label{cor:SC-free-preserving.initial}
      for all $s_0 \in Q_0$, $\wfg{\B}{s_0}$ is supercycle-free, and
\item \label{cor:SC-free-preserving.reachable-transitions}
   for all $\act \in \gamma$, there exists a supercycle-freedom preserving restriction $\BC$: $\BC(\B, Q_0, \act) = \ttt$ 
\en
Then for every reachable state $u$ of $(\B, Q_0)$:  $\wfg{\B}{u}$ is supercycle-free.
\eco
%
\bpr
Similar to the proof of \thm{SC-free-preserving.deadlock-free},  except that, for
the transition $s \la{\act} t$, use the  supercycle-freedom preserving restriction $\BC$
corresponding to $\act$.
\epr


%%%%%%%%%%%%%%%%%%%%%%%%%%%%%%%
\subsection{Overview of the four supercycle-freedom preserving restrictions}

The supercycle formation condition
(Proposition~\ref{prop:supercycle-formation}) tells us that, when a
supercycle $\SC$ is created, some component $\B_i$ that participates
in the interaction $\act$ whose execution created $\SC$, must be a
node of a strongly connected component $\CC$ of $\SC$, and moreover
$\CC$ is itself a supercycle in its own right. In a sense, $CC$ is the
``essential'' part of $\SC$.

Hence, for a BIP system $(\B, Q_0)$, our fundamental condition for the
prevention of supercycles is that for every reachable transition
$s \goesto[\act] t$ resulting from execution of $\act$, every
component $\B_i$ of $\act$ must exhibit a supercycle-violation
(Definition~\ref{def:supercycle.violation}) in state $t$ (the state
resulting from the execution of $a$). For a given BIP system
$(\B, Q_0)$ and interaction $\act$, we denote that condition
$\GAO(\B, Q_0, \act)$, and define it formally below.  This condition
is, in a sense, the ``most general'' condition for supercycle-freedom.

If $\GAO(\B, Q_0, \act)$ holds, and global state $s$ is
supercycle-free, and $s \goesto[\act] t$, then it follows (as we
establish below) that global state $t$ is also supercycle-free.  So,
by requiring (1) that all initial states are supercycle-free, and (2)
that $\GAO(\B, Q_0, \act)$ holds for all interactions
$\act \in \gamma$, we obtain, by straightforward induction on length
of executions, that every reachable state is supercycle-free.

It also follows that any condition which implies $\GAO(\B, Q_0, \act)$ is also sufficient to guarantee  supercycle-freedom, and
hence deadlock-freedom. We exploit this in two ways:
\bn

\item To provide a ``linear'' condition, $\GLin$, that is easier to evaluate than $\GAO$, since it requires only the
evaluation of lengths of wait-for-paths, \ie it does not have the ``alternating'' character of $\GAO$. 

\item To provide ``local variants'' of $\GAO$ and $\GLin$,  which can often be
evaluated in small subsystems of $(\B, Q_0)$, thereby avoiding state-explosion. The local conditions imply the
corresponding global ones, \ie they are sufficient but not necessary for deadlock-freedom.

\en

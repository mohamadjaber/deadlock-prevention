Our first global condition is the most general possible: simply assert that, after execution of interaction $\act$, some
$\B_i \in \cmps{\act}$ exhibits a supercycle-violation, as given by $\viol{\B_i}{d}{t}$
(Definition~\ref{def:supercycle.violation}). 


\bd[$\GAO(\B, Q_0, \act)$] \label{def:global.ANDOR-cond} \label{defn:global.ANDOR-cond} 
Let $s \goesto[\act] t$ be a reachable transition of $(\B, Q_0)$.
Then, in $t$, the following holds. 
For every component $\B_i \in \comps{\act}$, %$\B_i$ has a level-$d$ $SC$-violation for some $d$.
the formation violation condition holds.
Formally,\\
\ind  $\fas \B_i \in \comps{\act}, \formViol{\B_i}{t}$.
%\ind  $\fas \B_i \in \comps{\act}, \exs d \ge 1: \viol{\B_i}{d}{t}$.
\ed
We now show that $\GAO$ is supercycle-freedom preserving.


\bt \label{thm:GAO.SC-free-preserving}
$\GAO$ is supercycle-freedom preserving.
\et
\prf{
We must establish:
for every reachable transition $s \la{\act} t$,
$\wfg{\B}{s}$ is supercycle-free implies that $\wfg{\B}{t}$ is
supercycle-free. Our proof is by contradiction, so we assume the existence of a reachable transition
$s \la{\act} t$ such that $\wfg{\B}{s}$ is supercycle-free and $\wfg{\B}{t}$ contains a supercycle.

By Proposition~\ref{prop:supercycle-formation}
 there exists a component $\B_i \in \cmps{\act}$ such that $\B_i$ is in $\CC$, where 
$\CC$ is a strongly connected supercycle that is a subgraph of $\wfg{\B}{t}$.

Since $\CC$ is a strongly connected supercycle, we have,
 by Definition~\ref{def:sConn.violation}, that $\neg \connViol{\B_i}{t}$ holds.

Since $\CC$ is a supercycle, we have, by Proposition~\ref{prop:scViol-iff-notInSC}, 
that $\neg (\exs d \ge 1: \scViol{\B_i}{d}{t})$ holds.

Hence, by Definition~\ref{def:formation.violation}, $\neg \formViol{\B_i}{t}$
But, by Definition~\ref{def:global.ANDOR-cond}, we have $\formViol{\B_i}{t}$.
Hence, we have the desired contradiction, and so the theorem holds.
}



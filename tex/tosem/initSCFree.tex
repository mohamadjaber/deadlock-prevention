
Our deadlock-freedom theorem require that all initial states be sueprcycle-free. 
We assume that the number of initial states is small, so that we can check each explicitly. 


\begin{figure}[ht]
\setcounter{lctr}{0}
\begin{tabbing}\label{alg:compute-scViol}
mm\= mm\= mm\= mm\= mm\=\kill
\cInitSCFree{$Q_0$}\\
\cmnt\ returns true iff all initial states are supercycle-free\\
\lio{\FORALLC{s_0 \in Q_0}}
   \lit{\mbox{compute $\wfg{\B}{s_0}$}}
   \lit{\mbox{let $U$ be the result of removing from $\wfg{\B}{s_0}$ all nodes $v$ such that $(\ex d \ge 1: \viol{v}{d}{t})$}}
   \litc{\IFC{\mbox{$U$ is nonempty}}\ \THEN\ \RETURNE{\fff}}{\cmnt\ $s_0$ not supercycle-free, so return false}
\lion{\ELSE\ \RETURNE{\ttt}}
\end{tabbing}
\label{fig:checkInitSCFree}
\caption{Procedure to check that all initial states are supercycle-free}
\end{figure}


\bp
\cInitSCFree{$Q_0$} returns true iff all initial states are supercycle-free.
\ep
\prf{
Consider the execution of \cInitSCFree{$Q_0$} for an arbitrary $s_0 \in Q_0$.

Suppose that $U$ is nonempty. 
By Proposition~\ref{prop:notInSC-implies-scViol}, $U$ is a supercycle. 
Since $U \sub \wfg{\B}{s_0}$, we conclude that $s_0$ not supercycle-free, so false is the correct
result in this case.

Now suppose that $U$ is empty. Hence every node in $\wfg{\B}{s_0}$ has a supercycle violation, and so
by Proposition~\ref{prop:scViol-implies-notInSC}, no node of $\wfg{\B}{s_0}$ can be in a 
strongly-connected supercycle. Hence  $\wfg{\B}{s_0}$ does not contain a strongly-connected supercycle.
So, by Proposition~\ref{prop:supercycle:contains-mssc}, $\wfg{\B}{s_0}$ does not contain a supercycle.
}

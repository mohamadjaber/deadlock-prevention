
%That is, a system $(B, Q_0)$ whose set of states is finite. 
%Implementations of the method for infinite-state systems are a subject for future work.

%
\checkLin{$\B, Q_0$}\ iterates over each interaction $\act$ of ($\B, Q_0$), and checks 
$(\ex \l > 0: \LLin(B, Q_0, \act, \l))$ by starting with $\l=1$ and incrementing $\l$ until
either $\LLin(B, Q_0, \act, \l)$ is found to hold, or 
$\dsk{\act}{\l}$ has become the entire system and $\LLin(B, Q_0, \act, \l)$ does not hold. In the latter case, 
$\LLin(B, Q_0, \act, \l)$ does not hold for any finite $\l$, and, in practice, 
computation would halt 
before $\dsk{\act}{\l}$ had become the entire system, due to exhaustion of resources.

\checkLinIntDist{$\B, Q_0, \act, \l$} checks $\LLin(B, Q_0, \act, \l)$ by examining every reachable transition
that executes $a$, and checking that the final state satisfies
Definition~\ref{def:ldfc-k}. 

%If $\LDFC(a, \l)$ does not hold for any finite $\l$, then the
%iteration for $a$ must be halted by some predetermined time bound, or
%interactively, or when $\dsk{\act}{\l}$ has become the entire system.
\begin{figure}[H]
\setcounter{lctr}{0}
\begin{tabbing}\label{alg:check-df}
mm\= mm\= mm\= \kill
\checkLin{$\B, Q_0$},  where $\B \df \gamma(\B_1,\dots,\B_n)$\\
\lio{\FORALLC{\mbox{interactions $\act \in \gamma$}}}
   \lit{\IFC{\checkLinInt{\B, Q_0, \act} = \fff}\ \RETURNE{\fff} \ \FI}
\lio{\ENDFOR;}
\liocn{\RETURNE{\ttt}}{\cmnt\ return $\ttt$ if check succeeds for all $\act \in \gamma$}
\end{tabbing}

\setcounter{lctr}{0}
\begin{tabbing}\label{alg:checkInt}
mm\= mm\= mm\= \kill
\checkLinInt{$\B, Q_0, \act$},  where $\B \df \gamma(\B_1,\dots,\B_n), \act \in \gamma$\\
\cmnt\ check $(\ex \l > 0: \LLin(\B, Q_0, \act, \l))$\\
\lioc{\l \gts 1;}{\cmnt\ start with $\l = 1$}
\lio{\WHILEC{\ttt}}
   \litc{\IFC{\checkLinIntDist{\act, \l} = \ttt}\ \RETURNE{\ttt}\ \FI;}{\cmnt\ success, so return true}
   \litc{\IFC{\dsk{\act}{\l} = \gamma(\B_1,\dots,\B_n)}\ \RETURNE{\fff}\ \FI;}{\cmnt\ exhausted all subsystems, return false}
   \litc{\l \gts \l + 1}{\cmnt\ increment $\l$ until success or intractable or failure}
\lio{\ENDWHILE}
\end{tabbing}

\setcounter{lctr}{0}
\begin{tabbing}
\label{alg:eval-ldfc}
mm\= mm\= mm\= \kill
\checkLinIntDist{$\B, Q_0, \act, \l$}\\
\lio{\FORALLC{\mbox{reachable transitions $s_\act \goesto[\act] t_\act$ of $\dsk{\act}{\l}$}}}
   \lit{\IFC{\neg(\fa \B_i \in \cmps{\act}:  \widepth{\dsk{\act}{\l}}{\B_i}{t_\act} < 2\l - 1 \lor \wodepth{\dsk{\act}{\l}}{\B_i}{t_\act} < 2\l -1)}}
  	\lihc{\RETURNE{\fff}}{\cmnt\ check Definition~\ref{def:ldfc-k}}
  \lit{\FI}
\lio{\ENDFOR;}
\liocn{\RETURNE{\ttt}}{\cmnt\ return $\ttt$ if check succeeds for all transitions}
\end{tabbing}

\caption{Pseudocode for the implementation of the linear condition.}
\label{fig:implementation}
\label{fig:implementation-checkDF}
\end{figure}

\paragraph{Complexity} The running  time of our implementation is
also 
$O(\SUM_{a \in \gamma}\ |\Mdsk{\act}{\l_a}| * |\dsk{\act}{\l_a}|)$,
where $\l_a$ is the smallest value of $\l$ for which $\LLin(B, Q_0, \act, \l)$ holds, and where
$|\dsk{\act}{\l_a}|$, and $|\Mdsk{\act}{\l_a}|$ are as above.

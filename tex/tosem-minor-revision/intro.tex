
% Since deciding deadlock-freedom for finite-state concurrent systems is PSPACE-complete, our
% criterion gives up completeness in return for tractability of evaluation. Our criterion can be
% evaluated by model-checking subsystems of the overall large system. The size of these subsystems
% depends only on the local topology of direct interaction between components, and \emph{not} on the
% number of components in the overall system.

Deadlock freedom is a crucial property of concurrent and distributed systems. With increasing system
complexity, the challenge of assuring deadlock freedom and other correctness properties becomes even
greater.  In contrast to the alternatives of (1) deadlock detection and recovery, and (2) deadlock
avoidance, we advocate deadlock prevention: design the system so that deadlocks do not occur.
%during the normal functioning of the system.

Deciding deadlock freedom of finite-state concurrent programs is PSPACE-complete, in general
\cite[chapter 19]{papadimitriou1994computational}. To achieve tractability, we present a criterion
for deadlock-freedom that is evaluated by model-checking a set of subsystems of the overall
system. If the subsystems are small, the criterion can be checked
quickly. The criterion is sound (if
true, it implies deadlock-freedom) but not complete (if false, then it yields no information about
deadlock). If the subsystems are larger, then our criterion becomes more ``accurate'': roughly
speaking, there is less possibility for the criterion to evaluate to false when the system is
actually deadlock-free. In the limit, when the set of subsystems includes the entire system itself,
our criterion is complete, so that evaluation to false implies that the system is actually
deadlock-prone. Hence, our criterion only fails to resolve the question of deadlock-freedom 
when it's evaluation exhausts available computational resources, because
the subsystems being checked have become too large, and state-explosion has set in.

Our method thus combines the possibility of fast response together with theoretical completeness.
All deadlock-freedom checks given in the literature to date are, to our knowledge, incomplete in
principle, and so remain incomplete even if unlimited computational resources are available.
Hence these criteria could fail to resolve deadlock freedom for theoretical reasons, as well as for 
lack of computational resources.
%
The reason for this incompleteness is that existing criteria all characterize deadlock by the
occurrence of a wait-for cycle, \eg as stated by \citeN{AGR16}, discussion of related
work:
\begin{quote}
All these methods were designed, to some extent, around the principle that under reasonable
assumptions about the system, any deadlock state would contain a proper cycle of ungranted requests.
\end{quote}
In a model of concurrency which includes choice of actions
(\eg BIP, CSP, I/O automata, CCS), a wait-for cycle is an \emph{incomplete} characterization of
deadlock, since a process can be in a wait-for cycle, but not deadlocked, due to having a choice of
interaction with another process not in the wait-for cycle (see Figure \ref{fig:cycleOK}).

Our method, in contrast, characterizes deadlock by the occurrence of a \emph{supercycle} \cite{AE98,AC05}, which, very roughly, is the AND-OR analogue
of a wait-for cycle: a subset of processes constitutes a supercycle $\SC$ iff every possible action of every process in $\SC$ is blocked by another
process in $\SC$.
%
We show that supercycles are a sound and complete characterization of deadlock: a system is deadlock-prone iff a supercycle can arise in some
reachable state.
%
We then present our criterion, which prevents the occurrence of supercycles in reachable states of
the system. We first present a ``global'' version of our criterion, which is both sound and complete
\wrt absence of supercycles, and then a ``local'' version, which is sound \wrt absence of
supercycles, and can be evaluated over small subsystems.

In addition our criterion guarantees freedom from local (and therefore global) deadlock. A local deadlock occurs when 
a subsystem is deadlocked while the rest of the system can execute. Other criteria in the literature \cite{AGR16,Ma96,RD87,DFinder2,BR91,MM12,GS03,AB03} guarantee only global deadlock freedom.

This paper significantly extends a preliminary conference version \cite{FORTE13} as
follows: 
(1) we present an ``AND-OR'' criterion for deadlock-freedom, which exploits the AND-OR structure of supercycles, and is therefore complete for
deadlock-freedom in the limit, while our preliminary work \cite{FORTE13} gives a ``linear'' criterion, which is a special case in which the AND-OR structure is ignored, and
(2) experimental results show that the new criterion is more efficient
in practice, and also succeeds in cases where the linear criterion
fails.
We therefore have the best of both worlds: early stopping, and
therefore efficient verification of deadlock-freedom, in many cases,
together with theoretical completeness.
%
Our criterion is, to the best of our knowledge, the first criterion that is sound \emph{and complete} 
for global and \emph{local} deadlock-freedom in
concurrent programs with nondeterministic local choice, \ie a process can nondeterministically choose among enabled actions.

We present experimental results for dining philosophers and for a multi token-based resource allocation system, which generalizes Milner's token based
scheduler~\cite{milner}.  These show that our method compares favorably with existing approaches.



Section~\ref{s:bip} presents BIP \cite{BliudzeS08}.  Section~\ref{s:characterize} characterizes local and global deadlocks as the occurrence of a pattern of wait-for
edges called a supercycle (see discussion above).   Section~\ref{secn:globalSupercycles} considers ``global''
supercycles, \ie those that occur in the overall system being considered, characterizes these as the greatest fixpoint of a ``blocking'' operator, and
presents some structural properties of supercycles. Section~\ref{secn:globalDeadlockFreedom} presents global conditions for the prevention of the
formation of supercycles. Global means that these conditions are evaluated in the entire system.  Section~\ref{secn:localSupercycles} considers
``local'' supercycles, \ie those that occur in a given subsystem of the overall system being considered.  These are also characterized as the greatest
fixpoint of a ``local blocking'' operator, which pessimistically assumes that nodes on the boundary of the subsystem are blocked. This pessimistic
assumption ensures soundness, at the expense of completeness.  Section~\ref{secn:localDeadlockFreedom} presents local conditions for the prevention of
the formation of supercycles. These can be evaluated in (small) subsystems of the overall system, and are obtained by ``projecting'' the global
conditions onto a subsystem.  Section~\ref{s:results} presents the main soundness and completeness results of the paper, and gives the implication
relation among our various conditions for deadlock-freedom.  Section~\ref{s:impl} gives algorithms to evaluate the local conditions, and presents
experimental evaluation.  Section~\ref{s:discussion} discusses related work, further work, and concludes.





\begin{figure}[ht]
\setcounter{lctr}{0}
\begin{tabbing}\label{alg:compute-scViol}
aaa\= aa\= aa\= aa\= aa\=\kill
\cInitSCFree{$Q_0$}\\
\cmnt\ returns true iff all initial states are supercycle-free\\
\lio{\FORALLC{s_0 \in Q_0}}
   \lit{\mbox{compute $\wfg{\B}{s_0}$}}
   \lit{\mbox{let $U$ be the result of removing from $\wfg{\B}{s_0}$ all nodes $v$ such that
 $\scV{v}{s_0}$}}
   \litc{\IFC{\mbox{$U$ is nonempty}}\ \THEN\ \RETURNE{\fff}}{\cmnt\ $s_0$ not supercycle-free, so return false}
\lio{\ENDFOR;}
\liocn{\RETURNE{\ttt}}{\cmnt all initial states are supercycle-free}
\end{tabbing}
\caption{Procedure to check that all initial states are supercycle-free}
\label{fig:checkInitSCFree}
\end{figure}


\begin{proposition}
\cInitSCFree{$Q_0$} returns true iff all initial states are supercycle-free.
\end{proposition}
%
\begin{proof}
Consider the execution of \cInitSCFree{$Q_0$} for an arbitrary $s_0 \in Q_0$.

Suppose that $U$ is nonempty. 
%OLD By \prop{scViol-iff-notInSC} Proposition~\ref{prop:notInSC-implies-scViol}, $U$ is a supercycle. 
By Propositions~\ref{prop:scViol-iff-notInSC} and \ref{prop:supercycle:union}, $U$ is a supercycle. 
Since $U \sub \wfg{\B}{s_0}$, we conclude that $s_0$ not supercycle-free, so false is the correct
result in this case.

Now suppose that $U$ is empty. Hence every node in $\wfg{\B}{s_0}$ has a supercycle violation, and so
%OLD by Proposition~\ref{prop:scViol-implies-notInSC}, no node of $\wfg{\B}{s_0}$ can be in a 
by \prop{scViol-iff-notInSC}, no node of $\wfg{\B}{s_0}$ is in a 
supercycle. Hence  $\wfg{\B}{s_0}$ does not contain a supercycle, and so
$s_0$ is supercycle-free.
Hence the for loop continues on to the next initial state.
If all initial states are supercycle-free, the for loop terminates, and
\cInitSCFree{$Q_0$} returns true, as required.
\end{proof}

To implement our deadlock freedom criteria, we must:
\bn
\item \label{task:check-init} Check that all initial states are supercycle-free
\item \label{task:evaluate-lao} Evaluate $\LAO$
\item \label{task:evaluate-llin} Evaluate $\LLin$
\en
%
Tasks \ref{task:check-init} and \ref{task:evaluate-lao} require the computation of $\lfp{\lVFs}$.
\fig{algcomputeLFP} presents an algorithm that computes
$\lfp{\lVFs}$. Its correctness follows immediately from \defn{blocksLoc} and \defn{violFixLoc}.


%\subsection{Computing the least fixpoint of $\lVFs$}
   \label{secn:computeLFP}
%   


\newcommand{\fpt}{\mathit{fixpoint}}
\newcommand{\VLA}[5]{\ensuremath{\MATHIDN{V_{\dsk{#1}{#2},#3} [#4 #5]}}}    %sc violations array, #1 is a, #2 is \l, #3 is t_a, #4 is v, #5 is d
                                                                        % i.e., \VV{a}{\l}{t_a}{v}{d}

\begin{figure}[H]
\setcounter{lctr}{0}
\begin{tabbing}\label{alg:compute-scViol}
aa\= aa\= aa\= aa\= aa\=\kill
\cLFP{$\dsk{\act}{\l},t_\act$}\\
\cmnt\ compute least fixpoint of $\lVF$ in state $t_\act$ of $\dsk{\act}{\l}$\\
\cmnt\ Precondition: $\fas v \in \dsk{\act}{\l}: \VLA{\act}{\l}{t_\act}{v}{0} = \fff$ \\
\cmnt\ Postcondition: $\fas v \in \dsk{\act}{\l}: \VLA{\act}{\l}{t_\act}{v}{d}$ 
$ =
  \begin{cases}
    \ttt     & \quad \text{if }  v \in \lVFs^{d} (\ewfg)\\
    \fff     & \quad \text{otherwise}\\
  \end{cases}
$
\\
\lio{d \gts 1} 
\lio{\fpt \gts \ttt} 
\lio{\WHILE \neg \fpt}
   \lit{\mbox{invariant: $V$ computed correctly up to $d-1$}} 
   \lit{\fpt \gts \fff} 
   \lit{\FORALLC{v \in \dsk{\act}{\l}}}
      \lih{\IFC{\mbox{$v$ is an interior interaction $\actp$ and } 
                          \neg (\ex \B_i : \actp \ar \B_i \in \wfg{\dsk{\act}{\l}}{t_\act})}}
          \lifc{\VLA{\act}{\l}{t_\act}{v}{d} \gts \ttt}{\cmnt Base case: interaction with no outgoing wait-for-edges}  

     \lih{\ELSFC{\mbox{$v$ is an interior interaction $\actp$ and } 
                  (\fa \B_i : \actp \ar \B_i \in \wfg{\dsk{\act}{\l}}{t_\act} :  \VV{\act}{\l}{t_\act}{B_i}{})}}
          \lif{\VLA{\act}{\l}{t_\act}{v}{d} \gts \ttt}
     \lih{\ELSFC{\mbox{$v$ is a component $\B_i$ and }
                  (\ex \actp : \B_i \ar \actp \in \wfg{\dsk{\act}{\l}}{t_\act} : \VV{\act}{\l}{t_\act}{\actp}{})}}
          \lif{\VLA{\act}{\l}{t_\act}{v}{d} \gts \ttt}
    \lih{\FI}
    \lit{\IFC{\neg \VLA{\act}{\l}{t_\act}{v}{d-1} \land \VLA{\act}{\l}{t_\act}{v}{d}} \fpt \gts \fff} 
  \lit{\ENDFOR}
\lion{\ENDWHILE} 
\end{tabbing}
\vspace{-3ex}
\caption{Procedure to compute all supercycle-violations in state $t_\act$ of $\dsk{\act}{\l}$}
\label{fig:computeSCViolateLocDSK}
\label{fig:computeSCViolateLoc}
\end{figure}



\newcommand{\fpt}{\mathit{fixpoint}}
\newcommand{\VLA}[5]{\ensuremath{\MATHIDN{V_{\dsk{#1}{#2},#3} [#4 #5]}}}    %sc violations array, #1 is a, #2 is \l, #3 is t_a, #4 is v, #5 is d
                                                                        % i.e., \VV{a}{\l}{t_a}{v}{d}

\begin{figure}[H]
\setcounter{lctr}{0}
\begin{tabbing}\label{alg:compute-lfp}
aaa\= aa\= aa\= aa\= aa\=\kill
\cLFP{$\dsk{\act}{\l},t_\DS$}\\
\cmnt\ Precondition: $t_\DS$ is a local state of $\dsk{\act}{\l}$\\
\cmnt\ Postcondition: returns least fixpoint of $\lVFs$ in state $t_\DS$ of $\dsk{\act}{\l}$, \ie $\lfp{\lVFs} = \UN_{d \ge 0} \lVFs^{d} (\ewfg)$.\\

\lio{\XS \gts \ewfg;}

\lio{\REPEAT}
   \lit{\mbox{\cmnt loop invariant: on the $i$'th iteration, $\XS \equiv \lVFi{\ewfg}{i} \equiv  v \in \UN_{0 \le d \le i} \lVFi{\ewfg}{d}$}}
   \lit{\YS \gts \cVL{\dsk{\act}{\l}, t_\DS, \XS}}
   \lit{\XS \gts \YS}
\lio{\UNTIL\ \XS = \YS;}

\lion{\RETURNE{\XS}}
\end{tabbing}
\caption{Procedure to compute all supercycle-violations in state $t_\DS$ of $\dsk{\act}{\l}$}
\label{fig:algcomputeLFP}
\end{figure}




\begin{figure}[H]
\setcounter{lctr}{0}
\begin{tabbing}\label{alg:compute-lfp}
aaa\= aa\= aa\= aa\= aa\=\kill
\cVL{$\dsk{\act}{\l}, t_\DS, \XS$}\\
\cmnt\ precondition: $t_\DS$ is a state of $\dsk{\act}{\l}$ and $X \sub \dsk{\act}{\l}$\\
\cmnt\ postcondition: returns $\lVF{X}$ in state $t_\DS$ of $\dsk{\act}{\l}$,\\
\\
\lio{\mbox{compute $\wfg{\dsk{\act}{\l}}{t_\DS}$}}
\lio{\YS \gts \ewfg}

\lio{\FORALLC{\ndv \in \dsk{\act}{\l}}}

   \lit{\IFC{\mbox{$v$ is an interior interaction $\actp$ and } 
                          \neg (\ex \B_i : \actp \ar \B_i \in \wfg{\dsk{\act}{\l}}{t_\DS})}}
       \lihc{\YS \gts \YS \join \set{\ndv}}{\cmnt interaction with no outgoing wait-for-edges}  

    \lit{\ELSFC{\mbox{$v$ is an interior interaction $\actp$ and } 
     (\fa \B_i : \actp \ar \B_i \in \wfg{\dsk{\act}{\l}}{t_\DS} :  B_i \in \XS )}}
        \lihc{\YS \gts \YS \join \set{\ndv}}{\cmnt all outgoing wait-for-edges go to nodes in $\XS$}

    \lit{\ELSFC{\mbox{$v$ is a component $\B_i$ and }
     (\ex \actp : \B_i \ar \actp \in \wfg{\dsk{\act}{\l}}{t_\DS} : \actp \in \XS )}}
         \lihc{\YS \gts \YS \join \set{\ndv}}{\cmnt some outgoing wait-for-edge goes to a node in $\XS$}

    \lit{\FI}
\lio{\ENDFOR;}
\lion{\RETURNE{\YS}}
\end{tabbing}
\caption{Procedure to compute $\lVF{X}$}
\label{fig:algcomputeVL}
\end{figure}

Note that $\YS \join \set{\ndv}$ is the join of the wait-for-subgraph $\YS$ with the 
wait-for-subgraph $\ndv$ consisting of the single node $\ndv$. Recall
that the edges of $\YS \join \set{\ndv}$ are induced from $\lwfg{B}{t_\act}{\DS}$.

The time complexity of \cVL{$\dsk{\act}{\l}, t_\act, \XS$} is ......

The time complexity of \cLFP{$\dsk{\act}{\l},t_\act$} is ......






\thm{local.deadlock-free} requires that all initial states be supercycle-free. 
We assume that the number of initial states is small, so that we can check each explicitly. 
\fig{checkInitSCFree}  presents an algorithm that checks if a
given state is supercycle-free.

% \subsection{Checking that initial states are supercycle-free}
   \label{s:initSCFree}
%   
Our deadlock-freedom theorem require that all initial states be sueprcycle-free. 
We assume that the number of initial states is small, so that we can check each explicitly. 


\begin{figure}[ht]
\setcounter{lctr}{0}
\begin{tabbing}\label{alg:compute-scViol}
mm\= mm\= mm\= mm\= mm\=\kill
\cInitSCFree{$Q_0$}\\
\cmnt\ returns true iff all initial states are supercycle-free\\
\lio{\FORALLC{s_0 \in Q_0}}
   \lit{\mbox{compute $\wfg{\B}{s_0}$}}
   \lit{\mbox{let $U$ be the result of removing from $\wfg{\B}{s_0}$ all nodes $v$ such that $(\ex d \ge 1: \viol{v}{d}{t})$}}
   \litc{\IFC{\mbox{$U$ is nonempty}}\ \THEN\ \RETURNE{\fff}}{\cmnt\ $s_0$ not supercycle-free, so return false}
\lion{\ELSE\ \RETURNE{\ttt}}
\end{tabbing}
\label{fig:checkInitSCFree}
\caption{Procedure to check that all initial states are supercycle-free}
\end{figure}


\bp
\cInitSCFree{$Q_0$} returns true iff all initial states are supercycle-free.
\ep
\prf{
Consider the execution of \cInitSCFree{$Q_0$} for an arbitrary $s_0 \in Q_0$.

Suppose that $U$ is nonempty. 
%OLD By \prop{scViol-iff-notInSC} Proposition~\ref{prop:notInSC-implies-scViol}, $U$ is a supercycle. 
By \prop{scViol-iff-notInSC}, $U$ is a supercycle. 
Since $U \sub \wfg{\B}{s_0}$, we conclude that $s_0$ not supercycle-free, so false is the correct
result in this case.

Now suppose that $U$ is empty. Hence every node in $\wfg{\B}{s_0}$ has a supercycle violation, and so
%OLD by Proposition~\ref{prop:scViol-implies-notInSC}, no node of $\wfg{\B}{s_0}$ can be in a 
by \prop{scViol-iff-notInSC}, no node of $\wfg{\B}{s_0}$ can be in a 
strongly-connected supercycle. Hence  $\wfg{\B}{s_0}$ does not contain a strongly-connected supercycle.
So, by Proposition~\ref{prop:supercycle:contains-mssc}, $\wfg{\B}{s_0}$ does not contain a supercycle.
}



\begin{figure}[ht]
\setcounter{lctr}{0}
\begin{tabbing}\label{alg:compute-scViol}
aaa\= aa\= aa\= aa\= aa\=\kill
\cInitSCFree{$Q_0$}\\
\cmnt\ returns true iff all initial states are supercycle-free\\
\lio{\FORALLC{s_0 \in Q_0}}
   \lit{\mbox{compute $\wfg{\B}{s_0}$}}
   \lit{\US \gets \cLFP{\dsk{\act}{\l},t_\DS}}
%\mbox{let $U$ be the result of removing from $\wfg{\B}{s_0}$ all nodes $v$ such that $\scV{v}{s_0}$}}
   \litc{\IFC{U \ne \wfg{\B}{s_0}}\ \THEN\ \RETURNE{\fff}}{\cmnt\ $s_0$ not supercycle-free, so return false}
\lio{\ENDFOR;}
\liocn{\RETURNE{\ttt}}{\cmnt all initial states are supercycle-free}
\end{tabbing}
\caption{Procedure to check that all initial states are supercycle-free}
\label{fig:checkInitSCFree}
\end{figure}


\begin{proposition}
\cInitSCFree{$Q_0$} returns true iff all initial states are supercycle-free.
\end{proposition}
%
\begin{proof}
Consider the execution of \cInitSCFree{$Q_0$} for an arbitrary $s_0 \in Q_0$.

By its construction, $\US \subg \wfg{\B}{s_0}$.
Suppose that $\US \ne  \wfg{\B}{s_0}$.
%OLD By \prop{scViol-iff-notInSC} Proposition~\ref{prop:notInSC-implies-scViol}, $U$ is a supercycle. 
By Propositions~\ref{prop:scViol-iff-notInSC} and
\ref{prop:supercycle:union}, the nodes of $\wfg{\B}{s_0}$ that are not in $\US$ constitute a supercycle.
Hence $s_0$ not supercycle-free, and so false is the correct result in this case.

Now suppose that $\US = \wfg{\B}{s_0}$. Hence every node in $\wfg{\B}{s_0}$ has a supercycle violation, and so
%OLD by Proposition~\ref{prop:scViol-implies-notInSC}, no node of $\wfg{\B}{s_0}$ can be in a 
by \prop{scViol-iff-notInSC}, no node of $\wfg{\B}{s_0}$ is in a 
supercycle. Hence  $\wfg{\B}{s_0}$ does not contain a supercycle, and so
$s_0$ is supercycle-free.
Hence the for loop continues on to the next initial state.
If all initial states are supercycle-free, the for loop terminates, and
\cInitSCFree{$Q_0$} returns true, as required.
\end{proof}





   \subsection{Implementation of the Linear Condition $\LLin$}
   \label{s:implLin}
   
%That is, a system $(B, Q_0)$ whose set of states is finite. 
%Implementations of the method for infinite-state systems are a subject for future work.

%
\checkLin{$\B, Q_0$}\ iterates over each interaction $\act$ of ($\B, Q_0$), and checks 
$(\ex \l > 0: \LLin(B, Q_0, \act, \l))$ by starting with $\l=1$ and incrementing $\l$ until
either $\LLin(B, Q_0, \act, \l)$ is found to hold, or 
$\dsk{\act}{\l}$ has become the entire system and $\LLin(B, Q_0, \act, \l)$ does not hold. In the latter case, 
$\LLin(B, Q_0, \act, \l)$ does not hold for any finite $\l$, and, in practice, 
computation would halt 
before $\dsk{\act}{\l}$ had become the entire system, due to exhaustion of resources.

\checkLinIntDist{$\B, Q_0, \act, \l$} checks $\LLin(B, Q_0, \act, \l)$ by examining every reachable transition
that executes $a$, and checking that the final state satisfies
Definition~\ref{def:ldfc-k}. 

%If $\LDFC(a, \l)$ does not hold for any finite $\l$, then the
%iteration for $a$ must be halted by some predetermined time bound, or
%interactively, or when $\dsk{\act}{\l}$ has become the entire system.
\begin{figure}[H]
\setcounter{lctr}{0}
\begin{tabbing}\label{alg:check-df}
mm\= mm\= mm\= \kill
\checkLin{$\B, Q_0$},  where $\B \df \gamma(\B_1,\dots,\B_n)$\\
\lio{\FORALLC{\mbox{interactions $\act \in \gamma$}}}
   \lit{\IFC{\checkLinInt{\B, Q_0, \act} = \fff}\ \RETURNE{\fff} \ \FI}
\lio{\ENDFOR;}
\liocn{\RETURNE{\ttt}}{\cmnt\ return $\ttt$ if check succeeds for all $\act \in \gamma$}
\end{tabbing}

\setcounter{lctr}{0}
\begin{tabbing}\label{alg:checkInt}
mm\= mm\= mm\= \kill
\checkLinInt{$\B, Q_0, \act$},  where $\B \df \gamma(\B_1,\dots,\B_n), \act \in \gamma$\\
\cmnt\ check $(\ex \l > 0: \LLin(\B, Q_0, \act, \l))$\\
\lioc{\l \gts 1;}{\cmnt\ start with $\l = 1$}
\lio{\WHILEC{\ttt}}
   \litc{\IFC{\checkLinIntDist{\act, \l} = \ttt}\ \RETURNE{\ttt}\ \FI;}{\cmnt\ success, so return true}
   \litc{\IFC{\dsk{\act}{\l} = \gamma(\B_1,\dots,\B_n)}\ \RETURNE{\fff}\ \FI;}{\cmnt\ exhausted all subsystems, return false}
   \litc{\l \gts \l + 1}{\cmnt\ increment $\l$ until success or intractable or failure}
\lio{\ENDWHILE}
\end{tabbing}

\setcounter{lctr}{0}
\begin{tabbing}
\label{alg:eval-ldfc}
mm\= mm\= mm\= \kill
\checkLinIntDist{$\B, Q_0, \act, \l$}\\
\lio{\FORALLC{\mbox{reachable transitions $s_\act \goesto[\act] t_\act$ of $\dsk{\act}{\l}$}}}
   \lit{\IFC{\neg(\fa \B_i \in \cmps{\act}:  \widepth{\dsk{\act}{\l}}{\B_i}{t_\act} < 2\l - 1 \lor \wodepth{\dsk{\act}{\l}}{\B_i}{t_\act} < 2\l -1)}}
  	\lihc{\RETURNE{\fff}}{\cmnt\ check Definition~\ref{def:ldfc-k}}
  \lit{\FI}
\lio{\ENDFOR;}
\liocn{\RETURNE{\ttt}}{\cmnt\ return $\ttt$ if check succeeds for all transitions}
\end{tabbing}

\caption{Pseudocode for the implementation of the linear condition.}
\label{fig:implementation}
\label{fig:implementation-checkDF}
\end{figure}

\paragraph{Complexity.} The running  time of our implementation is
also 
$O(\SUM_{a \in \gamma}\ |\Mdsk{\act}{\l_a}| * |\dsk{\act}{\l_a}|)$,
where $\l_a$ is the smallest value of $\l$ for which $\LLin(B, Q_0, \act, \l)$ holds, and where
$|\dsk{\act}{\l_a}|$, and $|\Mdsk{\act}{\l_a}|$ are as above.
   
   \subsection{Implementation of the AND-OR Condition $\LAO$}
   \label{s:implANDOR}
   
Figure~\ref{fig:impl.locANDOR} presents the pseudocode for our algorithm to 
evaluate $\LAO$.
This uses the algorithm for computing supercycle violations based on $\dsk{\act}{\l}$,
given in Figure~\ref{fig:algcomputeLFP}.

\checkLAO{$\B, Q_0$} iterates over each interaction $\act$ of ($\B, Q_0$), and 
invokes \checkLAOInt{$\B, Q_0, \act$} to evaluate $(\ex \l > 0: \LAO(B, Q_0, \act, \l))$.
\checkLAOInt{$\B, Q_0, \act$}
starts with $\l=1$ and increments $\l$ until either $\LAO(B, Q_0, \act, \l)$ is found to hold, or
$\dsk{\act}{\l}$ has become the entire system and $\LAO(B, Q_0, \act, \l)$ does not hold. In the
latter case, $\LAO(B, Q_0, \act, \l)$ does not hold for any finite $\l$, and, in practice,
computation would halt before $\dsk{\act}{\l}$ had become the entire system, due to exhaustion of
resources.
Note that $\dsk{\act}{1}$ is the smallest system in which a
supercycle-violation can be confirmed. 

Evaluation of $\LAO(B, Q_0, \act, \l)$ is done by 
\checkLAOIntDist{$\B, Q_0, \act, \l$}, which invokes
$\cLFP{\dsk{\act}{\l},t_\act}$ to compute the supercycle violations.
The pseudocode is a straightforward translation of
Definitions~\ref{def:sConn.violation.loc} and \ref{def:locFormation.violation}.
%
Figure~\ref{fig:summaryProcedures} shows a summary of the procedures.

%%%%%%%%%%%%%%%%%%%%%%%%%%%%%%%%%%%%%%%%%%%%%%%%%%%%%%%%%%%%%%%%%%%%
\begin{figure}%[H]
{\normalsize
\begin{tabular}{|l|l|}
\hline
\checkLAO{$\B, Q_0$} & true iff $(\fa \act \in \gamma, \ex \l > 0: \LAO(B, Q_0, \act, \l))$\\ \hline
\checkLAOInt{$\B, Q_0, \act$} & true iff $(\ex \l > 0: \LAO(B, Q_0, \act, \l))$\\ \hline
\checkLAOIntDist{$\B, Q_0, \act, \l$} &  true iff $\LAO(\B, Q_0, \act, \l)$\\ \hline

\cLFV{$B_i, V, \dsk{\act}{\l},t_\act$} & true iff $B_i$ has local sc-formation violation \\
& in state $t_\act$ of $\dsk{\act}{\l}$, \ie $\locFormViol{B_i}{t_\act}{\act}{\l}$ holds\\ \hline

\cLconnScV{$B_i, \dsk{\act}{\l},t_\act$} & true iff $B_i$ has local strong connectedness
                                         violation \\ & in $t_{\act}$,  \ie $\locConnViol{B_i}{t_\act}{\act}{\l}$ holds\\ \hline

\cLFP{$\dsk{\act}{\l},t_\act$} & compute local supercycle violations \\ & in state $t_\act$ of $\dsk{\act}{\l}$, \ie $\lviol{v}{d}{t_\act}{\act}{\l}$ for all $v,d$\\
\hline
\end{tabular}
}
\caption{Summary of procedures}
\label{fig:summaryProcedures}
\end{figure}


%%%%%%%%%%%%%%%%%%%%%%%%%%%%%%%%%%%%%%%%%%%%%%%%%%%%%%%%%%%%%%%%%%%%%%%%%%%%%%%
 
\paragraph{Complexity} The running  time of our implementation is
$O(\SUM_{a \in \gamma}\  |\Mdsk{\act}{\l_a}| * |\dsk{\act}{\l_a}|)$, 
where 
$\Mdsk{\act}{\l_a}$ is the transition system of
$\dsk{\act}{\l_a}$, and $|\Mdsk{\act}{\l_a}|$ is the size (number of nodes plus number of edges) of 
$\Mdsk{\act}{\l_a}$, 
$|\dsk{\act}{\l_a}|$ is the size of the syntactic description of $\dsk{\act}{\l_a}$, and 
$\l_a$ is the smallest value of $\l$ for which $\LAO(B, Q_0, \act, \l)$ holds.


%%%%%%%%%%%%%%%%%%%%%%%%%%%%%%%%%%%%%%%%%%%%%%%%%%%%%%%%%%%%%%%%%%%%
\begin{figure}%[H]
\setcounter{lctr}{0}
\begin{tabbing}\label{alg:check.LAO}
aa\= aa\= aa\= \kill
\checkLAO{$\B, Q_0$},  where $\B \df \gamma(\B_1,\dots,\B_n)$\\
\cmnt\ returns $\ttt$ iff $(\fa \act \in \gamma, \ex \l > 0: \LAO(\act, \l))$\\
\lio{\FORALLC{\mbox{interactions $\act \in \gamma$}}}
   \lit{\IFC{\checkLAOInt{\B, Q_0, \act} = \fff}\ \RETURNE{\fff}\ \FI}
\lio{\ENDFOR;}
\liocn{\RETURNE{\ttt}}{\cmnt\ return $\ttt$ if check succeeds for all $a \in \gamma$}
\end{tabbing}

\setcounter{lctr}{0}
\begin{tabbing}\label{alg:check.LAO.Int}
aaa\= aa\= aa\= \kill
\checkLAOInt{$\B, Q_0, \act$},  where $\B \df \gamma(\B_1,\dots,\B_n), \act \in \gamma$\\
\cmnt\ returns $\ttt$ iff $(\ex \l > 0: \LAO(B, Q_0, \act, \l))$\\
\lioc{\l \gts 1;}{\cmnt\ start with $\l = 1$}
\lio{\WHILEC{\ttt}}
   \litc{\IFC{\checkLAOIntDist{\act, \l} = \ttt}\ \RETURNE{\ttt}\ \FI;}{\cmnt\ success, so return true}
   \litc{\IFC{\dsk{\act}{\l} = \gamma(\B_1,\dots,\B_n)}\ \RETURNE{\fff}\ \FI;}{\cmnt\ exhausted all subsystems, return false}
   \litc{\l \gts \l + 1}{\cmnt\ increment $\l$ until success or intractable or failure}
\lion{\ENDWHILE}
\end{tabbing}

\setcounter{lctr}{0}
\begin{tabbing}
\label{alg:eval-ldfc}
aaa\= aa\= aa\= aa\= aa\=\kill
\checkLAOIntDist{$\B, Q_0, \act, \l$}\\
\cmnt\ returns $\ttt$ iff $\LAO(\B, Q_0, \act, \l)$\\
\lio{\FORALLC{\mbox{reachable transitions $s_\act \goesto[\act] t_\act$ of $\dsk{\act}{\l}$}}}
   \litc{V \gets \cLFP{\dsk{\act}{\l},t_\act}}{\cmnt see \fig{algcomputeLFP}}
   \lit{\FORALLC{\B_i \in \cmps{\act}}}
      \lihc{\IF\ \neg \cLFV{B_i, V, \dsk{\act}{\l},t_\act}\ \THEN\ \RETURNE{\fff}\ \FI}{\cmnt return $\fff$ if no violation for $\B_i$}
%         \lifc{\IFC{\neg \LF{\act}{\l}{t_\act}{v}{\l}}\ 
   \lit{\ENDFOR}
\lio{\ENDFOR}
\liocn{\RETURNE{\ttt}}{\cmnt return $\ttt$ if all $\B_i \in \cmps{\act}$ violate local supercycle formation}
\end{tabbing}



\setcounter{lctr}{0}
\begin{tabbing}
\label{alg:computeLocForm}
aaa\= aa\= aa\= aa\= aa\=\kill
\cLFV{$B_i, V, \dsk{\act}{\l},t_\act$}\\
\cmnt\ returns true iff $\locFormViol{B_i}{t_\act}{\act}{\l}$ holds (Definition~\ref{def:locFormation.violation})\\
\cmnt\ \ie $B_i$ has a local supercycle formation violation in state $t_\act$ of subsystem $\dsk{\act}{\l}$\\
\lion{\RETURNE{V[B_i] \, \lor \, \mbox{\cLconnScV{$B_i, V, \dsk{\act}{\l}, t_\act$}}}}
\end{tabbing}


\setcounter{lctr}{0}
\begin{tabbing}\label{alg:compute-scViol}
aaa\= aa\= aa\= aa\= aa\=\kill
\cLconnScV{$B_i, V, \dsk{\act}{\l}, t_\act$}\\
\cmnt\ returns true iff $\locConnViol{B_i}{t_\act}{\act}{\l}$ holds (Definition~\ref{def:sConn.violation.loc})\\
\cmnt\ \ie $B_i$ has a local strong connectedness supercycle formation violation in state $t_\act$ of subsystem $\dsk{\act}{\l}$\\
%\cmnt\ compute local supercycle violations in state $t_\act$ of $\dsk{\act}{\l}$\\
%\cmnt\ Postcondition: $\fas v \in \set{\B_1,\ldots,\B_n} \un \gamma, d' = 1,\ldots,dd: \VV{\act}{\l}{t_\act}{v}{d'} = \lviol{v}{d'}{t_\act}{\act}{\l}$\\
\lio{\mbox{remove all nodes $v$ such that $V[v]=\true$, \ie with a local supercycle violation}}
\lio{\mbox{compute maximal strongly connected components of remaining wait-for graph}}
\lio{\FORALLC{\mbox{maximal strongly connected components $C$}}}
   \lit{\mbox{\IF\ $C$ contains a non-trivial strongly connected supercycle which contains $B_i$ as a node \THEN}}
      \lihc{\RETURNE{\fff} \, \FI}{\cmnt{Definition~\ref{def:sConn.violation.loc}, Clause~\ref{def:sConn.violation.loc:scc} holds here}} 

\lio{\FORALLC{\mbox{wait-for paths $\pi$ from $B_i$ to the border of  $\dsk{\act}{\l}$}}}
   \lit{\IF\ \mbox{some node $v$ of $\pi$ has $V[v]=\true$, \ie local supercycle violation}\ \THEN}
      \lihc{\RETURNE{\ttt}\ \FI}{\cmnt Clause~\ref{def:sConn.violation.loc:wait-for-out} holds}

\lio{\FORALLC{\mbox{wait-for paths $\pi'$ from the border of $\dsk{\act}{\l}$ to $B_i$}}}
   \lit{\IF\ \mbox{some node $v$ of $\pi'$ has $V[v]=\true$, \ie a local supercycle violation}\ \THEN}
       \lihc{\RETURNE{\ttt}\ \FI}{\cmnt Clause~\ref{def:sConn.violation.loc:wait-for-in} holds}

\liocn{\RETURNE{\fff}}{\cmnt  Definition~\ref{def:sConn.violation.loc}, Clause~\ref{def:sConn.violation.border} does not hold}
\end{tabbing}

\caption{Pseudocode for the implementation of the local AND-OR condition.}
\label{fig:impl.locANDOR}
\end{figure}

% eliminate ugly space after f
%%%%%%%%%%%%%%%%%%%%%%%%%%%%%%%%%%%%%%%%%%%%%%%%%%%%%%%%%%%%%%%%%%%%%%%%



   \subsection{Tool-set}
   \label{s:tools}
   %
We provide \deadlocktool{}, a suite of supporting tools that implement our method. \deadlocktool{} is around $\sim 2500$ Java LOC.
% and is available at  \href{http://todo}{http://todo}. 
%
\deadlocktool{} is equipped with a command line interface (see Figure~\ref{code:cmd-ldfc}) that accepts a set 
of configuration options. 
It takes the name of the input BIP file and other optional flags. 


\begin{lstlisting}[language=Bash, caption = {\deadlocktool{} Command Line Interface},label={code:cmd-ldfc}]
> java -jar ldfc.jar [options] input.bip 
and options are:
-condition <s> LLIN (local linear check) or LALT (local and/or check - default)
               (optional)
-debug         Prints useful information at each iteration of checking. 
               Example: selected interaction, depth length, etc.
               This information could be useful in case when the condition fails.

Examples:
  java -jar ldfc.jar -debug input.bip # deadlock freedom using default LALT
  java -jar ldfc.jar -condition=LLIN -debug input.bip # deadlock freedom using LLIN
\end{lstlisting}



   \subsection{Experimentation}
   \label{s:experiments}
   \subsubsection{Experiment: Dining Philosophers} 
We consider $n$ philosophers in a cycle, based on the components of Figure~\ref{fig:diningSpectrum}.

\begin{table}
\centering
\begin{tabular}{| l | l | l | l |}
\hline
Size & \LAO & \LLin & D-Finder \\ \hline \hline
$1000$ &         $0.46 s$  &   $0.7 s$       & $15 s$ \\ \hline
$2000$ &          $1.4 s$  &   $1.9 s$       & $60s$ \\ \hline
$3000$ &          $2.9 s$  &    $4$       & $2m:41s$ \\ \hline
$4000$ &          $4.8 s$  &    $7$        & $5m:37s$ \\ \hline
$5000$ &          $8.3 s$  &    $12$        & $12m:38s$ \\ \hline
$6000$ &          $13.0 s$ &    $17$         & $17m:48s$ \\ \hline
$7000$ &          $17.2 s$ &   $25$        & $30m:18s$ \\ \hline
$8000$ &          $25.6 s$ &   $34$        & $-$ \\ \hline
$9000$ &          $34.1 s$ &   $55$        & $-$ \\ \hline
$10000$ &          $47 s$  &   $62 s$          & $-$ \\ \hline 
\end{tabular}
\caption{Benchmarks: Dining Philosopher}
\label{bench:dining}
\end{table}


\begin{itemize}
\item number of states = $3^{N} \times 2^N$
\item $\LAO$: maximum $\l$ is 1, maximum number of states is 18, which is constant w.r.t. to the size of the system, i.e., $N$, (
\item $\LLin$: maximum $\l$ is 2, , maximum number of states is $648 = 2^3 \times 3^4$, which is constant w.r.t. to the size of the system, i.e., $N$
\item although \LLin is more efficient to compute but it requires more $\l$ (less complete)
\item DFinder2 (Most efficient implementation Incremental (IPM) - Incremental Positive Mapping - manually partitioning) 
\end{itemize}


\subsubsection{Experiment: Conflict-Resource Allocation System}


\begin{figure}[ht]
\begin{center}
\includegraphics[scale=1.2]{compiledfigures/client-crop.pdf}
\caption{Client}
\label{fig:client}
\end{center}
\end{figure}

\begin{figure}[ht]
\begin{center}
\includegraphics[scale=1.2]{compiledfigures/resource-crop.pdf}
\caption{Resource}
\label{fig:resourse}
\end{center}
\end{figure}

\begin{figure}[ht]
\begin{center}
\includegraphics[scale=1.2]{compiledfigures/token-crop.pdf}
\caption{Token Resource Manager}
\label{fig:conflict-token}
\end{center}
\end{figure}

\begin{figure}[ht]
\begin{center}
\includegraphics[scale=1.2]{compiledfigures/resourceallocation-crop.pdf}
\caption{Conflict-Resource Allocation System}
\label{fig:resourceallocation}
\end{center}
\end{figure}


**********************************************\\
Lesson 1: DFinder only global deadlock \\
5 clients and 5 resources \\
resourceMapping = {{0, 2}, {2, 0}, {1} , {3}, {4}};\\
conflictingResources = {{0, 1}, {2, 3}, {4}};\\
nbOfTokens = 3;\\\\


Local deadlock but not global deadlock \\
DFinder (deadlock-free)\\
however there exists a local deadlock (client0.RR, resource0.SR), whole system \\
**********************************************\\
Lesson 2: LALT more complete \\
5 clients and 5 resources; \\
resourceMapping = {{0, 2}, {0, 2}, {1} , {3}, {4}};\\
conflictingResources = {{0, 1}, {2, 3, 4}};\\
nbOfTokens = 2;\\

LALT no local and global deadlock \\
LLIN the system might contain deadlock \\
DFinder the system might contain deadlock\\
****************************************************
Lesson 3: Future work (a component may block forever, no local deadlock exists though)\\
Cannot find a subsystem s.t. when considered in isolation has a deadlock state (conspiracies)\\

5 clients and 5 resources; \\
resourceMapping = {{0, 1}, {1, 0}, {2} , {3}, {4}};\\
conflictingResources = {{0, 1}, {2, 3, 4}};\\
nbOfTokens = 2;\\

LALT no local and global deadlock \\
LLIN the system might contain deadlock  \\
DFinder the system might contain deadlock\\


\begin{table}
\centering
\begin{tabular}{| l | l | l | l |}
\hline
Size & \LAO & \LLin & D-Finder \\ \hline \hline
$10$ &          $148 s$ \\ \hline
$12$ &          $169 s$ \\ \hline
$14$ &          $189 s$ \\ \hline
$16$ &          $230 s$ \\ \hline
$18$ &          $254 s$  \\ \hline
$20$ &          $277 s$  \\ \hline 
$22$ &          $298 s$ \\ \hline 
$24$ &          $318 s$   \\ \hline 
$26$ &          $351 s$  \\ \hline 
$28$ &          $374 s$  \\ \hline
$30$ &          $430 s$   \\ \hline  
\end{tabular}
\caption{Benchmarks: Conflict-Resource Allocation}
\label{bench:resourceallocation}
\end{table}

TODO
\begin{itemize}
\item n clients, n resources , client i requests resource i, and n token 
\item number of states global system: $4^n \times 3^n \times 5^n$
\item maximum $\l$ is 2
\item \LAO linear w.r.t. n! although number of states is exponential w.r.t. n (12 components out of 3 * n), 23040000 states regardless of n
\item \LLin not complete (the system might contain deadlock)
\item D-Finder time limit (one hour) for n = 10. Try different combinations of partitions
\end{itemize}





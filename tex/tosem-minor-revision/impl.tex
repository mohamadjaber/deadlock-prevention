To implement our deadlock-freedom conditions, we must:
\bn
\item \label{task:check-init} Check that all initial states are supercycle-free
\item \label{task:evaluate-lao} Evaluate $\LAO$
\item \label{task:evaluate-llin} Evaluate $\LLin$
\en
%
Tasks \ref{task:check-init} and \ref{task:evaluate-lao} require the computation of $\lfp{\lVFsymb_{\sD}}$.
\fig{algcomputeLFP} presents an algorithm that does this. Its
correctness follows immediately from 
%\defn{blocksLoc} and \defn{violFixLoc}.
Definitions~\ref{defn:blocksLoc} and \ref{defn:violFixLoc}.

%\subsection{Computing the least fixpoint of $\lVFs$}
   \label{secn:computeLFP}
%   


\newcommand{\fpt}{\mathit{fixpoint}}
\newcommand{\VLA}[5]{\ensuremath{\MATHIDN{V_{\dsk{#1}{#2},#3} [#4 #5]}}}    %sc violations array, #1 is a, #2 is \l, #3 is t_a, #4 is v, #5 is d
                                                                        % i.e., \VV{a}{\l}{t_a}{v}{d}

\begin{figure}[H]
\setcounter{lctr}{0}
\begin{tabbing}\label{alg:compute-scViol}
aa\= aa\= aa\= aa\= aa\=\kill
\cLFP{$\dsk{\act}{\l},t_\act$}\\
\cmnt\ compute least fixpoint of $\lVF$ in state $t_\act$ of $\dsk{\act}{\l}$\\
\cmnt\ Precondition: $\fas v \in \dsk{\act}{\l}: \VLA{\act}{\l}{t_\act}{v}{0} = \fff$ \\
\cmnt\ Postcondition: $\fas v \in \dsk{\act}{\l}: \VLA{\act}{\l}{t_\act}{v}{d}$ 
$ =
  \begin{cases}
    \ttt     & \quad \text{if }  v \in \lVFs^{d} (\ewfg)\\
    \fff     & \quad \text{otherwise}\\
  \end{cases}
$
\\
\lio{d \gts 1} 
\lio{\fpt \gts \ttt} 
\lio{\WHILE \neg \fpt}
   \lit{\mbox{invariant: $V$ computed correctly up to $d-1$}} 
   \lit{\fpt \gts \fff} 
   \lit{\FORALLC{v \in \dsk{\act}{\l}}}
      \lih{\IFC{\mbox{$v$ is an interior interaction $\actp$ and } 
                          \neg (\ex \B_i : \actp \ar \B_i \in \wfg{\dsk{\act}{\l}}{t_\act})}}
          \lifc{\VLA{\act}{\l}{t_\act}{v}{d} \gts \ttt}{\cmnt Base case: interaction with no outgoing wait-for-edges}  

     \lih{\ELSFC{\mbox{$v$ is an interior interaction $\actp$ and } 
                  (\fa \B_i : \actp \ar \B_i \in \wfg{\dsk{\act}{\l}}{t_\act} :  \VV{\act}{\l}{t_\act}{B_i}{})}}
          \lif{\VLA{\act}{\l}{t_\act}{v}{d} \gts \ttt}
     \lih{\ELSFC{\mbox{$v$ is a component $\B_i$ and }
                  (\ex \actp : \B_i \ar \actp \in \wfg{\dsk{\act}{\l}}{t_\act} : \VV{\act}{\l}{t_\act}{\actp}{})}}
          \lif{\VLA{\act}{\l}{t_\act}{v}{d} \gts \ttt}
    \lih{\FI}
    \lit{\IFC{\neg \VLA{\act}{\l}{t_\act}{v}{d-1} \land \VLA{\act}{\l}{t_\act}{v}{d}} \fpt \gts \fff} 
  \lit{\ENDFOR}
\lion{\ENDWHILE} 
\end{tabbing}
\vspace{-3ex}
\caption{Procedure to compute all supercycle-violations in state $t_\act$ of $\dsk{\act}{\l}$}
\label{fig:computeSCViolateLocDSK}
\label{fig:computeSCViolateLoc}
\end{figure}



\newcommand{\fpt}{\mathit{fixpoint}}
\newcommand{\VLA}[5]{\ensuremath{\MATHIDN{V_{\dsk{#1}{#2},#3} [#4 #5]}}}    %sc violations array, #1 is a, #2 is \l, #3 is t_a, #4 is v, #5 is d
                                                                        % i.e., \VV{a}{\l}{t_a}{v}{d}

\begin{figure}[H]
\setcounter{lctr}{0}
\begin{tabbing}\label{alg:compute-lfp}
aaa\= aa\= aa\= aa\= aa\=\kill
\cLFP{$\dsk{\act}{\l},\sD$}\\
\cmnt\ Precondition: $\sD$ is a state of $\dsk{\act}{\l}$\\
\cmnt\ Postcondition: returns least fixpoint of $\lVFsymb_{\sD}$ \ie $\lfp{\lVFsymb_{\sD}} = \JOIN_{d \ge 1} \lVFsi{\sD}{\ewfg}{d}$\\
%in state $t_\DS$ of $\dsk{\act}{\l}$, \ie $\lfp{\lVFs} = \UN_{d \ge 0} \lVFs^{d} (\ewfg)$.\\

\lio{\XS \gts \ewfg;}

\lio{\WHILEC{\true}}
   \lit{\mbox{\cmnt loop invariant: at end of $i$'th iteration, $\XS = \lVFsi{\sD}{\ewfg}{i} =  \JOIN_{1 \le d \le i} \lVFsi{\sD}{\ewfg}{d}$}}
   \lit{\YS \gts \cVL{\dsk{\act}{\l}, \sD, \XS};}
   \litc{\IF \XS = \YS \ \RETURNE{\XS};}{\cmnt reached fixpoint}
   \litc{\XS \gts \YS}{\cmnt $\XS \ne \YS$, so keep iterating}
\lion{\ENDWHILE}
\end{tabbing}
\caption{Procedure to compute $\lfp{\lVFsymb_{\sD}}$.}
\label{fig:algcomputeLFP}
\end{figure}




\begin{figure}[H]
\setcounter{lctr}{0}
\begin{tabbing}\label{alg:compute-lfp}
aaa\= aa\= aa\= aa\= aa\=\kill
\cVL{$\dsk{\act}{\l}, \sD, \XS$}\\
\cmnt\ precondition: $\sD$ is a state of $\dsk{\act}{\l}$ and $X \subg \lwfg{\B}{\sD}{\dsk{\act}{\l}}$\\
\cmnt\ postcondition: returns $\lVFs{\sD}{X}$\\ %  in state $t_\DS$ of $\dsk{\act}{\l}$,\\
\\
\lio{\mbox{compute $\lwfg{\B}{\sD}{\dsk{\act}{\l}}$};}
\lio{\YS \gts \ewfg;}

\lio{\FORALLC{\ndv \in \lwfg{\B}{\sD}{\DS}}}

   \lit{\IFC{\neg \cLBL{$\ndv, \DS, \sD, \complLoc{\XS}$}}}
       \lihc{\YS \gts \YS \join \set{\ndv}}{\cmnt satisfies \defn{violFixLoc}}

\lio{\ENDFOR;}
\lion{\RETURNE{\YS}}
\end{tabbing}
\caption{Procedure to compute $\lVFs{\sD}{X}$}
\label{fig:algcomputeVL}
\end{figure}



\begin{figure}[H]
\setcounter{lctr}{0}
\begin{tabbing}\label{alg:compute-lfp}
aaa\= aa\= aa\= aa\= aa\=\kill
\cLBL{$\ndv, \dsk{\act}{\l}, \sD, \XS$}\\
\cmnt\ precondition: $\sD$ is a state of $\dsk{\act}{\l}$, $X \subg \lwfg{\B}{\sD}{\dsk{\act}{\l}}$, and $\ndv \in \lwfg{\B}{\sD}{\DS}$\\
\cmnt\ postcondition: returns  $\lblocks{\sD}{\ndv}{X}$\\


   \lit{\IFC{\mbox{$v$ is an border interaction of $\DS$}}}
       \lihc{\RETURNE{\true}}{\cmnt border interaction pessimistically assumed to be blocked}

    \lit{\ELSFC{\mbox{$v$ is an interior interaction $\act$ and } 
                        (\ex \B_i \in \XS : \act \ar \B_i \in \lwfg{\B}{\sD}{\DS})}}
        \lihc{\RETURNE{\true}}{\cmnt some outgoing wait-for edge goes to a component in $\XS$}

    \lit{\ELSFC{\mbox{$v$ is a component $\B_i$ and }
                          (\fa \act : \B_i \ar \act \in \lwfg{\B}{\sD}{\DS}  \imp \act \in \XS )}}
         \lihc{\RETURNE{\true}}{\cmnt all outgoing wait-for edges go to an interaction in $\XS$}

    \lit{\ELSE}
         \lihc{\RETURNE{\false}}{\cmnt all cases for blocking do not hold, cf.~\defn{blocksLoc}}

    \lit{\FI}
\end{tabbing}
\caption{Procedure to compute $\lblocks{\sD}{\ndv}{X}$}
\label{fig:algcomputeLBL}
\end{figure}








Note that $\YS \join \set{\ndv}$ is the join of the wait-for-subgraph $\YS$ with the 
wait-for-subgraph $\ndv$ consisting of the single node $\ndv$. Recall
that the edges of $\YS \join \set{\ndv}$ are induced from $\lwfg{B}{\sD}{\DS}$.


Let 
$\Adj_v$ be the outdegree of $\ndv$ in $\lwfg{\B}{\tD}{\DS}$,
$\cnodes{\DS}$ be the number of components and interactions in $\DS$, \ie the number of nodes in $\lwfg{\B}{\sD}{\DS}$ and
$\card{\DS}$ be the size of the syntactic description of $\DS$.
%$\card{\lwfg{\B}{\sD}{\DS}}$ is the size (number of nodes plus number of edges) of $\lwfg{\B}{\sD}{\DS}$,
%$\card{\DS}$ is the number of component and interactions in $\dsk{\act}{\l}$, which is also the number of nodes in $\lwfg{\B}{\tD}{\DS}$,
%
We assume that membership in $\XS$ and $\YS$ can be determined in constant time, \eg by using Boolean arrays, and that evaluation of 
transition guards in components takes time proportional to the length of the guards.
%
Then, the time complexity of \cLBL{$\ndv, \dsk{\act}{\l}, \sD, \XS$} is $O(\Adj_\ndv)$.
%
Hence the time complexity of \cVL{$\DS, \sD, \XS$} is 
$O(\card{\DS} + \card{\lwfg{\B}{\tD}{\DS}})$, since each edge in
$\lwfg{\B}{\tD}{\DS}$ is examined at most once, over all calls of
\cLBL{$\ndv, \dsk{\act}{\l}, \sD, \XS$},
and computing $\lwfg{\B}{\tD}{\DS}$ can be done in time $O(\card{\DS})$.
Since $\card{\lwfg{\B}{\tD}{\DS}}$ is $O(\card{\DS})$, this is just $O(\card{\DS})$.
%
%The time complexity of \cLFP{$\DS, \sD$} is then $O(\Fix \mult (\card{\lwfg{\B}{\sD}{\DS}} + \card{\DS} \mult \Adj))$
The time complexity of \cLFP{$\DS, \sD$} is then $O(\Fix \mult \card{\DS})$
where $\Fix$ is the number of iterations that  \cLFP{$\DS, \sD$} takes to reach a fixpoint.
$\Fix$ is $O(\cnodes{\DS})$, since each iteration either adds a node or reaches the least fixpoint.
Thus the time complexity of  \cLFP{$\DS, \sD$} is $O(\cnodes{\DS} \mult \card{\DS})$.
%





\thm{local.deadlock-free} requires that all initial states be supercycle-free. 
We assume that the number of initial states is small, so that we can check each explicitly. 
\fig{checkInitSCFree}  presents an algorithm which checks that all
initial states are supercycle-free.

% \subsection{Checking that initial states are supercycle-free}
   \label{s:initSCFree}
%   
Our deadlock-freedom theorem require that all initial states be sueprcycle-free. 
We assume that the number of initial states is small, so that we can check each explicitly. 


\begin{figure}[ht]
\setcounter{lctr}{0}
\begin{tabbing}\label{alg:compute-scViol}
mm\= mm\= mm\= mm\= mm\=\kill
\cInitSCFree{$Q_0$}\\
\cmnt\ returns true iff all initial states are supercycle-free\\
\lio{\FORALLC{s_0 \in Q_0}}
   \lit{\mbox{compute $\wfg{\B}{s_0}$}}
   \lit{\mbox{let $U$ be the result of removing from $\wfg{\B}{s_0}$ all nodes $v$ such that $(\ex d \ge 1: \viol{v}{d}{t})$}}
   \litc{\IFC{\mbox{$U$ is nonempty}}\ \THEN\ \RETURNE{\fff}}{\cmnt\ $s_0$ not supercycle-free, so return false}
\lion{\ELSE\ \RETURNE{\ttt}}
\end{tabbing}
\label{fig:checkInitSCFree}
\caption{Procedure to check that all initial states are supercycle-free}
\end{figure}


\bp
\cInitSCFree{$Q_0$} returns true iff all initial states are supercycle-free.
\ep
\prf{
Consider the execution of \cInitSCFree{$Q_0$} for an arbitrary $s_0 \in Q_0$.

Suppose that $U$ is nonempty. 
%OLD By \prop{scViol-iff-notInSC} Proposition~\ref{prop:notInSC-implies-scViol}, $U$ is a supercycle. 
By \prop{scViol-iff-notInSC}, $U$ is a supercycle. 
Since $U \sub \wfg{\B}{s_0}$, we conclude that $s_0$ not supercycle-free, so false is the correct
result in this case.

Now suppose that $U$ is empty. Hence every node in $\wfg{\B}{s_0}$ has a supercycle violation, and so
%OLD by Proposition~\ref{prop:scViol-implies-notInSC}, no node of $\wfg{\B}{s_0}$ can be in a 
by \prop{scViol-iff-notInSC}, no node of $\wfg{\B}{s_0}$ can be in a 
strongly-connected supercycle. Hence  $\wfg{\B}{s_0}$ does not contain a strongly-connected supercycle.
So, by Proposition~\ref{prop:supercycle:contains-mssc}, $\wfg{\B}{s_0}$ does not contain a supercycle.
}



\begin{figure}[ht]
\setcounter{lctr}{0}
\begin{tabbing}\label{alg:compute-scViol}
aaa\= aa\= aa\= aa\= aa\=\kill
\cInitSCFree{$Q_0$}\\
\cmnt\ returns true iff all initial states are supercycle-free\\
\lio{\FORALLC{s_0 \in Q_0}}
   \lit{\mbox{compute $\wfg{\B}{s_0}$}}
   \lit{\US \gets \cLFP{\B, s_0}}
%\mbox{let $U$ be the result of removing from $\wfg{\B}{s_0}$ all nodes $v$ such that $\scV{v}{s_0}$}}
   \litc{\IFC{U \ne \wfg{\B}{s_0}}\ \THEN\ \RETURNE{\fff}}{\cmnt\ $s_0$ not supercycle-free, so return false}
\lio{\ENDFOR;}
\liocn{\RETURNE{\ttt}}{\cmnt all initial states are supercycle-free}
\end{tabbing}
\caption{Procedure to check that all initial states are supercycle-free}
\label{fig:checkInitSCFree}
\end{figure}


\begin{proposition}
\cInitSCFree{$Q_0$} returns true iff all initial states are supercycle-free.
\end{proposition}
%
\begin{proof}
Consider the execution of \cInitSCFree{$Q_0$} for an arbitrary $s_0 \in Q_0$.

By its construction, $\US \subg \wfg{\B}{s_0}$.
Suppose that $\US \ne  \wfg{\B}{s_0}$.
%OLD By \prop{scViol-iff-notInSC} Proposition~\ref{prop:notInSC-implies-scViol}, $U$ is a supercycle. 
By Propositions~\ref{prop:scViol-iff-notInSC} and
\ref{prop:supercycle:union}, the nodes of $\wfg{\B}{s_0}$ that are not in $\US$ constitute a supercycle.
Hence $s_0$ not supercycle-free, and so false is the correct result in this case.

Now suppose that $\US = \wfg{\B}{s_0}$. Hence every node in $\wfg{\B}{s_0}$ has a supercycle violation, and so
%OLD by Proposition~\ref{prop:scViol-implies-notInSC}, no node of $\wfg{\B}{s_0}$ can be in a 
by \prop{scViol-iff-notInSC}, no node of $\wfg{\B}{s_0}$ is in a 
supercycle. Hence  $\wfg{\B}{s_0}$ does not contain a supercycle, and so
$s_0$ is supercycle-free.
Hence the for loop continues on to the next initial state.
If all initial states are supercycle-free, the for loop terminates, and
\cInitSCFree{$Q_0$} returns true, as required.
\end{proof}


Let
%$\card{\wfg{\B}{s_0}}$ be the size (number of nodes plus number of edges) of $\wfg{\B}{s_0}$,
$\card{\B}$ be the size of the syntactic description of $\B$, 
$\cnodes{\B}$ be the number of components and interactions in $\B$, which is also the number of nodes in $\wfg{\B}{s_0}$, and
$\card{Q_0}$ be the number of states in $Q_0$, \ie the number of initial states,
%$\Adj$ is the largest outdegree (\ie length of longest adjacency list) of $\wfg{\B}{s_0}$, and 
%$\Fix(s_0)$ is the  number of iterations that  \cLFP{$\B, s_0$} takes to reach a fixpoint.
%
Then time complexity of \cInitSCFree{$Q_0$} is 
$O(\card{Q_0} \mult \cnodes{\B} \mult \card{\B})$.

%$\Fix(s_0)$ is $O(nodes(\B))$ since each iteration either adds at least one node or reaches a fixpoint.
%Hence we have a running time of $O(\SUM_{s_0 \in Q_0}  (\card{\wfg{\B}{s_0}}^2 + \card{\B} \mult \Adj)$







   \subsection{Implementation of the Linear Condition $\LLin$}
   \label{s:implLin}
%   
%That is, a system $(B, Q_0)$ whose set of states is finite. 
%Implementations of the method for infinite-state systems are a subject for future work.

%
\checkLin{$\B, Q_0$}\ iterates over each interaction $\act$ of ($\B, Q_0$), and checks 
$(\ex \l > 0: \LLin(B, Q_0, \act, \l))$ by starting with $\l=1$ and incrementing $\l$ until
either $\LLin(B, Q_0, \act, \l)$ is found to hold, or 
$\dsk{\act}{\l}$ has become the entire system and $\LLin(B, Q_0, \act, \l)$ does not hold. In the latter case, 
$\LLin(B, Q_0, \act, \l)$ does not hold for any finite $\l$, and, in practice, 
computation would halt 
before $\dsk{\act}{\l}$ had become the entire system, due to exhaustion of resources.

\checkLinIntDist{$\B, Q_0, \act, \l$} checks $\LLin(B, Q_0, \act, \l)$ by examining every reachable transition
that executes $a$, and checking that the final state satisfies
Definition~\ref{def:ldfc-k}. 

%If $\LDFC(a, \l)$ does not hold for any finite $\l$, then the
%iteration for $a$ must be halted by some predetermined time bound, or
%interactively, or when $\dsk{\act}{\l}$ has become the entire system.
\begin{figure}[H]
\setcounter{lctr}{0}
\begin{tabbing}\label{alg:check-df}
mm\= mm\= mm\= \kill
\checkLin{$\B, Q_0$},  where $\B \df \gamma(\B_1,\dots,\B_n)$\\
\lio{\FORALLC{\mbox{interactions $\act \in \gamma$}}}
   \lit{\IFC{\checkLinInt{\B, Q_0, \act} = \fff}\ \RETURNE{\fff} \ \FI}
\lio{\ENDFOR;}
\liocn{\RETURNE{\ttt}}{\cmnt\ return $\ttt$ if check succeeds for all $\act \in \gamma$}
\end{tabbing}

\setcounter{lctr}{0}
\begin{tabbing}\label{alg:checkInt}
mm\= mm\= mm\= \kill
\checkLinInt{$\B, Q_0, \act$},  where $\B \df \gamma(\B_1,\dots,\B_n), \act \in \gamma$\\
\cmnt\ check $(\ex \l > 0: \LLin(\B, Q_0, \act, \l))$\\
\lioc{\l \gts 1;}{\cmnt\ start with $\l = 1$}
\lio{\WHILEC{\ttt}}
   \litc{\IFC{\checkLinIntDist{\act, \l} = \ttt}\ \RETURNE{\ttt}\ \FI;}{\cmnt\ success, so return true}
   \litc{\IFC{\dsk{\act}{\l} = \gamma(\B_1,\dots,\B_n)}\ \RETURNE{\fff}\ \FI;}{\cmnt\ exhausted all subsystems, return false}
   \litc{\l \gts \l + 1}{\cmnt\ increment $\l$ until success or intractable or failure}
\lio{\ENDWHILE}
\end{tabbing}

\setcounter{lctr}{0}
\begin{tabbing}
\label{alg:eval-ldfc}
mm\= mm\= mm\= \kill
\checkLinIntDist{$\B, Q_0, \act, \l$}\\
\lio{\FORALLC{\mbox{reachable transitions $s_\act \goesto[\act] t_\act$ of $\dsk{\act}{\l}$}}}
   \lit{\IFC{\neg(\fa \B_i \in \cmps{\act}:  \widepth{\dsk{\act}{\l}}{\B_i}{t_\act} < 2\l - 1 \lor \wodepth{\dsk{\act}{\l}}{\B_i}{t_\act} < 2\l -1)}}
  	\lihc{\RETURNE{\fff}}{\cmnt\ check Definition~\ref{def:ldfc-k}}
  \lit{\FI}
\lio{\ENDFOR;}
\liocn{\RETURNE{\ttt}}{\cmnt\ return $\ttt$ if check succeeds for all transitions}
\end{tabbing}

\caption{Pseudocode for the implementation of the linear condition.}
\label{fig:implementation}
\label{fig:implementation-checkDF}
\end{figure}

\paragraph{Complexity.} The running  time of our implementation is
also 
$O(\SUM_{a \in \gamma}\ |\Mdsk{\act}{\l_a}| * |\dsk{\act}{\l_a}|)$,
where $\l_a$ is the smallest value of $\l$ for which $\LLin(B, Q_0, \act, \l)$ holds, and where
$|\dsk{\act}{\l_a}|$, and $|\Mdsk{\act}{\l_a}|$ are as above.
  


%That is, a system $(B, Q_0)$ whose set of states is finite. 
%Implementations of the method for infinite-state systems are a subject for future work.

Figure~\ref{fig:impl.locLin} presents the pseudocode for our algorithm
\checkLin{$\B, Q_0$}\ to evaluate $\LLin$.
%
\checkLin{$\B, Q_0$}\ iterates over each interaction $\act$ of ($\B,Q_0$), and 
invokes \checkLinInt{$\B, Q_0, \act$} to evaluate $(\ex \l \ge 1: \LLin(\B, Q_0, \act, \l))$.
\checkLinInt{$\B, Q_0, \act$}
starts with $\l=1$ and increments $\l$ until either $\LLin(\B, Q_0, \act, \l)$ is found to hold, or
$\dsk{\act}{\l}$ has become the entire system and $\LLin(\B, Q_0, \act, \l)$ does not hold. In the
latter case, $\LLin(\B, Q_0, \act, \l)$ does not hold for any finite $\l$, and, in practice,
computation would halt before $\dsk{\act}{\l}$ had become the entire system, due to exhaustion of
resources. Evaluation of $\LLin(\B, Q_0, \act, \l)$ is done by 
\checkLinIntDist{$\B, Q_0, \act, \l$}, which examines every reachable transition
that executes $\act$, and checks that the final state satisfies
Definition~\ref{def:ldfc-k}. 

%If $\LDFC(a, \l)$ does not hold for any finite $\l$, then the
%iteration for $a$ must be halted by some predetermined time bound, or
%interactively, or when $\dsk{\act}{\l}$ has become the entire system.
\begin{figure}[H]
\setcounter{lctr}{0}
\begin{tabbing}\label{alg:check-df}
aaa\= aa\= aa\= \kill
\checkLin{$\B, Q_0$},  where $\B \df \gamma(\B_1,\dots,\B_n)$\\
\cmnt\ returns $\ttt$ iff $(\fa \act \in \gamma, \ex \l \ge 1: \LLin(\B, Q_0, \act, \l))$\\
\lio{\FORALLC{\mbox{interactions $\act \in \gamma$}}}
   \lit{\IFC{\checkLinInt{\B, Q_0, \act} = \fff}\ \RETURNE{\fff} \ \FI}
\lio{\ENDFOR;}
\lioc{\RETURNE{\ttt}}{\cmnt\ return $\ttt$ if check succeeds for all $\act \in \gamma$}
\end{tabbing}

\setcounter{lctr}{0}
\begin{tabbing}\label{alg:checkInt}
aaa \= aa\= aa\= \kill
\checkLinInt{$\B, Q_0, \act$},  where $\B \df \gamma(\B_1,\dots,\B_n), \act \in \gamma$\\
\cmnt\ returns $\ttt$ iff $(\ex \l \ge 1: \LLin(\B, Q_0, \act, \l))$\\
\lioc{\l \gts 1;}{\cmnt\ start with $\l = 1$}
\lio{\WHILEC{\ttt}}
%   \lit{\mbox{let $\dsk{\act}{\l} = \dsks{\act}{\l}$}}
   \litc{\IFC{\checkLinIntDist{\B, Q_0, \act, \l} = \ttt}\ \RETURNE{\ttt}\ \FI;}{\cmnt\ success, so return true}
   \litc{\IFC{\dsks{\act}{\l} = \gamma(\B_1,\dots,\B_n)}\ \RETURNE{\fff}\ \FI;}{\cmnt\ exhausted all subsystems, return false}
   \litc{\l \gts \l + 1}{\cmnt\ increment $\l$ until success (true) or intractable or failure (false)}
\lio{\ENDWHILE}
\end{tabbing}

\setcounter{lctr}{0}
\begin{tabbing}
\label{alg:eval-ldfc}
aaa\= aa\= aa\= \kill
\checkLinIntDist{$\B, Q_0, \act, \l$}\\
\cmnt\ returns $\ttt$ iff $\LLin(\B, Q_0, \act, \l))$\\
\lio{\mbox{let $\DS = \dsks{\act}{\l}$}}
\lio{\FORALLC{\mbox{reachable transitions $\tD \goesto[\act] \sD$ of $\DS$}}}
   \lit{\mbox{compute $\lwfg{\B}{\sD}{\DS}$};}
   \lit{\IFC{\neg(\fa \B_i \in \cmps{\act}:  \widepth{\dsk{\act}{\l}}{\B_i}{\sD} < 2\l - 1 \lor \wodepth{\dsk{\act}{\l}}{\B_i}{\sD} < 2\l -1)}}
  	\lihc{\RETURNE{\fff}}{\cmnt\ check Definition~\ref{def:ldfc-k}}
  \lit{\FI}
\lio{\ENDFOR;}
\liocn{\RETURNE{\ttt}}{\cmnt\ return $\ttt$ if check succeeds for all transitions}
\end{tabbing}

\caption{Pseudocode for the implementation of the linear condition.}

\label{fig:impl.locLin} 
\label{fig:implementation}
\label{fig:implementation-checkDF}
\end{figure}

\paragraph{Time complexity} 
Let 
$\l_\act$ be the smallest value of $\l$ for which $\LLin(\B, Q_0, \act, \l)$ holds, 
$\Mdsk{\act}{\l}$ be the transition system of $\dsks{\act}{\l}$, 
$|\Mdsk{\act}{\l}|$ be the size (number of nodes plus number of edges) of $\Mdsk{\act}{\l}$, and
$|\dsks{\act}{\l}|$ be the size of the syntactic description of $\dsks{\act}{\l}$.
%
Then the running  time of \checkLinIntDist{$\B, Q_0, \act, \l$} is $O( |\Mdsk{\act}{\l}| \mult |\dsks{\act}{\l}|)$, since 
computing $\lwfg{\B}{\sD}{\dsks{\act}{\l}}$ can be done in time $O(|\dsks{\act}{\l}|)$, and 
$\lwfg{\B}{\sD}{\dsks{\act}{\l}}$ has size in $O(|\dsks{\act}{\l}|)$, and computing in-depth and out-depth in 
$\lwfg{\B}{\sD}{\dsks{\act}{\l}}$ can be done in linear time using depth-first graph search.
%
The running  time of \checkLinInt{$\B, Q_0, \act$}, is $O( \SUM_{1 \le \l \le \l_\act}   |\Mdsk{\act}{\l}| \mult |\dsks{\act}{\l}|)$.
The running  time of \checkLin{$\B, Q_0$}\ is
$O(\SUM_{\act \in \gamma}    \SUM_{1 \le \l \le \l_\act}   \    |\Mdsk{\act}{\l}| \mult |\dsks{\act}{\l}|)$.



 
   \subsection{Implementation of the AND-OR Condition $\LAO$}
   \label{s:implANDOR}
%   
Figure~\ref{fig:impl.locANDOR} presents the pseudocode for our algorithm to 
evaluate $\LAO$.
This uses the algorithm for computing supercycle violations based on $\dsk{\act}{\l}$,
given in Figure~\ref{fig:algcomputeLFP}.

\checkLAO{$\B, Q_0$} iterates over each interaction $\act$ of ($\B, Q_0$), and 
invokes \checkLAOInt{$\B, Q_0, \act$} to evaluate $(\ex \l > 0: \LAO(B, Q_0, \act, \l))$.
\checkLAOInt{$\B, Q_0, \act$}
starts with $\l=1$ and increments $\l$ until either $\LAO(B, Q_0, \act, \l)$ is found to hold, or
$\dsk{\act}{\l}$ has become the entire system and $\LAO(B, Q_0, \act, \l)$ does not hold. In the
latter case, $\LAO(B, Q_0, \act, \l)$ does not hold for any finite $\l$, and, in practice,
computation would halt before $\dsk{\act}{\l}$ had become the entire system, due to exhaustion of
resources.
Note that $\dsk{\act}{1}$ is the smallest system in which a
supercycle-violation can be confirmed. 

Evaluation of $\LAO(B, Q_0, \act, \l)$ is done by 
\checkLAOIntDist{$\B, Q_0, \act, \l$}, which invokes
$\cLFP{\dsk{\act}{\l},t_\act}$ to compute the supercycle violations.
The pseudocode is a straightforward translation of
Definitions~\ref{def:sConn.violation.loc} and \ref{def:locFormation.violation}.
%
Figure~\ref{fig:summaryProcedures} shows a summary of the procedures.

%%%%%%%%%%%%%%%%%%%%%%%%%%%%%%%%%%%%%%%%%%%%%%%%%%%%%%%%%%%%%%%%%%%%
\begin{figure}%[H]
{\normalsize
\begin{tabular}{|l|l|}
\hline
\checkLAO{$\B, Q_0$} & true iff $(\fa \act \in \gamma, \ex \l > 0: \LAO(B, Q_0, \act, \l))$\\ \hline
\checkLAOInt{$\B, Q_0, \act$} & true iff $(\ex \l > 0: \LAO(B, Q_0, \act, \l))$\\ \hline
\checkLAOIntDist{$\B, Q_0, \act, \l$} &  true iff $\LAO(\B, Q_0, \act, \l)$\\ \hline

\cLFV{$B_i, V, \dsk{\act}{\l},t_\act$} & true iff $B_i$ has local sc-formation violation \\
& in state $t_\act$ of $\dsk{\act}{\l}$, \ie $\locFormViol{B_i}{t_\act}{\act}{\l}$ holds\\ \hline

\cLconnScV{$B_i, \dsk{\act}{\l},t_\act$} & true iff $B_i$ has local strong connectedness
                                         violation \\ & in $t_{\act}$,  \ie $\locConnViol{B_i}{t_\act}{\act}{\l}$ holds\\ \hline

\cLFP{$\dsk{\act}{\l},t_\act$} & compute local supercycle violations \\ & in state $t_\act$ of $\dsk{\act}{\l}$, \ie $\lviol{v}{d}{t_\act}{\act}{\l}$ for all $v,d$\\
\hline
\end{tabular}
}
\caption{Summary of procedures}
\label{fig:summaryProcedures}
\end{figure}


%%%%%%%%%%%%%%%%%%%%%%%%%%%%%%%%%%%%%%%%%%%%%%%%%%%%%%%%%%%%%%%%%%%%%%%%%%%%%%%
 
\paragraph{Complexity} The running  time of our implementation is
$O(\SUM_{a \in \gamma}\  |\Mdsk{\act}{\l_a}| * |\dsk{\act}{\l_a}|)$, 
where 
$\Mdsk{\act}{\l_a}$ is the transition system of
$\dsk{\act}{\l_a}$, and $|\Mdsk{\act}{\l_a}|$ is the size (number of nodes plus number of edges) of 
$\Mdsk{\act}{\l_a}$, 
$|\dsk{\act}{\l_a}|$ is the size of the syntactic description of $\dsk{\act}{\l_a}$, and 
$\l_a$ is the smallest value of $\l$ for which $\LAO(B, Q_0, \act, \l)$ holds.


%%%%%%%%%%%%%%%%%%%%%%%%%%%%%%%%%%%%%%%%%%%%%%%%%%%%%%%%%%%%%%%%%%%%
\begin{figure}%[H]
\setcounter{lctr}{0}
\begin{tabbing}\label{alg:check.LAO}
aa\= aa\= aa\= \kill
\checkLAO{$\B, Q_0$},  where $\B \df \gamma(\B_1,\dots,\B_n)$\\
\cmnt\ returns $\ttt$ iff $(\fa \act \in \gamma, \ex \l > 0: \LAO(\act, \l))$\\
\lio{\FORALLC{\mbox{interactions $\act \in \gamma$}}}
   \lit{\IFC{\checkLAOInt{\B, Q_0, \act} = \fff}\ \RETURNE{\fff}\ \FI}
\lio{\ENDFOR;}
\liocn{\RETURNE{\ttt}}{\cmnt\ return $\ttt$ if check succeeds for all $a \in \gamma$}
\end{tabbing}

\setcounter{lctr}{0}
\begin{tabbing}\label{alg:check.LAO.Int}
aaa\= aa\= aa\= \kill
\checkLAOInt{$\B, Q_0, \act$},  where $\B \df \gamma(\B_1,\dots,\B_n), \act \in \gamma$\\
\cmnt\ returns $\ttt$ iff $(\ex \l > 0: \LAO(B, Q_0, \act, \l))$\\
\lioc{\l \gts 1;}{\cmnt\ start with $\l = 1$}
\lio{\WHILEC{\ttt}}
   \litc{\IFC{\checkLAOIntDist{\act, \l} = \ttt}\ \RETURNE{\ttt}\ \FI;}{\cmnt\ success, so return true}
   \litc{\IFC{\dsk{\act}{\l} = \gamma(\B_1,\dots,\B_n)}\ \RETURNE{\fff}\ \FI;}{\cmnt\ exhausted all subsystems, return false}
   \litc{\l \gts \l + 1}{\cmnt\ increment $\l$ until success or intractable or failure}
\lion{\ENDWHILE}
\end{tabbing}

\setcounter{lctr}{0}
\begin{tabbing}
\label{alg:eval-ldfc}
aaa\= aa\= aa\= aa\= aa\=\kill
\checkLAOIntDist{$\B, Q_0, \act, \l$}\\
\cmnt\ returns $\ttt$ iff $\LAO(\B, Q_0, \act, \l)$\\
\lio{\FORALLC{\mbox{reachable transitions $s_\act \goesto[\act] t_\act$ of $\dsk{\act}{\l}$}}}
   \litc{V \gets \cLFP{\dsk{\act}{\l},t_\act}}{\cmnt see \fig{algcomputeLFP}}
   \lit{\FORALLC{\B_i \in \cmps{\act}}}
      \lihc{\IF\ \neg \cLFV{B_i, V, \dsk{\act}{\l},t_\act}\ \THEN\ \RETURNE{\fff}\ \FI}{\cmnt return $\fff$ if no violation for $\B_i$}
%         \lifc{\IFC{\neg \LF{\act}{\l}{t_\act}{v}{\l}}\ 
   \lit{\ENDFOR}
\lio{\ENDFOR}
\liocn{\RETURNE{\ttt}}{\cmnt return $\ttt$ if all $\B_i \in \cmps{\act}$ violate local supercycle formation}
\end{tabbing}



\setcounter{lctr}{0}
\begin{tabbing}
\label{alg:computeLocForm}
aaa\= aa\= aa\= aa\= aa\=\kill
\cLFV{$B_i, V, \dsk{\act}{\l},t_\act$}\\
\cmnt\ returns true iff $\locFormViol{B_i}{t_\act}{\act}{\l}$ holds (Definition~\ref{def:locFormation.violation})\\
\cmnt\ \ie $B_i$ has a local supercycle formation violation in state $t_\act$ of subsystem $\dsk{\act}{\l}$\\
\lion{\RETURNE{V[B_i] \, \lor \, \mbox{\cLconnScV{$B_i, V, \dsk{\act}{\l}, t_\act$}}}}
\end{tabbing}


\setcounter{lctr}{0}
\begin{tabbing}\label{alg:compute-scViol}
aaa\= aa\= aa\= aa\= aa\=\kill
\cLconnScV{$B_i, V, \dsk{\act}{\l}, t_\act$}\\
\cmnt\ returns true iff $\locConnViol{B_i}{t_\act}{\act}{\l}$ holds (Definition~\ref{def:sConn.violation.loc})\\
\cmnt\ \ie $B_i$ has a local strong connectedness supercycle formation violation in state $t_\act$ of subsystem $\dsk{\act}{\l}$\\
%\cmnt\ compute local supercycle violations in state $t_\act$ of $\dsk{\act}{\l}$\\
%\cmnt\ Postcondition: $\fas v \in \set{\B_1,\ldots,\B_n} \un \gamma, d' = 1,\ldots,dd: \VV{\act}{\l}{t_\act}{v}{d'} = \lviol{v}{d'}{t_\act}{\act}{\l}$\\
\lio{\mbox{remove all nodes $v$ such that $V[v]=\true$, \ie with a local supercycle violation}}
\lio{\mbox{compute maximal strongly connected components of remaining wait-for graph}}
\lio{\FORALLC{\mbox{maximal strongly connected components $C$}}}
   \lit{\mbox{\IF\ $C$ contains a non-trivial strongly connected supercycle which contains $B_i$ as a node \THEN}}
      \lihc{\RETURNE{\fff} \, \FI}{\cmnt{Definition~\ref{def:sConn.violation.loc}, Clause~\ref{def:sConn.violation.loc:scc} holds here}} 

\lio{\FORALLC{\mbox{wait-for paths $\pi$ from $B_i$ to the border of  $\dsk{\act}{\l}$}}}
   \lit{\IF\ \mbox{some node $v$ of $\pi$ has $V[v]=\true$, \ie local supercycle violation}\ \THEN}
      \lihc{\RETURNE{\ttt}\ \FI}{\cmnt Clause~\ref{def:sConn.violation.loc:wait-for-out} holds}

\lio{\FORALLC{\mbox{wait-for paths $\pi'$ from the border of $\dsk{\act}{\l}$ to $B_i$}}}
   \lit{\IF\ \mbox{some node $v$ of $\pi'$ has $V[v]=\true$, \ie a local supercycle violation}\ \THEN}
       \lihc{\RETURNE{\ttt}\ \FI}{\cmnt Clause~\ref{def:sConn.violation.loc:wait-for-in} holds}

\liocn{\RETURNE{\fff}}{\cmnt  Definition~\ref{def:sConn.violation.loc}, Clause~\ref{def:sConn.violation.border} does not hold}
\end{tabbing}

\caption{Pseudocode for the implementation of the local AND-OR condition.}
\label{fig:impl.locANDOR}
\end{figure}

% eliminate ugly space after f
%%%%%%%%%%%%%%%%%%%%%%%%%%%%%%%%%%%%%%%%%%%%%%%%%%%%%%%%%%%%%%%%%%%%%%%%




Figure~\ref{fig:impl.locANDOR} presents the pseudocode for our algorithm \checkLAO{$\B, Q_0$}\ to evaluate $\LAO$.  This uses the 
\cLFP{$\dsk{\act}{\l},\sD$} algorithm for computing local supercycle violations in $\dsk{\act}{\l}$, given in Figure~\ref{fig:algcomputeLFP}.

\checkLAO{$\B, Q_0$} iterates over each interaction $\act$ of ($\B, Q_0$), and 
invokes \checkLAOInt{$\B, Q_0, \act$} to evaluate $(\ex \l \ge 1: \LAO(\B, Q_0, \act, \l))$.
\checkLAOInt{$\B, Q_0, \act$}
starts with $\l=1$ and increments $\l$ until either $\LAO(\B, Q_0, \act, \l)$ is found to hold, or
$\dsk{\act}{\l}$ has become the entire system and $\LAO(\B, Q_0, \act, \l)$ does not hold. In the
latter case, $\LAO(\B, Q_0, \act, \l)$ does not hold for any finite $\l$, and, in practice,
computation would halt before $\dsk{\act}{\l}$ had become the entire system, due to exhaustion of
resources.
Note that $\dsk{\act}{1}$ is the smallest system in which a
supercycle-violation can be confirmed. 

Evaluation of $\LAO(\B, Q_0, \act, \l)$ is done by 
\checkLAOIntDist{$\B, Q_0, \act, \l$}, which invokes
$\cLFP{\dsk{\act}{\l},t_\DS}$ to compute the local supercycle violations and 
$\cLFV{\B_i, \VS, \dsk{\act}{\l},\sD}$ to compute the general local supercycle violations.
$\cLFV{\B_i, \VS, \dsk{\act}{\l},\sD}$ invokes \cLconnScV{$\B_i, \VS, \dsk{\act}{\l}, \sD$} to compute the 
 local strong connectedness violation.
The pseudocode is a straightforward translation of
Definitions~\ref{def:sConn.violation.loc} and \ref{def:locFormation.violation}.
%
Figure~\ref{fig:summaryProcedures} shows a summary of the procedures.

%%%%%%%%%%%%%%%%%%%%%%%%%%%%%%%%%%%%%%%%%%%%%%%%%%%%%%%%%%%%%%%%%%%%
\begin{figure}%[H]
{\normalsize
\begin{tabular}{|l|l|}
\hline
\checkLAO{$\B, Q_0$} & true iff $(\fa \act \in \gamma, \ex \l \ge 1: \LAO(B, Q_0, \act, \l))$\\ \hline
\checkLAOInt{$\B, Q_0, \act$} & true iff $(\ex \l \ge 1: \LAO(B, Q_0, \act, \l))$\\ \hline
\checkLAOIntDist{$\B, Q_0, \act, \l$} &  true iff $\LAO(\B, Q_0, \act, \l)$\\ \hline

\cLFV{$\B_i, \VS, \dsk{\act}{\l}, \sD$} & true iff $B_i$ has local sc-formation violation \\
& in state $\sD$ of $\dsk{\act}{\l}$, \ie $\locFormViol{B_i}{\sD}{\act}{\l}$ holds\\ \hline

\cLconnScV{$\B_i, \VS, \dsk{\act}{\l}, \sD$} & true iff $B_i$ has local strong connectedness
                                         violation \\ & in $t_{\act}$,  \ie $\locConnViol{B_i}{\sD}{\act}{\l}$ holds\\ \hline

\cLFP{$\dsk{\act}{\l},\sD$} & compute local supercycle violations \\ & in state $\sD$ of $\dsk{\act}{\l}$, \ie $\lviol{v}{d}{\sD}{\act}{\l}$ for all $v,d$\\
\hline
\end{tabular}
}
\caption{Summary of procedures}
\label{fig:summaryProcedures}
\end{figure}


%%%%%%%%%%%%%%%%%%%%%%%%%%%%%%%%%%%%%%%%%%%%%%%%%%%%%%%%%%%%%%%%%%%%%%%%%%%%%%%
 
\paragraph{Time complexity} 
Let 
$\l_\act$ be the smallest value of $\l$ for which  $\LAO(\B, Q_0, \act, \l)$ holds, 
$\Mdsk{\act}{\l}$ be the transition system of $\dsks{\act}{\l}$, 
$|\Mdsk{\act}{\l}|$ be the size (number of nodes plus number of edges) of $\Mdsk{\act}{\l}$, 
$|\dsks{\act}{\l}|$ be the size of the syntactic description of $\dsks{\act}{\l}$, and 
$\cnodes{\dsks{\act}{\l}}$ be the number of components and interactions in $\dsks{\act}{\l}$.
%
The time complexity of \cLconnScV{$\B_i, \VS, \dsks{\act}{\l}, \sD$} is $O(\cnodes{\dsks{\act}{\l}} \mult \card{\dsks{\act}{\l}})$, since maximal strongly connected components
are computable in linear time using Kosaraju's algorithm, and the existence of paths in $\WL$ from $\B_i$ to the border of $\dsks{\act}{\l}$ can be checked in linear time by 
using depth-first graph search. Also, checking for supercycles (via least-fixpoint computation) within the strongly connected component $C$ can be done in time 
$O(\cnodes{\dsks{\act}{\l}} \mult \card{\dsks{\act}{\l}})$, amortized over all such $C$.
%
The time complexity of \cLFV{$\B_i, \VS, \dsks{\act}{\l}, \sD$} is also $O(\cnodes{\dsks{\act}{\l}} \mult \card{\dsks{\act}{\l}})$.
%
The running time of \checkLAOIntDist{$\B, Q_0, \act, \l$} is $O(\card{\Mdsk{\act}{\l}} \mult \cnodes{\dsks{\act}{\l}} \mult \card{\dsks{\act}{\l}})$,
since \cLFP{$\dsks{\act}{\l}, \sD$} has time complexity in $O(\cnodes{\dsks{\act}{\l}} \mult \card{\dsks{\act}{\l}})$, and computing 
$\lwfg{\B}{\sD}{\dsks{\act}{\l}}$ can be done in time $O(\card{\dsks{\act}{\l}})$.
%
The running time of \checkLAOInt{$\B, Q_0, \act$} is $O(\SUM_{1 \le \l \le \l_\act} \card{\Mdsk{\act}{\l}} \mult \cnodes{\dsks{\act}{\l}} \mult \card{\dsks{\act}{\l}})$
since \checkLAOInt{$\B, Q_0, \act$} iterates \checkLAOIntDist{$\B, Q_0, \act, \l$} until $\l = \l_\act$.
%
The running  time of \checkLAO{$\B, Q_0$} is
$O(\SUM_{\act \in \gamma}  \SUM_{1 \le \l \le \l_\act} \card{\Mdsk{\act}{\l}} \mult \cnodes{\dsks{\act}{\l}} \mult \card{\dsks{\act}{\l}})$
since \checkLAO{$\B, Q_0$} calls \checkLAOInt{$\B, Q_0, \act$} for every $\act \in \gamma$.







%%%%%%%%%%%%%%%%%%%%%%%%%%%%%%%%%%%%%%%%%%%%%%%%%%%%%%%%%%%%%%%%%%%%
\begin{figure}%[H]
\setcounter{lctr}{0}
\begin{tabbing}\label{alg:check.LAO}
aa\= aa\= aa\= \kill
\checkLAO{$\B, Q_0$},  where $\B \df \gamma(\B_1,\dots,\B_n)$\\
\cmnt\ returns $\ttt$ iff $(\fa \act \in \gamma, \ex \l \ge 1: \LAO(\B, Q_0, \act, \l))$\\
\lio{\FORALLC{\mbox{interactions $\act \in \gamma$}}}
   \lit{\IFC{\checkLAOInt{\B, Q_0, \act} = \fff}\ \RETURNE{\fff}\ \FI}
\lio{\ENDFOR;}
\liocn{\RETURNE{\ttt}}{\cmnt\ return $\ttt$ if check succeeds for all $a \in \gamma$}
\end{tabbing}

\setcounter{lctr}{0}
\begin{tabbing}\label{alg:check.LAO.Int}
aaa\= aa\= aa\= \kill
\checkLAOInt{$\B, Q_0, \act$},  where $\B \df \gamma(\B_1,\dots,\B_n), \act \in \gamma$\\
\cmnt\ returns $\ttt$ iff $(\ex \l \ge 1: \LAO(\B, Q_0, \act, \l))$\\
\lioc{\l \gts 1;}{\cmnt\ start with $\l = 1$}
\lio{\WHILEC{\ttt}}
   \litc{\IFC{\checkLAOIntDist{\B, Q_0, \act, \l} = \ttt}\ \RETURNE{\ttt}\ \FI;}{\cmnt\ success, so return true}
   \litc{\IFC{\dsks{\act}{\l} = \gamma(\B_1,\dots,\B_n)}\ \RETURNE{\fff}\ \FI;}{\cmnt\ exhausted all subsystems, return false}
   \litc{\l \gts \l + 1}{\cmnt\ increment $\l$ until success or intractable or failure}
\lion{\ENDWHILE}
\end{tabbing}

\setcounter{lctr}{0}
\begin{tabbing}
\label{alg:eval-ldfc}
aaa\= aa\= aa\= aa\= aa\=\kill
\checkLAOIntDist{$\B, Q_0, \act, \l$}\\
\cmnt\ returns $\ttt$ iff $\LAO(\B, Q_0, \act, \l)$\\
\lio{\mbox{let $\DS = \dsks{\act}{\l}$}}
\lio{\FORALLC{\mbox{reachable transitions $\tD \goesto[\act] \sD$ of $\dsk{\act}{\l}$}}}
   \lit{\mbox{compute $\lwfg{\B}{\sD}{\DS}$};}
   \litc{\VS \gets \cLFP{\dsk{\act}{\l},\sD};}{\cmnt see \fig{algcomputeLFP}}
   \lit{\FORALLC{\B_i \in \cmps{\act}}}
      \lihc{\IFC{\neg \cLFV{\B_i, \VS, \dsk{\act}{\l},\sD}} \RETURNE{\fff}\ \FI}{\cmnt no violation for $\B_i$}
%         \lifc{\IFC{\neg \LF{\act}{\l}{t_\DS}{v}{\l}}\ 
   \lit{\ENDFOR}
\lio{\ENDFOR;}
\liocn{\RETURNE{\ttt}}{\cmnt return $\ttt$ if all $\B_i \in \cmps{\act}$ violate local supercycle formation}
\end{tabbing}



\setcounter{lctr}{0}
\begin{tabbing}
\label{alg:computeLocForm}
aaa\= aa\= aa\= aa\= aa\=\kill
\cLFV{$\B_i, \VS, \dsk{\act}{\l}, \sD$}\\
\cmnt\ returns true iff $\locFormViol{\B_i}{\sD}{\act}{\l}$ holds (Definition~\ref{def:locFormation.violation})\\
\cmnt\ \ie $\B_i$ has a general local supercycle violation in state $\sD$ of subsystem $\dsk{\act}{\l}$\\
\lion{\RETURNE{\B_i \in \VS \, \lor \, \mbox{\cLconnScV{$\B_i, \VS, \dsk{\act}{\l}, \sD$}}}}
\end{tabbing}


\setcounter{lctr}{0}
\begin{tabbing}\label{alg:compute-scViol}
aaa\= aa\= aa\= aa\= aa\=\kill
\cLconnScV{$\B_i, \VS, \dsk{\act}{\l}, \sD$}\\
\cmnt\ returns true iff $\locConnViol{\B_i}{\sD}{\act}{\l}$ holds (Definition~\ref{def:sConn.violation.loc})\\
\cmnt\ \ie $\B_i$ has a local strong connectedness violation in state $\sD$ of subsystem $\dsk{\act}{\l}$\\
%\cmnt\ compute local supercycle violations in state $t_\DS$ of $\dsk{\act}{\l}$\\
%\cmnt\ Postcondition: $\fas v \in \set{\B_1,\ldots,\B_n} \un \gamma, d' = 1,\ldots,dd: \VV{\act}{\l}{t_\DS}{v}{d'} = \lviol{v}{d'}{t_\DS}{\act}{\l}$\\
%\lio{\mbox{compute $\lwfg{B}{\sD}{\DS}$;}}
\lio{\mbox{$\WL \gets \lwfg{B}{\sD}{\DS} - \VS$, \ie remove from $\lwfg{B}{\sD}{\DS}$ all nodes with a local supercycle violation};}
\lio{\mbox{compute maximal strongly connected components of $\WL$};}
\lio{\FORALLC{\mbox{maximal strongly connected components $C$ of $\WL$}}}
   \lit{\IFC{\mbox{$C$ contains a non-trivial strongly connected supercycle which contains $\B_i$ as a node}}}
      \lihc{\RETURNE{\fff} \, \FI;}{\cmnt{\defn{sConn.violation.loc}, Clause~\ref{def:sConn.violation.loc:scc} violated}} 

%\lio{\FORALLC{\mbox{wait-for paths $\pi$ from $\B_i$ to the border of  $\dsk{\act}{\l}$}}}
%   \lit{\IFC{\mbox{some node $\ndv$ of $\pi$ is in $\VS$, \ie has a local supercycle violation}}}
%      \lihc{\RETURNE{\ttt}\ \FI}{\cmnt Clause~\ref{def:sConn.violation.loc:wait-for-out} holds}
%
\lio{\IFC{\mbox{there is no path in $\WL$ from $\B_i$ to a border node of $\dsk{\act}{\l}$}}} 
     \litc{\RETURNE{\ttt}\ \FI;}{\cmnt \defn{sConn.violation.loc}, Clause~\ref{def:sConn.violation.loc:wait-for-out} holds}


%\lio{\FORALLC{\mbox{wait-for paths $\pi'$ from the border of $\dsk{\act}{\l}$ to $\B_i$}}}
%   \lit{\IFC{\mbox{some node $\ndv$ of $\pi'$ is in $\VS$, \ie has a local supercycle violation}}}
%       \lihc{\RETURNE{\ttt}\ \FI}{\cmnt Clause~\ref{def:sConn.violation.loc:wait-for-in} holds}
%
\lio{\IFC{\mbox{there is no path in $\WL$ from some border node of $\dsk{\act}{\l}$ to $\B_i$}}}
     \litc{\RETURNE{\ttt}\ \FI;}{\cmnt \defn{sConn.violation.loc}, Clause~\ref{def:sConn.violation.loc:wait-for-in} holds}


\liocn{\RETURNE{\fff}}{\cmnt  \defn{sConn.violation.loc}, Clauses~\ref{def:sConn.violation.loc:scc} and \ref{def:sConn.violation.border} both violated}
\end{tabbing}

\caption{Pseudocode for the implementation of the local AND-OR condition.}
\label{fig:impl.locANDOR}
\end{figure}

% eliminate ugly space after f
%%%%%%%%%%%%%%%%%%%%%%%%%%%%%%%%%%%%%%%%%%%%%%%%%%%%%%%%%%%%%%%%%%%%%%%%






   \subsection{Tool-set}
   \label{s:tools}
%   %
We provide \deadlocktool{}, a suite of supporting tools that implement our method. \deadlocktool{} is around $\sim 2500$ Java LOC.
% and is available at  \href{http://todo}{http://todo}. 
%
\deadlocktool{} is equipped with a command line interface (see Figure~\ref{code:cmd-ldfc}) that accepts a set 
of configuration options. 
It takes the name of the input BIP file and other optional flags. 


\begin{lstlisting}[language=Bash, caption = {\deadlocktool{} Command Line Interface},label={code:cmd-ldfc}]
> java -jar ldfc.jar [options] input.bip 
and options are:
-condition <s> LLIN (local linear check) or LALT (local and/or check - default)
               (optional)
-debug         Prints useful information at each iteration of checking. 
               Example: selected interaction, depth length, etc.
               This information could be useful in case when the condition fails.

Examples:
  java -jar ldfc.jar -debug input.bip # deadlock freedom using default LALT
  java -jar ldfc.jar -condition=LLIN -debug input.bip # deadlock freedom using LLIN
\end{lstlisting}




%
We provide \deadlocktool{}, a suite of supporting tools that implement our method. \deadlocktool{} is around $\sim 2500$ Java LOC.
% and is available at  \href{http://todo}{http://todo}. 
%
\deadlocktool{} is equipped with a command line interface (see Figure~\ref{code:cmd-ldfc}) that accepts a set 
of configuration options. 
It takes the name of the input BIP file and other optional flags. 

\begin{figure}
\begin{lstlisting}[language=Bash]
> java -jar ldfc.jar [options] input.bip 
and options are:
-condition <s> LLIN (local linear check) or LALT (local and/or check - default)
               (optional)
-debug         Prints useful information at each iteration of checking. 
               Example: selected interaction, depth length, etc.
               This information could be useful in case when the condition fails.

Examples:
  java -jar ldfc.jar -debug input.bip # deadlock freedom using default LALT
  java -jar ldfc.jar -condition=LLIN -debug input.bip # deadlock freedom using LLIN
\end{lstlisting}
\caption{\deadlocktool{} Command Line Interface}
\label{code:cmd-ldfc}
\end{figure}




\subsection{Experimentation}
\label{s:experiments}
%\subsubsection{Experiment: Dining Philosophers} 
We consider $n$ philosophers in a cycle, based on the components of Figure~\ref{fig:diningSpectrum}.

\begin{table}
\centering
\begin{tabular}{| l | l | l | l |}
\hline
Size & \LAO & \LLin & D-Finder \\ \hline \hline
$1000$ &         $0.46 s$  &   $0.7 s$       & $15 s$ \\ \hline
$2000$ &          $1.4 s$  &   $1.9 s$       & $60s$ \\ \hline
$3000$ &          $2.9 s$  &    $4$       & $2m:41s$ \\ \hline
$4000$ &          $4.8 s$  &    $7$        & $5m:37s$ \\ \hline
$5000$ &          $8.3 s$  &    $12$        & $12m:38s$ \\ \hline
$6000$ &          $13.0 s$ &    $17$         & $17m:48s$ \\ \hline
$7000$ &          $17.2 s$ &   $25$        & $30m:18s$ \\ \hline
$8000$ &          $25.6 s$ &   $34$        & $-$ \\ \hline
$9000$ &          $34.1 s$ &   $55$        & $-$ \\ \hline
$10000$ &          $47 s$  &   $62 s$          & $-$ \\ \hline 
\end{tabular}
\caption{Benchmarks: Dining Philosopher}
\label{bench:dining}
\end{table}


\begin{itemize}
\item number of states = $3^{N} \times 2^N$
\item $\LAO$: maximum $\l$ is 1, maximum number of states is 18, which is constant w.r.t. to the size of the system, i.e., $N$, (
\item $\LLin$: maximum $\l$ is 2, , maximum number of states is $648 = 2^3 \times 3^4$, which is constant w.r.t. to the size of the system, i.e., $N$
\item although \LLin is more efficient to compute but it requires more $\l$ (less complete)
\item DFinder2 (Most efficient implementation Incremental (IPM) - Incremental Positive Mapping - manually partitioning) 
\end{itemize}


\subsubsection{Experiment: Conflict-Resource Allocation System}


\begin{figure}[ht]
\begin{center}
\includegraphics[scale=1.2]{compiledfigures/client-crop.pdf}
\caption{Client}
\label{fig:client}
\end{center}
\end{figure}

\begin{figure}[ht]
\begin{center}
\includegraphics[scale=1.2]{compiledfigures/resource-crop.pdf}
\caption{Resource}
\label{fig:resourse}
\end{center}
\end{figure}

\begin{figure}[ht]
\begin{center}
\includegraphics[scale=1.2]{compiledfigures/token-crop.pdf}
\caption{Token Resource Manager}
\label{fig:conflict-token}
\end{center}
\end{figure}

\begin{figure}[ht]
\begin{center}
\includegraphics[scale=1.2]{compiledfigures/resourceallocation-crop.pdf}
\caption{Conflict-Resource Allocation System}
\label{fig:resourceallocation}
\end{center}
\end{figure}


**********************************************\\
Lesson 1: DFinder only global deadlock \\
5 clients and 5 resources \\
resourceMapping = {{0, 2}, {2, 0}, {1} , {3}, {4}};\\
conflictingResources = {{0, 1}, {2, 3}, {4}};\\
nbOfTokens = 3;\\\\


Local deadlock but not global deadlock \\
DFinder (deadlock-free)\\
however there exists a local deadlock (client0.RR, resource0.SR), whole system \\
**********************************************\\
Lesson 2: LALT more complete \\
5 clients and 5 resources; \\
resourceMapping = {{0, 2}, {0, 2}, {1} , {3}, {4}};\\
conflictingResources = {{0, 1}, {2, 3, 4}};\\
nbOfTokens = 2;\\

LALT no local and global deadlock \\
LLIN the system might contain deadlock \\
DFinder the system might contain deadlock\\
****************************************************
Lesson 3: Future work (a component may block forever, no local deadlock exists though)\\
Cannot find a subsystem s.t. when considered in isolation has a deadlock state (conspiracies)\\

5 clients and 5 resources; \\
resourceMapping = {{0, 1}, {1, 0}, {2} , {3}, {4}};\\
conflictingResources = {{0, 1}, {2, 3, 4}};\\
nbOfTokens = 2;\\

LALT no local and global deadlock \\
LLIN the system might contain deadlock  \\
DFinder the system might contain deadlock\\


\begin{table}
\centering
\begin{tabular}{| l | l | l | l |}
\hline
Size & \LAO & \LLin & D-Finder \\ \hline \hline
$10$ &          $148 s$ \\ \hline
$12$ &          $169 s$ \\ \hline
$14$ &          $189 s$ \\ \hline
$16$ &          $230 s$ \\ \hline
$18$ &          $254 s$  \\ \hline
$20$ &          $277 s$  \\ \hline 
$22$ &          $298 s$ \\ \hline 
$24$ &          $318 s$   \\ \hline 
$26$ &          $351 s$  \\ \hline 
$28$ &          $374 s$  \\ \hline
$30$ &          $430 s$   \\ \hline  
\end{tabular}
\caption{Benchmarks: Conflict-Resource Allocation}
\label{bench:resourceallocation}
\end{table}

TODO
\begin{itemize}
\item n clients, n resources , client i requests resource i, and n token 
\item number of states global system: $4^n \times 3^n \times 5^n$
\item maximum $\l$ is 2
\item \LAO linear w.r.t. n! although number of states is exponential w.r.t. n (12 components out of 3 * n), 23040000 states regardless of n
\item \LLin not complete (the system might contain deadlock)
\item D-Finder time limit (one hour) for n = 10. Try different combinations of partitions
\end{itemize}




%% bip results
We evaluated \deadlocktool{} using several case studies including the dining philosopher example and multiple instances
of a configurable generalized {\em Resource Allocation System} that comprises 
a configurable multi token-based scheduler.
The different configurations of our resource allocation system subsume problems like the Milner's scheduler, 
data arbiters and the dining philosopher with a butler problem. 
We benchmarked the performance of \deadlocktool{} against DFinder \cite{DFinder2}
on two benchmarks: 
{\em Dining Philosopher} with an increasing number of philosophers and 
a deadlock free resource allocation system with an increasing number of clients and resources. 
%We provide both benchmarks at \href{http://}{}.

All experiments were conducted on a machine with Intel (R) $8$-Cores (TM) $i7$-$6700$, CPU @ $3.40$GHZ, $32$GB RAM, 
running a CentOS Linux distribution. 

\subsubsection{Dining philosophers case study} 
We consider the traditional dining philosopher problem as depicted in 
Figure~\ref{fig:diningSpectrum}, which shows $4$ philosophers and $4$ forks modeled in BIP. 

Each philosopher component has $2$ states, and each fork component has $3$ states. 
Thus, the total number of states is $2^n \times 3^n$. 
We evaluated \deadlocktool{} by increasing $n$ and applying both $\LAO$ and $\LLin$ methods and compared against the best configuration 
we could compute with DFinder2. 
DFinder2 allows for several techniques to be applied. The most efficient one is 
the Incremental Positive Mapping (IPM) technique \cite{DFinder2}. 
IPM requires a manual partitioning of the system to exploit its efficiency. 
We applied IPM on all structural partitions and we report on the best result which is consistent 
%(takes less time possibly for hardware related reasons) 
with the results reported in \citeN{DFinder2}. 

Table~\ref{bench:dining} shows the results. Both $\LAO$ and $\LLin$ outperform the best performance of DFinder2 by several orders of magnitude 
for $n\leq 3,000$. Both $\LAO$ and $\LLin$ successfully completed the deadlock freedom check for $3,000 \leq n \leq 10,000$ 
in less than one minute, where DFinder2 timed out (~1 Hour). The sole exception being that
$\LLin$ required $62$ seconds for $n=10,000$. 


Even though $\LLin$ is asymptotically more efficient than $\LAO$,
$\LAO$ outperforms $\LLin$ in all cases. This due to the following. 

\begin{itemize}
\item The largest subsystem that $\LAO$ had to consider was with depth $\l=1$. This corresponds to $18 = 2^1\times 3^2$ states regardless of $n$, the number of philosophers. 
\item The largest subsystem that $\LLin$ had to consider was with depth $\l=2$. This corresponds to $648 = 2^3 \times 3^4$ states regardless of $n$. 
\item For a given depth $\l$, \LLin is more efficient to compute than $\LAO$. 
 Since $\LAO$ performs a stronger check, it often terminates for smaller depths, which makes it
 more efficient than $\LLin$ in many cases.
\end{itemize}


\begin{table}
\tbl{Benchmarks: Dining Philosopher (seconds or minutes : seconds)}{
{\normalsize	
\begin{tabular}{| l | l | l | l |}
\hline
Size & \LAO & \LLin & D-Finder \\ \hline \hline
$1,000$ &         $0.46 $  &   $0.7 $       & $15$ \\ \hline
$2,000$ &          $1.4 $  &   $1.9 $       & $60$ \\ \hline
$3,000$ &          $2.9 $  &    $4$       & $2:41$ \\ \hline
$4,000$ &          $4.8 $  &    $7$        & $5:37$ \\ \hline
$5,000$ &          $8.3 $  &    $12$        & $12:38$ \\ \hline
$6,000$ &          $13.0 $ &    $17$         & $17:48$ \\ \hline
$7,000$ &          $17.2 $ &   $25$        & $30:18$ \\ \hline
$8,000$ &          $25.6 $ &   $34$        & $-$ \\ \hline
$9,000$ &          $34.1 $ &   $55$        & $-$ \\ \hline
$10,000$ &          $47 $  &   $62 $          & $-$ \\ \hline 
\end{tabular}
}}
\label{bench:dining}
\end{table}



\subsubsection{Resource allocation system case studies}

We evaluated \deadlocktool{} with a multi token-based resource allocation system. 
The system consists of $n$ clients, $m$ resources, $k$ tokens. 
The number of tokens specifies the maximum number of resources that
can be in use at a given time. 
The system allows to specify conflicting resources. 
Only one resource out of a set of conflicting resources can be in use at a given time.
For each set of conflicting resources, we create a resource manager.
Resource managers are connected in a ring where they pass tokens to neighboring resource managers or to resources. 

Given a configuration specifying $n$, $m$, $k$, a map of requests between clients and resources, and a set of sets of conflicting resources, 
we automatically generate a corresponding BIP model.

Figures~\ref{fig:client},
~\ref{fig:resource}, and
~\ref{fig:conflict-token}
show BIP atomic components for client, resource and manager components. 

The client in Figure~\ref{fig:client} requests resources $R_0$ and $R_2$ in sequence. It has $5$ ports. 
Ports $SR_0$ and $SR_2$ send requests for 
resources $R_0$ and $R_2$, respectively.
Ports $RG_0$ and $RG_2$ receive grants for 
resources $R_0$ and $R_2$, respectively.
Port $rel$ releases all resources. 
The behavior of the client depends on its request sequence. 

\begin{figure}[H]
\begin{center}
\includegraphics[scale=1.2]{compiledfigures/client-crop.pdf}
\caption{Client}
\label{fig:client}
\end{center}
\end{figure}

Figure~\ref{fig:resource} shows a resource component. 
A resource component waits for a request from a connected client on port $RR$. 
Once a request is received, the resource component transitions to a state where it is ready to 
receive a token from the corresponding resource manager using port $RTT$.
The resource transitions to a state where it grants the client request using port $STC$ and waits until it is released on port $done$. 
There, it returns the token back to the resource manager and transitions to the start state. 

\begin{figure}[H]
\begin{center}
\includegraphics[scale=1.2]{compiledfigures/resource-crop.pdf}
\caption{Resource}
\label{fig:resource}
\end{center}
\end{figure}

Figure~\ref{fig:conflict-token} shows a resource manager.
A resource manager $M$ has four states. 
\begin{itemize}
  \item State $T$ denotes that $M$ has a token. $M$ may send the token to either 
    (1) a resource on port $STR$ and transition to state $TwR$ (token with resource), or 
    (2) the next resource manager on port $STT$ and transition to state $N$ (no token).
  \item State $N$ denotes that $N$ has no token. 
    It may receive a token from a neighboring resource manager in the ring on port $RTT$ 
    and transition to state $T$. 
  \item State $TwR$ denotes that $M$ has already passed a token to one of its resources. 
    $M$ may either receive (1) the assigned token back from the resource using port $RTR$ and transition to state $T$, 
    or (2) another token from a neighboring manager using port $RTT$ and transition to state $TTwR$ (token and token with resource).
  \item State $TTwR$ denotes that $M$ has a token and has already passed a token to one of its resources. 
    In this state $M$ cannot send the token it has to a resource it manages to respect the conflict constraint. 
    $M$ may send the token to the next manager on port $STT$ and transition back to state $TwR$. 
\end{itemize}

\begin{figure}[H]
\begin{center}
\includegraphics[scale=1.2]{compiledfigures/token-crop.pdf}
\caption{Token Resource Manager}
\label{fig:conflict-token}
\end{center}
\end{figure}

The connections between a resource manager $M$ and its resources on ports $STR$ and $RTR$ specify that the 
resources are conflicting. 
A system should have at least $x$ resource managers where $x$ is the maximum between the number of sets of conflicting resources 
and $k$.
Note that $k$ resource managers start at state $T$ to denote the $k$ tokens; the rest start at state $N$. 

Figure \ref{fig:resourceallocation} shows a configuration system with $5$ clients and $5$ resources where:
\begin{itemize}
  \item Client $C_0$ requires resource $R_0$ then $R_2$,
  \item Client $C_1$ requires resource $R_2$ then $R_0$,
  \item Client $C_2$ requires resource $R_1$,
  \item Client $C_3$ requires resource $R_3$, and
  \item Client $C_4$ requires resource $R_4$.
\end{itemize}

The system has three resource managers to specify the conflicting resources. 
$RM_{01}$ manages conflicting resources $\{R_0,R_1\}$. 
$RM_{23}$ managers conflicting resources $\{R_2,R_3\}$.
$RM_{4}$ manages resource $R_4$. 

\begin{figure}[H]
\begin{center}
\includegraphics[scale=0.9]{compiledfigures/resourceallocation-crop.pdf}
\caption{Conflict-Resource Allocation System}
\label{fig:resourceallocation}
\end{center}
\end{figure}

We evaluated \deadlocktool{} with various configurations.
We highlight several lessons learned for specific systems as follows. 

\paragraph{Lesson 1} 
$\LAO$ verifies freedom from global and local deadlock where DFinder2 can only verify freedom from global deadlock.
Consider a system with $5$ clients, $3$ tokens, and $5$ resources.
Clients request resources $\langle 0, 2\rangle, \langle 2, 0\rangle, \langle 1 \rangle, \langle 3\rangle,$ and $\langle 4\rangle$, respectively.
Resource sets $\{ 0, 1\}, \{2,3\}$ are conflicting. 
This system is clearly global deadlock free. 
It has a local deadlock where client $C_0$ has resource $0$ and client $C_1$ has resource $2$. 
DFinder qualitatively can not detect such a local deadlock while $\LAO$ successfully does. 

\paragraph{Lesson 2} 
$\LAO$ is more complete than both $\LLin$ and DFinder2. For example, it can verify global and local deadlock freedom in cases where $\LLin$ fails. 
Consider a system with $5$ clients, $2$ tokens, and $5$ resources.
Clients request resources $\langle0, 2\rangle, \langle 0, 2\rangle, \langle 1 \rangle, \langle 3\rangle,$ and $\langle 4\rangle$, respectively.
Resource sets $\{ 0, 1\}, \{2,3,4\}$ are conflicting. 
This system is global and local deadlock free. 
Both DFinder2 and $\LLin$ report that the system might contain a deadlock. 
$\LAO$ successfully reports that the system is both global and local deadlock free. 

%PAUL: I removed this para since the given example is not a true conspiracy
% will discuss further, and consider for further work
%% \paragraph{Lesson 3:}
%% Our work can be extended to detect conspiracies \cite{AFG93}.  For
%% example, consider a system with $5$ clients, $2$ tokens, and $5$
%% resources.  Clients request resources $\langle 0, 1\rangle, \langle 1,
%% 0\rangle, \langle 2 \rangle, \langle 3\rangle,$ and $\langle
%% 4\rangle$, respectively.  Resource sets $\{ 0, 1\}, \{2,3,4\}$ are
%% conflicting.  Client $C_0$ may block forever in case it acquires
%% resource $0$ because resource $0$ is conflicting with resource $1$.
%% However, it is not possible to find a deadlocked subsystem containing
%% $C_0$ and resources $0$ and $1$ since that will also have to include
%% the resource manager $M_{01}$ managing conflicting resources $0$ and
%% $1$.  The latter can always exchange the second token with the
%% neighboring resource managers.
%% %
%% An extension of our work that consider subsystem boundaries at ports
%% and abstracts port enablement conditions with free Boolean variables
%% can help detect such scenarios.

\begin{table}
\tbl{Benchmarks: Time required for $\LAO$ on the resource allocation system}{
{\normalsize
\begin{tabular}{| l | 
  c |  c | c | c | 
  c |  c | c | c | 
  c |  c | c |  }
\hline

Size & $10$ & $12$ & $14$ & $16$ & $18$ & $20$ & $22$ & $24$ & $26$ & $28$ & $30$ \\ \hline
Time (sec) & $148 $ & $169 $ & $189 $ & $230 $ & $254 $ & $277 $ & $298 $ & $318 $ & $351 $ & $374 $ & $430 $ \\ \hline 

\end{tabular}}}
\label{bench:resourceallocation}
\end{table}

\paragraph{Benchmarking:}

We evaluated the performance of $\LAO$ on a deadlock free system with the following configuration. 
\begin{itemize}
\item $n$ clients each with $3$ states, $n$ resources each with $5$ states, and $n$ tokens,
\item Client $C_i, 0\leq i < n$ requests resource $i$, and 
\item No resources are in conflict, hence we have $n$ resource managers each with $4$ states. 
\end{itemize}

The system has a total of $4^n \times 3^n \times 5^n$ states. 
DFinder2 timed out within seven hours for $n=10$. 
%Try different combinations of partitions
$\LLin$ had to increase the subsystem up to the whole system and also timed out within seven hours for $n=10$. 
$\LAO$ was able to verify deadlock freedom. It has to check subsystems with $12$ components out of $3\times n$ components regardless of $n$. 
This resulted from inspecting subsystems corresponding to a depth $\l=2$ with $\leq 23,040,000=4^{6} \times 3^2\times 5^4$ states regardless of $n$.
The numbers in Table~\ref{bench:resourceallocation} show a
\emph{linear increase in time} required to check deadlock freedom 
using $\LAO$ with respect to $n$. This indicates that the number of subsystems to check is proportional to $n$. 

Our resource allocation system subsumes the token based Milner scheduler~\cite{milner} which 
is essentially a token ring with precisely one token present~\cite{AGR16}. 

%The technique presented in ~\cite{AGR16} fails to prove deadlock freedom for Milner Scheduler 
%PAUL: commented this out since we repeat it in the related work section
%\cite{AGR16} present a technique that fails to prove deadlock freedom for Milner Scheduler 
%because it requires a large subset of the system, 
%while $\LAO$ succeeds. 




%%%%%%%%%%%%%%%%%%%%%%%%%%
\subsection{The supercycle formation condition}

We use the structural properties of supercycles (\secn{supercycle-structural}) and the 
dynamics of wait-for graphs (\prop{wait-for-edge-preservation}) to define a condition that 
must hold whenever a supercycle is created. Negating this condition then implies the absence of
supercycles. 


\bp[Supercycle formation condition] \label{prop:supercycle-formation}
Assume that $s \goesto[\act] t$ is a transition of $(\B, Q_0)$, $\wfg{\B}{s}$ is supercycle-free, and that $\wfg{\B}{t}$
contains a supercycle.  Then, in $\wfg{\B}{t}$, there exists a $\CC$ such that
\bn
\item $\CC$ is a subgraph of $\wfg{\B}{t}$
\item $\CC$ is strongly connected
\item $\CC$ is a supercycle
\item  in $\wfg{\B}{t}$, there is no wait-for edge from a node in $\CC$ to a node outside of $\CC$
\item there exists a component $\B_i \in \cmps{\act}$ such that $\B_i$ is in $\CC$
\en
\ep
%
\bpr
By assumption, there is a supercycle $\SC$ that is a subgraph of $\wfg{\B}{t}$.
By Proposition~\ref{prop:supercycle:contains-mssc}, $\SC$ contains a
subgraph $\CC$ that is strongly connected, is itself a supercycle, and
such that there is no wait-for-edge from a node in $\CC$ to a node outside of $\CC$.
This establishes Clauses~1--4.

Now suppose $\B_i \not\in \CC$ for every $\B_i \in \cmps{\act}$. Then, no edge in $\CC$ is
$\B_i$-incident.  Hence, by Proposition~\ref{prop:wait-for-edge-preservation}, every edge in $\CC$
is an edge in $\wfg{\B}{s}$. Hence $\CC$ is a subgraph of $\wfg{\B}{s}$.
%
Now let $v$ be an arbitrary node in $\CC$.
%
Suppose $v$ is a component $\B_j$.  By assumption, $\B_j \not\in \cmps{\act}$, and so
$s \pj \B_j = t \pj \B_j$ by Definition~\ref{def.bip.composition}. Hence $\B_j$ enables the same set
of interactions in state $s$ as in state $t$. Also, in $\wfg{\B}{t}$, all of $\B_j$'s wait-for edges
must end in an interaction that is in $\CC$, since $\CC$ is a supercycle in $\wfg{\B}{t}$. Hence the
same holds in $\wfg{\B}{s}$.
%
If $v$ is an interaction, it must also have a wait-for-edge $e'$ to some component $\B_j \in \CC$,
since $\CC$ is a supercycle in $\wfg{\B}{t}$. Hence this also holds in $\wfg{\B}{s}$.
%
Hence $v$ has enough successors in $\CC$ to satisfy the supercycle definition (\defn{supercycle}).
%
We conclude that $\CC$ by itself is a supercycle in $\wfg{\B}{s}$, which contradicts the assumption
that $\wfg{\B}{s}$ is supercycle-free. Hence, $\B_i \in \CC$ for some $\B_i \in \cmps{\act}$, and so
Clause~5 is established.  
\epr



%\subsection{Violations of the supercycle formation condition}

\subsection{General supercycle violation condition}

We use \prop{supercycle-formation} to formulate a condition that prevents the formation of
supercycles. 
For transition $s \goesto[\act] t$, we determine for every component $\B_i \in \cmps{\act}$ whether
it is possible for $\B_i$ to be a node in a strongly-connected supercycle $\CC$ in $\wfg{\B}{t}$. 
%If not, then we say that $\B_i$ satisfies the \emph{supercycle violation} condition, and we
%formalize this condition below.
There are two ways for $\B_i$ to not be a node in a strongly-connected supercycle:
\bn
\item \textit{no supercycle membership}: $\B_i$ is not a node of any supercycle, \ie $\neg \scyc{\B}{s}{\B_i}$.

\item \textit{no strong-connectedness}: $\B_i$ is a node in a supercycle, but not a node in a \emph{strongly-connected} supercycle. 

\en

%\subsection{Strong connectedness condition}
% commentary: strong connectedness violation cannot be defined "locally" in a good way, like supercycle violation
% can, since it depends on reachability, which is a global property. For a supercycle, once the set
% of nodes is fixed, the required wait-for relations are all local properties. Hence, violation is
% also a local property. For strong connectedness, the best local property is "no incoming" or "no
% outgoing", which is just the linear condition of the FORTE 2013 paper. Hence a local condition for
% strong connectedness violation is no improvement over FORTE 2013.

%Given that $\B_i$ is a node in a supercycle, we wish to determine whether or not it is a node in a
%\emph{strongly-connected} supercycle. We do this by removing all nodes with supercycle-violations,
%and then finding the maximal strongly connected components of the resulting wait-for subgraph.

We formalize the second condition as follows.

\bd[Strong connectedness violation, $\connViol{v}{t}$]
\label{def:sConn.violation}
 Let $v$ be a node of $\wfg{\B}{t}$.   Then $\connViol{v}{t}$ holds iff there does not exist a 
strongly connected
 supercycle $SSC$ such that $v \in SSC$ and $SSC \sub \wfg{B}{t}$.
\ed






The general supercycle violation condition is then a disjunction of the supercycle violation condition
and the strong connectedness violation conditions.


\bd[General supercycle violation, $\genViol{v}{t}$]
\label{def:formation.violation} 
\label{defn:formation.violation} 
Let $v$ be a node of $\wfg{\B}{t}$.
Then 
$\genViol{v}{t}  \df \scV{v}{t}  \lor \connViol{v}{t}$.
%$\genViol{v}{t}  \df (\exs d \ge 1: \scViol{v}{d}{t}) \lor \connViol{v}{t}$.
\ed
%
Let $s \goesto[\act] t$ be a reachable transition. If, for every $\B_i \in \cmps{\act}$,
$\formViol{v}{t}$ holds, then, as we show below, $s \goesto[\act] t$ does not introduce a
supercycle, \ie if $s$ is supercycle-free, then so is $t$.  However, evaluating this condition over
all global transitions is subject to state explosion, and so we formulate below a ``local'' version
of the general condition, which can be evaluated in ``small subsystems'', and so we often avoid
state-explosion. Hence the advantage of the local versions is that they are usually efficiently
computable, as we show in the sequel.  We also formulate a ``linear'' condition (both global and
local), which is simpler (but ``more incomplete'') than the general condition, and so is easier to
evaluate.

We remark that, as shown above $\scV{v}{t}$ implies that $v$ cannot be in a
supercycle. Hence, $v$ cannot be in a strongly-connected supercycle.  Hence
$\scV{v}{t}$ implies $\connViol{v}{t}$. It is however convenient to state the
formation violation condition in this manner, since we will formulate a local version
for each of $\scV{v}{t}$ and $\connViol{v}{t}$, and the implication does not
necessarily hold for the local versions. 

We therefore now have four deadlock-freedom conditions: global general, local general, 
global linear, and local linear, which are all concrete instances of
the abstract version of 
the deadlock-freedom condition given in \secn{abstract-scfree-conditions}.


%%%%%%%%%%%%%%%%%%%%%%%%%%%%%%%
%\subsection{Overview of the four supercycle-freedom preserving restrictions}

The supercycle formation condition
(Proposition~\ref{prop:supercycle-formation}) tells us that, when a
supercycle $\SC$ is created, some component $\B_i$ that participates
in the interaction $\act$ whose execution created $\SC$, must be a
node of a strongly connected component $\CC$ of $\SC$, and moreover
$\CC$ is itself a supercycle in its own right. In a sense, $CC$ is the
``essential'' part of $\SC$.

Hence, for a BIP system $(\B, Q_0)$, our fundamental condition for the
prevention of supercycles is that for every reachable transition
$s \goesto[\act] t$ resulting from execution of $\act$, every
component $\B_i$ of $\act$ must exhibit a supercycle-violation
(Definition~\ref{def:supercycle.violation}) in state $t$ (the state
resulting from the execution of $a$). For a given BIP system
$(\B, Q_0)$ and interaction $\act$, we denote that condition
$\GAO(\B, Q_0, \act)$, and define it formally below.  This condition
is, in a sense, the ``most general'' condition for supercycle-freedom.

If $\GAO(\B, Q_0, \act)$ holds, and global state $s$ is
supercycle-free, and $s \goesto[\act] t$, then it follows (as we
establish below) that global state $t$ is also supercycle-free.  So,
by requiring (1) that all initial states are supercycle-free, and (2)
that $\GAO(\B, Q_0, \act)$ holds for all interactions
$\act \in \gamma$, we obtain, by straightforward induction on length
of executions, that every reachable state is supercycle-free.

It also follows that any condition which implies $\GAO(\B, Q_0, \act)$ is also sufficient to guarantee  supercycle-freedom, and
hence deadlock-freedom. We exploit this in two ways:
\bn

\item To provide a ``linear'' condition, $\GLin$, that is easier to evaluate than $\GAO$, since it requires only the
evaluation of lengths of wait-for-paths, \ie it does not have the ``alternating'' character of $\GAO$. 

\item To provide ``local variants'' of $\GAO$ and $\GLin$,  which can often be
evaluated in small subsystems of $(\B, Q_0)$, thereby avoiding state-explosion. The local conditions imply the
corresponding global ones, \ie they are sufficient but not necessary for deadlock-freedom.

\en






%%%%%%%%%%%%%%%%%%%%%%%%%%%%%%%%%%%%%%%%%%%%%%%%%%%%%%%%%%%%%%%%%%%%%%%%%%%%%%
   \subsection{A Global AND-OR Condition for Deadlock Freedom}
   \label{s:global.ANDOR}
   Our first global condition is the most general possible: simply assert that, after execution of interaction $\act$, some
$\B_i \in \cmps{\act}$ exhibits a supercycle-violation, as given by $\viol{\B_i}{d}{t}$
(Definition~\ref{def:supercycle.violation}). 


\bd[$\GAO(\B, Q_0, \act)$] \label{def:global.ANDOR-cond}
Let $s \goesto[\act] t$ be a reachable transition of $(\B, Q_0)$.
Then, in $t$, the following holds. 
For every component $\B_i \in \comps{\act}$, %$\B_i$ has a level-$d$ $SC$-violation for some $d$.
the formation violation condition holds.
Formally,\\
\ind  $\fas \B_i \in \comps{\act}, \formViol{\B_i}{t}$.
%\ind  $\fas \B_i \in \comps{\act}, \exs d \ge 1: \viol{\B_i}{d}{t}$.
\ed
We now show that $\GAO$ is supercycle-freedom preserving.


\bt \label{thm:GAO.SC-free-preserving}
$\GAO$ is supercycle-freedom preserving.
\et
\prf{
We must establish:
for every reachable transition $s \la{\act} t$,
$\wfg{\B}{s}$ is supercycle-free implies that $\wfg{\B}{t}$ is
supercycle-free. Our proof is by contradiction, so we assume the existence of a reachable transition
$s \la{\act} t$ such that $\wfg{\B}{s}$ is supercycle-free and $\wfg{\B}{t}$ contains a supercycle.

By Proposition~\ref{prop:supercycle-formation}
 there exists a component $\B_i \in \cmps{\act}$ such that $\B_i$ is in $\CC$, where 
$\CC$ is a strongly connected supercycle that is a subgraph of $\wfg{\B}{t}$.

Since $\CC$ is a strongly connected supercycle, we have,
 by Definition~\ref{def:sConn.violation}, that $\neg \connViol{\B_i}{t}$ holds.

Since $\CC$ is a supercycle, we have, by Proposition~\ref{prop:scViol-iff-notInSC}, 
that $\neg (\exs d \ge 1: \scViol{\B_i}{d}{t})$ holds.

Hence, by Definition~\ref{def:formation.violation}, $\neg \formViol{\B_i}{t}$
But, by Definition~\ref{def:global.ANDOR-cond}, we have $\formViol{\B_i}{t}$.
Hence, we have the desired contradiction, and so the theorem holds.
}







%%%%%%%%%%%%%%%%%%%%%%%%%%%%%%%%%%%%%%%%%%%%%%%%%%%%%%%%%%%%%%%%%%%%%%%%%%%%%%
   \subsection{A Global Linear Condition for Deadlock Freedom}
   \label{s:globCondition}
   \label{s:global.Linear}
   In some cases, a simpler condition suffices to guarantee deadlock-freedom. This simpler condition is ``linear'', \ie it lacks the AND-OR alternation
aspect of $\GAO$. After execution of a reachable transition $s \la{\act} t$ of $(\B, Q_0)$, 
we consider the in-depth and out-depth of the components $\B_i \in \cmps{\act}$. There are three cases:
%
\begin{itemize}

\item \emph{Case 1} \label{case:finite-in}
$\B_i$ has finite in-depth in $\wfg{\B}{t}$: then, if $\B_i \in
\SC$, it can be removed and still leave a supercycle $\SC'$, by
Proposition~\ref{prop:supercycle:essential-subgraph-of}. Hence 
$\SC'$ exists in $\wfg{\B}{s}$, and so 
$\B_i$ is not essential to the creation of a supercycle.

\item \emph{Case 2} \label{case:finite-out}
$\B_i$ has finite out-depth in $\wfg{\B}{t}$: 
by Proposition~\ref{prop:supercycle:no-finite-outdegree}, 
$\B_i$ cannot be part of a supercycle, and so $\SC \sub \wfg{\B}{s}$.

\item \emph{Case 3} \label{case:infinite-both}
$\B_i$ has infinite in-depth and infinite out-depth in
$\wfg{\B}{t}$: in this case, $\B_i$ is possibly an essential part of
$\SC$, \ie $\SC$ was created in going from $s$ to $t$.

\end{itemize}
We thus impose a condition which guarantees that only 
Case~1 %~\ref{case:finite-in} 
or Case~2 %\ref{case:finite-out}
occur. 



\begin{definition}[$\GLin(B,Q_0,\act)$] \label{def:global:dfc}
%Let $(B, Q_0)$ be a BIP system, with $B =\gamma(\B_1,\dots,\B_n)$, and 
%$a$ an interaction of $(B, Q_0)$, \ie $a \in \gamma$.
$\GLin(B,Q_0,\act)$ holds iff, for every reachable transition $s \goesto[\act] t$ of BIP-system $(\B, Q_0)$, 
the following holds in state $t$:\\
  $$\fa \B_i \in \cmps{a}: \widepth{\B}{\B_i}{t} < \omega \lor \wodepth{\B}{\B_i}{t} < \omega.$$
That is, for every component $\B_i$ of $\cmps{\act}$:  either $\B_i$ has finite in-depth, or finite out-depth, in $\wfg{\B}{t}$.
\end{definition}



\begin{proposition} \label{prop:indepth-finite-implies-scViol}
Assume that node $v$ of $\wfg{\B}{t}$ has a finite in-depth of $d$ in $\wfg{\B}{t}$, \ie 
$\widepth{\B}{v}{t} = d$. Then $\connViol{v}{t}$.
\end{proposition}
%
\begin{proof}
A node with finite in-depth cannot be in a wait-for-cycle, and
therefore cannot be in a strongly connected supercycle.
\end{proof}


% \prf{
% Proof is by induction on $d$.

% \noindent
% \ul{Base case, $d=1$.} Hence by Definitions~\ref{def:path} and \ref{def:depth},  
% $v$ has no incoming wait-for-edges in $\wfg{\B}{t}$. Hence by Definition~\ref{def:supercycle.violation},
% $\viol{v}{1}{t}$.

% \noindent
% \ul{Inductive step, $d > 1$.}
% By Definition~\ref{def:path} and \ref{def:depth}, all predecessors (\ie nodes $u$ such that $u \ar v \in \wfg{\B}{t}$)
% of $v$ have an in-depth of at most $d-1$. Hence, by the induction hypothesis applied to $d-1$, we obtain 
% $\viol{u}{d-1}{t}$ for all predecessors $u$ of $v$. Hence by Definition~\ref{def:supercycle.violation}, 
% Clauses~\ref{def:supercycle.violation.component.in} and \ref{def:supercycle.violation.interaction.in},
% $\viol{v}{1}{t}$.
% }




\begin{proposition} \label{prop:outdepth-finite-implies-scViol}
Assume that node $v$ of $\wfg{\B}{t}$ has a finite out-depth of $d$ in $\wfg{\B}{t}$, \ie 
$\wodepth{\B}{v}{t} = d$. Then $\viol{v}{d}{t}$.  % USED TO BE d+1. CHECK
\end{proposition}
%
\begin{proof}
Proof is by induction on $d$.\\

\noindent
\ul{Base case, $d=0$.} Hence by $\wodepth{\B}{v}{t} = 0$ and Definitions~\ref{def:path} and \ref{def:depth},  
$v$ has no outgoing wait-for-edges in $\wfg{\B}{t}$. Hence $\neg \blocks{v}{\wfg{\B}{t}}$, \ie $v$ is not blocked by the entire set of nodes in 
$\wfg{\B}{t}$. Hence $\neg \blocks{v}{\compl{\ewfg}}$, since $\wfg{\B}{t} = \compl{\ewfg}$. So by \defn{violFix}, 
$v \in \VF{\ewfg}$. By \defn{supercycle.violation}, $\scVd{v}{0}{s}$.\\

%OLD Hence by Definition~\ref{def:supercycle.violation}, $\viol{v}{1}{t}$.

\noindent
\ul{Inductive step, $d > 0$.}
Let $w$ be an arbitrary successor of $v$, \ie a node $w$ such that $v \ar w \in \wfg{\B}{t}$.
By Definitions~\ref{def:path} and \ref{def:depth}, $w$ has an out-depth $d'$ that is less than $d$. 
That is, $\wodepth{\B}{u}{t} = d' < d$.
By the induction hypothesis applied to $d'$, we obtain $\viol{w}{d'}{t}$, and so $w \in \VFs^{d'} (\ewfg)$ by \defn{supercycle.violation}.
Hence $w \in \VFs^{d-1} (\ewfg)$, since, by monotonicity of $\VFs$, we have $\VFs^{d'} (\ewfg) \subg \VFs^{d}$ when $d' \le d$.
Since  $w$ is an arbitrary successor of $v$, it follows that $v$ is only blocked by nodes in $\VFs^{d-1} (\ewfg)$.
Hence $\neg \blocks{v}{\compl{\VFs^{d-1} (\ewfg)}}$. 
By \defn{violFix}, $v \in \VF{\VFs^{d-1} (\ewfg)}$, \ie $v \in \VFs^{d} (\ewfg)$.
By \defn{supercycle.violation}, $\scVd{v}{d}{s}$.
% OLD Hence by Definition~\ref{def:supercycle.violation}, Clauses~\ref{def:supercycle.violation.component.out} and \ref{def:supercycle.violation.interaction.out}, $\viol{v}{d+1}{t}$.
\end{proof}




\begin{lemma} \label{lemma:glob-lin-implies-globANDOR} \label{GLinGAO}
$\fa \act \in \gamma: \GLin(\B, Q_0, \act) \imp \GAO(\B, Q_0, \act)$.  
\end{lemma}
%
\begin{proof}
Assume, for arbitrary $\act \in \gamma$, that $\GLin(\B, Q_0, \act)$ holds. That is, 
 
\ind For every reachable transition $s \goesto[\act] t$ of $(\B, Q_0)$,\\
\ind \ind  $\fas \B_i \in \cmps{\act}: \widepth{\B}{\B_i}{t} < \omega \lor \wodepth{\B}{\B_i}{t} < \omega$.

\noindent
By Propositions~\ref{prop:indepth-finite-implies-scViol} and \ref{prop:outdepth-finite-implies-scViol}, 

\ind For every reachable transition $s \goesto[\act] t$ of $(\B, Q_0)$,\\
\ind \ind  $\fas \B_i \in \cmps{\act}:  \connViol{\B_i}{t} \lor (\ex d \ge 1: \viol{\B_i}{d}{t})$.

\noindent
Hence by Definition~\ref{def:formation.violation}, \\
\ind For every reachable transition $s \goesto[\act] t$ of $(\B, Q_0)$,\\
\ind \ind  $\fas \B_i \in \cmps{\act} : \formViol{\B_i}{t}$

\noindent
Hence $\GAO(\B, Q_0, \act)$ holds.
\end{proof}


\begin{theorem} \label{thm:GLin.SC-free-preserving}
$\GLin$ is supercycle-freedom preserving
\end{theorem}
%
\begin{proof}
Follows immediately from Lemma~\ref{lemma:glob-lin-implies-globANDOR} and Theorem~\ref{thm:GAO.SC-free-preserving}.
\end{proof}


%Since $\GLin(\act)$ implies $\GAO(\act)$, and $(\fa \act \in \gamma: \GAO(\act))$ implies deadlock-freedom (assuming also that initial
%state are deadlock-free), we obtain:



%%%%%%%%%%%%%%%%%%%%%%%%%%%%%%%%%%%%%%%%%%%%%%%%%%%%%%%%%
\subsection{Deadlock freedom using global restrictions}


\begin{corollary}[Deadlock-freedom via $\GAO, \GLin$] 
\label{theorem:global.deadlock-free}
Assume that
\bn
\item \label{theorem:global.deadlock-free.initial}
      for all $s_0 \in Q_0$, $\wfg{\B}{s_0}$ is supercycle-free, and
\item \label{theorem:global.deadlock-free.scfPres}
      for all interactions $\act$ of $\B$ (\ie $\act \in \gamma$):  $\GAO(\B, Q_0, \act) \lor \GLin(\B, Q_0, \act)$ holds.
\en
Then for every reachable state $u$ of $(\B, Q_0)$:  $\wfg{\B}{u}$ is supercycle-free, and so 
$(\B, Q_0)$ is free of local deadlock.
\end{corollary}
%
\begin{proof}
Immediate from Theorems~\ref{thm:GAO.SC-free-preserving}, \ref{thm:GLin.SC-free-preserving} and Corollary~\ref{cor:SC-free-preserving.deadlock-free}.
\end{proof}



%% \bt[Deadlock-freedom via \GLin] 
%% \label{theorem:global:deadlock-free}
%% \label{theorem:global.linear.deadlock-free}
%% %Let $(B, Q_0)$ be a BIP system, with $B = \gamma(\B_1,\dots,\B_n)$.
%% Assume that
%% \bn
%% \item \label{theorem:global.linear.deadlock-free.initial} 
%%       for all $s_0 \in Q_0$, $\wfg{\B}{s_0}$ is supercycle-free, and
%% \item \label{theorem:global.linear.deadlock-free.globalLinear}
%%       for all interactions $\act$ of $B$ (\ie $\act \in \gamma$), $\GLin(\act)$ holds.
%% \en
%% Then for every reachable state $u$ of $(\B, Q_0)$:  $\wfg{\B}{u}$ is supercycle-free.
%% \et
%% %
%% \prf{
%% By Assumption~(\ref{theorem:global.linear.deadlock-free.globalLinear}) and 
%% Proposition~\ref{prop:glob-lin-implies-globANDOR}, we have 
%% $\fas \act \in \gamma: \GAO(\act)$. By this and
%% Assumption~(\ref{theorem:global.linear.deadlock-free.initial}), we obtain that the 
%% assumptions of Theorem~\ref{theorem:global.ANDOR.deadlock-free} hold. Hence, by applying 
%% Theorem~\ref{theorem:global.ANDOR.deadlock-free}, we obtain the conclusion

%% for every reachable state $u$ of $(\B, Q_0)$:  $\wfg{\B}{u}$ is supercycle-free.

%% and we are done.
%% }






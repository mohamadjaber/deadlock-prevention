%\section{BIP - Behavior Interaction Priority}

BIP is a component framework for constructing systems by superposing three layers of modeling: Behavior, Interaction,
and Priority.
%
A technical treatment of priority is beyond the scope of this paper. Adding priorities never introduces a deadlock,
since priority enforces a choice between possible transitions from a state, and deadlock-freedom means that there is at
least one transition from every (reachable) state.  Hence if a BIP system without priorities is deadlock-free, then the
same system with priorities added will also be deadlock-free.

\begin{definition}[Atomic Component]
An  {\em atomic component} $\B_i$ is a labeled transition system represented by a triple
$(Q_i, P_i, \goesto_i)$ where $Q_i$ is a set of {\em states}, $P_i$ is a set of {\em communication ports}, and
$\goesto_i\, \subseteq Q_i \times P_i \times Q_i$ is a set of {\em possible transitions}, each labeled by some port.
\end{definition}

\redbox{
For states $s_i, t_i \in Q_i$ and port $p_i \in P_i$, write $s_i  \goesto[p_i]_i t_i$, iff
$(s_i,p_i,t_i) \in\,\goesto_i$. When $p_i$ is clear from the context or not needed, we drop it from the transition and write $s_i \goesto_i t_i$. Similarly, $s_i \goesto[p_i]_i$ means that there
exists $t_i \in Q_i$ such that
$s_i \goesto[p_i]_i t_i$. In this case, $p_i$ is \emph{enabled} in state $s_i$.
Ports are used for communication between different components, as
discussed below.}

Figure~\ref{fig:components} shows atomic components for a philosopher $\Phl_i$ and a fork $F_i$ in dining philosophers.
%
A philosopher $\Phl_i$ that is hungry (in state $h_i$) can eat by executing $\get_i$ (to get its forks) and moving to state $e_i$ (eating). From $e_i$,
$\Phl_i$ releases its forks by executing $\putt_i$ (to put down its forks) and moving back to $h_i$.
%
Adding the thinking state does not change the deadlock behaviour of the system, since the thinking to hungry transition is internal to $\Phl_i$, and so we
omit it.
%
A fork $F_i$ is taken by either: (1) the left philosopher (transition $\use_i^\l$) and so moves to state $u_i^\l$ (used by left philosopher), or (2) the
right philosopher (transition $\use_i^r$) and so moves to state $u_i^r$ (used by right philosopher). From state $u_i^r$ (resp. $u_i^\l$), $F_i$ is released by
the right philosopher (resp. left philosopher) and so moves back to state $f_i$ (free).
%


\redbox{
In practice, we describe the transition system using some syntax, \eg involving local variables (BIP does not have shared variables).  We abstract
away from issues of syntactic description since we are only interested in enablement of ports and actions.
%
In BIP, the enablement of a port depends only on the local state of a component.  In particular, it cannot depend on the state of other components.
For example, state $e_i$ in atomic component $\Phl_i$ of Figure~\ref{fig:components} enables port $\putt_i$ as there exists a transition from
  $e_i$ labeled with port $\putt_i$.
%This assumption is a restriction on BIP, and we defer to subsequent work how to lift this restriction.  
Hence, there exists a predicate $\gd{p_i}{i}$ that holds in state $s_i$ of component $\B_i$ iff port $p_i$ is enabled in $s_i$, \ie
$s_i(\gd{p_i}{i}) = \true \mbox{ iff } s_i \goesto[p_i]_i$.
}

\begin{definition}[Interaction] For a given system built from a set of $n$ atomic components 
$\{\B_i = (Q_i, P_i, \goesto_i)\}_{i=1}^n$, we require that their respective sets of ports are
pairwise disjoint,
\ie for all $i, j$ such that $i, j \in \oneton \land i \ne j$, we have $P_i \ints P_j = \emptyset$.
An {\em interaction} is a set of ports not containing two or more ports from the same component.
That is, for an interaction $\act$ we have 
$\act \subseteq P \land (\fa i \in \set{1..n}: |\act \ints P_i| \le 1)$, where 
$P = \bigcup_{i=1}^n P_i$ is the set of all ports in the
system. When we write $\act = \{p_i\}_{i \in I}$, we assume that
$p_i \in P_i$ for all $i \in I$, where $I \subseteq \set{1..n}$.
\end{definition}


\redbox{The Connectors that connect ports in Figure~\ref{fig:diningbip} 
illustrate interactions. 
For example, the interaction $\Grab_0= \set{\get_0, \use_0^\l, \use_1^r}$ connects ports $\get_0$, $\use_0^\l$, and $\use_1^r$
from components $\Phl_0$, $F_0$, and $F_1$ respectively, and corresponds to 
philosopher component $\Phl_0$ acquiring both forks and moving to it's eating state. }
%
%\redbox{%For example, interaction $a_{001}=\{\get_0, \use_{l0}, \use_{r1}\}$ and 
%involves components $\cmps{a_{001}}=\{P_0, F_0, F_1\}$.}

%
Execution of an interaction
$\act = \{p_i\}_{i \in I}$
involves all the components which have
ports in $\act$.  
We denote by $\cmps{\act}$ the
set of atomic components participating in $\act$. Formally,
$\cmps{\act} = \{\B_i \st p_i \in \act\}$. 

\begin{figure}[t]
  \begin{center}
    \mbox{
      \subfigure[Philosopher $\Phl_i$ and fork $F_i$ atomic components.]{\label{fig:components}\scalebox{0.4}{\input{figs/philAndFork.pdf_t}}} \quad
      \subfigure[Dining philosophers composite component with four philosophers.]{\label{fig:diningbip}\scalebox{0.35}{\input{figs/diningbip.pdf_t}}}
      }
    \caption{Dining philosophers.}
    \label{fig:diningSpectrum}
  \end{center}
\end{figure}






\begin{definition}[Composite Component]\label{def.bip.composition} A {\em composite
  component} (or simply {\em component}) 
 $\B \df \gamma(\B_1,\dots,\B_n)$
is defined by a composition
operator parameterized by a set of interactions $\gamma \subseteq
2^P$.  $\B$ has a transition system
$(Q,\gamma, \goestog)$, where %$Q = \bigotimes_{i=1}^n Q_i$ and
$Q = Q_1 \times \cdots \times Q_n$ and
$\goestog \sub Q \times \gamma \times Q$ is the least set of transitions satisfying the rule
%
\begin{mathpar}
\inferrule
{
    \act = \{p_i\}_{i\in I}\in \gamma\\
    \forall i\in I: s_i \goesto[p_i]_i t_i\\
    \forall i\not\in I: s_i = t_i
}
{
    \tpl{s_1,\dots,s_n} \goestog[\act] \tpl{t_1,\dots,t_n}
}
\end{mathpar}
\end{definition}
%
This inference rule says that a composite component $\B = \gamma(\B_1,\dots,\B_n)$ can
execute an interaction $\act \in \gamma$, iff for each port $p_i \in \act$, the
corresponding atomic component $\B_i$ can execute a transition labeled with
$p_i$; the states of components that do not participate in the interaction stay
unchanged. 
\redbox{
Note that interactions are the only means of inter-component communication and synchronization in BIP. 
}


%
%Given an interaction $\act = \{p_i\}_{i \in I}$, we denote by $\cmps{\act}$ the
%set of atomic components participating in $\act$, formally:
%$\cmps{\act} = \{\B_i \st p_i \in \act\}$. 
%Given a component $\B_i$, we denote by $A_i$ the
%interactions that $\B_i$ participates in, formally: $A_i = \set{a \st P_i \ints a \ne \emptyset}$.
\begin{example}[Composite Component]
Figure \ref{fig:diningbip} shows a composite component consisting of
four philosophers and the four forks between them. Each philosopher $\Phl_i\ (0 \le i \le 3)$
and its two neighboring forks share two interactions: 
$\Grab_i = \set{\get_i, \use_i^\l, \use_{i+1}^r}$ in which the philosopher obtains the forks, and 
$\Rel_i = \set{\putt_i, \free_i^\l, \free_{i+1}^r}$ in which the philosopher releases the forks
(addition of indices being modulo 4).
\end{example}


\begin{definition}[Interaction enablement]\label{def.bip.enablement} 
An atomic component $\B_i = (Q_i,P_i,\goesto_i)$ enables a port $p_i \in P_i$ in state $s_i$ iff $s_i \goesto[p_i]_i$.
$\B_i$ enables interaction $\act$ in state $s_i$ iff $s_i \goesto[p_i]_i$, where $\set{p_i} = P_i \ints \act$ is the port of $\B_i$ involved in $\act$.
That is, $\B_i$ enables $\act$ in state $s_i$ iff $\B_i$ enables port $\act \ints P_i$ in state $s_i$. 

Recall that $\gd{p_i}{i}$ denotes the enablement condition for port $p_i$ in component $\B_i$, that is, $\gd{p_i}{i}$ holds iff
$s_i \goesto[p_i]_i$, where $s_i$ is the current state of $\B_i$.
Let $\gd{\act}{i}$ denote the enablement condition for interaction $\act$ in
component $\B_i$, that is,  $\gd{\act}{i} = \gd{p_i}{i}$ where $\set{p_i} = \act \ints P_i$.  
%$\gd{a}{i}$ holds iff the current state of $\B_i$ is an $s_i$ such that $s_i \goesto[p_i]_i$.

Let $\B = \gamma(\B_1,\dots,\B_n)$ be a composite component, let
$s = \tpl{s_1,\ldots,s_n}$ be a state of $\B$,
and let $\act = \{p_i\}_{i\in I}$ be an interaction of $\B$.
Then $\B$ enables $\act$ in $s$
iff every $\B_i \in \cmps{\act}$ enables $\act$ in $s_i$, \ie iff
$\AND_{i \in I} \gd{p_i}{i}$ holds in $s$.
\end{definition}
%
The definition of  interaction enablement is a consequence of 
Definition~\ref{def.bip.composition}. 
Interaction $\act$ being enabled in state $s$ means that executing
$\act$ is one of the possible transitions that can be taken from $s$.

To avoid pathological cases of deadlock due solely to a single component refusing to enable any interaction at all, 
we assume that every component always enables at least one interaction.
Structurally, this means that there is no local state with zero transitions, 
and every port labeling a transition is 
part of at least one interaction. 
\redbox{
Intuitively, component $\B_i$ in state $s$ must enable at least one interaction $a$. 
However, $a$ requires enablement from all components involved
in it to execute. That might not be the case as $a$ may be blocked by another 
component. Therefore the assumption is not enough to guarantee deadlock freedom. 
}

\begin{definition}[Local Enablement Assumption] \label{def.bip.local-enablement}
For every component  $\B_i = (Q_i,P_i,\goesto_i)$, the following holds. 
  In every $s_i \in Q_i$, $\B_i$ enables some
interaction $\act$.
\end{definition}

\begin{definition}[BIP System]\label{def.bip.system} Let $\B = \gamma(\B_1,\dots,\B_n)$ be a composite component with transition system $(Q,\gamma,
\goestog)$, and let $Q_0 \sub Q$ be a set of initial states. Then
$(\B,Q_0)$ is a BIP system.  
\end{definition}
%We usually refer to $B$ as a BIP system, understanding that $Q_0$ is implicit.

\noindent
Figure~\ref{fig:diningbip} gives a BIP-system with philosophers initially in state $h$ (hungry) and forks initially in
state $f$ (free).
%
To avoid tedious repetition, we fix, for the rest of the paper, an arbitrary BIP-system $(\B, Q_0)$, with
$\B \df \gamma(\B_1,\dots,\B_n)$, and transition system $(Q,\gamma, \goestog)$.

\begin{definition}[Execution]\label{def.bip.execution} Let $(\B, Q_0)$ be a BIP system
with transition system $(Q,\gamma, \goestog)$.  Let $\rho = s_0 \act_1 s_1 \ldots s_{j-1} \act_j s_j \ldots$ be an alternating sequence of
states of $\B$ and interactions of $\B$. Then $\rho$ is an execution of
$(\B, Q_0)$ iff (1) $s_0 \in Q_0$, and (2) $\fa j > 0: s_{j-1} \goesto[\act_j] s_j$. 
\end{definition}

\begin{definition}[Reachable state, transition]\label{def.bip.reachable}
A state or transition that occurs in some execution is called \emph{reachable}.
$\rstates{\B,Q_0}$ denotes the set of reachable states of $(\B, Q_0)$.
\end{definition}


\begin{definition}[State projection]\label{def.bip.state.projection} Let $(\B, Q_0)$ be a BIP system where $\B = \gamma(\B_1,\dots,\B_n)$ and let
$s = \tpl{s_1,\dots,s_n}$ be a state of $(\B, Q_0)$. Let $\set{\B_{i_1},\dots,\B_{i_k}} \sub \set{\B_1,\dots,\B_n}$. Then
$s \pj \set{\B_{i_1},\dots,\B_{i_k}} \df \tpl{s_{i_1},\dots,s_{i_k}}$. For a single $\B_i$, we write $s \pj \B_i = s_i$.
%
We extend state projection to sets of states element-wise.
\end{definition}

\begin{definition}[Subcomponent]\label{def.bip.subcomponent} Let $\B \df \gamma(\B_1,\dots,\B_n)$ be a composite component, and let $\set{\B_{i_1},\dots,\B_{i_k}}$
be a subset of $\set{\B_{1},\dots,\B_{n}}$.  Let $P' = P_{i_1} \un \cdots \un P_{i_k}$, \ie the union of the ports of $\set{\B_{i_1},\dots,\B_{i_k}}$.
Then the subcomponent $\B'$ of $\B$ based on $\set{\B_{i_1},\dots,\B_{i_k}}$ is as follows:
%
\bn
\item $\gamma' \df \set{ \act \ints P' \st \act \in \gamma \land \act \ints P' \ne \emptyset}$
\item $\B' \df \gamma'(\B_{i_1},\dots,\B_{i_k})$ 
\en
\end{definition}
%
That is, $\gamma'$ consists of those interactions in $\gamma$ that have at least one participant in 
$\set{\B_{i_1},\dots,\B_{i_k}}$, and restricted to the participants in $\set{\B_{i_1},\dots,\B_{i_k}}$,
\ie participants not in $\set{\B_{i_1},\dots,\B_{i_k}}$ are removed.

We write $s \pj \B'$ to indicate state projection onto $\B'$, and define 
$s \pj \B' \df  s \pj \set{\B_{i_1}, \dots, \B_{i_k}}$, where $\B_{i_1},\dots,\B_{i_k}$ are the atomic
components in $\B'$.


%MENTION THAT THE INTERACTION $a$ SHOULD ALSO BE PROJECTED. WHEN WE WRITE $a$ IN A TRANSITION $s_{k-1} \goestog[a] s_k$
%OF THE PROJECTED SUBSYSTEM, WE REALLY MEAN THE PROJECTION OF $a$. SINCE $a$ IS JUST USED AS A LABEL, THERE IS NO HARM
%DONE, BUT WE KEEP IN MIND THAT THE PROJECTION OF $a$ ONTO THE SUBSYSTEM IS MEANT



\begin{definition}[Subsystem]\label{def.bip.subsystem}
Let $(\B, Q_0)$ be a BIP system where $\B = \gamma(\B_1,\dots,\B_n)$, and let 
$\set{\B_{i_1},\dots,\B_{i_k}}$ be a subset of $\set{\B_{1},\dots,\B_{n}}$.
Then the subsystem $(\B', Q'_0)$ of  $(\B, Q_0)$ based on $\set{\B_{i_1},\dots,\B_{i_k}}$ is as follows:
\bn
\item $\B'$ is the subcomponent of $\B$ based on $\set{\B_{i_1},\dots,\B_{i_k}}$ 
\item $Q'_0 = Q_0 \pj \set{\B_{i_1},\dots,\B_{i_k}}$
\en
\end{definition}


\begin{definition}[Execution Projection]\label{def.bip.execution.projection}
Let $(\B, Q_0)$ be a BIP system where $\B = \gamma(\B_1,\dots,\B_n)$, and let $(\B', Q'_0)$, with $\B'
= \gamma'(\B_{i_1},\dots,\B_{i_k})$ be the subsystem of $(\B, Q_0)$ based on $\set{\B_{i_1},\dots,\B_{i_k}}$.
Let $P' = P_{i_1} \un \cdots \un P_{i_k}$, \ie $P'$ is the set of ports of $(\B', Q'_0)$.
%
Let $\rho = s_0 \act_1 s_1 \ldots s_{j-1} \act_j s_j \ldots$ be an execution of $(\B, Q_0)$.  Then, $\rho \pj (\B', Q'_0)$, the projection
of $\rho$ onto $(\B', Q'_0)$, is the sequence resulting from:
\bn
\item \label{def.clause.bip.execution.projection.state} 
replacing each $s_j$ by $s_j \pj \set{\B_{i_1},\dots,\B_{i_k}}$, \ie replacing each state by its projection onto $\set{\B_{i_1},\dots,\B_{i_k}}$

\item \label{def.clause.bip.execution.projection.action} removing all $\act_j s_j$ where $\act_j \ints P' = \emptyset$   %$\act_k \not\in \gamma'$\\

\item \label{def.clause.bip.execution.projection.port} replacing each $\act_j$ by $\act_j \ints P'$, \ie replacing each
interaction by its projection onto the port set $P'$ 

\en
\end{definition}



\begin{proposition} [Execution Projection]\label{prop.bip.execution.projection}
Let $(\B, Q_0)$ be a BIP system where $\B = \gamma(\B_1,\dots,\B_n)$, and let
$(\B', Q'_0)$, with $\B' = \gamma'(\B_{i_1},\dots,\B_{i_k})$ be the subsystem of $(\B, Q_0)$ based on $\set{\B_{i_1},\dots,\B_{i_k}}$.
Let $P' = P_{i_1} \un \cdots \un P_{i_k}$,  \ie the union of the ports of $\set{\B_{i_1},\dots,\B_{i_k}}$.
Let $\rho = s_0 \act_1 s_1 \ldots s_{j-1} \act_j s_j \ldots$ be an execution of $(\B, Q_0)$. 
Then, $\rho \pj (\B', Q'_0)$ is an execution of $(\B', Q'_0)$.
\end{proposition}

\begin{proof}
%Let $\rho = s_0 \act_1 s_1 \ldots s_{j-1} \act_j s_j \ldots$, be an execution of $(\B, Q_0)$,
By Definitions~\ref{def.bip.state.projection}, \ref{def.bip.subsystem}, and \ref{def.bip.execution.projection}, 
we have $\rho \pj (\B', Q'_0) = s'_0 \actb_1 s'_1 \actb_2 s'_2 \ldots$ for some $s'_0,\actb_1 s'_1 \actb_2 s'_2 \ldots$,
where $s'_j \in Q' = Q \pj \set{\B_{i_1},\dots,\B_{i_k}}$ for $j \ge 0$.
%
Also by Definitions~\ref{def.bip.state.projection}, \ref{def.bip.subsystem}, and \ref{def.bip.execution.projection}, 
we have $s'_0 \in Q'_0 = Q_0 \pj \set{\B_{i_1},\dots,\B_{i_k}}$, since $s'_0 = s_0 \pj \B'$, and $s_0 \in Q_0$, by Definition~\ref{def.bip.execution}.

Consider an arbitrary step $(s'_{j-1}, \actb_j, s'_j)$ of $\rho \pj (\B', Q'_0)$.
Since $\actb_j s'_j$ was not removed in Clause~\ref{def.clause.bip.execution.projection.action} of
Definition~\ref{def.bip.execution.projection}, we have 

\noindent
(1) $s'_j = s_{\l} \pj \set{\B_{i_1},\dots,\B_{i_k}}$ for some $\l > 0$ and such that $\act_{\l} \ints P' \ne \emptyset$\\                     %$\act_\l \in \gamma'$\\
(2) $\actb_j = \act_{\l} \ints P'$\\                                                              %$b_j = \act_k$
(3) $s'_{j-1} = s_m \pj \set{\B_{i_1},\dots,\B_{i_k}}$ for the smallest $m$ such that\\
   \ind $m < \l$ and
        $\fa m' : {m+1 \leq m' < \l} : \act_{m'} \ints P' = \emptyset$                              %\act_{m'} \not\in \gamma'$

\noindent
From  (3) we have $\fa m' : {m+1 \leq m' < \l} : \act_{m'} \ints P' = \emptyset$. So by Definitions~\ref{def.bip.composition} and \ref{def.bip.execution.projection},
    we have $s_m \pj \set{\B_{i_1},\dots,\B_{i_k}} = s_{\l-1} \pj \set{\B_{i_1},\dots,\B_{i_k}}$.
From (3) we have $s'_{j-1} = s_m \pj \set{\B_{i_1},\dots,\B_{i_k}}$. 
%From this and $s_m \pj \set{\B_{i_1},\dots,\B_{i_k}} = s_{\l-1} \pj \set{\B_{i_1},\dots,\B_{i_k}}$ just established, we obtain
Hence $s'_{j-1} = s_{\l-1} \pj \set{\B_{i_1},\dots,\B_{i_k}}$.

From $s_{\l-1} \goestog[\act_{\l}] s_{\l}$,    $\act_{\l} \ints P' \ne \emptyset$, and Definition~\ref{def.bip.composition}, we have  % $\act_k \in \gamma'$
        $s_{\l-1} \pj \set{\B_{i_1},\dots,\B_{i_k}} \goestog[\act_{\l} \ints P'] s_{\l} \pj \set{\B_{i_1},\dots,\B_{i_k}}$.
$s'_{j-1} = s_{\l-1} \pj \set{\B_{i_1},\dots,\B_{i_k}}$ was established above.
$s'_j = s_{\l} \pj \set{\B_{i_1},\dots,\B_{i_k}}$ is from (1).
$\actb_j = \act_{\l} \ints P'$ is from (2).
Hence we obtain $s'_{j-1} \goestog[\actb_j] s'_j$, \ie that $s'_{j-1}, \actb_j s'_j$ is a step of  $(\B', Q'_0)$.

Since $(s'_{j-1}, \actb_j, s'_j)$ was arbitrarily chosen, we conclude that
every step of $\rho \pj (\B', Q'_0)$  is a step of $(\B', Q'_0)$. This establishes Clause (2) of Definition~\ref{def.bip.execution}.
%
The first state of
$\rho \pj (\B', Q'_0)$  is $s'_0$, and $s'_0 \in Q'_0$ was shown above, so we establish Clause (1) of Definition~\ref{def.bip.execution}.

Since both clauses of Definition~\ref{def.bip.execution} are satisfied, we conclude that $\rho \pj (\B', Q'_0)$ is an execution of $(\B', Q'_0)$.
\end{proof}


\begin{corollary} \label{cor.bip.reachability} \label{cor:bip.reachability}
Let $(\B', Q'_0)$ be a subsystem of $(\B, Q_0)$, and let $P'$ be the port set of $(\B', Q'_0)$.
Let $s$ be a reachable state of $(\B, Q_0)$. Then $s \pj \B'$ is a reachable state of  $(\B',Q'_0)$. Let $s \la{\act} t$ be
a reachable transition of $(\B, Q_0)$, and let $\act \ints P'$ be an interaction of
 $(\B', Q'_0)$. Then  $s \pj \B' \la{\act \ints P'} t \pj \B'$ is a reachable transition of $(\B', Q'_0)$.
\end{corollary}
%
\begin{proof}
Immediate corollary of Proposition~\ref{prop.bip.execution.projection}.
\end{proof}



\redbox{
\begin{example}[Execution projection]
In the dining philosophers example of \fig{diningSpectrum},
let the (single) initial state be $(h_0,h_1,h_2,h_3,f_0,f_1,f_2,f_3)$, i.e., all philosophers are hungry and all forks are free. 
Also, $\Grab_0 = \set{\get_0, use_0^\l, use_1^r}$ (resp.  $\Grab_2 = \set{\get_2, use_2^\l, use_3^r}$) is the interaction in which $\Phl_0$ 
(resp. $\Phl_2$) picks up both forks, and $\Rel_0=\set{\putt_0, \free_0^\l, free_1^r}$ is the interaction in which $\Phl_0$ releases both forks. 
Consider the following execution: 
$\rho = 
(h_0,h_1,h_2,h_3,f_0,f_1,f_2,f_3)$
$\{\get_0, \use_0^\l, \use_1^r\}$  
$(e_0,h_1,h_2,h_3,u_0^\l,u_1^r,f_2,f_3)$  
$\{\get_2, \use_2^\l, \use_3^r\}$ 
$(e_0,h_1,e_2,h_3,u_0^\l,u_1^r,u_2^\l,u_3^r)$ 
$\{\putt_0, \free_0^\l, \free_1^r\}$  
$(h_0,h_1,e_2,h_3,f_0,f_1,u_2^\l,u_3^r) \cdots$.
%, i.e., philosopher $P_0$ and $P_2$ gets their left and right forks, respectively, then $P_0$ releases its left and right forks. 
The projection of this execution on the subsystem defined by subcomponent $\{\Phl_0, \Phl_1, F_0\}$, is equal to:
$\rho \pj (\{\Phl_0, \Phl_1, F_0\}, Q'_0) = (h_0,h_1,f_0) \, \{\get_0, \use_0^\l\} \, (e_0,h_1,u_0^\l) \, \{\putt_0, \free_0^\l\} \, (h_0,h_1,f_0) \cdots$.
In particular, we project the states and interactions with respect to the subcomponent. Notice that interaction $\Grab_2$ will disappear as its ports do not belong to the subcomponent.
Clearly,  $\rho \pj (\{\Phl_0, \Phl_1, F_0\}, Q'_0)$ is an execution of $(\{\Phl_0, \Phl_1, F_0\}, Q'_0)$. 
\end{example}
}








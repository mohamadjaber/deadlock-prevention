Recall that $(\B, Q_0)$ is an arbitrary fixed BIP-system, which
we use in all definitions, theorems etc.
%
We characterize a supercycle as a post-fixpoint of a ``blocking operator'' $\SFsymb$ (defined below) over the complete Boolean lattice formed from the
subgraphs of \wfg{\B}{s}, with $\subg$ (\defn{wsubgraph}) as the ordering.
%
Roughly, $\SFsymb$ maps a subset $\XS$ of the nodes of $\wfg{\B}{s}$ (\ie some subset of the components and interactions in $(\B, Q_0)$)
to a set of nodes $\YS$ whose execution is blocked by $\XS$. An interaction $\act$ in $\YS$ is blocked by $\XS$ if some participant of $\act$ is in $\XS$ and does
not enable $\act$. A component $\B_i$  in $\YS$ is blocked by $\XS$ if every interaction that $\B_i$ enables is in $\XS$. In terms of $\wfg{\B}{s}$,
$\act$ is blocked by $\XS$ if there is a wait-for edge from $\act$ to a node in $\XS$, and $\B_i$ is blocked by $\XS$ if every wait-for edge from $\B_i$ is to a
node in $\XS$. 

Since $\SFsymb$ is monotone, its greatest fixpoint $\SC$ exists.  If \wfg{\B}{s} is supercycle-free, then $\SC$ is the empty wait-for graph \ewfg.
Otherwise $\SC$ is the largest supercycle in \wfg{\B}{s}.  We define the dual $\VFsymb$ of $\SFsymb$, whose least fixpoints are the nodes that are not
members of any supercycle, and we say that such nodes have a \emph{supercycle violation}. Since 
$\VFsymb$ is monotone and continuous, and the underlying lattice is finite, its least fixpoint can be computed as usual by iterating $\VFsymb$, starting from \ewfg.
This provides a method of computing the nodes with supercycle violations, which is the basis for our deadlock-freedom criterion.



%%%%%%%%%%%%%%%%%%%%%%%%%%%%%%%%%%%%%%%%%%%%%
\subsection{\redbox{A fixpoint characterization of supercycles}}
\label{secn:supercycle-fixpoint}

\begin{definition}[Set of subgraphs] \label{defn:wsetOfSubgraphs}
$\wfgPow{\B}{s} = \set{ \XS \stt \XS \subg \wfg{\B}{s} }$.
\end{definition}
We include in $\wfgPow{\B}{s}$ the empty wait-for graph, which we denote by \ewfg.
Let $\nodes{\B} = \set{\B_1,\ldots,\B_n} \un \gamma$, \ie  $\nodes{\B}$ is the set of components and interactions in $\B$, and
let $\nodesPow{\B}$ be the powerset of $\nodes{\B}$.
Then $\wfgPow{\B}{s}$ is isomorphic to $\nodesPow{\B}$, where each $\XS \in \wfgPow{\B}{s}$ is mapped to the set of nodes that it contains.

\begin{definition}[Wait-for lattice] \label{defn:wflattice}
Define the partially ordered set $\lat{\B}{s} = \tpl{ \wfgS{\B}{s}, \subg }$ %\tpl{ \wfg{\B}{s}, \ewfg, \subg }$
whose elements are all the subgraphs of 
\wfg{\B}{s}, and where  $U \subg V$ is as in \defn{wsubgraph}.   
%iff $U$ is a subgraph of $V$, \ie $\ord$ is the ``is a subgraph of'' order relation.
\end{definition}

The following proposition follows immediately from the definitions; its proof is left to the reader.
\begin{proposition} \label{prop:isALattice}
$\lat{\B}{s} = \tpl{ \wfgS{\B}{s}, \subg }$ is a finite complete Boolean lattice as follows:
\be

\item meet is given by graph intersection: 
$\XS \meet \YS$ consists of the nodes that are present in both $\XS$ and $\YS$, together with the edges induced by \wfg{\B}{s}, 
\ie if $u \in \XS \meet \YS$, $v \in \XS \meet \YS$, and $u \ar v \in  \wfg{\B}{s}$, then $u \ar v \in  \XS \meet \YS$.

\item join is given by graph union: $\XS \join \YS$ consists of the nodes that are present in $\XS$, or in $\YS$, or in both, together with 
 the edges induced by \wfg{\B}{s}.
 Note that $\join$ is \emph{not} disjoint graph union: 
it is possible for $\XS$ and $\YS$ to have nodes and edges in common. Note also that $\XS \join \YS$ may contain edges not present in either $\XS$ nor $\YS$,
since the edges are those induced by \wfg{\B}{s}. 

\item \wfg{\B}{s}  is the top element 

\item  the empty wait-for graph, denoted by \ewfg, is the bottom element

\item the complement \compl{\XS} of $\XS$ is obtained by taking all the nodes of \wfg{\B}{s} that are not in $\XS$, together with the induced edges.
\ee
\end{proposition}
%
As noted, $\join, \meet$ and complement are determined entirely by the sets of nodes in the relevant subgraphs. The resulting edges are always those
that are induced by \wfg{\B}{s}.
Let $\tpl{ \nodesPow{\B}, \sub }$ be the lattice defined using the subset ordering $\sub$. 
Then $\lat{\B}{s} = \tpl{ \wfgPow{\B}{s}, \subg }$ is isomorphic to $\tpl{ \nodesPow{\B}, \sub }$, 
where each $\XS \in \wfgPow{\B}{s}$ is mapped to the set of nodes that it contains.


\begin{definition}[$\mathit{blocks}_s$] \label{defn:blocks} 
Let $\XS \subg \wfg{\B}{s}$ and $\act, \B_i$ be nodes in $\wfg{\B}{s}$. Then 
$\blocks{s}{\act}{\XS} \df (\ex \B_i \in \XS : \act \ar \B_i \in \wfg{\B}{s})$, and 
$\blocks{s}{\B_i}{\XS} \df (\fa \act : \B_i \ar \act \in \wfg{\B}{s} \imp \act \in \XS)$.
\end{definition}

\begin{definition}[$\SFsymb_s$]  \label{defn:scFix} 
Define $\SFsymb_s: \wfgPow{\B}{s} \to \wfgPow{\B}{s}$ as follows.
$\SFs{s}{\XS}$ is the subgraph with nodes $\set{v \stt \blocks{s}{v}{\XS} }$, together with their induced edges.
\end{definition}

\begin{definition}[$\VFsymb_s$]  \label{defn:violFix}
Define $\VFsymb_s: \wfgS{\B}{s} \to \wfgS{\B}{s}$ as follows.
$\VFs{s}{\XS}$ is the subgraph with nodes
$\set{v \stt \neg \blocks{s}{v}{\compl{\XS}}}$, together with their induced edges.
\end{definition}
%
Hence $\VFs{s}{\XS} = \compl{\SFs{s}{\compl{\XS}}}$, \ie $\VFsymb_s$ and $\SFsymb_s$ are duals.
Note that $\SFsymb_s$ and $\VFsymb_s$ are defined given both a particular BIP system 
$\B$ and a particular state $s$ of $\B$. Hence we should really write 
$\SFsymb_{\B,s}(\XS)$, $\VFsymb_{\B,s}(\XS)$ to indicate this functional dependence. 
Since however, 
$\B$ is a fixed BIP-system, we omit the $\B$ subscript to avoid notational clutter.
In giving examples, we usually omit the subscript for the state, since the state will be implicitly given by the example.


\begin{proposition} \label{prop:monotone}
$\SFsymb_s$ and $\VFsymb_s$ are monotone and continuous.
\end{proposition}
%
\begin{proof}
% show its monotnoe
We show first that $\SFsymb_s$ is monotone, \ie $\XS \subg \YS \imp \SFs{s}{\XS} \subg \SFs{s}{\YS}$.
Let $\ndv$ be an arbitrary node in $\SFs{s}{\XS}$, so that $\blocks{s}{\ndv}{\XS}$ holds. There are two cases.\\

\emph{Case of $\ndv$ is an interaction $\act$}. By Definitions~\ref{defn:blocks} and\ref{defn:scFix}, and  we have $\ex \B_i \in \XS  : \act \ar \B_i \in \wfg{\B}{s}$.
Since $\XS \subg \YS$, this same $\B_i$ is also a node of $\YS$, and so  $\ex \B_i \in \YS  : \act \ar \B_i \in \wfg{\B}{s}$. 
Hence  $\blocks{s}{\act}{\YS}$, and so $\act \in \SFs{s}{\YS}$. \\

\emph{Case of $\ndv$ is a component $\B_i$}. By Definitions~ \ref{defn:blocks} and \ref{defn:scFix}, and, we have $(\fa \act : \B_i \ar \act \in \wfg{\B}{s} \imp \act \in \XS)$.
Since $\XS \subg \YS$, we have  $(\fa \act : \B_i \ar \act \in \wfg{\B}{s} \imp \act \in \YS)$. 
Hence $\blocks{s}{\B_i}{\YS}$, and so $\B_i \in \SFs{s}{\YS}$. \\

In both cases, we have $\ndv \in \SFs{s}{\YS}$. Since $\ndv$ was chosen arbitrarily from $\SFs{s}{\XS} $, it follows that $\SFs{s}{\XS} \subg \SFs{s}{\YS}$. Hence
$\SFsymb_s$ is monotone.
%
Since the dual of a monotone mapping in a complete Boolean lattice is also monotone, we have that $\VFsymb_s$ is monotone.
%
Finally, since $\lat{\B}{s}$ is finite, it follows that $\SFsymb_s$ and $\VFsymb_s$ are continuous.
\end{proof}

Hence, by the Knaster-Tarski theorem, the least and greatest fixpoints of $\SFsymb_s$ and $\VFsymb_s$ exist.
%, and so their greatest and least fixpoints 

\begin{proposition} \label{prop:supercycleGFP}
Let $\XS \ne \ewfg$ and $\XS \subg \wfg{\B}{s}$, \ie $\XS$ is a non-empty subgraph of \wfg{\B}{s}. Then $\XS$ is a supercycle in \wfg{\B}{s} iff $\XS \subg \SFs{s}{\XS}$.
\end{proposition}
%
\begin{proof}
Let $\XS$ be a supercycle in \wfg{\B}{s}. By \defn{supercycle}, every node in $\XS$ is blocked by $\XS$, \ie 
$(\fa \ndx \in \XS: \blocks{s}{\ndx}{\XS})$. By \defn{scFix}, $\XS \subg \SFs{s}{\XS}$.
%
Conversely, suppose $\XS \subg \SFs{s}{\XS}$ for some subgraph $\XS$ of \wfg{\B}{s}. Hence 
$(\fa \ndx \in \XS: \ndx \in \SFs{s}{\XS})$, so by \defn{scFix}, $(\fa \ndx \in \XS: \blocks{s}{\ndx}{\XS})$.
Hence every node in $\XS$ is blocked by $\XS$, and so $\XS$ satisfies \defn{supercycle}, and is therefore a supercycle.
\end{proof}
%
Thus the supercycles of \wfg{\B}{s} are exactly the post-fixpoints of $\SFsymb$. %This implies that the union of two supercycles is also a supercycle. 


\begin{proposition} \label{prop:supercycle:union}
Let $\SC, \SC'$ be supercycles in $\wfg{\B}{s}$. Then $\SC \join \SC'$ is
a supercycle in $\wfg{\B}{s}$.
\end{proposition}
%
\begin{proof}
By \prop{supercycleGFP}, $\SC$ and $\SC'$ are post-fixpoints of $\SFsymb_s$. Since the join of post-fixpoints is a post-fixpoint, 
the proposition follows by applying \prop{supercycleGFP} again.
%Straightforward, since each node in  $\SC \un \SC'$ has enough successors that it waits for to satisfy  \defn{supercycle}. 
\end{proof}




\begin{proposition} \label{prop:GFPisLargestSC}
Let $\SC$ be the greatest post-fixpoint of $\SFsymb_s$. Then either
(a) \wfg{\B}{s} is supercycle-free and $\SC = \ewfg$, or 
(b) \wfg{\B}{s} contains supercycles, and $\SC$ is the largest supercycle in \wfg{\B}{s}.
\end{proposition}
%
\begin{proof}
By the Knaster-Tarski theorem, the greatest post-fixpoint is the join of all the post-fixpoints. 
If \wfg{\B}{s} is supercycle-free, then by \prop{supercycleGFP}, the only post-fixpoint of $\SFsymb_s$ is \ewfg
Hence $\SC = \ewfg$.
If \wfg{\B}{s} contains supercycles, then by \prop{supercycleGFP},  the set of post-fixpoints of $\SFsymb_s$ is exactly the set of 
supercycles of \wfg{\B}{s}. Hence $\SC$ is the join of all these supercycles. By \prop{supercycle:union}, $\SC$ is itself a supercycle.
Hence $\SC$ is the largest supercycle in \wfg{\B}{s}.
\end{proof}

Let $\MATHIDN{lfp}, \MATHIDN{gfp}$ denote the least fixpoint and greatest fixpoint operators, respectively.

\begin{proposition}  \label{prop:LFPisScViolations}
$\ndv \in  \lfp{\VFsymb_s}$ iff $\ndv$ is not a node in any supercycle of \wfg{\B}{s}.    
\end{proposition}
%
\begin{proof}
From the Park conjugate (dual) fixpoint theorem in complete Boolean lattices, we have 
\lfp{\VFsymb_s} = \compl{\gfp{\SFsymb_s}}.
By \prop{GFPisLargestSC}, \gfp{\SFsymb_s} is the largest supercycle in \wfg{\B}{s}. Hence the nodes not in 
\gfp{\SFsymb_s} are exactly the nodes that are not in any supercycle. These are exactly the nodes in \lfp{\VFsymb_s}.
\end{proof}

Define $\VFsi{s}{\XS}{1} = \VFs{s}{\XS}$, and for $d > 1$, $\VFsi{s}{\XS}{d} =  \VFs{s}{ \VFsi{s}{\XS}{d-1} }$, 
\ie a superscript indicates functional iteration of $\VFsymb$. Also let $\JOIN$ be the ``quantifier'' version of $\join$.
Note that $\VFsi{s}{\ewfg}{d} \subg \VFsi{s}{\ewfg}{d'}$ when $d \le d'$, since $\VFsymb$
is monotone.
Hence $\VFsi{s}{\ewfg}{1}, \VFsi{s}{\ewfg}{2}, \ldots$ is a non-decreasing sequence.

\begin{proposition}  \label{prop:computeLFP}
$\lfp{\VFsymb_s} = \JOIN_{d \ge 1} \VFsi{s}{\ewfg}{d}$.
\end{proposition}
%
\begin{proof}
By \prop{monotone}, $\VFsymb_s$ is continuous. Follows by standard results, \eg see the CPO fixpoint theorem I in 
\citeN{DP02}.
\end{proof}



\begin{definition}[Supercycle violation, $\scV{v}{s}$, $\scVd{v}{d}{s}$]
\label{def:supercycle-violation}
\label{def:supercycle.violation}
\label{defn:supercycle.violation}
Let $\ndv$ be a node of $\wfg{\B}{s}$. Define
$\scV{v}{s} \df v \in \lfp{\VFsymb_s}$ and, for $d \ge 1$,
$\scVd{v}{d}{s} \df v \in \VFsi{s}{\ewfg}{d}$.\footnote{Note that
we abuse notation by overloading $\SMATHID{viol}$, but no ambiguity arises since the two versions have
different parameter lists.}
\end{definition}

\begin{proposition}
\label{prop:globViol-equiv-globViolDist}
$\scV{\ndv}{s}$ iff  $(\ex d \ge 1: \scVd{\ndv}{d}{s})$.
\end{proposition}
%
\begin{proof}
By \defn{supercycle.violation}, $\scV{\ndv}{s} \ev \ndv \in \lfp{\VFsymb_s}$.
By \prop{computeLFP}, $\ndv \in \lfp{\VFsymb_s} \ev v \in \JOIN_{d \ge 1} \VFsi{s}{\ewfg}{d}$.
By \defn{supercycle.violation}, $\fa d \ge 1: \scVd{v}{d}{s} \ev v \in \VFsi{s}{\ewfg}{d}$.
Chaining these equivalences establishes the proposition.
\end{proof}
%
It follows from \prop{LFPisScViolations} that $\scV{v}{s}$ iff there does not exist $\SC$ such that $\SC$ is a supercycle and $v \in SC$.
We say that a node $v$ of \wfg{\B}{s} has a \emph{supercycle violation} %\footnote{In the sequel, we say ``sc-violation'' rather than ``supercycle  violation.''} 
iff $v$ is not a node in any supercycle of \wfg{\B}{s}, 
\ie iff $\scV{v}{s}$ holds. 
By \prop{computeLFP}, we can compute $\lfp{\VFsymb_s}$ (and therefore $\scV{v}{s}$) by iterating $\VFsymb_s$, starting from $\ewfg$, until there is no more change.
$\scVd{v}{d}{s}$ defines a supercycle violation that can be confirmed within $d$ iterations of $\VFsymb_s$, which we call a \emph{level-$d$ supercycle violation}.
$\scV{v}{s}$ requires, in general, the entire least fixed point of $\VFsymb_s$.


\vspace{1ex}
\begin{example}[Supercycle violation]
For example, consider the wait-for graph in \fig{SCnotCycle}. We show the set of nodes in each $\VFi{\ewfg}{d}$, since the induced subgraph is easily inferred from \fig{SCnotCycle}.
%$\VFsymb^{0} (\ewfg) = \ewfg$,
$\VFi{\ewfg}{1}  = \set{\acti}$, 
$\VFi{\ewfg}{2} = \set{\B_6, \acti}$, 
$\VFi{\ewfg}{3}  = \set{\acth, \B_6, \acti}$, 
$\VFi{\ewfg}{4}  = \set{\B_5, \acth, \B_6, \acti}$, 
$\VFi{\ewfg}{5}  = \set{\B_5, \acth, \B_6, \acti}$, 
as so $\lfp{\VFsymb} = \set{\B_5, \acth, \B_6, \acti}$.
%
For \fig{cycleOK}, it is easy to verify that $\lfp{\VFsymb}$ consists of all the nodes in the system, \ie the wait-for graph shown is supercycle-free.
\end{example}


\begin{example}[Supercycle violations in dining philosophers]
\label{exm:glob-dphils-viols}
Figure~\ref{fig:globalDphilsViolations} illustrates supercycle violations in four global states of the dining philosophers system of \fig{diningSpectrum}.
The states shown are the initial state, and the states resulting after execution of the indicated sequences of interactions.
For each node $\ndv$ (interaction or component), we include a small
positive integer after its name, giving the smallest $d$ such that $v \in \VFi{\ewfg}{d}$,
\ie the supercyle violation level.
\end{example}


\begin{figure*}[ht]
  \begin{center}
      \subfigure[Supercycle violations in initial state.]{\label{fig:violsInitial}\scalebox{0.4}{\input{figs/scvDiningInitial.pdf_t}}} \quad \quad
      \subfigure[Supercycle violations after execution of $\Grab_0$.]{\label{fig:violsGrab}\scalebox{0.4}{\input{figs/scvDining1.pdf_t}}} 
      \subfigure[Supercycle violations after execution of $\Grab_0; \Grab_2$.]{\label{fig:violsGrabGrab}\scalebox{0.4}{\input{figs/scvDining2.pdf_t}}} \quad \quad
      \subfigure[Supercycle violations after execution of $\Grab_0; \Grab_2; \Rel_0$.]{\label{fig:violsGrabGrabRel}\scalebox{0.4}{\input{figs/scvDining3.pdf_t}}} 
      \caption{Example supercycle violations for dining philosophers system of Figure~\ref{fig:diningSpectrum}.}
       \label{fig:globalDphilsViolations}
  \end{center}
\end{figure*}








\begin{definition}[Supercycle membership, $\scyc{\B}{s}{v}$]
\label{defn:supercycle.membership}
Let $\ndv$ be a node of $\wfg{\B}{s}$. Then $\scyc{\B}{s}{\ndv}$ holds iff there exists a supercycle $\SC \subg \wfg{\B}{s}$ such that
$\ndv \in \SC$.
\end{definition}





\begin{proposition} \label{prop:scViol-iff-notInSC}
Let $\ndv$ be a node of $\wfg{\B}{s}$. Then $\neg \scyc{\B}{s}{\ndv}$ iff $\scV{\ndv}{s}$ % $(\ex d \ge 1: \viol{v}{d}{t})$.
That is, a node is not in any supercycle iff it has a supercycle violation.
\end{proposition}
%
\begin{proof}
Immediate from \defn{supercycle.violation}, \defn{supercycle.membership}, and \prop{LFPisScViolations}.
\end{proof}

%\prf{Immediate from Propositions~\ref{prop:scViol-implies-notInSC} and
%  \ref{prop:notInSC-implies-scViol}.}







%%%%%%%%%%%%%%%%%%%%%%%%%%%%%%%%%%%%%%%%%%%%%
\subsection{Structural properties of supercycles}
\label{secn:supercycle-structural}

%\subsection{Supercycle Membership} 

We present some structural properties of supercycles, which
are central to our deadlock-freedom conditions.


Define
$\preds{\B}{s}{v} = \set{w \stt w \ar v \in \wfg{\B}{s}}$ and 
$\succs{\B}{s}{v} = \set{w \stt v \ar w \in \wfg{\B}{s}}$.
The definition of a supercycle (\defn{supercycle}) 
imposes certain constraints on supercycle membership of a node \wrt its predecessors and successors
in the wait-for-graph, as follows:

%this is not used anywhere, and is present only for its own sake
\begin{proposition}[Supercycle-membership constraints]
\label{prop:sc-membership}
Let $\act, \B_i$ be nodes of $\wfg{\B}{s}$. Then
\bn

\item \label{clause:sc-membership:comp-out}
$\scyc{\B}{s}{\B_i} \ifof (\fa \act \in \succs{\B}{s}{\B_i} : \scyc{\B}{s}{\act})$.

\item \label{clause:sc-membership:comp-in}
$\scyc{\B}{s}{\B_i} \imp (\fa \act \in \preds{\B}{s}{\B_i} : \scyc{\B}{s}{\act})$.

\item \label{clause:sc-membership:act-out}
$\scyc{\B}{s}{\act} \ifof (\ex \B_i \in \succs{\B}{s}{\act} : \scyc{\B}{s}{\B_i})$.

\item \label{clause:sc-membership:act-in}
$\scyc{\B}{s}{\act} \folf (\ex \B_i \in \preds{\B}{s}{\act} : \scyc{\B}{s}{\B_i})$.

\en
\end{proposition}
%
\begin{proof}
We deal with each clause in turn.


%%%%%%%%%%%%%%%%%%%%%
\textit{Proof of \clause{sc-membership:comp-out}}.
%
Assume $\scyc{\B}{s}{\B_i}$, and let $\SC \subg \wfg{\B}{s}$ be the supercycle containing $\B_i$.  Let
$\actp \in \succs{\B}{s}{\B_i}$.  By \defn{supercycle}, \clause{supercycle.component-blocked},
$\actp \in \SC$.  Hence $(\fa \act \in \succs{\B}{s}{\B_i} : \scyc{\B}{s}{\act})$.
We conclude
$\scyc{\B}{s}{\B_i} \imp (\fa \act \in \succs{\B}{s}{\B_i} : \scyc{\B}{s}{\act})$.
%
Now assume $(\fa \act \in \succs{\B}{s}{\B_i} : \scyc{\B}{s}{\act})$, and let 
$\SC$ be the join of all the supercycles containing all the $\act \in \succs{\B}{s}{\B_i}$. 
By \prop{supercycle:union}, $\SC \subg \wfg{\B}{s}$ is a supercycle.
Let $\SC'$ be $\SC$ with edge $\B_i \ar \act$ added, for all 
$\act \in \succs{\B}{s}{\B_i}$.
Then $\SC'$ is a supercycle by 
\defn{supercycle}, and also $\SC' \subg \wfg{\B}{s}$. Hence $\scyc{\B}{s}{\act}$.
We conclude 
$\scyc{\B}{s}{\B_i} \folf (\fa \act \in \succs{\B}{s}{\B_i} : \scyc{\B}{s}{\act})$.




%%%%%%%%%%%%%%%%%%%%%%%%%%
\textit{Proof of \clause{sc-membership:comp-in}}.
%
Assume $\scyc{\B}{s}{\B_i}$, so that $\SC \subg \wfg{\B}{s}$ is the supercycle containing $\B_i$.
Let  $\act \in \preds{\B}{s}{\B_i}$, and let $\SC'$ be $\SC$ with 
$\act \ar \B_i$ added. Hence $\SC'$ is a supercycle by \defn{supercycle},
  Clause~\ref{def:supercycle.action-blocked}.
Since $\act$ was chosen arbitrarily, we conclude 
$(\fa \act \in \preds{\B}{s}{\B_i} : \scyc{\B}{s}{\act})$.



%%%%%%%%%%%%%%
\textit{Proof of \clause{sc-membership:act-out}}.
%
Assume $\scyc{\B}{s}{\act}$, and let $\SC \subg \wfg{\B}{s}$ be the supercycle containing $\act$.  By
\defn{supercycle}, \clause{supercycle.action-blocked}, there exists some
$\B_i \in \succs{\B}{s}{\act}$ such that $\B_i \in \SC$.  Hence $\scyc{\B}{s}{\B_i}$.
We conclude
$\scyc{\B}{s}{\act} \imp (\ex \B_i \in \succs{\B}{s}{\act} : \scyc{\B}{s}{\B_i})$.
%
Now assume $(\ex \B_i \in \succs{\B}{s}{\act} : \scyc{\B}{s}{\B_i})$, and let 
$\SC \subg \wfg{\B}{s}$ be the supercycle containing some $\B_i \in \succs{\B}{s}{\act}$. 
Let $\SC'$ be $\SC$ with $\act \ar \B_i$ added. Then $\SC'$ is a supercycle by 
\defn{supercycle}, and also $\SC' \subg \wfg{\B}{s}$. Hence $\scyc{\B}{s}{\act}$.
We conclude 
$\scyc{\B}{s}{\act} \folf (\ex \B_i \in \succs{\B}{s}{\act} : \scyc{\B}{s}{\B_i})$.


%%%%%%%%%%%%%%%%%%%%%%%%
\textit{Proof of \clause{sc-membership:act-in}}.  
%
Assume $\neg \scyc{\B}{s}{\act}$, so that $\act$ is not in any supercycle of $\wfg{\B}{s}$.
Let $\B_i \in \preds{\B}{s}{\act}$. 
By \defn{supercycle}, \clause{supercycle.component-blocked}, 
$\B_i$ cannot be in any supercycle of $\wfg{\B}{s}$, since all $\actp \in \succs{\B}{s}{\B_i}$ must
also be in the supercycle. Hence $\neg \scyc{\B}{s}{\B_i}$.
Since $\B_i$ was chosen arbitrarily, we conclude
$\neg \scyc{\B}{s}{\act} \imp  (\fa \B_i \in \preds{\B}{s}{\act} : \neg \scyc{\B}{s}{\B_i})$, the
contrapositive of \clause{sc-membership:act-in}.  
\end{proof}


Note that \clause{sc-membership:comp-in} cannot be strengthened to an equivalence: if all
the interactions that wait for a component $\B_i$ are in a supercycle, then $\B_i$ itself may or may
not be in a supercycle, depending on whether $\B_i$ is waiting for some other interaction $\actp$ that is not in a
supercycle.
%
Likewise, \clause{sc-membership:act-in} cannot be strengthened to an equivalence: if $\act$
is in a supercycle, then any component $\B_i$ that waits for $\act$ may or may not be in a 
supercycle, depending on whether $\B_i$ is waiting for some other interaction $\actp$ that is not in a supercycle. 

While \prop{sc-membership} gives relationships between supercycle membership of a node and both its
successors and predecessors, nevertheless \defn{supercycle} implies that the ``causality'' of
supercycle-membership of a node $v$ is from the successors of $v$ to $v$, \ie membership of $v$ in a
supercycle is caused only by membership of $v$'s successors in a supercycle. Repeating this step, we
infer that $v$'s supercycle-membership is caused by the subgraph of the wait-for graph that is
reachable from $v$.






% The next two results concern the structure of supercycle. The first
% shows that a supercycle contains at least one strongly connected
% component. The second shows that removing nodes with only simple paths
% leading into them leaves a resulting graph that is also a supercycle,
% \ie that such nodes are not ``essential'' elements of a
% supercycle. This idea is central to our deadlock-freedom condition.

\begin{proposition} \label{prop:supercycle:contains-twoNodes}
Every supercycle $\SC$ contains at least two nodes.
\end{proposition} 

\begin{proof}
By \defn{supercycle}, $\SC$ is nonempty, and so contains at least one node $v$.
If $v$ is an interaction $\act$, then by \defn{supercycle}, $\SC$ also contains some component $\B_i$ such that 
$\act \ar \B_i$. 
If $v$ is a  component $\B_i$, then, by assumption, $\B_i$ enables at least one interaction $\act$, and by 
\defn{supercycle}, every interaction that $\B_i$ enables must be in $\SC$.
Hence in both cases, $\SC$ contains at least two nodes.
\end{proof}





\begin{proposition} \label{prop:supercycle:contains-mssc}
Every supercycle $\SC$ contains a maximal strongly connected component $\CC$
such that (1) $\CC$ is itself a supercycle, and (2) there is no wait-for edge from a node in $\CC$ to a node outside of $\CC$.
\end{proposition}
%
\begin{proof}
$\SC$ is a directed graph, and so consider the decomposition of $\SC$
into its maximal strongly connected components (MSCC). Let $\mscc{\SC}$ be the graph resulting
from replacing each MSCC by a single node. By its construction,  $\mscc{\SC}$ is acyclic, and so contains at least one
node $x$ with no outgoing edges. Let $\CC$ be the MSCC corresponding to $x$.
%
It follows from the construction of $\CC$ that no node in $\CC$ has a wait-for edge going to a node outside of
$\CC$, and so Clause (2) of the Proposition is established.


It also follows from the construction of $\CC$ that $\CC$ is nonempty, and
hence $\CC$ satisfies clause (1) of \defn{supercycle}.
Let $v$ be an arbitrary node in $\CC$. Since $\CC \subg \SC$, $v$ is a node of $\SC$. Let $w$ be an arbitrary successor of
$v$ in $\SC$. Since no node in $\CC$ has an edge going to a node outside of $\CC$, it follows that $w$ is a node of $\CC$.
Hence $v$ has the same successors in $\CC$ as in $\SC$. 
Now since $\SC$ is a supercycle, every vertex $v$ in $\SC$ has enough successors in $\SC$ to satisfy clauses (2) and (3)
of \defn{supercycle}. It follows that every vertex $v$ in $\CC$ has enough successors in
$\CC$ to satisfy clauses (2) and (3) of \defn{supercycle}.  
%
Hence, by \defn{supercycle}, $\CC$ is itself a supercycle, and so Clause (1) of the Proposition is established.
\end{proof}

Note also that by Proposition~\ref{prop:supercycle:contains-twoNodes}, $\CC$ contains at least two nodes. Hence $\CC$ is
not a trivial strongly connected component.






\begin{definition}[Path, path length] \label{def:path} \label{defn:path}
Let $G$ be a directed graph and $v$ a vertex in $G$. A path $\pi$ in $G$ is a \emph{finite} sequence
$v_0, v_1, \ldots,v_n$ such that $(v_i, v_{i+1})$ is an edge in $G$ for all $i \in \rng{0}{n-1}$.
Write $\pth{G}{\pi}$ iff $\pi$ is a path in $G$.
Define $\first{\pi} = v_0$ and $\last{\pi} = v_n$. 
%
Let $|\pi|$ denote the length of $\pi$, which we define as follows:
\be
\item if $\pi$ is simple, \ie all $v_i$, $0 \le i \le n$, are distinct, then $|\pi| = n$, \ie the
number of edges in $\pi$
\item if $\pi$ contains a cycle, \ie there exist $v_i, v_j$ such that $i \ne j$ and $v_i = v_j$, then
$|\pi| = \omega$ ($\omega$ for ``infinity'').
\ee
\end{definition}

\begin{definition}[In-depth, Out-depth] \label{def:depth} \label{defn:depth} 
Let $G$ be a directed graph and $v$ a vertex in $G$. Define the in-depth of $v$ in $G$, notated as
$\idepth{G}{v}$, as follows:
\be
\item if there exists a path $\pi$ in $G$ that contains a cycle and ends in $v$, \ie $|\pi| = \omega
  \land \last{\pi} = v$, then $\idepth{G}{v} = \omega$,
%THIS DEFINITION OF INFINITE IN-DEGREEE IS STRANGE, SINCE YOU HAVE ``INFINITE' PATHS THAT NEVERTHELESS END IN A NODE!

\item otherwise, let $\pi$ be a longest (simple) path ending in $v$. Then $\idepth{G}{v} = |\pi|$.
\ee
Formally, $\idepth{G}{v} = (\MAX\ \pi : \pth{G}{\pi} \land \last{\pi} = v : |\pi|)$.

Likewise define the out-depth of $v$ in $G$, notated as
$\odepth{G}{v}$, as follows:
\be
\item if there exists a path $\pi$ in $G$ that contains a cycle and starts in $v$, \ie $|\pi| = \omega
  \land \first{\pi} = v$, then $\odepth{G}{v} = \omega$,

\item otherwise, let $\pi$ be a longest (simple) path starting in $v$. Then $\odepth{G}{v} = |\pi|$.
\ee
Formally, $\odepth{G}{v} = (\MAX\ \pi : \pth{G}{\pi} \land \first{\pi} = v : |\pi|)$.
\end{definition}

\noindent
We use $\widepth{\B}{v}{s}$ for $\idepth{\wfg{\B}{s}}{v}$, and also
$\wodepth{\B}{v}{s}$ for $\odepth{\wfg{\B}{s}}{v}$.
%
A node with finite in-depth is not reachable from any non-trivial (\ie consisting of more than one node) MSCC, and a node with finite out-depth cannot
reach any non-trivial MSCC.







\begin{proposition} \label{prop:outdepth-finite-implies-scViol}
Assume that node $v$ of $\wfg{\B}{s}$ has a finite out-depth of $d \ge 1$ in $\wfg{\B}{s}$, \ie 
$\wodepth{\B}{v}{s} = d$. Then $\viol{v}{d+1}{s}$. 
\end{proposition}
%
\begin{proof}
Proof is by induction on $d$.\\

\noindent
\ul{Base case, $d=0$.} Hence by $\wodepth{\B}{v}{s} = 0$ and Definitions~\ref{def:path} and \ref{def:depth},  
$v$ has no outgoing wait-for edges in $\wfg{\B}{s}$. Hence $\neg \blocks{s}{v}{\wfg{\B}{s}}$, \ie $v$ is not blocked by the entire set of nodes in 
$\wfg{\B}{s}$. Hence $\neg \blocks{s}{v}{\compl{\ewfg}}$, since $\wfg{\B}{s} = \compl{\ewfg}$. So by \defn{violFix}, 
$v \in \VFs{s}{\ewfg}$. By \defn{supercycle.violation}, $\scVd{v}{1}{s}$.\\

\noindent
\ul{Inductive step, $d > 0$.}
Let $w$ be an arbitrary successor of $v$, \ie a node $w$ such that $v \ar w \in \wfg{\B}{s}$.
By Definitions~\ref{def:path} and \ref{def:depth}, $w$ has an out-depth $d'$ that is less than $d$. 
That is, $\wodepth{\B}{u}{s} = d' < d$.
By the induction hypothesis applied to $d'$, we obtain $\viol{w}{d'+1}{s}$, and so $w \in \VFsi{s}{\ewfg}{d'+1}$ by \defn{supercycle.violation}.
Hence $w \in \VFsi{s}{\ewfg}{d}$, since, by monotonicity of $\VFsymb_s$, we have 
$ \VFsi{s}{\ewfg}{d'} \subg  \VFsi{s}{\ewfg}{d}$ when $d' \le d$.
Since  $w$ is an arbitrary successor of $v$, it follows that $v$ is only blocked by nodes in $\VFsi{s}{\ewfg}{d}$.
Hence $\neg \blocks{s}{v}{\compl{ \VFsi{s}{\ewfg}{d} } }$.
By \defn{violFix}, $v \in \VFs{s}{\VFsi{s}{\ewfg}{d}}$, \ie $v \in  \VFsi{s}{\ewfg}{d+1}$.
By \defn{supercycle.violation}, $\scVd{v}{d+1}{s}$.
\end{proof}




\begin{corollary} 
\label{cor:supercycle:no-finite-outdepth}
A supercycle $\SC$ contains no nodes with finite out-depth.
\end{corollary}
%
\begin{proof} 
Let $v$ be a node in $\SC$ with finite out-depth $d$.
By \prop{outdepth-finite-implies-scViol}, $\scVd{v}{d+1}{s}$.
By \defn{supercycle.violation}, $\scV{v}{s}$.
By \prop{scViol-iff-notInSC} $\neg \scyc{\B}{t}{v}$. Hence $v$ cannot be a node of any supercycle, and we have a contradiction.
\end{proof}








\begin{proposition} \label{prop:supercycle:contains-cycle}
Every supercycle $\SC$ contains at least one cycle.
\end{proposition} 
%
\begin{proof}
By contradiction. Suppose that $\SC$ is a supercycle and is also acyclic. Then every path in $\SC$ is simple, and therefore finite.  Hence every
node in $\SC$ has finite out-depth. By \prop{outdepth-finite-implies-scViol}, $\SC$ cannot be a supercycle.
\end{proof}


\begin{proposition} \label{prop:supercycle:essential-subgraph-of} 
%Let $\B = \gamma(\B_1,\dots,\B_n)$ be a composite component and $s$ a state of $\B$.
Let $\SC$ be a supercycle in $\wfg{\B}{s}$, and let $\SC'$ be the
graph obtained from $SC$ by removing all vertices of finite in-depth
and their incident edges. Then $\SC'$ is also a supercycle in
$\wfg{\B}{s}$. 
\end{proposition} 
%
\begin{proof}
A vertex with finite in-depth cannot lie on a cycle in $\SC$.  Hence
by Proposition~\ref{prop:supercycle:contains-cycle}, $\SC' \neq
\emptyset$. Thus $\SC'$ satisfies clause (1) of the supercycle
definition~(\ref{def:supercycle}).
%
Let $v$ be an arbitrary vertex of $\SC'$.  Thus $v \in \SC$ and $\idepth{\SC}{v} = \omega$ by definition of $\SC'$. Let
$w$ be an arbitrary successor of $v$ in $\SC$, \ie $v \ar w \in \SC$.
Hence $\idepth{\SC}{w} = \omega$ by \defn{depth}. Hence $w \in \SC'$, by definition of $SC'$.
Furthermore, $v \ar w \in \SC'$, since $\SC'$ consists of
\emph{all} nodes of $\SC$ with infinite in-depth. Hence the successors of $v$ in $\SC'$ are
the same as the successors of $v$ in $\SC$
%
Now since $\SC$ is a supercycle, every vertex $v$ in $\SC$ has enough successors in $\SC$ to satisfy clauses (2) and (3)
of the supercycle definition~(\ref{def:supercycle}). It follows that every vertex $v$ in $\SC'$ has enough successors in
$\SC'$ to satisfy clauses (2) and (3) of the supercycle definition~(\ref{def:supercycle}).  
Hence $\SC'$ is a supercycle in $\wfg{\B}{s}$. 
\end{proof}

%%%%%%%%%%%%%%%%%%%%%%%%%%%%%%%%%%%%%%%%%%%%%
\subsection{A fixpoint characterization of supercycles}
\label{secn:supercycle-fixpoint}

\begin{definition}[Set of subgraphs] \label{defn:wsetOfSubgraphs}
$\wfgS{B}{s} = \set{ W \stt W \subg \wfg{B}{s} }$.
\end{definition}


\begin{definition}[Wait-for lattice] \label{defn:wflattice}
Define the partially ordered set $\lat{\B}{s} = \tpl{ \wfgS{B}{s}, \subg }$ %\tpl{ \wfg{B}{s}, \ewfg, \subg }$
whose elements are all the subgraphs of 
\wfg{B}{s}, and where  $U \subg V$ is as in \defn{wsubgraph}.   
%iff $U$ is a subgraph of $V$, \ie $\ord$ is the ``is a subgraph of'' order relation.
\end{definition}

The following proposition follows immediately from the definitions; its proof is left to the reader.
\begin{proposition} \label{prop:isALattice}
$\lat{\B}{s} = \tpl{ \wfgS{B}{s}, \subg }$ is a complete boolean lattice as follows:
\be

\item meet is given by graph intersection: 
$\US \meet \VS$ consists of the nodes that are present in both $\US$ and $\VS$, together with their induced edges

\item join is given by graph union: $\US \join \VS$ consists of the nodes and edges that are present in $\US$, or in $\VS$, or in both, together with their
  induced edges.
 Note that $\join$ is \emph{not} disjoint graph union: 
it is possible for $\US$ and $\VS$ to have nodes and edges in common. Note also that $\US \join \VS$ may contain edges not present in either $\US$ nor $\VS$,
since the edges are those induced by  \wfg{B}{s}. 

\item \wfg{B}{s}  is the top element 

\item  the empty wait-for graph, denoted by \ewfg, is the bottom element

\item the complement of $\US$ is \compl{\US} obtained by taking all the nodes of $\US$ that are not in \wfg{B}{s}, together with the induced edges.
\ee
\end{proposition}
%
As noted, $\join, \meet$ and complement are determined entirely by the sets of nodes in the relevant subgraphs. The resulting edges are always those
that are induced by \wfg{B}{s}.



\begin{definition}[$\mathit{blocks}, \SFs, \VFs$] \label{defn:blocks} 
Let $\US \subg \wfg{B}{s}$ and $\act, B_i$ be nodes in $\wfg{B}{s}$. Then 
$\blocks{\act}{\US} \df (\ex B_i \in \US : \act \ar B_i \in \wfg{B}{s})$, and 
$\blocks{\B_i}{\US} \df (\fa \act : \B_i \ar \act \in \wfg{B}{s} \imp \act \in \US)$.
\end{definition}

\begin{definition} \label{defn:scFix} 
Define $\SFs: \wfgS{B}{s} \to \wfgS{B}{s}$ as follows.
$\SF{\US}$ is the subgraph with nodes $\set{v \stt \blocks{v}{\US} }$, together with their induced edges.
\end{definition}

\begin{definition} \label{defn:violFix}
Define $\VFs: \wfgS{B}{s} \to \wfgS{B}{s}$ as follows.
$\VF{\US}$ is the subgraph with nodes ${\set{v \stt \neg \blocks{v}{\compl{\US}}}}$, together with their induced edges.
\end{definition}
%
Hence $\VF{\UU} = \overline{\SF{\overline{\US}}}$, \ie $\VFs$ and $\SFs$ are duals.

\begin{proposition} \label{prop:monotone}
$\SFs$ and $\VFs$ are monotone and continuous.
\end{proposition}
%
\begin{proof}
% show its monotnoe
We show first that $\SFs$ is monotonic, \ie $\US \subg \VS \imp \SF{\US} \subg \SF{\VS}$.
Let $\ndv$ be an arbitrary node in $\SF{\US}$, so that $\blocks{\ndv}{\US}$ holds. There are two cases.\\

\emph{Case of $\ndv$ is an interaction $\act$}. By Definitions~\ref{defn:scFix} and \ref{defn:blocks}, we have $\ex \B_i \in \US  : \act \ar \B_i \sub \wfg{B}{s}$.
Since $\US \subg \VS$, this same $\B_i$ is also a node of $\VS$, and so  $\ex \B_i \in \VS  : \act \ar \B_i \sub \wfg{B}{s}$. 
Hence  $\blocks{\act}{\VS}$, and so $\act \in \SF{\VS}$. \\

\emph{Case of $\ndv$ is a component $\B_i$}. By Definitions~\ref{defn:scFix} and \ref{defn:blocks}, we have $(\fa \act : \B_i \ar \act \in \wfg{B}{s} \imp \act \in \US)$.
Since $\US \subg \VS$, we have  $(\fa \act : \B_i \ar \act \in \wfg{B}{s} \imp \act \in \VS)$. 
Hence $\blocks{\B_i}{\VS}$, and so $\B_i \in \SF{\VS}$. \\

In both cases, we have $\ndv \in \SF{\VS}$. Since $\ndv$ was chosen arbitrarily from $\SF{\US} $, it follows that $\SF{\US} \subg \SF{\VS}$. Hence
$\SFs$ is monotone.
%
Since the dual of a monotone mapping in a complete boolean lattice is also monotone, we have that $\VFs$ is monotone.
%
Finally, since $\lat{\B}{s}$ is finite, it follows that $\SFs$ and $\VFs$ are continuous.
\end{proof}

Hence, by the Knaster-Tarski theorem, the least and greatest fixpoints of $\SFs$ and $\VFs$ exist.
%, and so their greatest and least fixpoints 

\begin{proposition} \label{prop:supercycleGFP}
Let $\US \subg \wfg{B}{s}$. Then $\US$ is a supercycle in \wfg{B}{s} iff $\US \subg \SF{\US}$.
\end{proposition}
%
\begin{proof}
Let $\US$ be a supercycle in \wfg{B}{s}. By \defn{supercycle}, every node in $\US$ is blocked by $\US$, \ie 
$(\fa \ndv \in \US: \blocks{\B_i}{\US})$. By \defn{scFix}, $\US \subg \SF{\US}$.
%
Conversely, suppose $\US \subg \SF{\US}$ for some subgraph $\US$ of \wfg{B}{s}. Hence 
$(\fa \ndv \in \US: \ndv \in \SF{\US})$, so by \defn{scFix}, $(\fa \ndv \in \US: \blocks{\B_i}{\US})$.
Hence every node in $\US$ is blocked by $\US$, and so $\US$ satisfies \defn{supercycle}, and is therefore a supercycle.
\end{proof}
%
Thus the supercycles of \wfg{B}{s} are exactly the post-fixpoints of $\SFs$. %This implies that the union of two supercycles is also a supercycle. 

\begin{proposition} \label{prop:GFPisLargestSC}
Let $\SC$ be the greatest fixpoint of $\SFs$. Then either
(a) \wfg{B}{s} is supercycle-free and $\SC = \ewfg$, or 
(b) \wfg{B}{s} contains supercycles, and $\SC$ is the largest supercycle in \wfg{B}{s}
\end{proposition}
%
\begin{proof}
By the Knaster-Tarski theorem, the greaterst fixpoint is the join of all the post-fixpoints. 
If \wfg{B}{s} is supercycle-free, then by \prop{supercycleGFP}, the set of post-fixpoints of $\SFs$ is empty. 
Hence $\SC = \ewfg$.
If \wfg{B}{s} contains supercycles, then by \prop{supercycleGFP},  the set of post-fixpoints of $\SFs$ is exactly the set of 
supercycles of \wfg{B}{s}, and so $\SC$ is the join of all these supercycles, and so $\SC$ is the largest supercycle in \wfg{B}{s}.
\end{proof}

Let $\MATHIDN{lfp}, \MATHIDN{gfp}$ denote the least-fixed-point and greatest-fixed-point operators, respectively.

\begin{proposition}  \label{prop:LFPisScViolations}
Let  $V= \lfp{\VFs}$, \ie $V$ is the least fixpoint of $\VFs$. Then $v \in V$ iff $v$ is not a node in any supercycle of \wfg{B}{s}.    
%the nodes of $V$ are exactly the nodes in \wfg{B}{s} that have supercycle violations. 
\end{proposition}
%
\begin{proof}
%Follows from \prop{GFPisLargestSC} are the Park conjugate (dual) fixpoint theorem in complete boolean lattices.
From the Park conjugate (dual) fixpoint theorem in complete boolean lattices, we have 
\lfp{\VFs} = \compl{\gfp{\SFs}}.
By \prop{GFPisLargestSC} , \gfp{\SFs} is the largest supercycle in \wfg{B}{s}. Hence the nodes not in 
\gfp{\SFs} are exactly the nodes that have supercycle violations. These are exactly the nodes in \lfp{\VFs}.
\end{proof}


\begin{proposition}  \label{prop:computeLFP}
$\lfp{\VFs} = \JOIN_{d \ge 0} \VFs^{d} (\ewfg)$.
\end{proposition}
%
\begin{proof}
By \prop{monotone}, $\VFs$ is continuous. Follows by standard results, \eg see the CPO fixpoint theorem I in 
\cite{DP02}.
\end{proof}



\begin{definition}[Supercycle violation, $\scV{v}{s}$, $\scVd{v}{d}{s}$]
\label{def:supercycle-violation}
\label{def:supercycle.violation}
\label{defn:supercycle.violation}
$\scV{v}{s} \df v \in \lfp{\VFs}$ and 
$\scVd{v}{d}{s} \df v \in \VFs^{d} (\ewfg)$.
\end{definition}
%
It immediately follows that $\scV{v}{s}$ iff  $(\ex d \ge 0: \scVd{v}{d}{s})$.
Also, it follows from \prop{LFPisScViolations} that $\scV{v}{s}$ iff there does not exist $\SC$ such that $\SC$ is a supercycle and $v \in SC$.
We say that a node $v$ of \wfg{B}{s} has a \emph{supercycle violation}\footnote{In the sequel, we say ``sc-violation'' rather than ``supercycle violation.''} iff $v$ is not a node in any supercycle of \wfg{B}{s}, 
\ie iff $\scV{v}{s}$ holds. 
By \prop{computeLFP}, we can compute $\lfp{\VFs}$ (and therefore $\scV{v}{s}$) by iterating $\VFs$, starting from $\ewfg$, until there is no more change.
%
For example, consider Figure~\ref{fig:SCnotCycle}. We have 
$\VFs^{0} (\ewfg) = \ewfg$,
$\VFs^{1} (\ewfg) = \set{i}$, 
$\VFs^{2} (\ewfg) = \set{B_6, i}$, 
$\VFs^{3} (\ewfg) = \set{h, B_6, i}$, 
$\VFs^{4} (\ewfg) = \set{B_5, h, B_6, i}$, 
$\VFs^{5} (\ewfg) = \set{B_5, h, B_6, i}$, 
as so $\lfp{\VFs} = \set{B_5, h, B_6, i}$.


%In the sequel, we say sc-violation rather than ``supercycle violation.''  The crucial result is
%that, if $v$ has a level-$d$ sc-violation, for some $d \ge 1$, then $v$ cannot be a node of a
%supercycle.



\begin{definition}[Supercycle membership, $\scyc{\B}{s}{v}$]
\label{defn:supercycle.membership}
Let $v$ be a node of $\wfg{\B}{s}$. Then
$\scyc{\B}{s}{v}$ holds iff there exists a supercycle $SC \sub
\wfg{\B}{s}$ such that $v \in SC$. 
%Define $\scyc{\B}{s}{\B_i}$, $\scyc{\B}{s}{\act}$ to mean that $\B_i$,
%$\act$, respectively, are nodes of a supercycle of $\wfg{\B}{s}$.
\end{definition}

If a component or interaction is not a node of a supercycle, then we say that it has a
\emph{SC-violation}, \ie a supercycle-violation.
%



\begin{proposition} \label{prop:scViol-iff-notInSC}
$\neg \scyc{\B}{t}{v}$ iff $\scV{v}{t}$ % $(\ex d \ge 1: \viol{v}{d}{t})$.
\end{proposition}
%
\begin{proof}
Immediate from \defn{supercycle.membership} and \prop{LFPisScViolations}.
\end{proof}

%\prf{Immediate from Propositions~\ref{prop:scViol-implies-notInSC} and
%  \ref{prop:notInSC-implies-scViol}.}







%%%%%%%%%%%%%%%%%%%%%%%%%%%%%%%%%%%%%%%%%%%%%
\subsection{Structural properties of supercycles}
\label{secn:supercycle-structural}

%\subsection{Supercycle Membership} 

We present some structural properties of supercycles, which
are central to our deadlock-freedom conditions.


\begin{proposition} \label{prop:supercycle:union}
Let $\SC, \SC'$ be supercycles in $\wfg{B}{s}$. Then $\SC \join \SC'$ is
a supercycle in $\wfg{B}{s}$.
\end{proposition}
%
\begin{proof}
By \prop{supercycleGFP}, $\SC$ and $\SC'$ are post-fixpoints of $\SFs$. Since the join of post-fixpoints is a post-fixpoint, 
the proposition follows by applying \prop{supercycleGFP} again.
%Straightforward, since each node in  $\SC \un \SC'$ has enough successors that it waits for to satisfy  \defn{supercycle}. 
\end{proof}





Define
$\preds{\B}{s}{v} = \set{w \stt w \ar v \in \wfg{\B}{s}}$ and 
$\succs{\B}{s}{v} = \set{w \stt v \ar w \in \wfg{\B}{s}}$.
The definition of a supercycle (\defn{supercycle}) 
imposes certain constraints on supercycle membership of a node \wrt its predecessors and successors
in the wait-for-graph, as follows:

%this is not used anywhere, and is present only for its own sake
\begin{proposition}[Supercycle-membership constraints]
\label{prop:sc-membership}
Let $\act, \B_i$ be nodes of $\wfg{\B}{s}$. Then
\bn

\item \label{clause:sc-membership:comp-out}
$\scyc{\B}{s}{\B_i} \ifof (\fa \act \in \succs{\B}{s}{\B_i} : \scyc{\B}{s}{\act})$.

\item \label{clause:sc-membership:comp-in}
$\scyc{\B}{s}{\B_i} \imp (\fa \act \in \preds{\B}{s}{\B_i} : \scyc{\B}{s}{\act})$.

\item \label{clause:sc-membership:act-out}
$\scyc{\B}{s}{\act} \ifof (\ex \B_i \in \succs{\B}{s}{\act} : \scyc{\B}{s}{\B_i})$.

\item \label{clause:sc-membership:act-in}
$\scyc{\B}{s}{\act} \folf (\ex \B_i \in \preds{\B}{s}{\act} : \scyc{\B}{s}{\B_i})$.

\en
\end{proposition}
%
\begin{proof}
We deal with each clause in turn.


%%%%%%%%%%%%%%%%%%%%%
\textit{Proof of \clause{sc-membership:comp-out}}.
%
Assume $\scyc{\B}{s}{\B_i}$, and let $\SC \subg \wfg{B}{s}$ be the supercycle containing $\B_i$.  Let
$\actp \in \succs{\B}{s}{\B_i}$.  By \defn{supercycle}, \clause{supercycle.component-blocked},
$\actp \in \SC$.  Hence $(\fa \act \in \succs{\B}{s}{\B_i} : \scyc{\B}{s}{\act})$.
We conclude
$\scyc{\B}{s}{\B_i} \imp (\fa \act \in \succs{\B}{s}{\B_i} : \scyc{\B}{s}{\act})$.
%
Now assume $(\fa \act \in \succs{\B}{s}{\B_i} : \scyc{\B}{s}{\act})$, and let 
$\SC$ be the join of all the supercycles containing all the $\act \in \succs{\B}{s}{\B_i}$. 
By \prop{supercycle:union}, $\SC \subg \wfg{B}{s}$ is a supercycle.
Let $\SC'$ be $\SC$ with $\B_i \ar \act$ added, for all 
$\act \in \succs{\B}{s}{\B_i}$.
Then $\SC'$ is a supercycle by 
\defn{supercycle}, and also $\SC' \subg \wfg{B}{s}$. Hence $\scyc{\B}{s}{\act}$.
We conclude 
$\scyc{\B}{s}{\B_i} \folf (\fa \act \in \succs{\B}{s}{\B_i} : \scyc{\B}{s}{\act})$.




%%%%%%%%%%%%%%%%%%%%%%%%%%
\textit{Proof of \clause{sc-membership:comp-in}}.
%
Assume $\scyc{\B}{s}{\B_i}$, so that $\SC \subg \wfg{B}{s}$ is the supercycle containing $\B_i$.
Let  $\act \in \preds{\B}{s}{\B_i}$, and let $\SC'$ be $\SC$ with 
$\act \ar \B_i$ added. Hence $\SC'$ is a supercycle 
by Definition~\ref{def:supercycle},
  Clause~\ref{def:supercycle.action-blocked}.
Since $\act$ was chosen arbitrarily, we conclude 
$(\fa \act \in \preds{\B}{s}{\B_i} : \scyc{\B}{s}{\act})$.



%%%%%%%%%%%%%%
\textit{Proof of \clause{sc-membership:act-out}}.
%
Assume $\scyc{\B}{s}{\act}$, and let $\SC \subg \wfg{B}{s}$ be the supercycle containing $\act$.  By
\defn{supercycle}, \clause{supercycle.action-blocked}, there exists a
$\B_i \in \succs{\B}{s}{\act}$ such that $\B_i \in \SC$.  Hence $\scyc{\B}{s}{\B_i}$.
We conclude
$\scyc{\B}{s}{\act} \imp (\ex \B_i \in \succs{\B}{s}{\act} : \scyc{\B}{s}{\B_i})$.
%
Now assume $(\ex \B_i \in \succs{\B}{s}{\act} : \scyc{\B}{s}{\B_i})$, and let 
$\SC \subg \wfg{B}{s}$ be the supercycle containing some $\B_i \in \succs{\B}{s}{\act}$. 
Let $\SC'$ be $\SC$ with $\act \ar \B_i$ added. Then $\SC'$ is a supercycle by 
\defn{supercycle}, and also $\SC' \subg \wfg{B}{s}$. Hence $\scyc{\B}{s}{\act}$.
We conclude 
$\scyc{\B}{s}{\act} \folf (\ex \B_i \in \succs{\B}{s}{\act} : \scyc{\B}{s}{\B_i})$.


%%%%%%%%%%%%%%%%%%%%%%%%
\textit{Proof of \clause{sc-membership:act-in}}.  
%
Assume $\neg \scyc{\B}{s}{\act}$, so that $\act$ is not in any supercycle of $\wfg{B}{s}$.
Let $\B_i \in \preds{\B}{s}{\act}$. 
By \defn{supercycle}, \clause{supercycle.component-blocked}, 
$\B_i$ cannot be in any supercycle of $\wfg{B}{s}$, since all $\actp \in \succs{\B}{s}{\B_i}$ must
also be in the supercycle. Hence $\neg \scyc{\B}{s}{\B_i}$.
Since $\B_i$ was chosen arbitrarily, we conclude
$\neg \scyc{\B}{s}{\act} \imp  (\fa \B_i \in \preds{\B}{s}{\act} : \neg \scyc{\B}{s}{\B_i})$, the
contrapositive of \clause{sc-membership:act-in}.  
\end{proof}


Note that \clause{sc-membership:comp-in} cannot be strengthened to an equivalence: if all
the interactions that wait for a component $\B_i$ are in a supercycle, then $\B_i$ itself may or may
not be in a supercycle, depending on whether $\B_i$ is waiting for some $\actp$ that is not in a
supercycle.
%
Likewise, \clause{sc-membership:act-in} cannot be strengthened to an equivalence: if $\act$
is in a supercycle, then any component $\B_i$ that waits for $\act$ may or may not be in a 
supercycle, depending on whether $\B_i$ is waiting for some $\actp$ that is not in a supercycle. 




While \prop{sc-membership} gives relationships between supercycle membership of a node and both its
successors and predecessors, nevertheless \defn{supercycle} implies that the ``causality'' of
supercycle-membership of a node $v$ is from the successors of $v$ to $v$, \ie membership of $v$ in a
supercycle is caused only by membership of $v$'s successors in a supercycle. Repeating this step, we
infer that $v$'s supercycle-membership is caused by the subgraph of the wait-for graph that is
reachable from $v$.






% The next two results concern the structure of supercycle. The first
% shows that a supercycle contains at least one strongly connected
% component. The second shows that removing nodes with only simple paths
% leading into them leaves a resulting graph that is also a supercycle,
% \ie that such nodes are not ``essential'' elements of a
% supercycle. This idea is central to our deadlock-freedom condition.

\begin{proposition} \label{prop:supercycle:contains-twoNodes}
Every supercycle $\SC$ contains at least two nodes.
\end{proposition} 

\begin{proof}
By Definition~\ref{def:supercycle}, $\SC$ is nonempty, and so contains at least one node $v$.
If $v$ is an interaction $\act$, then by Definition~\ref{def:supercycle}, $\SC$ also contains some component $\B_i$ such that 
$\act \ar \B_i$. 
If $v$ is a  component $\B_i$, then, by assumption, $\B_i$ enables at least one interaction $\act$, and by 
Definition~\ref{def:supercycle}, every interaction that $\B_i$ enables must be in $\SC$.
Hence in both cases, $\SC$ contains at least two nodes.
\end{proof}


\begin{proposition} \label{prop:supercycle:contains-cycle}
Every supercycle $\SC$ contains at least one cycle.
\end{proposition} 
%
\prfs{Suppose not. Then $\SC$ is an acyclic supercycle. Hence every node in $\SC$
 has finite out-depth, which contradicts
 Proposition~\ref{prop:supercycle:no-finite-outdegree}. }
%
\prf{By contradiction. Suppose that $\SC$ is a supercycle and is also
  acyclic. Then every path in $\SC$ is simple, and therefore finite.
  Hence every node in $\SC$ has finite out-depth. By
  Proposition~\ref{prop:supercycle:no-finite-outdegree}, $\SC$ cannot
  be a supercycle.}



\begin{proposition} \label{prop:supercycle:essential-subgraph-of} Let $\B =
\gamma(\B_1,\dots,\B_n)$ be a composite component and $s$ a state of
$\B$.  Let $SC$ be a supercycle in $\wfg{\B}{s}$, and let $SC'$ be the
graph obtained from $SC$ by removing all vertices of finite in-depth
and their incident edges. Then $SC'$ is also a supercycle in
$\wfg{\B}{s}$. 
\end{proposition} 
%
% \prfs{
% By Proposition~\ref{prop:supercycle:contains-cycle}, $\SC'$ is nonempty.
% Thus $SC'$ satisfies clause (1) of Definition~\ref{def:supercycle}.
% %
% Let $v$ be an arbitrary vertex of $\SC'$. Hence $v$ has infinite in-depth, and
% therefore so do all of $v$'s successors in $\SC$. Hence all of these successors
% are in $\SC'$. 
% Hence every vertex $v$ in $SC'$ has successors
% in $\SC'$ that satisfy clauses (2) and (3) of Definition~\ref{def:supercycle}.
% }
%
\begin{proof}
A vertex with finite in-depth cannot lie on a cycle in $\SC$.  Hence
by Proposition~\ref{prop:supercycle:contains-cycle}, $\SC' \neq
\emptyset$. Thus $\SC'$ satisfies clause (1) of the supercycle
definition~(\ref{def:supercycle}).
%
Let $v$ be an arbitrary vertex of $\SC'$.  Thus $v \in \SC$ and $\idepth{\SC}{v} = \omega$ by definition of $SC'$. Let
$w$ be an arbitrary successor of $v$ in $\SC$. $\idepth{\SC}{w} = \omega$ by Definition~\ref{def:depth}. Hence
$w \in \SC'$, by definition of $SC'$.  Furthermore, $w$ is a successor of $v$ in $\SC'$, since $SC'$ consists of
\emph{all} nodes of $SC$ with infinite in-depth. Hence the successors of $v$ in $\SC'$ are
the same as the successors of $v$ in $\SC$
%
%Thus every vertex $v$ of $\SC'$ is also a vertex of $\SC$, and 
%
Now since $\SC$ is a supercycle, every vertex $v$ in $\SC$ has enough successors in $\SC$ to satisfy clauses (2) and (3)
of the supercycle definition~(\ref{def:supercycle}). It follows that every vertex $v$ in $\SC'$ has enough successors in
$\SC'$ to satisfy clauses (2) and (3) of the supercycle definition~(\ref{def:supercycle}).  
\end{proof}

\begin{proposition} \label{prop:supercycle:contains-mssc}
Every supercycle $\SC$ contains a maximal strongly connected component $\CC$
such that (1) $\CC$ is itself a supercycle, and (2) there is no wait-for-edge from a node in $\CC$ to a node outside of $\CC$.
\end{proposition}
%
\begin{proof}
$\SC$ is a directed graph, and so consider the decomposition of $\SC$
into its maximal strongly connected components (MSCC). Let $\overline{\SC}$ be the graph resulting
from replacing each MSCC by a single node. By its construction,  $\overline{\SC}$ is acyclic, and so contains at least one
node $x$ with no outgoing edges. Let $\CC$ be the MSCC corresponding to $x$. It follows that $\CC$ is nonempty, and
hence $\CC$ satisfies clause (1) of the supercycle definition~(\ref{def:supercycle}).
It also follows from the construction of $\CC$ that no node in $\CC$ has a wait-for-edge going to a node outside of
$\CC$, and so Clause (2) of the Proposition is established.

Let $v$ be an arbitrary node in $\CC$. Since $\CC \sub \SC$, $v$ is a node of $\SC$. Let $w$ be an arbitrary successor of
$v$ in $\SC$. Since no node in $\CC$ has an edge going to a node outside of $\CC$, it follows that $w$ is a node of $\CC$.
Hence $v$ has the same successors in $\CC$ as in $\SC$. 
Now since $\SC$ is a supercycle, every vertex $v$ in $\SC$ has enough successors in $\SC$ to satisfy clauses (2) and (3)
of the supercycle definition~(\ref{def:supercycle}). It follows that every vertex $v$ in $\CC$ has enough successors in
$\CC$ to satisfy clauses (2) and (3) of the supercycle definition~(\ref{def:supercycle}).  

Hence, by Definition~\ref{def:supercycle}, $\CC$ is itself a supercycle, and so Clause (1) of the Proposition is established.
\end{proof}

Note also that by Proposition~\ref{prop:supercycle:contains-twoNodes}, $\CC$ contains at least two nodes. Hence $\CC$ is
not a trivial strongly connected component.






\begin{definition}[Path, path length] \label{def:path} \label{defn:path}
Let $G$ be a directed graph and $v$ a vertex in $G$. A path $\pi$ in $G$ is a \emph{finite} sequence
$v_0, v_1, \ldots,v_n$ such that $(v_i, v_{i+1})$ is an edge in $G$ for all $i \in \rng{0}{n-1}$.
Write $\pth{G}{\pi}$ iff $\pi$ is a path in $G$.
Define $\first{\pi} = v_0$ and $\last{\pi} = v_n$. 
%
Let $|\pi|$ denote the length of $\pi$, which we define as follows:
\be
\item if $\pi$ is simple, \ie all $v_i$, $0 \le i \le n$, are distinct, then $|\pi| = n$, \ie the
number of edges in $\pi$
\item if $\pi$ contains a cycle, \ie there exist $v_i, v_j$ such that $i \ne j$ and $v_i = v_j$, then
$|\pi| = \omega$ ($\omega$ for ``infinity'').
\ee
\end{definition}

\begin{definition}[In-depth, Out-depth] \label{def:depth} \label{defn:depth} 
Let $G$ be a directed graph and $v$ a vertex in $G$. Define the in-depth of $v$ in $G$, notated as
$\idepth{G}{v}$, as follows:
\be
\item if there exists a path $\pi$ in $G$ that contains a cycle and ends in $v$, \ie $|\pi| = \omega
  \land \last{\pi} = v$, then $\idepth{G}{v} = \omega$,
%THIS DEFINITION OF INFINITE IN-DEGREEE IS STRANGE, SINCE YOU HAVE ``INFINITE' PATHS THAT NEVERTHELESS END IN A NODE!

\item otherwise, let $\pi$ be a longest (simple) path ending in $v$. Then $\idepth{G}{v} = |\pi|$.
\ee
Formally, $\idepth{G}{v} = (\MAX\ \pi : \pth{G}{\pi} \land \last{\pi} = v : |\pi|)$.

Likewise define the out-depth of $v$ in $G$, notated as
$\odepth{G}{v}$, as follows:
\be
\item if there exists a path $\pi$ in $G$ that contains a cycle and starts in $v$, \ie $|\pi| = \omega
  \land \first{\pi} = v$, then $\odepth{G}{v} = \omega$,

\item otherwise, let $\pi$ be a longest (simple) path starting in $v$. Then $\odepth{G}{v} = |\pi|$.
\ee
Formally, $\odepth{G}{v} = (\MAX\ \pi : \pth{G}{\pi} \land \first{\pi} = v : |\pi|)$.
\end{definition}

\noindent
We use $\widepth{\B}{v}{s}$ for $\idepth{\wfg{\B}{s}}{v}$, and also
$\wodepth{\B}{v}{s}$ for $\odepth{\wfg{\B}{s}}{v}$.



\begin{proposition} \label{prop:supercycle:no-finite-outdegree}
\label{prop:supercycle:no-finite-outdepth} A supercycle $SC$
contains no nodes with finite out-depth.
\end{proposition}
%
% \prfs{By contradiction. Let $v$ be a node in $\SC$ with finite out-depth.
% Hence all outgoing paths from $v$ end in a
% sink node. By assumption, all atomic components are individually
% deadlock-free, \ie they always enable at least one interaction. Hence
% these sink nodes are all interactions, and therefore they violate
% clause~\ref{clause:supercycle:action-blocked} in Definition~\ref{def:supercycle}.
% }
%
\begin{proof} By contradiction. Let $v$ be a node in $\SC$ with finite out-depth.
Hence by Definition~\ref{def:depth}  all outgoing paths from $v$ are
simple (and finite), and end in a
sink node $w$, so $w$ has no outgoing wait-for-edges.
By assumption, all atomic components are individually
deadlock-free, \ie they always enable at least one interaction. So if
$w$ is an atomic component $\B_i$, we have a wait-for-edge $\B_i \ar \act$
for some interaction $\act$, contradicting the fact that $w$ is a sink node.
Hence $w$ is some interaction $\act$.
Since $\act$ has no outgoing edges, it violates
clause~\ref{def:supercycle.action-blocked} in
Definition~\ref{def:supercycle}, contradicting the assumption that
$\SC$ is a supercycle.
\end{proof}

\begin{abstract}
We present a criterion for checking local and global deadlock freedom
of finite state systems expressed in BIP: a component-based framework
for the construction of complex distributed systems.  Our criterion is
evaluated by model-checking a set of subsystems of the overall large
system. If satisfied in small subsystems, it implies deadlock-freedom
of the overall system. If not
satisfied, then we re-evaluate over larger subsystems, which
improves the accuracy of the check.
When the subsystem being checked becomes the entire system, 
our criterion becomes complete for deadlock-freedom.  Hence our criterion
can only fail to decide deadlock-freedom because
of computational limitations: state-space explosion sets in when the
subsystems being checked become too large.
Our method thus combines the possibility of fast
response together with theoretical completeness.
Other criteria for deadlock-freedom, in contrast, are incomplete in
principle, and so may fail to decide deadlock-freedom even if
unlimited computational resources are available.  Also, our
criterion certifies freedom from local deadlock, in which a subsystem
is deadlocked while the rest of the system executes.  We present
experimental results for dining philosophers and for a multi
token-based resource allocation system, which subsumes several data
arbiters and schedulers, including Milner's token based scheduler.
\end{abstract}




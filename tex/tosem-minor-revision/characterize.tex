%\section{Deadlock-freedom}


\begin{definition}[Global Deadlock-freedom]
\label{def:static:deadlock-free}
\label{def:global.deadlock-free} 
\label{defn:static:deadlock-free}
\label{defn:global.deadlock-free} 
A BIP-system $(\B, Q_0)$ is \emph{free of global deadlock} iff 
in every reachable state $s$ of $(\B, Q_0)$, some interaction $\act$ is enabled.
Formally, $\fa s \in \rstates{\B,Q_0}, \ex \act : s \enb{\act}{\B}$.
%% A BIP-system $(\B, Q_0)$ is \emph{global deadlock prone} iff there exists a reachable state
%% $s$ of $(\B, Q_0)$ in which no interaction $a$ is enabled.
%% A BIP-system $(\B, Q_0)$ is \emph{free of global deadlock} iff it is not global deadlock prone.
\end{definition}


\begin{definition}[Local Deadlock-freedom]
\label{def:local.deadlock-free} 
\label{defn:local.deadlock-free} 
A BIP-system $(\B, Q_0)$ is \emph{free of local deadlock} iff 
for every subsystem $(\B', Q'_0)$ of  $(\B, Q_0)$, and every reachable state $s$ of $(\B, Q_0)$,
$(\B', Q'_0)$ has some interaction enabled in state $s \pj \B'$.
Formally:\\
\ind for every subsystem $(\B', Q'_0)$ of  $(\B, Q_0)$:\\
\ind \ind $\fa s \in \rstates{\B,Q_0},  \ex \act : s \pj \B' \enb{\act}{\B'}$.
%A BIP-system $(\B, Q_0)$ is \emph{local deadlock prone}  iff there exists a subsystem $(\B', Q'_0)$ of  $(\B, Q_0)$
%and a reachable state $s$ of $(\B, Q_0)$ such that $(\B', Q'_0)$ has no interaction enabled in state $s \pj \B'$.
\end{definition}

\redbox{
Intuitively, every reachable state $s'$ of subsystem $\B'$ (within the context of the overall system $\B$) is a projection of a reachable state $s$ of $\B$, i.e., $s' = s\pj \B'$. Hence, if $s$ enables an interaction $a$ of $\B'$, then so does $s' = s\pj \B'$.
%  Intuitively, the state space $Q'$ of each susbsystem $\B'$ of $\B$ 
%  overapproximates the state space $Q$ of $\B$. 
%  Consequently, local deadlock freedom in subsystem $\B'$ 
%  requires only those states 
%  that are reachable in $\B$ to enable interactions. 
%  These states $s\in \rstates{B,Q_0}$ are captured by their 
%  projections $s\pj \B'$ in $\B'$.  
}


%%%%%%%%%%%%%%%%%%%%%%%%%%%%%%%%%%%%%%%%%%%%%%%%%%%%%%%%%%%%%%
\subsection{Wait-for graphs}
\label{secn:wait-for-graphs}

The wait-for graph for a state $s$ is a directed bipartite and-or graph which contains as nodes the atomic components $\B_1, \ldots, \B_n$, and all
the interactions $\gamma$.  Edges in the wait-for-graph are from a component $\B_i$ to all the interactions that $\B_i$ enables (in $s$), and from an
interaction $\act$ to all the components that participate in $\act$ and which do not enable it (in $s$).


\begin{definition}[Wait-for graph  $\wfg{\B}{s}$]
\label{def:static:wait-for-graph} 
\label{defn:static:wait-for-graph} 
Let $\B = \gamma(\B_1,\dots,\B_n)$ be a BIP composite component, and let
$s = \tpl{s_1,\ldots,s_n}$ be an arbitrary state of $\B$.
The {\em wait-for graph} $\wfg{\B}{s}$ of $s$ is a directed bipartite and-or graph, where
\begin{nlst1}

\item \label{def:static:wait-for-graph:nodes} the nodes of $\wfg{\B}{s}$ are as follows:
%
   \begin{nlst2}
   \item the and-nodes are the atomic components $\B_i$, $i \in \oneton$,
   \item the or-nodes are the interactions $\act \in \gamma$,
% such that for some  $i \in \oneton$: $B_i \in C_a$ and $s_i \enb{a}{B_i}$  OMIT this
   \end{nlst2}

\item \label{def:static:wait-for-graph:edges-aut-action} 
   there is an edge in $\wfg{\B}{s}$ from $\B_i$ to every node 
   $\act$ such that $\B_i \in \cmps{\act}$ and $s_i(\gd{\act}{i}) = \true$, \ie from $\B_i$ to every interaction
   which $\B_i$ enables in $s_i$,

\item  \label{def:static:wait-for-graph:edges-action-aut}
   there is an edge in $\wfg{\B}{s}$ from $\act$ to every 
   $\B_i$ such that $\B_i \in \cmps{\act}$ and $s_i(\gd{\act}{i}) = \false$, \ie from $\act$ to every component
   $\B_i$ which participates in $\act$ but does not enable it, in state $s_i$.
                  
\end{nlst1}
\end{definition}

A component $\B_i$ is an and-node since all of its successor actions (or-nodes) must be disabled for
$\B_i$ to be incapable of executing.  An interaction $\act$ is an or-node since it is disabled if
any of its participant components do not enable it.  An edge (path) in a wait-for graph is called a
wait-for edge (wait-for path). 
% done in Local Enablement Assumption above
%We assume, in the sequel, that every component $B_i$ always enables at least one interaction, \ie 
%in all reachable states $s$,  $B_i \ar a$ for at least $a$. This prevents pathological situations where a component
%causes a deadlock by disabling every interaction that it participates in.


\begin{definition}[Subgraph of a wait-for graph] \label{defn:wsubgraph}
$\US$ is a subgraph of $\wfg{\B}{s}$ iff the nodes of $\US$ are a subset of the nodes of $\wfg{\B}{s}$ and the edges of $\US$ are the induced edges from 
$\wfg{\B}{s}$, \ie if $u,v$ are nodes of $\US$ and $u \ar v$ is an edge of $\wfg{\B}{s}$, then $u \ar v$ is an edge of $\US$.
Write $\US \subg \wfg{\B}{s}$ when $\US$ is a subgraph of $\wfg{\B}{s}$, and extend the definition of $\subg$ to subgraphs of $\wfg{\B}{s}$ in the obvious
manner, so that $\US \subg \VS$ means that $\US$ is a subgraph of $\VS$.
\end{definition}

Write $\act \ar \B_i$ ($\B_i \ar \act$ respectively) for a
wait-for edge from $\act$ to $\B_i$ ($\B_i$ to $\act$ respectively).  We abuse notation by writing
$v \in \wfg{\B}{s}$ to indicate that $v$ is a node of $\wfg{\B}{s}$, and 
$e \in \wfg{\B}{s}$ to indicate that $e$ (either $\act \ar \B_i$ or $\B_i \ar \act$) is an edge in
$\wfg{\B}{s}$. 
Also $\B_i \ar \act \ar \B'_i \in \wfg{\B}{s}$ for
$\B_i \ar \act \in \wfg{\B}{s} \land \act \ar \B'_i \in \wfg{\B}{s}$, \ie for a wait-for path of
length 2, and similarly for longer wait-for paths.
Likewise use $v \in \US$, $e \in \US$, where $\US$ is a subgraph of $\wfg{\B}{s}$.



Consider the dining philosophers system given in Figure~\ref{fig:diningSpectrum}.
Figure~\ref{fig:wfg0} shows its wait-for graph in its sole initial state.  Figure~\ref{fig:wfg1}
shows the wait-for graph after execution of $\Get_0$.  Edges from components to interactions are
shown solid, and edges from interactions to components are shown dashed.

\begin{figure*}[ht]
  \begin{center}
      \subfigure[Wait-for-graph in initial state.]{\label{fig:wfg0}\scalebox{0.4}{\input{figs/wfgDiningInitial.pdf_t}}} \quad \quad
      \subfigure[Wait-for-graph after execution of $\Get_0$.]{\label{fig:wfg1}\scalebox{0.4}{\input{figs/wfgDining1.pdf_t}}} 
      \caption{Example wait-for-graphs for dining philosophers system of Figure~\ref{fig:diningSpectrum}.}
       \label{fig:wfg}
  \end{center}
\end{figure*}





A key principle of the dynamics of the change of wait-for graphs is that 
wait-for edges not involving some interaction
$\act$ and its participants $\B_i \in \cmps{\act}$ are unaffected by the execution
of $\act$.  Say that edge $e$ in a wait-for-graph
is \emph{$\B_i$-incident} iff $\B_i$ is one of the endpoints of $e$.


\begin{proposition}[Wait-for edge preservation] \label{prop:wait-for-edge-preservation}
Let $s \la{\act} t$ be a transition of composite component $\B =
\gamma(\B_1,\dots,\B_n)$, and let $e$ be a wait-for edge in $\wfg{\B}{s}$
that is not $\B_i$-incident, for every $\B_i \in \cmps{\act}$. Then $e \in
\wfg{\B}{s}$ iff $e \in \wfg{\B}{t}$. 
\end{proposition}
%
% \prfs{
% Components not involved in the execution of $\act$ do not change state
% along $s \la{\act} t$. Hence the endpoint of $e$ that is a
% component has the same state in $s$ as in $t$. The proposition then
% follows from Definition~\ref{def:static:wait-for-graph}. 
% }
%
\begin{proof}
Fix $e$ to be an arbitrary wait-for-edge that is not
$\B_i$-incident. $e$ is either $\B_j \ar b$ or $b \ar \B_j$, for some
component $\B_j$ of $\B$ that is not in $\cmps{\act}$, and an interaction $\actb$
(different from $\act$) that $\B_j$ participates in.
%
Now $s \pj \B_j = t \pj \B_j$, since $s \la{\act} t$ and $\B_j \not\in \cmps{\act}$. Hence $s(\gd{b}{j}) =
t(\gd{b}{j})$. It follows from Definition~\ref{def:static:wait-for-graph} that 
$e \in \wfg{\B}{s}$ iff $e \in \wfg{\B}{t}$.
\end{proof}



%%%%%%%%%%%%%%%%%%%%%%%%%%%%%%%%%%%%%%%%%%%%%%%%%%%%%%%%%%%%%%%%%%%%%%%%%%%%%%%%%%%%%%%%%%%%%%%%%%%%%
\subsection{Supercycles and deadlock-freedom}

We characterize a deadlock as the existence in the wait-for-graph of a
graph-theoretic construct that we call a {\em supercycle}.

\begin{definition}[Supercycle]
\label{def:supercycle} 
\label{defn:supercycle} 
Let $\B = \gamma(\B_1,\dots,\B_n)$ be a composite component and $s$ be a state of $\B$.
A subgraph $\SC$ of $\wfg{\B}{s}$ is a \emph{supercycle} in $\wfg{\B}{s}$ if and only if all of the following hold:
\begin{nlst1}
   \item \label{def:supercycle.nonempty} \label{clause:supercycle.nonempty} 
$\SC$ is nonempty, \ie contains at least one node,

   \item \label{def:supercycle.component-blocked} \label{clause:supercycle.component-blocked} 
if $\B_i$ is a node in $\SC$, then for all interactions $\act$ such that
there is an edge in $\wfg{\B}{s}$ from $\B_i$ to $\act$:
      \begin{nlst2}
      \item $\act$ is a node in $\SC$, and 
      \item there is an edge in $\SC$ from $\B_i$ to $\act$,
      \end{nlst2}
that is, $\B_i \ar \act \in \wfg{\B}{s}$ implies $\B_i \ar \act \in \SC$,

   \item \label{def:supercycle.action-blocked}  \label{clause:supercycle.action-blocked}  
if $\act$ is a node in $\SC$, then there exists a $\B_j$ such that:
      \begin{nlst2}
      \item $\B_j$  is a node in $\SC$, and
      \item there is an edge from $\act$ to $\B_j$ in $\wfg{\B}{s}$, and
      \item there is an edge from $\act$ to $\B_j$ in $\SC$,
      \end{nlst2}
that is, $\act \in \SC$ implies $\ex \B_j : \act \ar \B_j \in \wfg{\B}{s} \land \act \ar \B_j \in \SC$.

\end{nlst1}
\end{definition}
%where $\act \in SC$ means that $\act$ is a node in $\SC$, etc. 
%Also, write $SC \sub \wfg{\B}{s}$ when $\SC$ is a subgraph of $\wfg{\B}{s}$.
Intuitively, $\SC$ is a supercycle iff every node is $\SC$ is blocked from executing by other nodes in $\SC$. 

\begin{definition}[Supercycle-free]
\label{def:supercycle-free}
\label{defn:supercycle-free}
$\wfg{\B}{s}$ is \emph{supercycle-free} iff 
there does not exist a supercycle $\SC$ in $\wfg{\B}{s}$. 
In this case, say that state $s$ is supercycle-free.
%Formally, we define the predicate 
%    $\scf{B}{s} \df \neg \ex \SC: \SC \sub \wfg{\B}{s} \ \mbox{and $\SC$ is a supercycle}$. 
\end{definition}

\begin{figure*}[ht]
  \begin{center}
   \scalebox{0.4}{\input{figs/wfgDiningCycle.pdf_t}}
   \caption{Example supercycle for dining philosophers system of Figure~\ref{fig:diningSpectrum}.}
   \label{fig:sc}
  \end{center}
\end{figure*}


Figure~\ref{fig:sc} shows an example supercycle (with edges in bold) for the dining philosophers
system of Figure~\ref{fig:diningSpectrum}.
$\Phl_0$ waits for (enables) a single interaction, $\Grab_0$. 
$\Grab_0$ waits for (is disabled by) fork $F_0$, which waits for interaction $\Rel_0$.
$\Rel_0$ in turn waits for $\Phl_0$. However, this supercycle occurs when $\Phl_0$ is in state $h_0$
and $F_0$ is in state $u_0^\l$. This state is not reachable from the initial state. 


Figure~\ref{fig:SCnotCycle} shows an example of a supercycle that is not a simple cycle.  The ``essential'' part of the supercycle, consisting of
components $\B_1, \B_2, \B_3$, and their interactions $\act, \actb, \actc, \actd$, is in bold.  
The supercycle can be extended to contain $\B_4$, but neither $\B_5$ nor
$\B_6$: $\B_6$ is enabled, and $\B_5$ is ready to execute the interaction $\acth$, which waits only for $\B_6$.
%
\begin{figure}[ht]
\begin{center}
\scalebox{0.6}{\input{figs/SCnotCycle.pdf_t}}
\caption{Example supercycle that is not a simple cycle}
\label{fig:SCnotCycle}
\end{center}
\end{figure}
%
Figure~\ref{fig:cycleOK} shows that deleting the wait-for edge from $\actd$ to $\B_1$ in \fig{SCnotCycle} results in 
an example where there is
a cycle of wait-for-edges, without there being a supercycle. This shows
that a cycle does not necessarily imply a supercycle, and hence
deadlock. 
%
\begin{figure}[ht]
\begin{center}
\scalebox{0.6}{\input{figs/cycleOK.pdf_t}}
\caption{Example where a wait-for cycle does not imply deadlock}
\label{fig:cycleOK}
\end{center}
\end{figure}


The existence of a supercycle is sufficient and necessary for the occurrence of
a deadlock, and so checking for supercycles gives a sound and complete check for
deadlocks.  
%
\prop{static:supercycle-is-sufficient} states that the
existence of a supercycle implies a local deadlock: all components in
the supercycle are blocked forever.

\begin{proposition}
\label{prop:static:supercycle-is-sufficient}
Let $s$ be a state of $\B$.
If $\SC \subg \wfg{\B}{s}$ is a supercycle, then all components $\B_i$ in $\SC$ cannot execute a transition in any state reachable
from $s$, including $s$ itself.
\end{proposition}
%
\begin{proof}
Let $\B_i$ be an arbitrary component in $\SC$. By Definition~\ref{def:supercycle}, every interaction that $\B_i$ enables
has a wait-for edge to some other component $\B_j$ in $\SC$ and so cannot be executed in state $s$. Hence in any
transition from $s$ to another global state $t$, all of the components $\B_i$ in $\SC$ remain in the same local state. 
Hence $\SC \subg \wfg{\B}{t}$, \ie the same supercycle $\SC$ remains in global state $t$. Repeating this argument from state $t$
and onwards leads us to conclude that $\SC \subg \wfg{\B}{u}$ for any state $u$ reachable from $s$.
\end{proof}

\prop{static:supercycle-is-necessary} states that the existence of a supercycle is necessary for a local deadlock to occur: if a set of components,
\emph{considered in isolation}, are blocked, then there exists a supercycle consisting of exactly those components, together with the interactions
that each component enables.
%
\begin{proposition}
\label{prop:static:supercycle-is-necessary} 
\label{prop:supercycle-is-necessary}
Let $\B'$ be a subcomponent of $\B$, and let $s$ be an arbitrary state of $\B$ such that $\B'$, when considered in isolation, has no enabled
interaction in state $s \pj \B'$.
%
Then, $\wfg{\B}{s}$ contains a supercycle.
\end{proposition}
%
% \prfs{Every atomic component $\B_i$ in $\B'$ is individually deadlock free, by assumption, 
% and so there is at least one interaction $a_i$ which
% $\B_i$ enables. Now $a_i$ is not enabled in $\B'$, by the antecedent of the proposition. Hence $a_i$ has some
% outgoing wait-for-edge in $\wfg{\B}{s}$. The subgraph of $\wfg{\B}{s}$ induced by all
% the $\B_i$ and all their (locally) enabled interactions is therefore a supercycle.
% }
%
\begin{proof}
Let $\B_i$ be an arbitrary atomic component in $\B'$, and let $\act_i$ be any interaction that $\B_i$ enables. Since $\B'$ has no
enabled interaction, it follows that $\act_i$ is not enabled in $\B'$, and
therefore has a wait-for-edge to some atomic component $\B_j$ in
$\B'$. Let $SC$ be the subgraph of $\wfg{\B}{s}$ induced by:
\bn
\item the atomic components of $\B'$, 
\item the interactions $\act$ that each atomic component $\B_i$ enables,
     and the edges $\B_i \ar \act$, and
\item  the edges $\act \ar \B_j$ from each interaction to some atomic
     component $\B_j$ in $\B'$ that does not enable $\B_j$.
\en
$\SC$ satisfies Definition~\ref {def:supercycle} and so is a supercycle.
\end{proof}

We consider subcomponent $\B'$ in isolation to avoid other phenomena that prevent interactions from executing, \eg
conspiracies \cite{AFG93}.  
Now the converse of Proposition~\ref{prop:supercycle-is-necessary} is that absence of
supercycles in $\wfg{\B}{s}$ means there is no locally deadlocked subsystem. 



\begin{corollary}[Supercycle-free implies free of local deadlock]
\label{cor:static:dead-free}
%Let $(B,Q_0)$ be a BIP system, with $B = \gamma(\B_1,\dots,\B_n)$. 
If, for every reachable state $s$ of $(\B,Q_0)$, $\wfg{\B}{s}$ is supercycle-free, then
$(\B,Q_0)$ is free of local deadlock.
\end{corollary}
%
\begin{proof}
We establish the contrapositive.
Suppose that $(\B, Q_0)$ is not free of local deadlock. Then there exists a subsystem $(\B', Q'_0)$ of $(\B, Q_0)$, and
a reachable state $s$ of $(\B', Q'_0)$, such that $\B'$ enables no interaction in state  $s \pj \B'$.
By Proposition~\ref{prop:supercycle-is-necessary}, $\wfg{\B}{s}$ contains a supercycle.
\end{proof}



%%%%%%%%%%%%%%%%%%%%%%%%%%%%%%%%%%%%%%%

\subsection{\redbox{Subsystems and Supercycles}}
%
We formally discuss the notion of \emph{local supercycle} in \secn{localSupercycles}.  Our method provides a local version of the supercycle-freedom check.
For each interaction in the system, the local check computes a subsystem which includes the interaction and also other components at a given ``distance''
from the interaction.  It then checks whether any of the subsystem components is involved in a local supercycle.
%
Figure~\ref{fig:sc-local-dining} illustrates the wait-for-graph (in the initial state) of the dining philosopher subsystem corresponding to the
$\Grab_0$ interaction with a distance of $1$, i.e., its components and their interactions.  The subsystem includes the $\Grab_0$ interaction,
the components $\Phl_0, F_0,$ and $F_1$ that participate in $\Grab_0$, and the interactions $\Rel_0, \Grab_1,\Rel_1, \Grab_3,$ and $\Rel_3$ which are connected to these
components.  We notice that no component in Figure~\ref{fig:sc-local-dining} is involved in a supercycle.
%
The interactions  $\Grab_1,\Rel_1, \Grab_3,$ and $\Rel_3$ (underlined in the figure) are ``border interactions'', since they have participating components that are outside the subsystem.
The enablement of border interaction cannot be determined from the subsystem in isolation. Hence, to ensure soundness of our supercycle-freedom check,
we must assume pessimistically  that the border interactions are not enabled. This pessimism may yield a false negative, thereby causing our check to
be incomplete. The further the distance is increased, the ``more complete'' the local check becomes, as the local states of the larger subsytem give a
better overapproximation to the global states of the entire system.

\begin{figure*}[ht]
  \begin{center}
   \scalebox{0.4}{\input{figs/wfgDiningInitial-Local.pdf_t}}
   \caption{Wait-for graph in the intial state of the dining philosophers subsystem for $\Grab_0$ and distance 1.}
   \label{fig:sc-local-dining}
  \end{center}
\end{figure*}

%%%%%%%%%%%%%%%%%%%%%%%%%%%%%%%%%%%%%%%





In the sequel, we say ``deadlock-free'' to mean ``free of local deadlock''.
%
We wish to check whether supercycles can be formed or not.
% \ie whether the supercycle formation condition is satisfied or not.
In principle, we could check directly whether $\wfg{\B}{s}$ contains a supercycle, for each
reachable state $s$. However, this approach is subject to state-explosion, and so is usually
unlikely to be viable in practice.  Instead, we formulate global conditions for supercycle-freedom,
and then ``project'' these conditions onto small subsystems, to obtain local versions of these
conditions that are (1) efficiently checkable, and (2) imply the global versions.
To formulate these conditions, we characterize the static (structural) and dynamic
(formation) properties of supercycles, in Sections~\ref{secn:globalSupercycles} and \ref{secn:globalDeadlockFreedom}, respectively.




%%%%%%%%%%%%%%%%%%%%%%%%%%%%%%%%%%%%%%%%%%%%%
\subsection{Abstract supercycle freedom conditions}
\label{secn:abstract-scfree-conditions}

Since we will present several conditions for supercycle-freedom, we now present an abstract
definition of the essential properties that all such conditions must have.  The key idea is that
execution of an interaction $\act$ does not create a supercycle, and so any condition which implies 
this for $\act$ is sufficient. If a different condition implies the same for another interaction
$\actp$, this presents no problem \wrt establishing deadlock-freedom. Hence, it is sufficient to
have one such condition for each interaction in  $(\B,Q_0)$. Since each condition restricts the
behavior of interaction execution, we call it a ``behavioral restriction condition''.

%Moreover, since implication of $\GAO(\act)$ can be on a ``per interaction'' basis, with different
%conditions being applied to different interactions, we have:


\begin{definition}[Behavioral restriction condition]
A \introdf{behavioral restriction condition} $\BC$ is a predicate
$\BC: (\B, Q_0, \act) \to \set{\ttt, \fff}$.
\end{definition}
%
$\BC$ is a predicate on the effects of a particular interaction $\act$ within a given system
$(\B, Q_0)$.

\begin{definition}[Supercycle-freedom preserving] \label{def:SC-free-preserving}
A behavioral restriction condition $\BC$ is \introdf{supercycle-freedom preserving} iff, for every system 
$(\B, Q_0)$ and $\act \in \gamma$ such that $\BC(\B, Q_0, \act) = \ttt$, the following holds:\\%[0.5ex]
%
%\ind for all interactions $\act$ of $\B$ (\ie $a \in \gamma$),\\
\ind \ind for every reachable transition $s \goesto[\act] t$ of $(\B, Q_0)$\\
\ind \ind \ind if $s$ is supercycle-free, then $t$ is supercycle-free.
\end{definition}


\begin{theorem}[Deadlock-freedom via supercycle-freedom preserving restriction]
\label{theorem:SC-free-preserving.deadlock-free}
\label{thm:SC-free-preserving.deadlock-free}
Assume that
\bn
\item \label{theorem:SC-free-preserving.initial}
      for all $s_0 \in Q_0$, $\wfg{\B}{s_0}$ is supercycle-free, and
\item \label{theorem:SC-free-preserving.reachable-transitions}
   there exists a supercycle-freedom preserving restriction $\BC$ such that,
   for all $\act \in \gamma$: $\BC(\B, Q_0, \act) = \ttt$. 
\en
Then for every reachable state $u$ of $(\B, Q_0)$:  $\wfg{\B}{u}$ is supercycle-free.
\end{theorem}
%
\begin{proof} Let $u$ be an arbitrary reachable state. The proof is by induction on the length of the finite
execution $\al$ that ends in $u$.  Assumption~\ref{theorem:SC-free-preserving.initial} provides the
base case, for $\al$ having length 0, and so $u \in Q_0$.  For the induction step, we establish: for
every reachable transition $s \la{\act} t$, $\wfg{\B}{s}$ is supercycle-free implies that
$\wfg{\B}{t}$ is supercycle-free. This is immediate from
Assumption~\ref{theorem:SC-free-preserving.reachable-transitions}, and
Definition~\ref{def:SC-free-preserving}. 
\end{proof}

Since the above proof does not make any use of the requirement that there is a single restriction
$\BC$ for all interactions, we immediately have:


\begin{corollary}[Deadlock-freedom via several supercycle-freedom preserving restrictions]

\label{cor:SC-free-preserving.deadlock-free}
Assume that
\bn
\item \label{cor:SC-free-preserving.initial}
      for all $s_0 \in Q_0$, $\wfg{\B}{s_0}$ is supercycle-free, and
\item \label{cor:SC-free-preserving.reachable-transitions}
   for all $\act \in \gamma$, there exists a supercycle-freedom preserving restriction $\BC$ such that $\BC(\B, Q_0, \act) = \ttt$.
\en
Then for every reachable state $u$ of $(\B, Q_0)$:  $\wfg{\B}{u}$ is supercycle-free.
\end{corollary}
%
\begin{proof}
Similar to the proof of \thm{SC-free-preserving.deadlock-free},  except that, for
the transition $s \la{\act} t$, use the  supercycle-freedom preserving restriction $\BC$
corresponding to $\act$.
\end{proof}












%% REMOVED DUE TO FIXPOINT CHARACTERIZATION
% Hence, we follow outgoing wait-for edges in computing supercycle-membership. Actually, it turns out
% to be easier to compute the negation of supercycle membership, which we call \intro{supercycle
%   violation}. This is because supercycle-violation has a base case: when a node has no outgoing
% wait-for edges. We need a base case, and an inductive definition, because 
% a node that is not in any supercycle may nevertheless be a node of a wait-for cycle, since a cycle
% of wait-for-edges does not necessarily imply a supercycle. Hence, to compute supercycle violation
% properly, we introduce a notion of the \emph{level} of a violation. A node with no
% outgoing wait-for edges has a level-1 violation. A node whose violation is based on outgoing edges
% to neighbors whose violation level is at most $d-1$, has itself a level-$d$ violation. We
% formalize the notion of \emph{level-$d$ supercycle violation} as the predicate $\viol{v}{d}{t}$,
% defined by induction on $d$.


%% REMOVED DUE TO FIXPOINT CHARACTERIZATION
% \bd[Supercycle violation, $\scViol{v}{d}{t}$]
% \label{def:supercycle-violation}
% \label{def:supercycle.violation}
% Let $t$ be a state of $(\B, Q_0)$, $v$ be a node of $\wfg{\B}{t}$, and $d$ an integer $\ge 1$.
% We define the predicate $\viol{v}{d}{t}$ by induction on $d$, as follows. We indicate the
% justification for each clause of the definition.

% \noindent
% \ul{Base case, $d=1$.} $\viol{v}{1}{t}$ iff $v$ is an interaction $\act$ and it has no outgoing wait-for-edges, otherwise $\neg \viol{v}{1}{t}$.
% %
% Justification: if $v$ has no outgoing wait-for-edges, then it cannot be in a supercycle.  Note that $v$ must be an
% interaction in this case, since a component must have at least one outgoing wait-for edge at all times.

% \noindent
% \ul{Inductive step, $d > 1$.}  $\viol{v}{d}{t}$ iff any of the following cases hold. Otherwise $\neg \viol{v}{d}{t}$.
% %
% \bn

% \item  \label{def:supercycle.violation.component.out}
% $v$ is a component $\B_i$ and there exists interaction $\act$ such that $\B_i \ar \act \in \wfg{\B}{t}$ and $(\ex d' : 1 \le d' < d: \viol{\act}{d'}{t})$.
% That is, $\B_i$ enables an interaction $\act$ which has a level-$d'$ supercycle-violation, for some $d' < d$.
% Justification is \prop{sc-membership}, \clause{sc-membership:comp-out}.


% \item \label{def:supercycle.violation.interaction.out}
% $v$ is an interaction $\act$ and for all components $\B_i$ such that $\act \ar \B_i \in \wfg{\B}{t}$, we have $(\ex d' : 1 \le d' < d: \viol{\B_i}{d'}{t})$.
% That is, each component $\B_i$ that $\act$ waits for has a level-$d'$ supercycle-violation, for some $d' < d$.
% Justification is \prop{sc-membership}, \clause{sc-membership:act-out}.

% \en
% \ed
% %
% Figure~\ref{fig:scViolate} gives a formal, recursive definition of $\viol{v}{d}{t}$. 
% The notation $v = \B_i$ means that $v$ is some component $\B_i$. Likewise, 
% $v = \act$ means that $v$ is some interaction $\act$.
% Line 0 corresponds to the base case, line 1 corresponds to item 1 of the inductive case, and line 2 corresponds to item 2 of the inductive case.
% Line 3 handles all cases that do not return true.
%
% \begin{figure}[t]
% \setcounter{lctr}{-1}
% \horline
% \begin{tabbing}\label{alg:check-scViol}
% aa\= aa\= aa\= \kill
% $\viol{v}{d}{t}$\\
% \lioc{\IFC{d = 1 \land v = \act \land \neg (\ex \B_i : \act \ar \B_i \in \wfg{\B}{t})}\ \RETURNE{\ttt}\ \FI}{\cmnt base case for \ttt\ result}
% \lio{\IFC{v = \B_i \land (\ex \act : \B_i \ar \act \in \wfg{\B}{t} : (\ex d' : 1 \le d' < d : \viol{\act}{d'}{t}))} \ \RETURNE{\ttt}\ \FI}
% \lio{\IFC{v = \act \land (\fa \B_i : \act \ar \B_i \in \wfg{\B}{t} : (\ex d' : 1 \le d' < d : \viol{\B_i}{d'}{t}))} \  \RETURNE{\ttt}\ \FI}
% \lioc{\RETURNE{\fff}}{\cmnt no case for \ttt\ result, so result is \fff}
% \end{tabbing}
% \vspace{-3ex}
% \horline
% \caption{Formal definition of $\viol{v}{d}{t}$}
% \label{fig:scViolate}
% \end{figure}


%In the sequel, we say sc-violation rather than ``supercycle violation.''  The crucial result is
%that, if $v$ has a level-$d$ sc-violation, for some $d \ge 1$, then $v$ cannot be a node of a
%supercycle.


% %% REMOVED DUE TO FIXPOINT CHARACTERIZATION
% \bp[Soundness of supercycle violation \wrt supercycle non-membership]
%  \label{prop:supercycle-violation} \label{prop:scViol-implies-notInSC}
% If $(\ex d \ge 1: \viol{v}{d}{t})$ then  $\neg \scyc{\B}{t}{v}$,
% \ie supercycle violation implies supercycle non-membership.
% %$v$ is not a node of a supercycle in $\wfg{\B}{t}$.
% \ep
% \prf{
% Proof is by induction in $d$. 

% \noindent
% \textit{Base case, $d=1$}. $v$ has no outgoing edges. Hence  $v$ cannot be in a supercycle.

% \noindent
% \textit{Induction step, $d >1$}. Assume that $v$ has a level $d$ $SC$-violation. We have two cases. 

% \topcase{1}{$v$ is a component $\B_i$}   %\scase{1.1}{$v$ has an out-violation}
% Hence there exists an interaction $\act$ such that $\B_i \ar \act \in \wfg{\B}{t}$ and $\act$ has a level-$(d-1)$
% $SC$-violation. By the induction hypothesis, $\neg \scyc{\B}{t}{\act}$.
% By \prop{sc-membership}, \clause{sc-membership:comp-out}, 
% $\neg \scyc{\B}{t}{\B_i}$.

% \topcase{2}{$v$ is an interaction $\act$}    %\scase{2.1}{$v$ has an out-violation}
% Hence for all components $\B_i$ such that $\act \ar \B_i \in \wfg{\B}{t}$, $\B_i$ has a level-$(d-1)$
% $SC$-violation. By the induction hypothesis, 
% $(\fa \B_i : \act \ar \B_i \in \wfg{\B}{t}: \neg \scyc{\B}{t}{\B_i})$.
% By \prop{sc-membership}, \clause{sc-membership:act-out}, 
% $\neg \scyc{\B}{t}{\act}$.
% }


%In the other direction, we have a slightly weaker result: if $v$ has no level-$d$ scsc-violation,
%for all $d \ge 1$, then $v$ is a node of a supercycle.


%% REMOVED DUE TO FIXPOINT CHARACTERIZATION
% \bp[Completeness of supercycle violation \wrt supercycle non-membership]
%  \label{prop:notInSC-implies-scViol}
% If  $\neg \scyc{\B}{t}{v}$ then $(\ex d \ge 1: \viol{v}{d}{t})$,
% \ie supercycle non-membership implies supercycle violation.
% \ep
% %
% \prf{
% %
% We establish the contrapositive $(\fas d \ge 1: \neg \viol{v}{d}{t})$ then $\scyc{\B}{t}{v}$.
% Let $V$ be the set of nodes in $\wfg{\B}{t}$ with a supercycle-violation, \ie
% $V = \set{w \stt w \in \wfg{\B}{t} \land (\exs d: \viol{w}{d}{t})}$.  Let $\bV$ be the remaining nodes, \ie all nodes
% in $\wfg{\B}{t}$ that do not have a supercycle-violation, so 
% $\bV = \set{w \stt w \in \wfg{\B}{t} \land (\fas d \ge 1: \neg \viol{v}{d}{t})}$.

% If $\bV$ is empty then the proposition holds vacuously and we are done. So assume that $\bV$ is non-empty and let 
% $v$ be an arbitrary node in $\bV$.

% \topcase{1}{$v$ is a component $\B_i$}  
% Suppose that there is a wait-for-edge from $v$ to some interaction $\act$ that is in $V$.
% Then, by Definition~\ref{def:supercycle.violation}, $v$ has a supercycle violation, which contradicts the choice of $v$
% as a member of $\bV$. Hence all wait-for-edges starting in $v$ must end in a node in $\bV$.

% \topcase{2}{$v$ is an interaction $\act$}  
% Suppose that every wait-for-edge from $v$ to some component  $\B_i$ that is in $V$.
% Then, by Definition~\ref{def:supercycle.violation}, $v$ has a supercycle violation, which contradicts the choice of $v$
% as a member of $\bV$. Hence some wait-for-edge starting in $v$ must end in a node in $\bV$.


% Hence we have that $\bV$ satisfies all three clauses of Definition~\ref{def:supercycle}: it is
% nonempty, each component in $\bV$ has all its enabled interactions also in $\bV$, and each
% interaction in $\bV$ waits for a component in $\bV$.  Hence $\bV$ as a whole is a supercycle. Since
% the nodes of $\bV$ are, by definition of $\bV$, exactly the nodes $v$ such that
% $(\fas d \ge 1: \neg \viol{v}{d}{t})$, we have that any such node $v$ is a node of a supercycle in
% $\wfg{\B}{t}$, \ie $\scyc{\B}{t}{v}$. Hence the Proposition is established.  }

%Note that the above implies that there are no wait for edges from component in $U$ to an interaction outside $U$???



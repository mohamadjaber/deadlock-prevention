% !TEX root = answers.tex
\section{Reviewer 3}
This paper presents a method for checking for the ABSENCE of local as well as global deadlocks in finite state systems. The method is based on examining subsystems of the given system of processes; in essence, if none of the examined subsystems exhibits a deadlock, neither does the full system. Experiments show that this method can establish deadlock-freedom by examining only small subsystems, a significant savings in time and space over an alternative method which considers the full system.

~

As individual processes have a choice of actions, the traditional criterion for a deadlock, that of a wait-for cycle, must be generalized to that of a wait-for supercycle. This concept was examined in earlier work. The contribution of this work is to design abstractions that allow checking for this property on subsystems of processes. In a subsystem, the behavior of processes outside the subsystem is considered pessimistically, so that any reachable wait-for supercycle witnessing a local deadlock in the full system appears as a reachable wait-for supercycle in at least one sub-system. It follows (from the contrapositive) that if all subsystems are free of supercycles, there is no local deadlock in the full system and, therefore, there is no global deadlock either. The paper actually formulates two criteria for the subsystems: one detects supercycles with pessimistic assumptions on the external processes; the other detects strongly connected supercycles. 

~

Completeness is trivial. The subsystems are chosen to be those in a circle of radius of l (roughly); by successively increasing l, eventually the subsystem becomes the entire system, so detection is precise.

~


The paper is written carefully and is self-contained, giving all necessary background on deadlock and supercycles. The experimental results are interesting and impressive: they show that for two example protocols, deadlock freedom can be established by considering small subsystems, which means a corresponding order-of-magnitude reduction in analysis time over methods that directly consider the entire system. 

~

\paragraph{Suggestions:}

\begin{enumerate}
\item It would help a reader to see a simple but non-trivial example of this method at work at the very beginning of the paper. One now has to read through nearly 25 pages before getting to the heart of the local method, described in Section 6 and even further to see the examples used to test the method.

  \answer{The paper illustrated the BIP and the wait-for-graph definitions with the dining philosopher example. We added an illustration of the wait-for-graph of a subsystem directly therein to present the local method as early as possible. } 

\item The analysis of supercycles can, I believe, be considerably simplified.

~

  A first observation is that the definition (3.5) of a supercycle is in the form of a post-fixpoint, of the form [SC $\Rightarrow$ F(SC)],  where the conditions (2) and (3) define the monotone function F(SC). From the Knaster-Tarski theorem, it follows that there is a largest SC, which is the greatest fixpoint of F. This, in turn, implies that the condition for NOT being in any supercycle is given by the negation of the greatest fixpoint, i.e., a least fixpoint. The supercycle violation conditions (Definition 4.3) are just the per-stage approximations of this least fixpoint. This considerably simplifies the development, for instance, Prop. 3.17 (union of supercycles) is a trivial consequence of the monotonicity of F. And Prop. 4.5 (Completeness of supercycle violation) is just the fact that the negation of a greatest fixpoint results in a least fixpoint. I believe that a careful examination of Section 4 from this point of view will reveal many such opportunities for simplifying the presentation and clarify the formulation of the violation conditions.

~

  A second observation is that the analysis of the strongly connected structure of the wait-for graph is a very good starting point for definitions such as the in- and out-depth. This analysis is done in Prop. 3.16. The SCC being analyzed in Prop. 3.16 is a terminal SCC. Any terminal SCC of a supercycle must be non-trivial. Moreover, nodes with finite in-depth are "higher up" interior nodes in the SCC DAG, while those with infinite out-depth are any that reach a non-trivial SCC. 

~

I strongly recommend that the authors consider reworking the section to incorporate the simplifications that follow from taking these two observations as fundamental. I found in my reading that keeping these two observations in mind simplified and illuminated much of the subsequent discussion.

    \answer{We thank the reviewer for the very valuable insight. The paper has been majorly rewritten to use fixpoints and that helped simplify the presentation as expected by the reviewer.}
\end{enumerate}

\paragraph{Requirements:}
\begin{enumerate}
\item The paper has a good discussion of related work on deadlock detection.
 However, it is, I believe, missing a significant set of prior research on 
 abstraction methods, in particular those related to compositional 
 reasoning, which can be used to prove global safety and deadlock-freedom. 
 Here are a selection of references. The three papers below formulate 
 methods to infer global invariants from compositional analysis. 
 These invariants can be used to check for the absence of global deadlock 
 and possibly for the absence of local deadlock as well. 
 (There is much work on compositional analysis, these papers should be 
 taken as starting points.)
~
Amir Pnueli, Sitvanit Ruah, Lenore D. Zuck:
Automatic Deductive Verification with Invisible Invariants. TACAS 2001: 82-97

%\answer{
%The paper discusses a specific type of bounded data parametrized systems with parameter $N$. It argues that for certain assertions $\exists N_0$ such that if 
%the assertion is valid for every $N, 1< N \leq N_0$ then the assertion is valid for any parameter. It also argues that $N_0$ is logarithmic in the number of 
%local states. 
%%    
%The systems discussed communicate by sharing variables and do not synchronize with locks or similar mechanisms. 
%Although the paper gives an example of ``request'' and ``release'' implemented in a spin lock fashion, 
%the concept of deadlock is not directly covered in the $R-$assertions. 
%%
%Our method also differs in that it considers systems that are not necessarily parametrized. 
%For instance, in the resource allocation example, clients may request resources in different sequence orders. 
%The deadlock condition depends highly on those sequences and not on the number of clients, resources, and managers. 
%%    
%Lastly, our method of constructing a subsystem starts with an interaction and increments the subsystem with a distance $\ell$. 
%Then $\ell$ is incremented in case a supercycle containing a component is found. 
%If not, this implies that the execution of the interaction does not cause a deadlock. 
%The analysis has to cover all interactions; and each maybe with a different termination distance $\ell$. 
%This differs from the $N_0$ selection approach in~\cite{PuneliTACAS2001}. 
%}

~

Ariel Cohen, Kedar S. Namjoshi:
Local Proofs for Global Safety Properties. CAV 2007: 55-67
%\answer{
%Similar to our work, this paper produces a global proof using several local proofs. 
%It splits an invariant into local process invariants and produces a proof for each in an incrementally complete manner. 
%The processes here also communicate through shared variables. 
%Our work, proves deadlock freedom one interaction at a time. 
%%
%Paper~\cite{CohenCav2007} splits the invariant based on shared and local variables therein.
%We differ in that the deadlock local checks are based on the structure of the BIP 
%interactions, ports, and components. 
%%
%The derived local invariants in~\cite{CohenCav2007} are symbolic over-approximations of the 
%reachable state space. 
%Our method differs in that the local and global checks work on the structure of the 
%wait-for-graphs that are extracted for each interaction. 
%These checks are independent from the 
%transition systems that govern the execution of each component. 
%In contrast, a deadlock invariant in~\cite{CohenCav2007} will have to visit the internals of 
%the processes' transition systems as the locks are expressed in request and release 
%statements invoked therein. 
%Currently, the paper illustrates its techniques using mutual exclusion safety conditions
%(race condition avoidance).
%%
%Similarly the refinement step in~\cite{CohenCav2007} includes refining the 
%over-approximation with predicates reasoning about additional local variables. 
%The refinement step in our work increases the size of the 
%subsystem instead. 
%}
~

Parosh Aziz Abdulla, Frédéric Haziza, Lukás Holík:
All for the Price of Few. VMCAI 2013: 476-495
%\answer{
%  parametrized systems.
%  linear, ring, tree-like communication topologies between processes. 
%  cut-off points: beyound which the verfication procedure need not to continue
%  view abstraction function:  parameter k: approximates a configuration by the set of all its projection containing at most k processes. 
%  1. forward reachability among  system configurations of size k.
%  2. symbolic forward reachability in the abstract domain of sets of views of size at most k. ==> overapprximation of all views of size up to k on all 
%  reachable configurations (of all size-- reachable configurations). 
%  Crucial to the computation is a bound on the number of processes involved
%  at each state such that the post image (typically infinite)
%  can be computed using successor operations of size (k+\ell) where \ell 
%  is a small constant.
%  This limits the techniques to specific kind of systems that abide by the 
%  characteristic: arrays, rings, trees, and multisets. 
%%
%  Interactions in this paper happen through global state rules with order 
%  (<,>) and difference (\not=) relations. 
%  In this paper all parametrized processes have the same states and transition rules, and the main difference across processes is defined by the index
%  of the process. 
%}

~

The following work describes la technique developed by Kurshan that is very similar in spirit to the abstraction done here. In essence, the method cuts "wires" connecting a sub-system to the rest, the wires become pessimistic free inputs.

~

Edmund M. Clarke, Robert P. Kurshan, Helmut Veith:
The Localization Reduction and Counterexample-Guided Abstraction Refinement. Essays in Memory of Amir Pnueli 2010: 61-71

%\answer{
%  The paper \cite{} reviews the abstraction techniques that 
%  use counterexample guided refinements and focus on the early
%  history of the method. 
%  It also discusses some alternatives that are not counterexample
%  based. 
%  The reviewer pointed out to the technique developed by Kurshan that 
%  is described in the paper. 
%  The technique builds a dependency graph for the variables referenced 
%  in the safety property.
%  The technique starts by verifying the property using a model 
%  checker. In case it passes, then the property is a tautaulogy.
%  Otherwise, there is an error track. 
%  A greedy routine attempts to lift the error track to the full design. 
%  In case that fails (no input configuration of the full design 
%  can cause a transition from a reachable state in the abstract design 
%  to the next state of the error track), 
%  a set of blocking variables are identified as the reason for the failure. 
%  These variables lead paths to active variables (variables with full 
%  assignments in the error track) in the dependency graph.
%  The refined abstraction shall now contain the property, the blocking 
%  variables, and the active variables with those variables at the boundary
%  of the dependency graph freed; the latter for the free fence. 
%  This localization refinement continues either to find a liftable counter
%  example, the property passes in some abstraction refinement, or 
%  the computation exhausts resources. 
%  The choice of the blocking variables to minimize the free fence was 
%  key to the efficacy of the routine. 
%%
%  Our work differs from the counterexample based techniques described in 
%  tin that it characterizes deadlock with a structural
%  property, defines a local version of the property, and looks for 
%  subsystems that violate that property to declare deadlock freedom. 
%  The refinement is not counterexample based. 
%  %
%}

\answer{
  We thank the reviewer for bringing the following papers on 
  safety property model checking to our attention. 
  We addressed these papers on abstraction, compositional reasoning 
  and localization reduction using symbolic model checking in the related
  work section. 
  We also addressed similar localization refinement work with structural 
  analysis and bounded model checking. 
  The added paragraphs in the related work section follow.  \\
%
Abstraction methods related to compositional reasoning
\cite{AbdullaHH13VMCAI,PnueliRZ01TACAS,CohenN09FMSD}
that target safety properties can be used to to prove
global deadlock freedom. 
%
The work in~\cite{AbdullaHH13VMCAI,PnueliRZ01TACAS,CohenN09FMSD}
target parametrized systems composed of $N$ communicating processes $P_{\{i\}}$ 
via over-approximating the reachable state space of a system with 
arbitrary $N$ with symbolic states from a system of size $K<N$ such that
if the property holds for the system of size $K$ it holds for any arbitrary $N$.
%
Crucial to the work in~\cite{AbdullaHH13VMCAI} is a bound on the number of processes
involved at each state such that the post image (typically infinite) can be 
computed using successor operations of size $K+\ell$ where $\ell$ is a small 
constant. This limits the completeness of the technique to systems with specific
array, ring and tree like topologies. 
%
The work in~\cite{CohenN09FMSD} provides a global proof using 
several local proofs. 
It splits a target system invariant into local process invariants across 
local and shared variables and attempts to prove these invariants. 
The derived local invariants are symbolic over-approximations of the reachable 
state space of the system. 
The abstraction refinement step refines the invariants with predicates reasoning 
about additional local variables. \\
%
The work in~\cite{PnueliRZ01TACAS} targets a specific type of bounded data 
parametrized systems with parameter $N$ where $N$ is the number of processes
where safety is expressed using a specific type of assertions called $R$-assertions.
The work shows that for a given such system, $\exists N_0$ such that an
$R-$assertion $\phi$ is preserved by any step of the system for every $N>1$
iff it is proven valid for every $N\le N_0$. 
Then the work shows how to handle such systems with model checking and deductive
reasoning techniques. 
% 
The survey paper~\cite{ClarkeKV10Memory} describes several abstraction 
techniques that use counterexamples to guide the refinement steps. 
It also describes a localization reduction technique~\cite{Kurshan94} 
a dependency graph of the variables referenced in the safety property. 
The first abstraction is the property itself. 
In case, it passes model checking, then it is a tautology. 
In case an error track is produced, either the tack is feasible and the property
fails, or the track is analyzed and linked to 
a group of blocking variables that could not be assigned to satisfy the track. 
These variables lead dependency graph paths to active variables that have full 
assignments in the error track. 
All these variables consist the next refinement step where the border variables
are considered free and termed as the free fence. 
Key to the efficacy of the technique is the choice of the blocking variables
such that they minimize the free fence at each abstraction step. \\
Localization refinement is also used synergistically with input 
reparameterization to attain maximal input reduction in 
sequential netlists using min-cut analysis 
in a structural manner~\cite{BaumgartnerM05Charme}.  
%
The reduced netlists are then subject to verification using several 
techniques including decomposing the netlist into several sub-netlists each 
with a bounded state transition diameter and applying bounded model checking
~\cite{BaumgartnerK04Date,BaumgartnerKA02CAV} on each sub-netlist. 
%
Our work differs from the above techniques in that 
(1) we do not limit our technique to 
parametrized systems, 
(2) we characterize deadlock freedom with a structural supercycle property
that governs the wait-for-graph of the system interactions, 
(3) we compute our local systems based on interactions, 
(4) we establish deadlock freedom by performing the structural super cycle violation
check for each interaction using its local subsystems,  and
(5) our technique is complete for BIP systems. 
In the future, we would like to explore whether the local supercycle violation check
is enough to prove parametrized systems. 
We would like to consider characterizing other interesting safety properties with
similar structural checks in the context of BIP. 
}

\item Questions

Could your method be used to establish deadlock freedom for an entire 
parameterized family? E.g., for the dining philosophers' protocol i
for all N? This would seem to require some use of symmetry in i
combination with local analysis, as is done in the three papers on i
invariant generation listed above.


\item Typos


"boldened" ... not sure that is a word.

~

"beither" $\rightarrow$ either
\end{enumerate}

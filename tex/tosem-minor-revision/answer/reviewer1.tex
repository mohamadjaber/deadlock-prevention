\section{Reviewer 1}
This work presents the details of a technique to determine whether a multi-process 
system
is free from deadlock. In order to avoid the problems of state-explosion in model 
checking
multi-component systems, the technique is designed to apply first to smaller 
sub-systems
of the larger system. 
Freedom from deadlock in the smaller system implies freedom
from deadlock in the larger system.  If the smaller system is not free 
from deadlock,
one can expand the model to include more components and perform the check again on
a larger sub-system, until the point where either the system is shown to 
satisfy deadlock
freedom, the entire system is found not to be deadlock free,  
or state explosion prevents
the further application of model checking.


Evaluation: the paper is well motivated and well written; the topic of
avoiding state explosion in model checking multi-component systems
is both interesting and important.

I have a few concerns about the presentation of the results, and about the
comparison with related work sections.  I feel these are relatively minor
questions and that assuming the authors address the questions the paper should
then be accepted for publication.

Detailed comments/questions:
\begin{enumerate}

\item there has been some work on showing how to use CTL-like logics to answer
questions about deadlock freedom.  E.g.
  Computation Tree Logic with Deadlock Detection
  Rob van Glabbeek1,2, Bas Luttik3,4 \& Nikola Trˇcka, that appeared in
  Logical Methods in Computer Science
  Vol. 5 (4:5) 2009, pp. 1–24
Can you comment on the use of those logics in the setting of detecting deadlocks in
BIP?

~

\answer{
  We added a paragraph in the related work section to discuss the paper.
}

\item  pg 3, second para from bottom: what does `irrelevant' mean in this context?

~

\answer{
We clarified the meaning by updating the sentence with: "when the port is clear from the context or not needed, we drop it on the transition and write". 
}

\item pg 4, top of page: "we assume that enablement of a port depends only on the
local state of a component" --- I found this a bit confusing, especially as it related
to models of the dining philosophers or token passing models.  
    
    ~
    
\answer{
The statement discusses port enablement.  
A port here is an interface of
a component to other components. 
An interaction connects components through ports. 
An interaction is enabled iff all its ports are enables. 
A port is enabled iff its component is in a state that enables the port, i.e. 
the state has an outgoing transition labeled with the port and the predicate
enabling the port is $true$. 
For example, state $e$ in component Philosopher enables the $\mathit{release}$
port. 
We addressed the issue by moving the description of the atomic component
example before the confusing statement, and then gave an example of port enablement
in the confusing statement. 
}

~ 

More specifically, I
did not follow the reasoning about how one analyzed multi-process systems
where shared variable interaction was critical to showing deadlock behavior.

~

\answer{
The paragraph starts by stating that we abstract away from syntax and variables
therein and analyze interactions, ports, components and states. 
Deadlock is characterized by the absence of an enabled interaction. 
This is due to the semantics of BIP where a system executes through interaction
enablement. 
Interaction enablement depends on port enablement, and port enablement is a local 
property of the component containing the port and can be expressed through
a predicate. 
In a sense, port and interaction enablements are predicates that abstract 
syntactic BIP issues away and are enough to characterize deadlock. 
We address this issue by adding an example after Definition 2.2 that illustrates 
an interaction since that definition is needed for the discussion of deadlocks. 
}

~


It would be very helpful if in the paper the authors could `work through' 
the application
of the local reasoning for both a version of the dinning philosophers and 
a version of a single token ring like protocol where the processes require 
the token to make
any progress.

~

\answer{
We added reference to the example of dining philosophers across some definitions. 
We hope this helps improve the understanding of the concepts. 
We did not do the same for the token ring for brevity concerns. 
}

~

In the dining philosophers or the token passing model, what do the reduced systems
look like? Individual components?  Can you show how the smaller reduced systems
contain sufficient information about the behavior of the larger system to draw conclusions
about the reachable state space?

~

\answer{
  We added to Section 3.3 an illustration of the wait-for-graph for the subsystem
    corresponding to interaction $get_{0}$ 
    from the dining philosopher example.
  We discussed how the wait-for-graph contains the necessary components and 
    interactions to judge that the execution of the interaction will not result
    in a deadlock. 
}
~


\item  definition 2.5. pg 5:  local enablement assumption --- for systems 
  that satisfy
this assumption, does that imply freedom from deadlock?   
Can you comment on how and where
this definition will be used?  what implications does it have for 
the systems that satisfy it?

~

\answer{
The local enablement assumption in Definition 2.5 simply states that a component 
$B$ in state $s$ enables at least one interaction $a$. 
However, for $a$ to execute, it requires enablement from all components involved
in it. That might not be the case as $a$ may be blocked by another component. 
Therefore the assumption is not enough to guarantee deadlock freedom. 
We reflected that in the paper directly after the definition.
}

\item  pg. 6 \& 7, application of proposition 2.13 --- detailing how this proposition works
on the dining philosophers model and the token passing model would be very helpful
for understanding the technique.

~

\answer{
We added an example that applies proposition 2.13 on the dining philosopher model. 
}

\item pg 7; def. 3.1 and 3.2 --- definition 3.1 of global deadlock seems straightforward, but
def. 3.2 is not.  A more detailed description of the intuition behind definition 3.1 would
be very useful.  Maybe a small example would help.

~

\answer{
We added a paragraph to describe the intuition behind definition 3.2 directly
after the definition. 
}

~

\item pg 27, def. 6.11 --- it would be very helpful to provide a bit more
intuition as to what this definition is describing --- in particular it was not
clear to me how to interpret the use of the definition in understanding the
application on page 32..

~

\answer{We added, in Section 3, an illustration of subsystem and local 
supercycle to present the local method.} 

\item  related work: there has been extensive work on compositional reasoning
technique applied to multi-component systems, both in terms of analyzing
safety properties, and in terms of analyzing ltl and ctl formulae. 
Could you comment
on the relationship of that work to this current submission?

~

\answer{
  We added several paragraphs to the related work section to address compositional 
  reasoning alternatives to safety properties.
  We referred to papers suggested by reviewer 3, reviewer 1 and other related 
  papers. 
}

\end{enumerate}




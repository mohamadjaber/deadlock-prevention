
   \label{s:local.preamble}
%   
Evaluating the global restrictions $\GAO(B,Q_0,\act)$, $\GLin(B,Q_0,\act)$ requires checking all reachable transitions of
$(\B, Q_0)$, which is, in general, subject to state-explosion.
We need restrictions which imply a global restriction, and 
and which can be checked efficiently.
%
To this end, we first develop some terminology, and a projection result, for relating the waiting-behavior in a
subsystem of $(B,Q_0)$ to that in $(B,Q_0)$ overall.




Evaluating the global restrictions $\GAO(\B,Q_0,\act)$, $\GLin(\B,Q_0,\act)$ requires checking all reachable transitions of
$(\B, Q_0)$, which is, in general, subject to state-explosion.
We need restrictions which imply a global restriction, 
and which can be checked efficiently.
%
To this end, we first develop some terminology, and a projection result, for relating the waiting-behavior in a
subsystem of $(\B,Q_0)$ to that in $(\B,Q_0)$ overall.



\subsection{Projection onto Subsystems}
\label{s:projection}

\begin{definition}[Structure Graph $\sg{\B}$, $\ssg{\act}{\l}$] \label{def:structure-graph} The structure
graph $\sg{\B}$ of composite component $\B = \gamma(\B_1,\dots,\B_n)$ is a
bipartite graph whose nodes are the $\B_1, \ldots, \B_n$ and all the
$\act \in \gamma$.  There is an edge between $\B_i$ and
interaction $\act$ iff $\B_i$ participates in $\act$, \ie $\B_i \in \cmps{\act}$.  Define the
\emph{distance} between two nodes to be the number of edges in a shortest path
between them.  Let $\ssg{\act}{\l}$ be the subgraph
of $\sg{\B}$ that contains $\act$ and all nodes of $\sg{\B}$
that have a distance to $\act$ which is less than or equal to $\l$.
\end{definition}


\begin{definition}[Deadlock-checking subsystem, $\dsks{\act}{\l}$] \label{def:dsk}
Define $\dsks{\act}{\l}$, the \emph{deadlock-checking subsystem for interaction $\act$ and
radius $\l$}, to be the subsystem of $(\B, Q_0)$ based on the set of % depth --> radius
components in $\ssg{\act}{2\l}$.  (See \defn{bip.subsystem}).
\end{definition}

\begin{definition}[Border node, interior node of $\dsks{\act}{\l}$]  \label{def:dsk.border-and internal}
A node $v$ of $\dsks{\act}{\l}$ is a \emph{border-node} iff it has an
edge in $\sg{\B}$ to a node outside of $\dsks{\act}{\l}$.
If node $v$ of $\dsk{\act}{\l}$ is not a border node, then it is an \emph{internal node}.
\end{definition}
Note that all border nodes of  $\dsks{\act}{\l}$ are interactions,
since $2\l$ is even. Hence all component nodes of $\dsks{\act}{\l}$ are
interior nodes.
% TBD: EXPLAIN WHY....


\begin{proposition}[Wait-for-edge projection] \label{prop:edge-projection}
%Let $(B, Q_0)$ be a BIP system, and 
Let $(\B', Q'_0)$ be a subsystem of
$(\B, Q_0)$. Let $s$ be a state of $(\B, Q_0)$, and $s' = s \pj \B'$.
Let $\act$ be an interaction of $(\B', Q'_0)$, and $\B_i \in \cmps{\act}$ be an atomic component of $(\B', Q'_0)$.
Then 
(1) $\act \ar \B_i \in \wfg{\B}{s}$ iff $\act \ar \B_i \in \wfg{\B'}{s'}$, and
(2) $\B_i \ar \act \in \wfg{\B}{s}$ iff $\B_i \ar \act \in \wfg{\B'}{s'}$.
\end{proposition}
%
% \prfs{
% Since $s' = s \pj \B'$, all port enablement conditions of components in $\B'$ have the same value in $s$ and in
% $s'$. The proposition then follows by straightforward application of Definition~\ref{def:static:wait-for-graph}.
%}
\begin{proof}
By Definition~\ref{def:static:wait-for-graph}, $\act \ar \B_i \in \wfg{\B}{s}$ iff $s \pj i (\gd{\act}{\B_i}) = \false$.
Since $s' = s \pj \B'$, we have $s' \pj i = s \pj i$. Hence
$s \pj i (\gd{\act}{\B_i}) = s' \pj i (\gd{\act}{\B_i})$.
By Definition~\ref{def:static:wait-for-graph}, 
$a \ar \B_i \in \wfg{\B'}{s'}$ iff $s' \pj i (\gd{\act}{\B_i}) = \false$.
Putting together these three equalities gives us Clause (1).

By Definition~\ref{def:static:wait-for-graph},
$\B_i \ar \act \in \wfg{\B}{s}$ iff 
$s \pj i (\gd{\act}{\B_i}) = \true$.
Since $s' = s \pj \B'$, we have $s' \pj i = s \pj i$. Hence
$s \pj i (\gd{\act}{\B_i}) = s' \pj i (\gd{\act}{\B_i})$.
By Definition~\ref{def:static:wait-for-graph},
$\B_i \ar \act \in \wfg{\B'}{s'}$ iff $s' \pj i (\gd{\act}{\B_i}) = \true$.
Putting the above three equalities together gives us clause (2).
\end{proof}

In the sequel, we fix a particular subsystem $\dsks{\act}{\l}$, which
we refer to simply as \DS, with $\act$ and $\l$ being implicit (to
avoid notational clutter with double-sub/superscripts). 
We write $\DS.action = \act$ and $\DS.radius = \l$. 


\begin{definition}[Projection of a wait-for graph] \label{defn:projWgraph}
$\wfg{\B}{s} \pj \DS$ is the wait-for graph whose nodes are the
components and interactions in $\DS$, and whose edges are the induced
edges from $\wfg{\B}{s}$, \ie for nodes $v, w$ of $\wfg{\B}{s} \pj \DS$,
$v \ar w$ is an edge in $\wfg{\B}{s} \pj \DS$ iff $v \ar w$ is an edge in $\wfg{\B}{s}$.

Write $\lwfg{B}{s}{\DS}$ for $\wfg{\B}{s} \pj \DS$.
Also, if $\sD = s \pj \DS$, then define $\lwfg{B}{\sD}{\DS} \df \wfg{\B}{s} \pj \DS$, since $\wfg{\B}{s} \pj \DS$ depends only
on the projection of state $s$ onto $\DS$.
%$\lwfg{B}{s}{\act,\l} \df \wfg{\B}{s} \pj \dsk{\act}{\l}$.
\end{definition}


%
\subsubsection{Fixpoint characterization of supercycles in a  subsystem}

We carry over the defintion of subgraph $\subg$ from Section ...., and develop the analogous definitions for an arbitrary subsystem of $\B$.

\bd[Projection of a wait-for graph] \label{defn:projWgraph}
$\lwfg{B}{s}{\DS} \df \wfg{B}{s} \pj \DS$.
%$\lwfg{B}{s}{\act,\l} \df \wfg{B}{s} \pj \dsk{\act}{\l}$.
\ed

\bd[Set of subgraphs] \label{defn:wsetOfSubgraphs}
$\lwfgS{B}{s}{\DS} \df  \set{ W \stt W \subg \lwfg{B}{s}{\DS} }$.
%$\lwfgS{B}{s}{\act,\l} \df  \set{ W \stt W \subg \lwfg{B}{s}{\act,\l} }$.
\ed

\bd[Wait-for lattice] \label{defn:wflattice}
Define the partially ordered set 
$\llat{\B}{s}{\DS}  = \tpl{ \lwfgS{B}{s}{\DS}  \subg }$ 
%$\llat{\B}{s}{\act,\l}  = \tpl{ \lwfgS{B}{s}{\act,\l}  \subg }$ 
whose elements are all the subgraphs of 
\lwfgS{B}{s}{\DS}, and where  $U \subg V$ is as in \defn{wsubgraph}.   
%iff $U$ is a subgraph of $V$, \ie $\ord$ is the ``is a subgraph of'' order relation.
\ed


\bp \label{prop:isALatticeLoc}
$\llat{\B}{s}{\DS}  = \tpl{ \lwfgS{B}{s}{\DS}  \subg }$
%$\llat{\B}{s}{\act,\l}  = \tpl{ \lwfgS{B}{s}{\act,\l}  \subg }$
 is a complete boolean lattice, with $\meet$, $\join$, and complementation as in 
\prop{isALattice}, top element $\lwfgS{B}{s}{\act,\l}$, and bottom element \ewfg.
\ep


\bd \label{defn:blocksLoc}
%Let $U \subg \lwfgS{B}{s}{\act,\l}$ and $a, B_i$ be nodes in $\wfg{B}{s}$. Then 
Let $U \subg \lwfgS{B}{s}{\DS}$ and $a, B_i$ be nodes in $\wfg{B}{s}$. Then 
$\lblocks{a}{U} \df (\ex B_i \in U : a \ar B_i \sub \wfg{B}{s})$ or $a$ is a border interaction, and 
$\lblocks{B_i}{U} \df (\fa a : B_i \ar a \sub\wfg{B}{s} \imp a \in U \land \lblocks{a}{U})$.
\ed



\bd \label{defn:scFixL}
%Define $\lSFs:  \lwfgS{B}{s}{\act,\l}  \to  \lwfgS{B}{s}{\act,\l}$ as follows.
Define $\lSFs:  \lwfgS{B}{s}{\DS}  \to  \lwfgS{B}{s}{\DS}$ as follows.
$\lSF{W}$ is the subgraph with nodes $\set{w \stt \lblocks{w}{W} }$, together with their induced edges.
\ed


\bd \label{defn:violFixL}
%Define $\lVFs: \lwfgS{B}{s}{\act,\l}  \to  \lwfgS{B}{s}{\act,\l}$ as follows.
Define $\lVFs: \lwfgS{B}{s}{\DS}  \to  \lwfgS{B}{s}{\DS}$ as follows.
$\lVF{W}$ is the subgraph with nodes \compl{\set{w \stt \neg \lblocks{w}{W} }}, together with their induced edges.
\ed


\bp \label{prop:monotoneL}
$\lSFs$ and $\lVFs$ are monotone. and continuous.
\ep


\bp \label{prop:supercycleLocGFP}
Let $\UU \subg \lwfgS{B}{s}{\DS}$. Then $\UU$ is a supercycle in $\lwfgS{B}{s}{\DS}$ implies that $\UU \subg \lSF{\UU}$.
\ep
%
\begin{proof}
....%Since $\SF{\UU} \subg \UU$ by \defn{scFix}, we have $U = \SF{U}$. Hence, ...
\end{proof}


\bp \label{prop:borderLocGFP}
Let $\UU \subg \lwfgS{B}{s}{\DS}$. If $\UU$ consists entirely of border nodes, then $\UU \subg \lSF{\UU}$.
\ep


\bp \label{prop:locGFP}
Let $\UU \subg \lwfgS{B}{s}{\DS}$. Then, $\UU$ is a union of supercycles and border nodes iff $\UU \subg \lSF{\UU}$.
\ep

\bp \label{prop:GFPisLargestLocSC}
Let $\UU$ be the greatest fixpoint of $\lSFs$. Then $\UU$ is the union of all border nodes and all supercycles in 
$\lwfgS{B}{s}{\DS}$.
\ep

\bp \label{prop:LFPisLocScViolations}
Let  $\VS= \lfp{\lVFs}$, \ie $\VS$ is the least fixpoint of $\lVFs$. Then $v \in \VS$ iff $v$ is not a border node of 
$\lwfgS{B}{s}{\DS}$, and $v$ is not a node in any supercycle of $\lwfgS{B}{s}{\DS}$.
%the nodes of $V$ are exactly the nodes in \wfg{B}{s} that have supercycle violations. 
\ep


\bp \label{prop:computeLocLFP}
$\lfp{\lVFs} = \JOIN_{d \ge 0} \lVFs^{d} (\ewfg)$.
\ep
\begin{proof}
$\VFs$ is continuous. Follows by standard results, \eg see the CPO fixpoint theorem I in 
\cite{DP02}.
\end{proof}







%%%%%%%%%%%%%%%%%%%%%%%%%%%%%%%%%%%%%%%%%%%%%%%%%%%%%%%%%%%%%%%%%%%%%%%%%%%%%%%%%%%%%%%%%%



We define the predicate $\lviol{v}{d}{t}{\act}{\l}$ to hold iff node $v$ in $\wfg{B}{t}$ has a level-$d$ supercycle-violation
\emph{that can be confirmed within $\dsk{\act}{\l}$}.

\bd[Local supercycle violation, $\locScViol{v}{d}{t_\act}{\act}{\l}$]
\label{def:supercycle.violation.local}
Let $t_\act$ be a state of $\dsk{\act}{\l}$ and $v$ be a node of $\dsk{\act}{\l}$.
Define $\lviol{v}{d}{t_\act}{\act}{\l} \df v \in \lVFs^{d} (\ewfg)$.
\ed




%%% REMOVED DUE TO FIXPOINT CHARACTERIZATION
% We define $\lviol{v}{d}{t_\act}{\act}{\l}$ by induction on $d$, as follows.

% \noindent
% \ul{Base case, $d=1$.} $\lviol{v}{1}{t_\act}{\act}{\l}$  iff
% $v$ is an interaction $\actp$ and 
% $\actp$ is an interior node of $\dsk{\act}{\l}$ that has no outgoing wait-for edges in $\wfg{\dsk{\act}{\l}}{t_\act}$.
% Otherwise $\neg \lviol{v}{1}{t_\act}{\act}{\l}$. 

% \noindent
% \ul{Inductive step, $d > 1$.} $\lviol{v}{d}{t_\act}{\act}{\l}$ iff either of the following two cases hold. Otherwise $\neg \lviol{v}{d}{t_\act}{\act}{\l}$.

% \bn

% \item \ul{$v$ is a component $\B_i$} and there exists an interaction $\actp$ such that $B_i \ar \actp \in \wfg{\dsk{\act}{\l}}{t_\act}$ and 
%     $(\ex d': 1 \le d' < d : \lviol{\actp}{d'}{t_\act}{\act}{\l})$.
%     That is, $\B_i$ enables an interaction $\actp$ which has a level-$d'$ supercycle-violation in $\dsk{\act}{\l}$, for some $d' < d$. 
% %    It does not matter whether $\B_i$ is border or interior. COMPONENTS ARE ALWAYS INTERIOR NOW


% \item \ul{$v$ is an interaction $\actp$ and an internal node of $\dsk{\act}{\l}$} and
%     for all components $\B_i$ such that $\actp \ar \B_i \in \wfg{\dsk{\act}{\l}}{t_\act}$, we have 
%     $(\ex d' : 1 \le d' < d : \lviol{\B_i}{d'}{t_\act}{\act}{\l})$.
%     That is, each component $\B_i$ that $\actp$ waits for has a level-$d'$ supercycle-violation in $\dsk{\act}{\l}$, for some $d' < d$

% \en
% \ed
% %
% Note that if $v$ is an interaction $\actp$ and a border node, then
% $\lviol{\actp}{d}{t_\act}{\act}{\l}$ is false, for all $d$.  This is because $\actp$ has some
% component that is outside $\dsk{\act}{\l}$, and so this component cannot be checked.  A component
% cannot have a level-1 supercycle-violation since it must
% have at least one outgoing wait-for edge at all times.
% %
% Figure~\ref{fig:scViolateLoc} gives a formal, recursive definition of $\lviol{v}{d}{t_\act}{\act}{\l}$.
% The notation $v = \B_i$ means that $v$ is some component $\B_i$. Likewise, 
% $v = \actp$ means that $v$ is some interaction $\act$, and 
% ``$v = \actp$ is interior'' means that  $v$ is an interaction $\act$ and also an internal node.
% Line 0 corresponds to the base case, line 1 corresponds to item 1 of the inductive case, and line 2 corresponds to item 2 of the inductive case.
% Line 3 handles all cases that do not return true.

% \begin{figure}[ht]
% \setcounter{lctr}{-1}
% \begin{tabbing}
% aa\= aa\= aa\= mm\= mm\=\kill
% $\lviol{v}{d}{t_\act}{\act}{\l}$\\
% \cmnt\ Precondition: $v$ is a node of $\dsk{\act}{\l}$ and $d \ge 1$\\
% \lio{\IFC{d = 1 \land \mbox{$v = \actp$ is interior} \land \neg (\ex \B_i : \actp \ar \B_i \in \wfg{\dsk{\act}{\l}}{t_\act})}  \ \RETURNE{\ttt};}
% %\cmnt\ here $d > 1$\\
% \lio{\IFC{\mbox{$v = \actp $ is interior} \land (\fa \B_i : \actp \ar \B_i \in \wfg{\dsk{\act}{\l}}{t_\act} : (\ex d' : 1 \le d' < d : \lviol{\B_i}{d'}{t_\act}{\act}{\l}))}}
%     \>\>{\RETURNE{\ttt};}\\ 
% \lio{\IFC{\mbox{$v = \B_i$} \land (\ex \actp : \B_i \ar \actp \in \wfg{\dsk{\act}{\l}}{t_\act} : (\ex d' : 1 \le d' < d :\lviol{\actp}{d'}{t_\act}{\act}{\l}))}  \ \RETURNE{\ttt};}
% \lio{\RETURNE{\fff}}
% \end{tabbing}
% \vspace{-6ex}
% \caption{Formal definition of $\lviol{v}{d}{t_\act}{\act}{\l}$.}
% \label{fig:scViolateLoc}
% \end{figure}





% \begin{figure}[ht]
% \setcounter{lctr}{0}
% \begin{tabbing}
% mm\= mm\= mm\= mm\= mm\=\kill
% $\lviol{v}{d}{t_\act}{\act}{\l}$\\
% \cmnt\ Precondition: $v$ is a node of $\dsk{\act}{\l}$ and $d \ge 1$\\

% \lio{\IFC{d = 1}}
%        \lit{\IFC{\mbox{$v$ is an interior interaction $\actp$ and }
%               \neg (\ex \B_i : \actp \ar \B_i \in \wfg{\dsk{\act}{\l}}{t_\act})}}
%                     \lihc{\RETURNE{\ttt}}{\cmnt no outgoing wait-for-edges}
%        \lit{\ELSE\ \RETURNE{\fff}}
%        \lit{\FI}
% \lio{\FI}

% \cmnt\ here $d > 1$\\

% \lio{\IFC{\mbox{$v$ is an interior interaction $\actp$ and } 
%                  (\fa \B_i : \actp \ar \B_i \in \wfg{\dsk{\act}{\l}}{t_\act} : \lviol{\B_i}{d-1}{t_\act}{\act}{\l})}}
%         \lit{\RETURNE{\ttt}}

% \lio{\ELSFC{\mbox{$v$ is a component $\B_i$ and }
%             (\ex \actp : \B_i \ar \actp \in \wfg{\dsk{\act}{\l}}{t_\act} : \lviol{\actp}{d-1}{t_\act}{\act}{\l})}}
%       \lit{\RETURNE{\ttt}}

% \lio{\ELSE\ \RETURNE{\fff}}
% \lio{\FI}
% \end{tabbing}
% \caption{Formal definition of $\lviol{v}{d}{t_\act}{\act}{\l}$.}
% \label{fig:scViolateLoc}
% %\label{alg:check-scViol}
% \end{figure}


We now show that a local supercycle-violation implies (global) supercycle-violation.


\bp
(a) Let $\UU \subg \lwfgS{B}{s}{\DS}$. Then $\lVF{Z} \subg \VF{Z}$.\\
(b) $\lVFs^{d} (\ewfg) \subg \VFs^{d} (\ewfg)$.
\ep


\bp
\label{prop:locScViol-implies-scViol}
 \label{prop:lviol-implies-viol}
Let $t$ be an arbitrary reachable state of BIP-system $(\B, Q_0)$.
For all interactions $\act \in \gamma$, and $\l \ge 1$, let $t_\act = t \pj \dsk{\act}{\l}$.
Then\\
\ind $\fa d \ge 1: \locScViol{v}{d}{t_\act}{\act}{\l} \imp \scViol{v}{d}{t}$.
\ep



% \prf{
% Proof is by induction on $d$. 

% \noindent
% \ul{Base case, $d=1$.} Assume $\lviol{v}{1}{t_\act}{\act}{\l}$ for some node $v$. Then, by 
% Figure~\ref{fig:scViolateLoc}, 
% $v$ is an interior node and an interaction $\actp$ of
% $\dsk{\act}{\l}$, and has no outgoing 
% wait-for edges. Therefore, in $\wfg{\B}{t}$, it is still the case that $v$ has no outgoing 
% wait-for edges. Hence $\viol{v}{1}{t}$ holds.


% \noindent
% \ul{Inductive step, $d > 1$.}
% Assume $\lviol{v}{d}{t_\act}{\act}{\l}$ for some node $v$ and some $d > 1$. 
% We proceed by cases on Figure~\ref{fig:scViolateLoc}.

% \bn

% \item \ul{$v$ is an interior interaction $\actp$ and} \\
% \ul{$(\fa \B_i : \actp \ar \B_i \in \wfg{\dsk{\act}{\l}}{t_\act} : (\ex d' : 1 \le d' < d : \lviol{\B_i}{d'}{t_\act}{\act}{\l}))$}.

% Choose an arbitrary $\B_i$ such that $\actp \ar \B_i \in \wfg{\dsk{\act}{\l}}{t_\act}$.
% By the induction hypothesis applied to $\lviol{\B_i}{d'}{t_\act}{\act}{\l}$, we have $\viol{\B_i}{d'}{t}$ for some $d' < d$.
% Since $\wfg{\dsk{\act}{\l}}{t_\act} \sub \wfg{\B}{t}$ by construction, we have 
% $\actp \ar B_i \in \wfg{\B}{t}$ and $\viol{\B_i}{d'}{t}$.
% Hence by Definition~\ref{def:supercycle-violation}, Clause~\ref{def:supercycle.violation.component.out}, 
% we have $\viol{v}{d}{t}$.


% \item \ul{$v$ is a component $\B_i$ and}\\
% \ul{$(\ex \actp : \B_i \ar \actp \in \wfg{\dsk{\act}{\l}}{t_\act} : (\ex d' : 1 \le d' < d :\lviol{\actp}{d'}{t_\act}{\act}{\l}))$}.

% By the induction hypothesis applied to $\lviol{\actp}{d'}{t_\act}{\act}{\l}$, we have $\viol{\actp}{d'}{t}$ for some $d' < d$.
% Since $\wfg{\dsk{\act}{\l}}{t_\act} \sub \wfg{\B}{t}$ by construction, we have 
% $\B_i \ar \actp \in \wfg{\B}{t}$ and $\viol{\actp}{d'}{t}$.
% Hence by Definition~\ref{def:supercycle-violation}, Clause~\ref{def:supercycle.violation.component.out}, 
% we have $\viol{v}{d}{t}$.

% \en
% }




\subsection{\redbox{Fixpoint characterization of  local supercycles in a  subsystem}}
\label{s:local.fixpoint}

We now develop a local version of the sequence of defnitions and propositions given in \secn{supercycle-fixpoint}.  Local means that they apply to any
subsystem $(\B', Q'_0)$ of $(\B, Q_0)$. A subsystem has, in general, border nodes, \ie those nodes with a neighbor outside of the
subsystem. The supercycle membership of these nodes cannot be determined with certainty, by inspecting just the subsystem.
Hence we pessimistically assume that border nodes are in a supercycle. When false, this assumption may produce a false negative, and so we sacrifice
completeness of our deadlock-freedom criterion. We do however, avoid false positives (that may result if we assume a border node is not in a supercycle when in fact it
is), and so we maintain soundness of our criterion. 


\begin{definition}[Local supercycle]
\label{defn:supercycleLoc} 
Let $\B = \gamma(\B_1,\dots,\B_n)$ be a composite component, $s$ be a state of $\B$, and $\DS$ a subsystem
of $\B$.
A subgraph $\SC$ of  $\lwfg{B}{s}{\DS}$ is a local supercycle in $\lwfg{B}{s}{\DS}$ if and only if all of the following hold:
\begin{nlst1}
   \item $\SC$ is nonempty, \ie contains at least one node,

   \item if $\B_i$ is a node in $\SC$, then for all interactions $\act$ such that
there is an edge in $\wfg{\B}{s}$ from $\B_i$ to $\act$:
      \begin{nlst2}
      \item $\act$ is a node in $\SC$, and 
      \item there is an edge in $\SC$ from $\B_i$ to $\act$,
      \end{nlst2}
that is, $\B_i \ar \act \in \wfg{\B}{s}$ implies $\B_i \ar \act \in \SC$,

   \item 
if $\act$ is a node in $\SC$, then, either $\act$ is a border interaction of $\DS$, or  there exists a $\B_j$ such that:
      \begin{nlst2}
      \item $\B_j$  is a node in $\SC$, and
      \item there is an edge from $\act$ to $\B_j$ in $\wfg{\B}{s}$, and
      \item there is an edge from $\act$ to $\B_j$ in $\SC$,
      \end{nlst2}
that is, $\act \in \SC$ implies $\ex \B_j : \act \ar \B_j \in \wfg{\B}{s} \land \act \ar \B_j \in \SC$.

\end{nlst1}
\end{definition}
%where $\act \in SC$ means that $\act$ is a node in $\SC$, etc. 
%Also, write $SC \sub \wfg{\B}{s}$ when $\SC$ is a subgraph of $\wfg{\B}{s}$.
Intuitively, $\SC$ is a supercycle iff every node in $\SC$ is blocked from executing by other nodes in $\SC$, or is a border interaction. We pessimistically consider a border
interaction $\act$ to be blocked, since the subsystem $\DS$ cannot provide information about the participant components of $\act$ that are outside of
$\DS$. In particular, one or more border interactions necessarily form a local supercycle. Yet, it is important to notice that a blocked border interaction $\act$ does
not necessarily imply a global supercyle.


We carry over the defintion of subgraph $\subg$ from \secn{supercycle-fixpoint}, and develop the analogous definitions for an arbitrary subsystem \DS
of $\B$.


\begin{definition}[Set of subgraphs] \label{defn:wsetOfSubgraphsLoc}
Let $\sD$ be a state of $\DS$. Then 
$\lwfgS{B}{\sD}{\DS} \df  \set{ \XS \stt \XS \subg \lwfg{B}{\sD}{\DS} }$.
\end{definition}

\begin{definition}[Wait-for lattice] \label{defn:wflatticeLoc}
Define the partially ordered set 
$\llat{\B}{\sD}{\DS}  = \tpl{ \lwfgS{B}{\sD}{\DS},  \subg }$ 
%$\llat{\B}{s}{\act,\l}  = \tpl{ \lwfgS{B}{s}{\act,\l}  \subg }$ 
whose elements are all the subgraphs of 
\lwfgS{B}{\sD}{\DS}, and where  $\US \subg \VS$ is as in \defn{wsubgraph}.   
%iff $U$ is a subgraph of $V$, \ie $\ord$ is the ``is a subgraph of'' order relation.
\end{definition}

\begin{proposition} \label{prop:isALatticeLoc}
$\llat{\B}{\sD}{\DS}  = \tpl{ \lwfgS{B}{\sD}{\DS},  \subg }$
%$\llat{\B}{s}{\act,\l}  = \tpl{ \lwfgS{B}{s}{\act,\l}  \subg }$
 is a finite complete boolean lattice, with $\meet$, $\join$, and complementation as in 
\prop{isALattice}, top element $\lwfgS{B}{\sD}{\DS}$, and bottom element \ewfg.
\end{proposition}

\begin{definition}[$\mathit{lblocks}$] \label{defn:blocksLoc}
Let $\XS \subg \lwfg{\B}{\sD}{\DS}$ and $\act, \B_i$ be nodes in $\lwfgS{B}{\sD}{\DS}$. Then 
$\lblocks{\act}{\XS} \df
[ (\ex \B_i \in \XS : \act \ar \B_i \in \lwfgS{B}{\sD}{\DS})
   \mbox{ or $\act$ is a border interaction of $\DS$} ]$, and 
$\lblocks{\B_i}{\XS} \df (\fa \act : \B_i \ar \act \in \lwfg{B}{\sD}{\DS} \imp \act \in \XS)$.
\end{definition}
Hence a border interaction $\act$ is pessimistically considered to be always blocked, since
the subsystem $\DS$ does not contain enough information about the enablement of $\act$.
A non-border interaction $\act$ is (as usual) blocked by a set of nodes $\XS$ if some participant $\B_i$ of 
$\act$ is in $\XS$, and $\B_i$ does not enable $\act$.
A component $\B_i$ is blocked by $\XS$ if all of the interactions that $\B_i$ enables are in $\XS$.


\begin{definition}[$\lSFs$] \label{defn:scFixLoc}
%Define $\lSFs:  \lwfgS{B}{s}{\act,\l}  \to  \lwfgS{B}{s}{\act,\l}$ as follows.
Define $\lSFs:  \lwfgS{B}{s_D}{\DS}  \to  \lwfgS{B}{s}{\DS}$ as follows.
$\lSF{\XS}$ is the subgraph with nodes $\set{\ndv \stt \lblocks{\ndv}{\XS} }$, together with the edges induced by \lwfg{B}{\sD}{\DS}.
\end{definition}

\begin{definition}[$\lVFs$] \label{defn:violFixLoc}
%Define $\lVFs: \lwfgS{B}{s}{\act,\l}  \to  \lwfgS{B}{s}{\act,\l}$ as follows.
Define $\lVFs: \lwfgS{B}{s}{\DS}  \to  \lwfgS{B}{s}{\DS}$ as follows.
$\lVF{\XS}$ is the subgraph with nodes $\set{\ndv \stt \neg \lblocks{\ndv}{\compl{\XS}} }$, together with the edges induced by \lwfg{B}{\sD}{\DS},
where we take the complement $\compl{\XS}$ with respect to $\DS$.
\end{definition}
%
Hence $\lVF{\XS} = \overline{\lSF{\overline{\XS}}}$, \ie $\lVFs$ and $\lSFs$ are duals.
Note that (as for $\SFs$ and $\VFs$) $\lSFs$ and $\lVFs$ are defined given a particular
subsystem $\DS$ of the system 
$\B$, and a particular state $s$ of $\DS$. We will always make explicit the particular
subsystem and state that $\lVFs$ and $\lSFs$ ae defined over.





\begin{proposition} \label{prop:monotoneLoc}
$\lSFs$ and $\lVFs$ are monotone and continuous.
\end{proposition}
%
\begin{proof}
We show first that $\lSFs$ is monotone, \ie $\XS \subg \YS \imp \lSF{\XS} \subg \lSF{\YS}$.
Let $\ndv$ be an arbitrary node in $\SF{\XS}$, so that $\lblocks{\ndv}{\XS}$ holds. There are three cases.\\

\emph{Case of $\ndv$ is a border interaction of $\DS$}. Then $\lblocks{\ndv}{\YS}$ by \defn{blocksLoc}, and so $\ndv \in  \lSF{\YS}$ by
\defn{scFixLoc}.\\

\emph{Case of $\ndv$ is a non-border interaction $\act$}. By Definitions~\ref{defn:scFixLoc} and \ref{defn:blocksLoc}, we have
$\ex \B_i \in \XS  : \act \ar \B_i \in \lwfg{B}{s}{\DS}$.
Since $\XS \subg \YS$, this same $\B_i$ is also a node of $\YS$, and so  $\ex \B_i \in \YS  : \act \ar \B_i \in \lwfg{B}{s}{\DS}$. 
Hence  $\lblocks{\act}{\YS}$, and so $\act \in \lSF{\YS}$. \\

\emph{Case of $\ndv$ is a component $\B_i$}. By Definitions~\ref{defn:scFixLoc} and \ref{defn:blocksLoc}, we have
$(\fa \act : \B_i \ar \act \in \lwfg{\B}{s}{\DS} \imp \act \in \XS)$.
Since $\XS \subg \YS$, we have  $(\fa \act : \B_i \ar \act \in \lwfg{B}{s}{\DS} \imp \act \in \YS)$. 
Hence, $\lblocks{\B_i}{\YS}$, and so $\B_i \in \lSF{\YS}$. \\

In all three cases, we have $\ndv \in \lSF{\YS}$. Since $\ndv$ was chosen arbitrarily from $\lSF{\XS} $, it follows that $\lSF{\XS} \subg \lSF{\YS}$. Hence, $\lSFs$ is monotone.
%
Since the dual of a monotone mapping in a complete boolean lattice is also monotone, we have that $\lVFs$ is monotone.
%
Finally, since $\llat{\B}{s}{\DS}$ is finite, it follows that $\lSFs$ and $\lVFs$ are continuous.
\end{proof}




\begin{proposition} \label{prop:locGFP} \label{prop:supercycleGFPLoc}
Let $\XS \ne \ewfg$ and $\XS \subg \lwfg{B}{s}{\DS}$, \ie $\XS$ is a non-empty subgraph of $\lwfg{\B}{s}{\DS}$.
Then, $\XS$ is a local supercycle in $\lwfg{\B}{s}{\DS}$ iff $\XS \subg \lSF{\XS}$.
\end{proposition}
%
\begin{proof}
Let $\XS$ be a local supercycle in $\lwfg{B}{s}{\DS}$. By \defn{supercycleLoc}, every node in $\XS$ is blocked by $\XS$ or is a border interaction, \ie 
$(\fa \ndx \in \XS: \lblocks{\ndx}{\XS})$. By \defn{scFix}, $\XS \subg \SF{\XS}$.

Conversely, suppose $\XS \subg \SF{\XS}$ for some subgraph $\XS$ of $\lwfg{B}{s}{\DS}$. Hence 
$(\fa \ndx \in \XS: \ndx \in \SF{\XS})$, so by \defn{scFixLoc}, $(\fa \ndx \in \XS: \lblocks{\ndx}{\XS})$.
Hence every node in $\XS$ is blocked by $\XS$ or is a border interaction, and so $\XS$ satisfies \defn{supercycleLoc}, and is therefore a local supercycle.
\end{proof}

\begin{proposition} \label{prop:supercycleLoc:union}
Let $\SC, \SC'$ be local supercycles in $\lwfg{B}{s}{\DS}$. Then $\SC \join \SC'$ is
a local supercycle in $\lwfg{B}{s}{\DS}$.
\end{proposition}
%
\begin{proof}
By \prop{locGFP}, $\SC$ and $\SC'$ are post-fixpoints of $\lSFs$. Since the join of post-fixpoints is a post-fixpoint, 
the proposition follows by applying \prop{locGFP} again.
%Straightforward, since each node in  $\SC \un \SC'$ has enough successors that it waits for to satisfy  \defn{supercycle}. 
\end{proof}


\begin{proposition} \label{prop:GFPisLargestSCLoc}
Let $\SC$ be the greatest fixpoint of $\lSFs$. Then either
(a) $\lwfg{\B}{\sD}{\DS}$ is supercycle-free and $\SC = \ewfg$, or 
(b) $\lwfg{\B}{\sD}{\DS}$ contains supercycles, and $\SC$ is the largest supercycle in $\lwfg{\B}{\sD}{\DS}$ 
\end{proposition}
%
\begin{proof}
By the Knaster-Tarski theorem, the greatest fixpoint is the join of all the post-fixpoints. 
If $\lwfg{B}{s}{\DS}$ is supercycle-free, then by \prop{supercycleGFPLoc}, the set of post-fixpoints of $\lSFs$ is empty. 
Hence $\SC = \ewfg$ (this is possible since there may be no border interactions). %CHECK DEF BORDER INTERACTION
If $\lwfg{B}{s}{\DS}$ contains supercycles, then by \prop{supercycleGFPLoc},  the set of post-fixpoints of $\lSFs$ is exactly the set of 
local supercycles of $\lwfg{B}{s}{\DS}$, and so $\SC$ is the join of all these local supercycles, and so $\SC$ is the largest local supercycle in 
$\lwfg{\B}{\sD}{\DS}$.
\end{proof}



\begin{proposition} \label{prop:LFPisLocScViolations}
Let  $\VS= \lfp{\lVFs}$, \ie $\VS$ is the least fixpoint of $\lVFs$. Then $v \in \VS$ iff 
$v$ is not a node in any local supercycle of $\lwfgS{B}{s}{\DS}$.
%the nodes of $V$ are exactly the nodes in \wfg{\B}{s} that have supercycle violations. 
\end{proposition}
%
\begin{proof}
%Follows from \prop{GFPisLargestSC} are the Park conjugate (dual) fixpoint theorem in complete boolean lattices.
From the Park conjugate (dual) fixpoint theorem in complete boolean lattices, we have 
\lfp{\VFs} = \compl{\gfp{\SFs}}.
By \prop{GFPisLargestSC}, \gfp{\SFs} is the largest local supercycle in \wfg{\B}{s}. Hence the nodes not in 
\gfp{\SFs} are exactly the nodes that have local supercycle violations. These are exactly the nodes in \lfp{\VFs}.
\end{proof}

Define $\lVFi{\XS}{1} = \lVF{\XS}$, and for $d > 1$, 
$\lVFi{\XS}{d} = \lVFi{\lVF{\XS}}{d-1}$, \ie a superscript indicates functional iteration
of $\lVFs$. Note that 
$\lVFi{\ewfg}{d} \subg \lVFi{\ewfg}{d'}$ when $d \le d'$, since $\lVFs$
is monotone.
Hence $\lVFi{\ewfg}{1}, \lVFi{\ewfg}{2}, \ldots$ is a non-decreasing sequence.

\begin{proposition} \label{prop:computeLocLFP}
$\lfp{\lVFs} = \JOIN_{d \ge 1} \lVFs^{d} (\ewfg)$.
\end{proposition}
%
\begin{proof}
$\lVFs$ is continuous. Follows by standard results, \eg see the CPO fixpoint theorem I in 
\cite{DP02}.
\end{proof}







%%%%%%%%%%%%%%%%%%%%%%%%%%%%%%%%%%%%%%%%%%%%%%%%%%%%%%%%%%%%%%%%%%%%%%%%%%%%%%%%%%%%%%%%%%



\begin{definition}[Local supercycle violation, $\scVL{v}{t_\DS}{\act}{\l}, \scVLd{v}{d}{t_\DS}{\act}{\l}$]
\label{def:supercycle.violation.local}
\label{defn:supercycle.violation.local}
Let $t_\DS$ be a state of $\dsk{\act}{\l}$ and $v$ be a node of $\dsk{\act}{\l}$.
Define 
$\scVL{v}{t_\DS}{\act}{\l} \df v \in \lfp{\lVFs}$,
%\JOIN_{d \ge 0} \lVFs^{d} (\ewfg),
and, for $d \ge 1$, $\scVLd{v}{d}{t_\DS}{\act}{\l} \df v \in \lVFs^{d} (\ewfg)$.
\end{definition}

%We define the predicate $\lviol{v}{d}{t}{\act}{\l}$ to hold iff node $v$ in $\wfg{\B}{t}$ has a level-$d$ supercycle-violation
%\emph{that can be confirmed within $\dsk{\act}{\l}$}.

\begin{proposition}
\label{prop:locViol-equiv-locViolDist}
$\scVL{v}{t_\DS}{\act}{\l} \ev (\ex d \ge 1: \scVLd{v}{d}{t_\DS}{\act}{\l})$.
\end{proposition}
%
\begin{proof}
By \defn{supercycle.violation.local}, $\scVL{\ndv}{t_\DS}{\act}{\l} \ev v \in \lfp{\lVFs}$.
By \prop{computeLocLFP}, $v \in \lfp{\lVFs} \ev v \in \JOIN_{d \ge 1} \lVFs^{d} (\ewfg)$.
By \defn{supercycle.violation.local}, $\fa d \ge 1: (\scVLd{v}{d}{t_\DS}{\act}{\l} \ev v \in \lVFs^{d} (\ewfg))$.
Chaining these equivalences establishes the proposition.
\end{proof}
%
$\scVLd{v}{d}{t_\DS}{\act}{\l}$ defines a local supercycle violation that can be confirmed within $d$ iterations of $\lVFs$, which we call a
\emph{level-$d$ local supercycle violation}.
$\scVL{v}{t_\DS}{\act}{\l}$ requires, in general, the entire least fixed point of $\lVFs$.



\begin{example}[Local supercycle violations in dining philosophers]
\label{exm:loc-dphils-viols}
Figures~\ref{fig:violsLocGrab}, \ref{fig:violsLocGrabGrab}, and \ref{fig:violsLocGrabGrabRel} illustrate local supercycle violations corresponding to
Figures~\ref{fig:violsGrab}, \ref{fig:violsGrabGrab}, \ref{fig:violsGrabGrabRel} respectively. The subsystem used, in each case, is based on the last interaction
executed, \ie $\Grab_0$, $\Grab_2$, and $\Rel_0$, respectively, and with a distance of 1 in all cases.
The border interactions are shown underlined, and 
for each node $\ndv$ (interaction or component), we include a small positive integer after its name, giving the smallest $d$ such that $v \in \lVFi{\ewfg}{d}$, 
\ie the local supercycle violation level.
\end{example}



\begin{figure*}[ht]
  \begin{center}
%      \subfigure[Supercycle violations in initial state.]{\label{fig:violsInitial}\scalebox{0.4}{\input{figs/scvDiningInitial.pdf_t}}} \quad \quad
      \subfigure[Local supercycle violations after execution of $\Grab_0$.]{\label{fig:violsLocGrab}\scalebox{0.4}{\input{figs/scvLocDining1.pdf_t}}}
      \quad \quad
      \subfigure[Local supercycle violations after execution of $\Grab_0; \Grab_2$.]{\label{fig:violsLocGrabGrab}\scalebox{0.4}{\input{figs/scvLocDining2.pdf_t}}} \quad \quad
      \subfigure[Local supercycle violations after execution of $\Grab_0; \Grab_2; \Rel_0$.]{\label{fig:violsLocGrabGrabRel}\scalebox{0.4}{\input{figs/scvLocDining3.pdf_t}}} 
      \caption{Example supercycle violations for dining philosophers system of Figure~\ref{fig:diningSpectrum}.}
       \label{fig:localDphilsViolations}
  \end{center}
\end{figure*}









We now show that a local supercycle violation implies (global) supercycle violation.

\begin{proposition}
\label{prop:lV-subg-V}
(a) Let $\XS \subg \lwfg{B}{s}{\DS}$. Then $\lVF{\XS} \subg \VF{\XS}$.\\
(b) Let $d \ge 1$. Then $\lVFs^{d} (\ewfg) \subg \VFs^{d} (\ewfg)$.
\end{proposition}
%
\begin{proof}
For (a), let $\ndv \in \lVF{\XS}$. By \defn{violFixLoc}, $\neg \lblocks{\ndv}{\complL{\XS}}$.
Now $\ndv$ is either an interaction $\act$ or a component $\B_i$. 

By \defn{blocksLoc}, if $\ndv$ is an interaction $\act$, then it is not a border interaction, and furthermore there is no
component $\B_i \in \complL{\XS}$ such that $\act \ar \B_i \in \lwfg{\B}{s}{\DS}$. 
Since $\complL{\XS} \subg \compl{\XS}$, we conclude $\neg \blocks{\ndv}{\compl{\XS}}$, and so $\ndv \in \VF{\XS}$.

By \defn{blocksLoc}, if $\ndv$ is a component $\B_i$, then there exists an interaction $\act$ such that 
$\B_i \ar \act \in \lwfg{\B}{s}{\DS}$ and $\act \not\in \complL{\XS}$. Hence $\act \in \XS$, and so $\act \not\in \compl{\XS}$.
Hence  $\neg \blocks{\ndv}{\compl{\XS}}$, and so $\ndv \in \VF{\XS}$.

In both cases, the arbitrary element $\ndv$ of $\lVF{\XS}$ is also an element of $\VF{\XS}$, and so $\lVF{\XS} \subg \VF{\XS}$.

We establish (b) by induction on $d$. The base case is $d=1$, which is given by (a).
%$d=0$, for which $\lVFs^{d} (\ewfg) = \VFs^{d} (\ewfg) = \ewfg$.
For the induction step, $d > 1$, we have the induction hypothesis $\lVFs^{d-1} (\ewfg) \subg \VFs^{d-1} (\ewfg)$. 
Hence $\lVF{\lVFs^{d-1} (\ewfg)} \subg \lVF{\VFs^{d-1} (\ewfg)}$ since $\lVFs$ is monotone.
By (a) $\lVF{\VFs^{d-1} (\ewfg)} \subg \VF{\VFs^{d-1} (\ewfg)}$. 
Hence  $\lVF{\lVFs^{d-1} (\ewfg)} \subg \VF{\VFs^{d-1} (\ewfg)}$, \ie
  $\lVFs^{d} (\ewfg) \subg \VFs^{d} (\ewfg)$, and so (b) is established.
\end{proof}


\begin{proposition}
\label{prop:locScViol-implies-scViol}
\label{prop:lviol-implies-viol}
Let $t$ be an arbitrary reachable state of BIP-system $(\B, Q_0)$.
For all interactions $\act \in \gamma$ and $\l \ge 1$, let $\DS = \dsks{\act}{\l}$ and $t_\DS = t \pj \dsk{\act}{\l}$.
Then\\
\ind (a) $\fa d \ge 1:  (\scVLd{\ndv}{d}{t_\DS}{\act}{\l} \imp \scVd{v}{d}{t}) $, and\\
\ind (b) $\scVL{\ndv}{t_\DS}{\act}{\l} \imp \scV{v}{t}$.
\end{proposition}
%
\begin{proof}
For (a), assume $\scVLd{\ndv}{d}{t_\DS}{\act}{\l}$ for some arbitrary $d \ge 1$.
By \defn{supercycle.violation.local},  $\ndv \in \lVFs^{d} (\ewfg)$.
By \prop{lV-subg-V},  $\ndv  \in \VFs^{d} (\ewfg)$.
By \defn{supercycle.violation}, $\scVd{\ndv}{d}{t}$.

For (b), assume $\scVL{\ndv}{t_\DS}{\act}{\l}$. By \prop{locViol-equiv-locViolDist},
$\scVLd{\ndv}{d}{t_\DS}{\act}{\l}$ for some $d \ge 1$. By (a), we have $\scVd{v}{d}{t}$.
By \prop{globViol-equiv-globViolDist}, we have $\scV{v}{t}$.
\end{proof}

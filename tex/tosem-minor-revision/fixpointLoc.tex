
\subsubsection{Fixpoint characterization of supercycles in a  subsystem}

We carry over the defintion of subgraph $\subg$ from Section ...., and develop the analogous definitions for an arbitrary subsystem of $\B$.

\bd[Projection of a wait-for graph] \label{defn:projWgraph}
$\lwfg{B}{s}{\DS} \df \wfg{B}{s} \pj \DS$.
%$\lwfg{B}{s}{\act,\l} \df \wfg{B}{s} \pj \dsk{\act}{\l}$.
\ed

\bd[Set of subgraphs] \label{defn:wsetOfSubgraphs}
$\lwfgS{B}{s}{\DS} \df  \set{ W \stt W \subg \lwfg{B}{s}{\DS} }$.
%$\lwfgS{B}{s}{\act,\l} \df  \set{ W \stt W \subg \lwfg{B}{s}{\act,\l} }$.
\ed

\bd[Wait-for lattice] \label{defn:wflattice}
Define the partially ordered set 
$\llat{\B}{s}{\DS}  = \tpl{ \lwfgS{B}{s}{\DS}  \subg }$ 
%$\llat{\B}{s}{\act,\l}  = \tpl{ \lwfgS{B}{s}{\act,\l}  \subg }$ 
whose elements are all the subgraphs of 
\lwfgS{B}{s}{\DS}, and where  $U \subg V$ is as in \defn{wsubgraph}.   
%iff $U$ is a subgraph of $V$, \ie $\ord$ is the ``is a subgraph of'' order relation.
\ed


\bp \label{prop:isALatticeLoc}
$\llat{\B}{s}{\DS}  = \tpl{ \lwfgS{B}{s}{\DS}  \subg }$
%$\llat{\B}{s}{\act,\l}  = \tpl{ \lwfgS{B}{s}{\act,\l}  \subg }$
 is a complete boolean lattice, with $\meet$, $\join$, and complementation as in 
\prop{isALattice}, top element $\lwfgS{B}{s}{\act,\l}$, and bottom element \ewfg.
\ep


\bd \label{defn:blocksLoc}
%Let $U \subg \lwfgS{B}{s}{\act,\l}$ and $a, B_i$ be nodes in $\wfg{B}{s}$. Then 
Let $U \subg \lwfgS{B}{s}{\DS}$ and $a, B_i$ be nodes in $\wfg{B}{s}$. Then 
$\lblocks{a}{U} \df (\ex B_i \in U : a \ar B_i \sub \wfg{B}{s})$ or $a$ is a border interaction, and 
$\lblocks{B_i}{U} \df (\fa a : B_i \ar a \sub\wfg{B}{s} \imp a \in U \land \lblocks{a}{U})$.
\ed



\bd \label{defn:scFixL}
%Define $\lSFs:  \lwfgS{B}{s}{\act,\l}  \to  \lwfgS{B}{s}{\act,\l}$ as follows.
Define $\lSFs:  \lwfgS{B}{s}{\DS}  \to  \lwfgS{B}{s}{\DS}$ as follows.
$\lSF{W}$ is the subgraph with nodes $\set{w \stt \lblocks{w}{W} }$, together with their induced edges.
\ed


\bd \label{defn:violFixL}
%Define $\lVFs: \lwfgS{B}{s}{\act,\l}  \to  \lwfgS{B}{s}{\act,\l}$ as follows.
Define $\lVFs: \lwfgS{B}{s}{\DS}  \to  \lwfgS{B}{s}{\DS}$ as follows.
$\lVF{W}$ is the subgraph with nodes \compl{\set{w \stt \neg \lblocks{w}{W} }}, together with their induced edges.
\ed


\bp \label{prop:monotoneL}
$\lSFs$ and $\lVFs$ are monotone. and continuous.
\ep


\bp \label{prop:supercycleLocGFP}
Let $\UU \subg \lwfgS{B}{s}{\DS}$. Then $\UU$ is a supercycle in $\lwfgS{B}{s}{\DS}$ implies that $\UU \subg \lSF{\UU}$.
\ep
%
\begin{proof}
....%Since $\SF{\UU} \subg \UU$ by \defn{scFix}, we have $U = \SF{U}$. Hence, ...
\end{proof}


\bp \label{prop:borderLocGFP}
Let $\UU \subg \lwfgS{B}{s}{\DS}$. If $\UU$ consists entirely of border nodes, then $\UU \subg \lSF{\UU}$.
\ep


\bp \label{prop:locGFP}
Let $\UU \subg \lwfgS{B}{s}{\DS}$. Then, $\UU$ is a union of supercycles and border nodes iff $\UU \subg \lSF{\UU}$.
\ep

\bp \label{prop:GFPisLargestLocSC}
Let $\UU$ be the greatest fixpoint of $\lSFs$. Then $\UU$ is the union of all border nodes and all supercycles in 
$\lwfgS{B}{s}{\DS}$.
\ep

\bp \label{prop:LFPisLocScViolations}
Let  $\VS= \lfp{\lVFs}$, \ie $\VS$ is the least fixpoint of $\lVFs$. Then $v \in \VS$ iff $v$ is not a border node of 
$\lwfgS{B}{s}{\DS}$, and $v$ is not a node in any supercycle of $\lwfgS{B}{s}{\DS}$.
%the nodes of $V$ are exactly the nodes in \wfg{B}{s} that have supercycle violations. 
\ep


\bp \label{prop:computeLocLFP}
$\lfp{\lVFs} = \JOIN_{d \ge 0} \lVFs^{d} (\ewfg)$.
\ep
\begin{proof}
$\VFs$ is continuous. Follows by standard results, \eg see the CPO fixpoint theorem I in 
\cite{DP02}.
\end{proof}







%%%%%%%%%%%%%%%%%%%%%%%%%%%%%%%%%%%%%%%%%%%%%%%%%%%%%%%%%%%%%%%%%%%%%%%%%%%%%%%%%%%%%%%%%%



We define the predicate $\lviol{v}{d}{t}{\act}{\l}$ to hold iff node $v$ in $\wfg{B}{t}$ has a level-$d$ supercycle-violation
\emph{that can be confirmed within $\dsk{\act}{\l}$}.

\bd[Local supercycle violation, $\locScViol{v}{d}{t_\act}{\act}{\l}$]
\label{def:supercycle.violation.local}
Let $t_\act$ be a state of $\dsk{\act}{\l}$ and $v$ be a node of $\dsk{\act}{\l}$.
Define $\lviol{v}{d}{t_\act}{\act}{\l} \df v \in \lVFs^{d} (\ewfg)$.
\ed




%%% REMOVED DUE TO FIXPOINT CHARACTERIZATION
% We define $\lviol{v}{d}{t_\act}{\act}{\l}$ by induction on $d$, as follows.

% \noindent
% \ul{Base case, $d=1$.} $\lviol{v}{1}{t_\act}{\act}{\l}$  iff
% $v$ is an interaction $\actp$ and 
% $\actp$ is an interior node of $\dsk{\act}{\l}$ that has no outgoing wait-for edges in $\wfg{\dsk{\act}{\l}}{t_\act}$.
% Otherwise $\neg \lviol{v}{1}{t_\act}{\act}{\l}$. 

% \noindent
% \ul{Inductive step, $d > 1$.} $\lviol{v}{d}{t_\act}{\act}{\l}$ iff either of the following two cases hold. Otherwise $\neg \lviol{v}{d}{t_\act}{\act}{\l}$.

% \bn

% \item \ul{$v$ is a component $\B_i$} and there exists an interaction $\actp$ such that $B_i \ar \actp \in \wfg{\dsk{\act}{\l}}{t_\act}$ and 
%     $(\ex d': 1 \le d' < d : \lviol{\actp}{d'}{t_\act}{\act}{\l})$.
%     That is, $\B_i$ enables an interaction $\actp$ which has a level-$d'$ supercycle-violation in $\dsk{\act}{\l}$, for some $d' < d$. 
% %    It does not matter whether $\B_i$ is border or interior. COMPONENTS ARE ALWAYS INTERIOR NOW


% \item \ul{$v$ is an interaction $\actp$ and an internal node of $\dsk{\act}{\l}$} and
%     for all components $\B_i$ such that $\actp \ar \B_i \in \wfg{\dsk{\act}{\l}}{t_\act}$, we have 
%     $(\ex d' : 1 \le d' < d : \lviol{\B_i}{d'}{t_\act}{\act}{\l})$.
%     That is, each component $\B_i$ that $\actp$ waits for has a level-$d'$ supercycle-violation in $\dsk{\act}{\l}$, for some $d' < d$

% \en
% \ed
% %
% Note that if $v$ is an interaction $\actp$ and a border node, then
% $\lviol{\actp}{d}{t_\act}{\act}{\l}$ is false, for all $d$.  This is because $\actp$ has some
% component that is outside $\dsk{\act}{\l}$, and so this component cannot be checked.  A component
% cannot have a level-1 supercycle-violation since it must
% have at least one outgoing wait-for edge at all times.
% %
% Figure~\ref{fig:scViolateLoc} gives a formal, recursive definition of $\lviol{v}{d}{t_\act}{\act}{\l}$.
% The notation $v = \B_i$ means that $v$ is some component $\B_i$. Likewise, 
% $v = \actp$ means that $v$ is some interaction $\act$, and 
% ``$v = \actp$ is interior'' means that  $v$ is an interaction $\act$ and also an internal node.
% Line 0 corresponds to the base case, line 1 corresponds to item 1 of the inductive case, and line 2 corresponds to item 2 of the inductive case.
% Line 3 handles all cases that do not return true.

% \begin{figure}[ht]
% \setcounter{lctr}{-1}
% \begin{tabbing}
% aa\= aa\= aa\= mm\= mm\=\kill
% $\lviol{v}{d}{t_\act}{\act}{\l}$\\
% \cmnt\ Precondition: $v$ is a node of $\dsk{\act}{\l}$ and $d \ge 1$\\
% \lio{\IFC{d = 1 \land \mbox{$v = \actp$ is interior} \land \neg (\ex \B_i : \actp \ar \B_i \in \wfg{\dsk{\act}{\l}}{t_\act})}  \ \RETURNE{\ttt};}
% %\cmnt\ here $d > 1$\\
% \lio{\IFC{\mbox{$v = \actp $ is interior} \land (\fa \B_i : \actp \ar \B_i \in \wfg{\dsk{\act}{\l}}{t_\act} : (\ex d' : 1 \le d' < d : \lviol{\B_i}{d'}{t_\act}{\act}{\l}))}}
%     \>\>{\RETURNE{\ttt};}\\ 
% \lio{\IFC{\mbox{$v = \B_i$} \land (\ex \actp : \B_i \ar \actp \in \wfg{\dsk{\act}{\l}}{t_\act} : (\ex d' : 1 \le d' < d :\lviol{\actp}{d'}{t_\act}{\act}{\l}))}  \ \RETURNE{\ttt};}
% \lio{\RETURNE{\fff}}
% \end{tabbing}
% \vspace{-6ex}
% \caption{Formal definition of $\lviol{v}{d}{t_\act}{\act}{\l}$.}
% \label{fig:scViolateLoc}
% \end{figure}





% \begin{figure}[ht]
% \setcounter{lctr}{0}
% \begin{tabbing}
% mm\= mm\= mm\= mm\= mm\=\kill
% $\lviol{v}{d}{t_\act}{\act}{\l}$\\
% \cmnt\ Precondition: $v$ is a node of $\dsk{\act}{\l}$ and $d \ge 1$\\

% \lio{\IFC{d = 1}}
%        \lit{\IFC{\mbox{$v$ is an interior interaction $\actp$ and }
%               \neg (\ex \B_i : \actp \ar \B_i \in \wfg{\dsk{\act}{\l}}{t_\act})}}
%                     \lihc{\RETURNE{\ttt}}{\cmnt no outgoing wait-for-edges}
%        \lit{\ELSE\ \RETURNE{\fff}}
%        \lit{\FI}
% \lio{\FI}

% \cmnt\ here $d > 1$\\

% \lio{\IFC{\mbox{$v$ is an interior interaction $\actp$ and } 
%                  (\fa \B_i : \actp \ar \B_i \in \wfg{\dsk{\act}{\l}}{t_\act} : \lviol{\B_i}{d-1}{t_\act}{\act}{\l})}}
%         \lit{\RETURNE{\ttt}}

% \lio{\ELSFC{\mbox{$v$ is a component $\B_i$ and }
%             (\ex \actp : \B_i \ar \actp \in \wfg{\dsk{\act}{\l}}{t_\act} : \lviol{\actp}{d-1}{t_\act}{\act}{\l})}}
%       \lit{\RETURNE{\ttt}}

% \lio{\ELSE\ \RETURNE{\fff}}
% \lio{\FI}
% \end{tabbing}
% \caption{Formal definition of $\lviol{v}{d}{t_\act}{\act}{\l}$.}
% \label{fig:scViolateLoc}
% %\label{alg:check-scViol}
% \end{figure}


We now show that a local supercycle-violation implies (global) supercycle-violation.


\bp
(a) Let $\UU \subg \lwfgS{B}{s}{\DS}$. Then $\lVF{Z} \subg \VF{Z}$.\\
(b) $\lVFs^{d} (\ewfg) \subg \VFs^{d} (\ewfg)$.
\ep


\bp
\label{prop:locScViol-implies-scViol}
 \label{prop:lviol-implies-viol}
Let $t$ be an arbitrary reachable state of BIP-system $(\B, Q_0)$.
For all interactions $\act \in \gamma$, and $\l \ge 1$, let $t_\act = t \pj \dsk{\act}{\l}$.
Then\\
\ind $\fa d \ge 1: \locScViol{v}{d}{t_\act}{\act}{\l} \imp \scViol{v}{d}{t}$.
\ep



% \prf{
% Proof is by induction on $d$. 

% \noindent
% \ul{Base case, $d=1$.} Assume $\lviol{v}{1}{t_\act}{\act}{\l}$ for some node $v$. Then, by 
% Figure~\ref{fig:scViolateLoc}, 
% $v$ is an interior node and an interaction $\actp$ of
% $\dsk{\act}{\l}$, and has no outgoing 
% wait-for edges. Therefore, in $\wfg{\B}{t}$, it is still the case that $v$ has no outgoing 
% wait-for edges. Hence $\viol{v}{1}{t}$ holds.


% \noindent
% \ul{Inductive step, $d > 1$.}
% Assume $\lviol{v}{d}{t_\act}{\act}{\l}$ for some node $v$ and some $d > 1$. 
% We proceed by cases on Figure~\ref{fig:scViolateLoc}.

% \bn

% \item \ul{$v$ is an interior interaction $\actp$ and} \\
% \ul{$(\fa \B_i : \actp \ar \B_i \in \wfg{\dsk{\act}{\l}}{t_\act} : (\ex d' : 1 \le d' < d : \lviol{\B_i}{d'}{t_\act}{\act}{\l}))$}.

% Choose an arbitrary $\B_i$ such that $\actp \ar \B_i \in \wfg{\dsk{\act}{\l}}{t_\act}$.
% By the induction hypothesis applied to $\lviol{\B_i}{d'}{t_\act}{\act}{\l}$, we have $\viol{\B_i}{d'}{t}$ for some $d' < d$.
% Since $\wfg{\dsk{\act}{\l}}{t_\act} \sub \wfg{\B}{t}$ by construction, we have 
% $\actp \ar B_i \in \wfg{\B}{t}$ and $\viol{\B_i}{d'}{t}$.
% Hence by Definition~\ref{def:supercycle-violation}, Clause~\ref{def:supercycle.violation.component.out}, 
% we have $\viol{v}{d}{t}$.


% \item \ul{$v$ is a component $\B_i$ and}\\
% \ul{$(\ex \actp : \B_i \ar \actp \in \wfg{\dsk{\act}{\l}}{t_\act} : (\ex d' : 1 \le d' < d :\lviol{\actp}{d'}{t_\act}{\act}{\l}))$}.

% By the induction hypothesis applied to $\lviol{\actp}{d'}{t_\act}{\act}{\l}$, we have $\viol{\actp}{d'}{t}$ for some $d' < d$.
% Since $\wfg{\dsk{\act}{\l}}{t_\act} \sub \wfg{\B}{t}$ by construction, we have 
% $\B_i \ar \actp \in \wfg{\B}{t}$ and $\viol{\actp}{d'}{t}$.
% Hence by Definition~\ref{def:supercycle-violation}, Clause~\ref{def:supercycle.violation.component.out}, 
% we have $\viol{v}{d}{t}$.

% \en
% }




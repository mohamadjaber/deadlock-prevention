

   \subsection{A local AND-OR condition for deadlock-freedom}
   \label{s:ANDORcond}
%   
We now seek a local condition, which we evaluate in $\dsk{\act}{\l}$, and which implies \GAO.
We define local versions of both $\scViol{v}{d}{t}$ and $\connViol{v}{t}$.

To achieve a local and conservative approximation of $\scViol{v}{d}{t}$, we make the ``pessimistic'' assumption that the violation status of border
nodes of $\dsk{\act}{\l}$ cannot be known, since it depends on nodes outside of $\dsk{\act}{\l}$.
Now, if an internal node $v$ of $\dsk{\act}{\l}$ can be marked with a level $d$ sc-violation, by applying 
Definition~\ref{def:supercycle-violation} only within
$\dsk{\act}{\l}$, and with the border nodes marked as non-violating,
then it is also the case, as we show below, that $v$ has a level $d$ sc-violation overall.

To achieve a local and conservative approximation of
$\connViol{v}{t}$, we project onto a subsystem.




\subsubsection{Local supercycle violation condition}

We define the predicate $\lviol{v}{d}{t}{\act}{\l}$ to hold iff node $v$ in $\wfg{B}{t}$ has a level-$d$ supercycle-violation
\emph{that can be confirmed within $\dsk{\act}{\l}$}.

\bd[Local supercycle violation, $\locScViol{v}{d}{t_\act}{\act}{\l}$]
\label{def:supercycle.violation.local}
Let $t_\act$ be a state of $\dsk{\act}{\l}$ and $v$ be a node of $\dsk{\act}{\l}$.
We define $\lviol{v}{d}{t_\act}{\act}{\l}$ by induction on $d$, as follows.

\noindent
\ul{Base case, $d=1$.} $\lviol{v}{1}{t_\act}{\act}{\l}$  iff
$v$ is an interaction $\actp$ and 
$\actp$ is an interior node of $\dsk{\act}{\l}$ that has no outgoing wait-for edges in $\wfg{\dsk{\act}{\l}}{t_\act}$.
Otherwise $\neg \lviol{v}{1}{t_\act}{\act}{\l}$. 

\noindent
\ul{Inductive step, $d > 1$.} $\lviol{v}{d}{t_\act}{\act}{\l}$ iff either of the following two cases hold. Otherwise $\neg \lviol{v}{d}{t_\act}{\act}{\l}$.

\bn

\item \ul{$v$ is a component $\B_i$} and there exists an interaction $\actp$ such that $B_i \ar \actp \in \wfg{\dsk{\act}{\l}}{t_\act}$ and 
    $(\ex d': 1 \le d' < d : \lviol{\actp}{d'}{t_\act}{\act}{\l})$.
    That is, $\B_i$ enables an interaction $\actp$ which has a level-$d'$ supercycle-violation in $\dsk{\act}{\l}$, for some $d' < d$. 
%    It does not matter whether $\B_i$ is border or interior. COMPONENTS ARE ALWAYS INTERIOR NOW


\item \ul{$v$ is an interaction $\actp$ and an internal node of $\dsk{\act}{\l}$} and
    for all components $\B_i$ such that $\actp \ar \B_i \in \wfg{\dsk{\act}{\l}}{t_\act}$, we have 
    $(\ex d' : 1 \le d' < d : \lviol{\B_i}{d'}{t_\act}{\act}{\l})$.
    That is, each component $\B_i$ that $\actp$ waits for has a level-$d'$ supercycle-violation in $\dsk{\act}{\l}$, for some $d' < d$

\en
\ed
%
Note that if $v$ is an interaction $\actp$ and a border node, then
$\lviol{\actp}{d}{t_\act}{\act}{\l}$ is false, for all $d$.  This is because $\actp$ has some
component that is outside $\dsk{\act}{\l}$, and so this component cannot be checked.  A component
cannot have a level-1 supercycle-violation since it must
have at least one outgoing wait-for edge at all times.
%
Figure~\ref{fig:scViolateLoc} gives a formal, recursive definition of $\lviol{v}{d}{t_\act}{\act}{\l}$.
The notation $v = \B_i$ means that $v$ is some component $\B_i$. Likewise, 
$v = \actp$ means that $v$ is some interaction $\act$, and 
``$v = \actp$ is interior'' means that  $v$ is an interaction $\act$ and also an internal node.
Line 0 corresponds to the base case, line 1 corresponds to item 1 of the inductive case, and line 2 corresponds to item 2 of the inductive case.
Line 3 handles all cases that do not return true.

\begin{figure}[ht]
\setcounter{lctr}{-1}
\begin{tabbing}
aa\= aa\= aa\= mm\= mm\=\kill
$\lviol{v}{d}{t_\act}{\act}{\l}$\\
\cmnt\ Precondition: $v$ is a node of $\dsk{\act}{\l}$ and $d \ge 1$\\
\lio{\IFC{d = 1 \land \mbox{$v = \actp$ is interior} \land \neg (\ex \B_i : \actp \ar \B_i \in \wfg{\dsk{\act}{\l}}{t_\act})}  \ \RETURNE{\ttt};}
%\cmnt\ here $d > 1$\\
\lio{\IFC{\mbox{$v = \actp $ is interior} \land (\fa \B_i : \actp \ar \B_i \in \wfg{\dsk{\act}{\l}}{t_\act} : (\ex d' : 1 \le d' < d : \lviol{\B_i}{d'}{t_\act}{\act}{\l}))}}
    \>\>{\RETURNE{\ttt};}\\ 
\lio{\IFC{\mbox{$v = \B_i$} \land (\ex \actp : \B_i \ar \actp \in \wfg{\dsk{\act}{\l}}{t_\act} : (\ex d' : 1 \le d' < d :\lviol{\actp}{d'}{t_\act}{\act}{\l}))}  \ \RETURNE{\ttt};}
\lio{\RETURNE{\fff}}
\end{tabbing}
\vspace{-6ex}
\caption{Formal definition of $\lviol{v}{d}{t_\act}{\act}{\l}$.}
\label{fig:scViolateLoc}
\end{figure}





% \begin{figure}[ht]
% \setcounter{lctr}{0}
% \begin{tabbing}
% mm\= mm\= mm\= mm\= mm\=\kill
% $\lviol{v}{d}{t_\act}{\act}{\l}$\\
% \cmnt\ Precondition: $v$ is a node of $\dsk{\act}{\l}$ and $d \ge 1$\\

% \lio{\IFC{d = 1}}
%        \lit{\IFC{\mbox{$v$ is an interior interaction $\actp$ and }
%               \neg (\ex \B_i : \actp \ar \B_i \in \wfg{\dsk{\act}{\l}}{t_\act})}}
%                     \lihc{\RETURNE{\ttt}}{\cmnt no outgoing wait-for-edges}
%        \lit{\ELSE\ \RETURNE{\fff}}
%        \lit{\FI}
% \lio{\FI}

% \cmnt\ here $d > 1$\\

% \lio{\IFC{\mbox{$v$ is an interior interaction $\actp$ and } 
%                  (\fa \B_i : \actp \ar \B_i \in \wfg{\dsk{\act}{\l}}{t_\act} : \lviol{\B_i}{d-1}{t_\act}{\act}{\l})}}
%         \lit{\RETURNE{\ttt}}

% \lio{\ELSFC{\mbox{$v$ is a component $\B_i$ and }
%             (\ex \actp : \B_i \ar \actp \in \wfg{\dsk{\act}{\l}}{t_\act} : \lviol{\actp}{d-1}{t_\act}{\act}{\l})}}
%       \lit{\RETURNE{\ttt}}

% \lio{\ELSE\ \RETURNE{\fff}}
% \lio{\FI}
% \end{tabbing}
% \caption{Formal definition of $\lviol{v}{d}{t_\act}{\act}{\l}$.}
% \label{fig:scViolateLoc}
% %\label{alg:check-scViol}
% \end{figure}


We now show that a local supercycle-violation implies (global) supercycle-violation.
\bp
\label{prop:locScViol-implies-scViol}
 \label{prop:lviol-implies-viol}
Let $t$ be an arbitrary reachable state of BIP-system $(\B, Q_0)$.
For all interactions $\act \in \gamma$, and $\l \ge 1$, let $t_\act = t \pj \dsk{\act}{\l}$.
Then\\
\ind $\fa d \ge 1: \locScViol{v}{d}{t_\act}{\act}{\l} \imp \scViol{v}{d}{t}$.
\ep
\prf{
Proof is by induction on $d$. 

\noindent
\ul{Base case, $d=1$.} Assume $\lviol{v}{1}{t_\act}{\act}{\l}$ for some node $v$. Then, by 
Figure~\ref{fig:scViolateLoc}, 
$v$ is an interior node and an interaction $\actp$ of
$\dsk{\act}{\l}$, and has no outgoing 
wait-for edges. Therefore, in $\wfg{\B}{t}$, it is still the case that $v$ has no outgoing 
wait-for edges. Hence $\viol{v}{1}{t}$ holds.


\noindent
\ul{Inductive step, $d > 1$.}
Assume $\lviol{v}{d}{t_\act}{\act}{\l}$ for some node $v$ and some $d > 1$. 
We proceed by cases on Figure~\ref{fig:scViolateLoc}.

\bn

\item \ul{$v$ is an interior interaction $\actp$ and} \\
\ul{$(\fa \B_i : \actp \ar \B_i \in \wfg{\dsk{\act}{\l}}{t_\act} : (\ex d' : 1 \le d' < d : \lviol{\B_i}{d'}{t_\act}{\act}{\l}))$}.

Choose an arbitrary $\B_i$ such that $\actp \ar \B_i \in \wfg{\dsk{\act}{\l}}{t_\act}$.
By the induction hypothesis applied to $\lviol{\B_i}{d'}{t_\act}{\act}{\l}$, we have $\viol{\B_i}{d'}{t}$ for some $d' < d$.
Since $\wfg{\dsk{\act}{\l}}{t_\act} \sub \wfg{\B}{t}$ by construction, we have 
$\actp \ar B_i \in \wfg{\B}{t}$ and $\viol{\B_i}{d'}{t}$.
Hence by Definition~\ref{def:supercycle-violation}, Clause~\ref{def:supercycle.violation.component.out}, 
we have $\viol{v}{d}{t}$.


\item \ul{$v$ is a component $\B_i$ and}\\
\ul{$(\ex \actp : \B_i \ar \actp \in \wfg{\dsk{\act}{\l}}{t_\act} : (\ex d' : 1 \le d' < d :\lviol{\actp}{d'}{t_\act}{\act}{\l}))$}.

By the induction hypothesis applied to $\lviol{\actp}{d'}{t_\act}{\act}{\l}$, we have $\viol{\actp}{d'}{t}$ for some $d' < d$.
Since $\wfg{\dsk{\act}{\l}}{t_\act} \sub \wfg{\B}{t}$ by construction, we have 
$\B_i \ar \actp \in \wfg{\B}{t}$ and $\viol{\actp}{d'}{t}$.
Hence by Definition~\ref{def:supercycle-violation}, Clause~\ref{def:supercycle.violation.component.out}, 
we have $\viol{v}{d}{t}$.

\en
}










\subsubsection{Local strong connectedness condition}

We now present the local version of the strong connectedness violation condition, given above in Definition~\ref{def:sConn.violation}.


\bd[Local strong connectedness violation, $\locConnViol{v}{t_\act}{\act}{\l}$]
\label{def:sConn.violation.loc}

Let $L$ be the nodes of $\wfg{\dsk{a}{\l}}{t_\act}$ that have no local
supercycle violation, \ie $L = \set{v \stt v \in \dsk{a}{\l} \land \neg (\ex d \ge 1: \locScViol{v}{d}{t_\act}{\act}{\l}) }$.
Let $v$ be an arbitrary node in $L$. 
Let $WL = \wfg{\dsk{a}{\l}}{t_\act} \pj L$, \ie $WL$ is the subgraph of $\wfg{\dsk{a}{\l}}{t_\act}$ consisting of the
nodes in $L$, and the edges between those nodes that are also edges in $\wfg{\dsk{a}{\l}}{t_\act}$.

Then, $\locConnViol{v}{t_\act}{\act}{\l}$ holds iff:
\bn

\item \label{def:sConn.violation.loc:scc}
there does not exist a nontrivial strongly connected supercycle $SSC$ such that $v \in SSC$ and $SSC \sub WL$, and

%\item \label{def:sConn.violation.loc:wait-for-out} every wait-for path $\pi$ from $v$ to a border node
%  of $\dsk{a}{\l}$ contains at least one node with a local supercycle violation

\item \label{def:sConn.violation.border}
either
    \bn

    \item \label{def:sConn.violation.loc:wait-for-out} every wait-for path $\pi$ from $v$ to a border node
      of $\dsk{a}{\l}$ contains at least one node with a local supercycle violation

     or

    \item \label{def:sConn.violation.loc:wait-for-in} every wait-for path $\pi'$ from a border node
      of $\dsk{a}{\l}$ to $v$ contains at least one node with a local supercycle violation

    \en

\en
\ed


We show that the local strong connectedness condition implies the global strong connectedness condition.

\bp
\label{prop:locConnViol-implies-ConnViol}
 \label{prop:locConnViol-implies-connViol}
Let $t$ be an arbitrary reachable state of BIP-system $(\B, Q_0)$.
For all interactions $\act \in \gamma$, and $\l > 0$, let $t_\act = t \pj \dsk{\act}{\l}$.
Then\\
\ind $\locConnViol{v}{t_\act}{\act}{\l} \imp \connViol{v}{t}$.
\ep
%
\bpr
By contradiction. Assume there exists a node $v$ in $\dsk{a}{\l}$ such that $\locConnViol{v}{t_\act}{\act}{\l} \land \neg \connViol{v}{t}$.
By $\neg \connViol{v}{t}$ and Definition~\ref{def:sConn.violation}, there exists a strongly connected
supercycle $SSC$ such that $v \in SSC$ and $SSC \sub \wfg{B}{t}$. Then, there are two cases:
%
\bn
\item $SSC \sub \wfg{\dsk{a}{\l}}{t_\act}$: let $x$ be any node in $SSC$. Since $x$ is a node in a supercycle, we have by
  Proposition~\ref{prop:scViol-implies-notInSC}, that $\neg (\ex d \ge 1: \scViol{x}{d}{t})$. Hence 
  $(\fa d \ge 1: \neg \scViol{x}{d}{t})$. Hence by Proposition~\ref{prop:locScViol-implies-scViol}, 
  we have $(\fa d \ge 1: \neg \locScViol{x}{d}{t_\act}{\act}{\l})$. Let $L, WL$ be as given in Definition~\ref{def:sConn.violation.loc}.
  Then $x \in L$, and since $x$ is an arbitrary node of $SSC$, we have $SSC \sub WL$. 
  Thus Clause~\ref{def:sConn.violation.loc:scc} of Definition~\ref{def:sConn.violation.loc} is violated.

\item $SSC \not\sub \wfg{\dsk{a}{\l}}{t_\act}$: then there exists a node $x \in SSC -
  \dsk{a}{\l}$. Since $v \in SSC$, there must exist a wait-for path $\pi$
  from $v$ to $x$ and a wait-for path $\pi'$ from $x$ to
  $v$. Since $v \in \dsk{a}{\l}$ and $x \not\in \dsk{a}{\l}$, it
  follows that both $\pi$, $\pi'$  cross a border node of
  $\dsk{a}{\l}$. Furthermore, since $\pi$, $\pi'$ are part of $SSC$, every node
  along $\pi$, $\pi'$ is in a supercycle, and so cannot have a supercycle violation.
  By Proposition~\ref{prop:locScViol-implies-scViol}, the nodes on
  $\pi$, $\pi'$  cannot have a local supercycle violation.
  Hence Clauses~\ref{def:sConn.violation.loc:wait-for-out} and
  \ref{def:sConn.violation.loc:wait-for-in} of Definition~\ref{def:sConn.violation.loc} are violated,
  since they require that at least one node along $\pi$, $\pi'$ respectively, have a local supercycle violation.
  
\en
In both cases,  Definition~\ref{def:sConn.violation.loc} is violated. 
But  Definition~\ref{def:sConn.violation.loc} must hold, since we have $\locConnViol{v}{t_\act}{\act}{\l}$. 
Hence the desired contradiction.
\epr


% Let $v$ be a node that is in a supercycle of $\wfg{\dsk{a}{\l}}{t_\act}$, and $t_\act$ a state of $\dsk{\act}{\l}$.
% Let $W$ be the result of removing from $\wfg{\dsk{a}{\l}}{t}$ every node $u$ such that 
% $(\ex d \ge 1: \locScViol{u}{d}{t_\act}{\act}{\l})$. Let $V$ be the maximal strongly connected component of $W$ that
% contains $v$. Then $\locConnViol{v}{t_\act}{\act}{\l}$ holds iff $V$ (by itself) is not a supercycle.
% For technical convenience, we also define $\locConnViol{v}{t_\act}{\act}{\l}$ to be false when $(\ex d \ge 1:
% \locScViol{v}{d}{t_\act}{\act}{\l})$, \ie when $v$ is not in a supercycle.
% Hence $\locConnViol{v}{t_\act}{\act}{\l}$ is always well-defined.




\subsubsection{Local formation violation condition}


We showed above that local supercycle violation implies global supercycle violation, and local strong connecteness violation implies global string conectedness
violation.
The global supercycle formation condition is the disjunction of global supercycle violation and global strong connecteness violation.
Hence we formulate the local supercycle formation condition as the disjunction of local supercycle violation and local strong connecteness violation.
It follows that the local supercycle formation condition implies the global supercycle formation condition.



\bd[Local Formation violation, $\locFormViol{v}{t_\act}{\act}{\l}$]
\label{def:locFormation.violation}
Let $v$ be a node of $\dsk{\act}{\l}$.
Then $\locFormViol{v}{t_\act}{\act}{\l}  \df  (\exs d \ge 1: \locScViol{v}{d}{t_\act}{\act}{\l}) \lor \locConnViol{v}{t_\act}{\act}{\l}$.
\ed
%Let $s \goesto[\act] t$ be a reachable transition. If, for every $\B_i \in \cmps{\act}$, 
%$\formViol{v}{t}$ holds, then $s \goesto[\act] t$ does not introduce a supercycle, \ie if $s$ is
%supercycle-free, then so is $t$. We establish this in the sequel.
%


\bp \label{prop:locFromViol-implies-formViol}
\label{prop:locformviol-implies-formviol}
Let $t$ be an arbitrary reachable state of BIP-system $(\B, Q_0)$.
For all interactions $\act \in \gamma$, and $\l > 0$, let $t_\act = t \pj \dsk{\act}{\l}$.
Then\\
%\ind $\fa d \ge 1: $\locFormViol{v}{t_\act}{\act}{\l}\locFormViol{v}{d}{t_\act}{\act}{\l} \imp \formViol{v}{d}{t}$.
\ind $ \locFormViol{v}{t_\act}{\act}{\l} \imp \formViol{v}{t}$.
\ep
%
\bpr
Assume that $\locFormViol{v}{t_\act}{\act}{\l}$ holds. Then, by Definition~\ref{def:formation.violation}, 
$(\exs d \ge 1: \locScViol{v}{d}{t_\act}{\act}{\l}) \lor \locConnViol{v}{t_\act}{\act}{\l}$.
We proceed by cases:
\bn
\item $(\exs d \ge 1: \locScViol{v}{d}{t_\act}{\act}{\l})$: hence $(\exs d \ge 1: \scViol{v}{d}{t})$ by Proposition~\ref{prop:locScViol-implies-scViol}.
\item $\locConnViol{v}{t_\act}{\act}{\l}$: hence $\connViol{v}{t}$ by Proposition~\ref{prop:locConnViol-implies-connViol}.
\en
By Definition~\ref{def:formation.violation},  $\formViol{v}{t}  \df (\exs d \ge 1: \scViol{u}{d}{t}) \lor \connViol{v}{t}$.
Hence we conclude that $\formViol{v}{t}$ holds.
\epr




\subsubsection{Local AND-OR Condition}

The actual local condition, \LAO, is given by applying the local supercycle formation condition to every reachable transition 
of the subsystem $\dsk{\act}{\l}$ being considered, and to every component $B_i \in \cmps{\act}$.

\bd[$\LAO(\B, Q_0, \act, \l)$] \label{def:lao}
Let $\l > 0$, and let $s_\act \goesto[\act] t_\act$ be an arbitrary reachable transition of $\dsk{\act}{\l}$.
Then, in $t_\act$, the following holds. 
For every component $\B_i$ of $\cmps{\act}$:  
$\B_i$ has a supercycle formation violation that can be confirmed within $\dsk{\act}{\l}$.
Formally,\\
\ind  $\fa \B_i \in \cmps{\act} : \locFormViol{\B_i}{t_\act}{\act}{\l}$.
\ed
%
We showed previously that $\GAO$ implies deadlock-freedom, and so it remains to establish that $\LAO$ implies $\GAO$. 



\bl \label{lemma:loc.ANDOR.implies.glob.AND-OR}
Let $\act \in \gamma$ be an interaction of BIP-system $(\B, Q_0)$. Then\\
\ind $(\ex \l > 0: \LAO(\B, Q_0, \act, \l))$ implies $\GAO(\B, Q_0, \act)$
\el
%
%\prf{Immediate from Proposition~\ref{prop:locformviol-implies-formviol} and Definitions~\ref{def:global.ANDOR-cond}, \ref{def:lao}.}

\bpr
Assume $\LAO(\B, Q_0, \act, \l)$ for some $\l > 0$. 
%
Let $s \goesto[\act] t$ be an arbitrary reachable transition of BIP-system $(\B, Q_0)$, and let 
$s_\act \goesto[\act] t_\act$ be the projection of $s \goesto[\act] t$ onto $\dsk{\act}{\l}$.
By Corollary~\ref{cor:bip.reachability}, $s_\act \goesto[\act] t_\act$ is a reachable transition of $\dsk{\act}{\l}$.

\noindent
By Definition~\ref{def:lao}, we have for some $\l > 0$:\\
\ind for every reachable transition $s_\act \goesto[\act] t_\act$ of $\dsk{\act}{\l}$:\\
\ind \ind $\fa \B_i \in \cmps{\act} : \locFormViol{\B_i}{t_\act}{\act}{\l}$.
%\ind \ind  $\fa \B_i \in \cmps{\act}, \exs d \ge 1: \lviol{\B_i}{d}{t_\act}{\act}{\l}$. 

\noindent
From this and Proposition~\ref{prop:locFromViol-implies-formViol},\\
\ind for every reachable transition $s \goesto[\act] t$ of  $(\B, Q_0)$:\\ 
\ind \ind $\fa \B_i \in \cmps{\act} : \formViol{B_i}{t}$

\noindent
Hence, by Definition~\ref{def:global.ANDOR-cond}, $\GAO(\B, Q_0, \act)$ holds.
\epr



\bt \label{thm:LAO.SC-free-preserving}
$\LAO$ is supercycle-freedom preserving
\et
\prf{
Follows immediately from Lemma~\ref{lemma:loc.ANDOR.implies.glob.AND-OR} and Theorem~\ref{thm:GAO.SC-free-preserving}.
}



%%%%%%%%%%%%%%%%%%%%%%%%%%%%%%%%%%%%%%%%%%%%%%%%%%%%%%%%%%%%%%%%%%%%
\begin{figure}[ht]

\begin{tabular}{|l|l|}
\hline
$\scViol{v}{d}{t}$  & $v$ confirmed at depth $d$ to not be in supercycle\\ 
              %supercycle violation condition
$\locScViol{v}{d}{t_\act}{\act}{\l}$ & $v$ locally determined to not be in a supercycle\\
              % local supercycle violation condition:

$\connViol{v}{t}$ & $v$ not in a strongly connected supercycle \\
              %strongly connected supercycle violation: 

$\locConnViol{v}{t_\act}{\act}{\l}$ & $v$ locally determined to not be in a strongly connected supercycle \\
               %local strongly connected supercycle violation 

$\formViol{v}{t}$ & $v$ does not contribute to a supercycle\\
               %supercycle formation violation: 

$\locFormViol{v}{t_\act}{\act}{\l}$ & $v$ locally determined to not contribute to a supercycle\\
                %local supercycle formation violation condition

\hline
\end{tabular}

\caption{Summary of predicates}
\label{fig:summaryPredicates}
\end{figure}






We now seek a local condition, which we evaluate in $\dsk{\act}{\l}$, and which implies \GAO.
We define local versions of both $\scViol{v}{d}{s}$ and $\connViol{v}{s}$.

To achieve a local and conservative approximation of $\scViol{v}{d}{s}$, we make the ``pessimistic'' assumption that the violation status of border
nodes of $\dsk{\act}{\l}$ cannot be known, since it depends on nodes outside of $\dsk{\act}{\l}$.  Now, if an internal node $v$ of $\dsk{\act}{\l}$
can be marked with a level-$d$ \emph{local} supercycle-violation, by applying \defn{supercycle.violation.local} to $\dsk{\act}{\l}$, and with the
border nodes marked as non-violating, then it is also the case, as we show below, that $v$ also has a level-$d$ \emph{global} supercycle-violation, as per \defn{supercycle.violation}.

To achieve a local and conservative approximation of
$\connViol{v}{s}$, we project onto the subsystem \DS.


\subsubsection{Local strong connectedness condition}

We now present the local version of the strong connectedness violation condition, given above in \defn{sConn.violation}.

\begin{definition}[Local strong connectedness violation, $\locConnViol{v}{\sD}{\act}{\l}$]
\label{def:sConn.violation.loc}
\label{defn:sConn.violation.loc}

Let $L$ be the nodes of $\wfg{\dsk{a}{\l}}{\sD}$ that have no local supercycle violation, \ie
 $L = \set{ \ndv \stt \ndv \in \dsk{a}{\l} \land \neg \scVL{\ndv}{\sD}{\act}{\l} }$.
%
Let $\WL = \wfg{\dsk{a}{\l}}{\sD} \pj L$, \ie $\WL$ is the subgraph of $\wfg{\dsk{a}{\l}}{\sD}$ consisting of the
nodes with no local supercycle violation, and the edges between those nodes that are also edges in $\wfg{\dsk{a}{\l}}{\sD}$.

Let $\ndv$ be an arbitrary node in $\WL$.  Then, $\locConnViol{\ndv}{\sD}{\act}{\l}$ holds iff:
\bn

\item \label{def:sConn.violation.loc:scc}
there does not exist a nontrivial strongly connected supercycle $\SSC$ such that $v \in \SSC$ and $\SSC \subg \WL$, and

%\item \label{def:sConn.violation.loc:wait-for-out} every wait-for path $\pi$ from $v$ to a border node
%  of $\dsk{a}{\l}$ contains at least one node with a local supercycle violation

\item \label{def:sConn.violation.border}
either
    \bn

    \item \label{def:sConn.violation.loc:wait-for-out} there is no path in $\WL$ from $\ndv$ to a border node of $\dsk{a}{\l}$

     or

    \item \label{def:sConn.violation.loc:wait-for-in} there is no path in $\WL$ from  a border node of $\dsk{a}{\l}$ to $\ndv$.

    \en

\en
\end{definition}
%
Note that Clause~\ref{def:sConn.violation.loc:wait-for-out} means that 
every wait-for path $\pi$ in $\wfg{\dsk{a}{\l}}{\sD}$  from $\ndv$ to a border node
of $\dsk{a}{\l}$ contains at least one node $\ndw$ with a local supercycle violation, \ie $\scVL{\ndw}{\sD}{\act}{\l}$.
Also 
Clause~\ref{def:sConn.violation.loc:wait-for-in} means that every wait-for path $\pi'$ in $\wfg{\dsk{a}{\l}}{\sD}$ from a border node
of $\dsk{a}{\l}$ to $\ndv$ contains at least one node $\ndw$ with a local supercycle violation,  \ie $\scVL{\ndw}{\sD}{\act}{\l}$.


We show that the local strong connectedness condition implies the global strong connectedness condition.

\begin{proposition}
\label{prop:locConnViol-implies-ConnViol}
 \label{prop:locConnViol-implies-connViol}
Let $s$ be an arbitrary state of $\B$.
For all interactions $\act \in \gamma$, and $\l \ge 1$, let $\DS = \dsks{\act}{\l}$, $\sD = s \pj \dsk{\act}{\l}$, and let $\ndv$ be an arbitrary node in $\dsk{\act}{\l}$.
Then\\
\ind $\locConnViol{v}{\sD}{\act}{\l}\ \impliess\ \connViol{v}{s}$.
\end{proposition}
%
\begin{proof}
By contradiction. Assume for some state $s$ of $\B$ and some node $v$ in $\dsk{a}{\l}$ that $\locConnViol{v}{\sD}{\act}{\l} \land \neg
\connViol{v}{s}$ holds.
By $\neg \connViol{v}{s}$ and \defn{sConn.violation}, there exists a strongly connected
supercycle $\SSC$ such that $v \in \SSC$ and $\SSC \subg \wfg{\B}{s}$. Then, there are two cases:
%
\bn
\item $\SSC \subg \wfg{\dsk{a}{\l}}{\sD}$: let $\ndx$ be any node in $\SSC$. Since $\ndx$ is a node in a supercycle, we have by
  \defn{supercycle.membership} and \prop{scViol-iff-notInSC}, that $\neg \scV{\ndx}{s}$. 
   Hence, by \prop{locScViol-implies-scViol},  we have $\neg \scVL{\ndx}{\sD}{\act}{\l}$. 
   Let $\WL$ be as given in \defn{sConn.violation.loc}.
  Then $\ndx \in \WL$, and since $\ndx$ is an arbitrary node of $\SSC$, we have $\SSC \subg \WL$. 
  Thus Clause~\ref{def:sConn.violation.loc:scc} of Definition~\ref{def:sConn.violation.loc} is violated.

\item $\SSC \not\subg \wfg{\dsk{a}{\l}}{\sD}$: then there exists a node $x \in \SSC - \wfg{\dsk{a}{\l}}{\sD}$.
  Now $v \in \SSC$ and $\SSC$ is strongly connected. Hence there must exist a wait-for path $\pi$ in $\lwfg{\B}{\sD}{\DS}$
  from $v$ to $x$ and a wait-for path $\pi'$ in $\lwfg{\B}{\sD}{\DS}$ from $x$ to
  $v$. Since $v \in \dsk{a}{\l}$ and $x \not\in \dsk{a}{\l}$, it
  follows that both $\pi$ and $\pi'$  cross a border node of
  $\dsk{a}{\l}$. Furthermore, since $\pi$, $\pi'$ are paths in $\SSC$, every node $\ndw$ that is
  in $\pi$ or in $\pi'$ must be in a supercycle, and so cannot have a supercycle violation, \ie $\neg \scV{\ndw}{s}$. 
  By Proposition~\ref{prop:locScViol-implies-scViol}, every node $\ndw$ that is
  in $\pi$ or in $\pi'$ cannot have a local supercycle violation, \ie $\neg \scVL{\ndw}{\sD}{\act}{\l}$.
  Hence, Clauses~\ref{def:sConn.violation.loc:wait-for-out} and
  \ref{def:sConn.violation.loc:wait-for-in} of Definition~\ref{def:sConn.violation.loc} are violated,
  since they require that at least one node in $\pi$ and at least one node in $\pi'$ has a local supercycle violation.
  
\en
In both cases,  Definition~\ref{def:sConn.violation.loc} is violated. 
But  Definition~\ref{def:sConn.violation.loc} must hold, since we have $\locConnViol{v}{\sD}{\act}{\l}$. 
Hence, the desired contradiction.
\end{proof}



\subsubsection{General local violation condition}

We showed above that local supercycle violation implies global supercycle violation, and local
strong connectedness violation implies global strong connectedness violation.  The general global
supercycle violation condition is the disjunction of global supercycle violation and global strong
connectedness violation.  Hence, we formulate the general local supercycle violation condition as the
disjunction of local supercycle violation and local strong connectedness violation.  It follows that
the general local supercycle violation condition implies the general global supercycle violation condition.


\begin{definition}[General local supercycle violation, $\locFormViol{v}{\sD}{\act}{\l}$]
\label{def:locFormation.violation}
\label{defn:locFormation.violation}
Let $\ndv$ be an arbitrary node of $\dsk{\act}{\l}$ and $\sD$ be an arbitrary  state of $\dsk{\act}{\l}$.
Then $\locFormViol{v}{\sD}{\act}{\l}  \df \scVL{v}{\sD}{\act}{\l} \lor \locConnViol{v}{\sD}{\act}{\l}$.
\end{definition}


\begin{proposition}[Local violation implies global violation] 
\label{prop:locFromViol-implies-formViol}
\label{prop:locformviol-implies-formviol}
Let $s$ be an arbitrary state of BIP composite component $\B$.
For all interactions $\act \in \gamma$, and $\l \ge 1$, let $\DS = \dsks{\act}{\l}$ and $\sD = s \pj \dsk{\act}{\l}$.
Also let $\ndv$ be an arbitrary node of $\dsk{\act}{\l}$.
Then\\
\ind $ \locFormViol{v}{\sD}{\act}{\l}\ \impliess\ \genViol{v}{s}$.
\end{proposition}
%
\begin{proof}
Assume that $\locFormViol{v}{\sD}{\act}{\l}$ holds. Then, by \defn{locFormation.violation},
$\scVL{v}{\sD}{\act}{\l} \lor \locConnViol{v}{\sD}{\act}{\l}$.
We proceed by cases:
\bn
\item $\scVL{v}{\sD}{\act}{\l}$: hence $\scV{v}{s}$ by \prop{locScViol-implies-scViol}.
\item $\locConnViol{v}{\sD}{\act}{\l}$: hence $\connViol{v}{s}$ by \prop{locConnViol-implies-connViol}.
\en
By \defn{formation.violation},  $\genViol{v}{s}  \df \scV{v}{s}  \lor \connViol{v}{s}$.
Hence we conclude that $\genViol{v}{s}$ holds.
\end{proof}




\subsubsection{Local AND-OR Condition}

The actual local condition, \LAO, is given by applying the general local supercycle violation condition to every reachable transition 
of the subsystem $\dsk{\act}{\l}$ being considered, and to every component $B_i \in \cmps{\act}$.

\begin{definition}[$\LAO(\B, Q_0, \act, \l)$] \label{def:lao} \label{defn:lao}
Let $\l \ge 1$, $\DS = \dsks{\act}{\l}$, $\QDS = Q_0 \pj \DS$.
Let $\tD \goesto[\act] \sD$ be an arbitrary reachable transition of the subsystem $(\DS, \QDS)$. 
Then, in $\sD$, the following holds. 
For every $\B_i \in \cmps{\act}$:  
$\B_i$ has a general local supercycle violation that can be confirmed within $\dsk{\act}{\l}$.
Formally,\\
\ind  $\fa \B_i \in \cmps{\act} : \locFormViol{\B_i}{\sD}{\act}{\l}$.
\end{definition}
%

%
We showed previously that $\GAO$ implies deadlock-freedom, and so it remains to establish that $\LAO$ implies $\GAO$. 


\begin{lemma}
\label{lem:loc.ANDOR.implies.glob.AND-OR}
\label{lemma:loc.ANDOR.implies.glob.AND-OR}
\label{LAOGAO}
Let $\act \in \gamma$ be an interaction of BIP-system $(\B, Q_0)$. Then\\
\ind $(\ex \l \ge 1: \LAO(\B, Q_0, \act, \l))\ \impliess\ \GAO(\B, Q_0, \act)$. 
\end{lemma}
%
\begin{proof}
Assume $\LAO(\B, Q_0, \act, \l)$ for some $\l \ge 1$, and let $\DS = \dsks{\act}{\l}$, $\QDS = Q_0 \pj \DS$.
%
Let $t \goesto[\act] s$ be an arbitrary reachable transition of BIP-system $(\B, Q_0)$, and let 
$\tD = t \pj \DS$, $\sD = s \pj \DS$, so that 
$\tD \goesto[\act] \sD$ is the projection of $t \goesto[\act] s$ onto $\dsk{\act}{\l}$.
By \cor{bip.reachability}, $\tD \goesto[\act] \sD$ is a reachable transition of $(\DS, \QDS)$.

\noindent
By \defn{lao},\\
\ind for every reachable transition $\tD \goesto[\act] \sD$ of $(\DS, \QDS)$:\\
\ind \ind $\fa \B_i \in \cmps{\act} : \locFormViol{\B_i}{\sD}{\act}{\l}$.

\noindent
From this and Proposition~\ref{prop:locFromViol-implies-formViol},\\
\ind for every reachable transition $t \goesto[\act] s$ of  $(\B, Q_0)$:\\ 
\ind \ind $\fa \B_i \in \cmps{\act} : \formViol{\B_i}{s}$

\noindent
Hence, by \defn{global.ANDOR-cond}, $\GAO(\B, Q_0, \act)$ holds.
\end{proof}



\begin{theorem} \label{thm:LAO.SC-free-preserving}
$\LAO$ is supercycle-freedom preserving.
\end{theorem}
%
\begin{proof}
Follows immediately from \thm{GAO.SC-free-preserving} and \lem{loc.ANDOR.implies.glob.AND-OR}.
\end{proof}
%
Notice that \defn{lao} calls $\locFormViol{v}{\sD}{\act}{\l}$ on components, which by definition should be connected to at least one non-border interaction. 
As such, the trivial local supercycles, \ie consisting only of border interactions, have no effect on supercycle formation. 

%%%%%%%%%%%%%%%%%%%%%%%%%%%%%%%%%%%%%%%%%%%%%%%%%%%%%%%%%%%%%%%%%%%%
\begin{figure}[t]

\begin{tabular}{|l|l|}
\hline
$\scViol{v}{d}{s}$  & $v$ determined at depth $d$ to not be in supercycle\\ 
              %supercycle violation condition
$\locScViol{v}{d}{\sD}{\act}{\l}$ & $v$ locally determined at depth $d$ to not be in a supercycle\\
              % local supercycle violation condition:

$\connViol{v}{s}$ & $v$ not in a strongly connected supercycle \\
              %strongly connected supercycle violation: 

$\locConnViol{v}{\sD}{\act}{\l}$ & $v$ locally determined to not be in a strongly connected supercycle \\
               %local strongly connected supercycle violation 

$\formViol{v}{s}$ & $v$ does not contribute to a supercycle\\
               %supercycle formation violation: 

$\locFormViol{v}{\sD}{\act}{\l}$ & $v$ locally determined to not contribute to a supercycle\\
                %local supercycle formation violation condition

\hline
\end{tabular}

\caption{Summary of predicates.}
\label{fig:summaryPredicates}
\end{figure}











   \subsection{A local linear condition for deadlock-freedom}
   \label{s:condition}
%   We now formulate a local version of $\GLin$.
Observe that if
$\widepth{\B}{\B_i}{t} < \omega \lor \wodepth{\B}{\B_i}{t} < \omega$,
then there is some finite $\l$ such that 
$\widepth{\B}{\B_i}{t} = \l \lor \wodepth{\B}{\B_i}{t} = \l$.


% OMIT THIS AND LEAVE TO ANOTHER PAPER
% %containing $\B_i$ and all nodes(components and interactions) have distance from $\B_i$ of up to $\l+1$.
% Then $\widepth{\B}{\B_i}{t} = \l \lor \wodepth{\B}{\B_i}{t} = \l$ can be verified in 
% the wait-for-graph of 
% $\ssg{i}{\l+1}$, since we verify either that all wait-for-paths ending in $\B_i$
% have length $\le \l$, or that 
% all wait-for-paths starting in $\B_i$ 
% have length $\le \l$.
% These conditions can be checked in $\ssg{i}{\l+1}$, since $\ssg{i}{\l+1}$
% contains every node in a
% wait-for-path of length $\l+1$ or less and which starts or ends in $\B_i$.
% %
% Since $\ssg{i}{\l+1} \sub \ssg{\act}{\l+2}$ for $\B_i \in \cmps{\act}$, we use 
% $\ssg{\act}{\l+2}$ instead of the set of subsystems 
% $\set{\ssg{i}{\l+1} : \B_i \in \cmps{\act}}$. 
% %% DISCUSS tradeoff: one larger $\ssg{a}{\l+2}$ instead of several smaller 
% %$\ssg{i}{\l+1}?
% %We leave analysis of the tradeoff between using one larger system
% %($\ssg{a}{\l+2}$) versus several smaller ones ($\ssg{i}{\l+1}$) to another paper.




\bd[$\LLin(\B, Q_0, \act, \l)$] \label{def:ldfc-k}
%Let $(B, Q_0)$ be a BIP system, with $B =\gamma(\B_1,\dots,\B_n)$.
%Let $a$ be an interaction of $(B, Q_0)$, \ie $a \in \gamma$.
Let $\l \ge 2$ and $s_\act \goesto[\act] t_\act$ be a reachable transition of $\dsk{\act}{\l}$.
Then, in $t_\act$, the following holds. 
For every component $\B_i$ of $\cmps{\act}$:  
either $\B_i$ has in-depth at most $\l$, or out-depth at most $\l$, in $\wfg{\dsk{\act}{\l}}{t_\act}$. 
Formally,\\
\ind  $\fa \B_i \in \cmps{\act}: 
\widepth{\dsk{\act}{\l}}{\B_i}{t_\act} \le \l-2 \lor \wodepth{\dsk{\act}{\l}}{\B_i}{t_\act} \le \l-2$.
\ed


To infer deadlock-freedom in $(\B, Q_0)$ by checking $\LLin(\B, Q_0, a, \l)$,
% which is a condition on the behavior of the subsystem $\ds{a}$,
 we show that wait-for behavior in $\B$ ``projects down''
to any subcomponent $\B'$, and that wait-for behavior in $\B'$ ``projects up'' to $\B$.

% Let $e$ be a wait-for-edge with both endpoints in $B'$. Then if $e$ is
% present in the wait-for-graph of $B$ for state $s$, then $e$ is also present in the wait-for-graph
% $B'$ for $s \pj B'$ ($e$ projects down), and vice-versa ($e$ projects up).

%Thus $\wfg{B'}{s \pj B'}$ is a ``projection'' of $\wfg{\B}{s}$ onto $B'$.


\vspace{0.5ex}

Since wait-for-edges project up and down, it follows that wait-for-paths
project up and down, provided that the subsystem contains the entire wait-for-path.

\bp[In-projection, Out-projection] \label{prop:in-out-projection}
%Let $(B, Q_0)$ be a BIP system, 
Let $\l \ge 1$, let $\B_i$ be an atomic component of $\B$, and let 
$(\B', Q'_0)$ be a subsystem of $(\B, Q_0)$ which is based on a superset of $\ssg{i}{\l}$.
Let $s$ be a state of $(B, Q_0)$, and $s' = s \pj B'$. Then
(1) $\widepth{\B}{\B_i}{s} \le \l-1$ iff $\widepth{\B'}{\B_i}{s'} \le \l-1$, and
(2) $\wodepth{\B}{\B_i}{s} \le \l-1$ iff $\wodepth{\B'}{\B_i}{s'} \le \l-1$.
\ep
\prfs{Follows from Defintion~\ref{def:depth},
Proposition~\ref{prop:edge-projection}, and the observation that $\wfg{\B'}{s'}$
contains all wait-for-paths of length $\le \l$ that start or end in $\B_i$.}
%
\prf{
We establish clause (1). The proof of clause (2) is analogous, except we replace paths ending in
$\B_i$ by paths starting from $\B_i$.
The proof of clause (1) is by double implication.


$\widepth{\B}{\B_i}{s} \le \l-1$ implies $\widepth{\B'}{\B_i}{s'} \le \l-1$:
Let $\pi$ be an arbitrary wait-for-path in $\wfg{\B'}{s'}$ ending in $\B_i$. 
Since $(\B', Q'_0)$ is a subsystem of $(\B, Q_0)$, 
by Definition~\ref{def:static:wait-for-graph} and $s' = s \pj B'$, 
$\wfg{\B'}{s'}$ is a subgraph of $\wfg{\B}{s}$.
Hence $\pi$ is a wait-for-path in $\wfg{\B}{s}$.
By $\widepth{\B}{\B_i}{s} \le \l-1$, we have $|\pi| \le \l-1$. 
Hence $\widepth{\B'}{\B_i}{s'} \le \l-1$ since $\pi$ was arbitrarily chosen.


$\widepth{\B}{\B_i}{s} \le \l-1$ follows from $\widepth{\B'}{\B_i}{s'} \le \l-1$:
Let $\pi$ be an arbitrary wait-for-path ending in $\B_i$. 
Suppose that $|\pi| > \l-1$.
Since $(\B', Q'_0)$ is a subsystem of $(\B, Q_0)$, 
by Definition~\ref{def:static:wait-for-graph} and $s' = s \pj \B'$, 
$\wfg{\B'}{s'}$ is a subgraph of $\wfg{\B}{s}$. Hence every edge $e$ in $\pi$ that is within distance
$\l$ from $\B_i$ is also en edge in $\wfg{\B'}{s'}$, since 
$(\B', Q'_0)$ is based on a superset of $\ssg{i}{\l}$.
Hence 
$\widepth{\B}{\B_i}{s} > \l-1$ implies $\widepth{\B'}{\B_i}{s'} > \l-1$.
The contrapositive is our desired result.
}



\vspace{0.5ex}

\noindent
We now show that $\LLin(\B, Q_0, \act, \l)$ implies $\GLin(\B, Q_0, \act)$, which in turn implies deadlock-freedom.  

\bl
\label{lemma:loc-implies-glob}
\label{lemma:locLinear-implies-globlinear}
%Let $(B, Q_0)$ be a BIP system, with $B = \gamma(\B_1,\dots,\B_n)$, and 
Let $\act$ be an interaction of $\B$, \ie $\act \in \gamma$.
If $\LLin(\B, Q_0, \act, \l)$ holds for some finite $\l \ge 2$, then $\GLin(\B, Q_0, \act)$ holds.
\el
%
%% \prfs{
%% Let $s \la{\act} t$ be a reachable transition of $(B, Q_0)$ and let 
%% $s_\act = s \pj \dsk{\act}{\l}$, $t_\act = t \pj \dsk{\act}{\l}$.
%% Then $s_\act \la{\act} t_\act$ is a reachable transition of $\dsk{\act}{\l}$ by 
%% Corollary~\ref{cor.bip.reachability}.
%% By $\LLin(\B, Q_0, \act, \l)$,
%% $\widepth{\dsk{\act}{\l}}{\B_i}{t_\act} \le \l \lor \wodepth{\dsk{\act}{\l}}{\B_i}{t_\act} \le \l$.
%% Hence by Proposition~\ref{prop:in-out-projection},
%% $\widepth{\B}{\B_i}{t} \le \l \lor \wodepth{\B}{\B_i}{t} \le \l$.
%% So
%% $\widepth{\B}{\B_i}{t} < \omega \lor \wodepth{\B}{\B_i}{t} < \omega$.
%% Hence $\GLin(\B, Q_0, \act)$ holds.}
%
\prf{
Let $s \la{\act} t$ be a reachable transition of $(\B, Q_0)$ and let 
$\B_i \in \cmps{\act}$, 
$s_\act = s \pj \dsk{\act}{\l}$, $t_\act = t \pj \dsk{\act}{\l}$.
Then $s_\act \la{\act} t_\act$ is a reachable transition of $\dsk{\act}{\l}$ by 
Corollary~\ref{cor.bip.reachability}.
By $\LLin(\B, Q_0, \act, \l)$, 
$\widepth{\dsk{\act}{\l}}{\B_i}{t_\act} \le \l-2 \lor \wodepth{\dsk{\act}{\l}}{\B_i}{t_\act} \le \l-2$.
Hence by Proposition~\ref{prop:in-out-projection},
$\widepth{\B}{\B_i}{t} \le \l-2 \lor \wodepth{\B}{\B_i}{t} \le \l-2$.
So
$\widepth{\B}{\B_i}{t} < \omega \lor \wodepth{\B}{\B_i}{t} < \omega$.
Hence $\GLin(\B, Q_0, \act)$.
}


\bt \label{thm:LLin.SC-free-preserving}
$\LLin$ is supercycle-freedom preserving
\et
\prf{
Follows immediately from Lemma~\ref{lemma:locLinear-implies-globlinear} and Theorem~\ref{thm:GLin.SC-free-preserving}.
}




%% \bt[Deadlock-freedom via \LLin] 
%% \label{theorem:local:deadlock-free}
%% \label{theorem:local.linear.deadlock-free}
%% %Let $(B, Q_0)$ be a BIP system, with $B = \gamma(\B_1,\dots,\B_n)$.
%% Assume that
%% \bn
%% \item \label{theorem:local.linear.deadlock-free.initial} 
%%       for all $s_0 \in Q_0$, $\wfg{\B}{s_0}$ is supercycle-free, and
%% \item \label{theorem:local.linear.deadlock-free.localLinear}
%%       for all interactions $a$ of $B$ ($a \in \gamma$), $\LDFC(a,\l)$ holds for some $\l \ge 0$.
%% \en 
%% Then for every reachable state $u$ of $(B, Q_0)$:  $\wfg{\B}{u}$ is supercycle-free.
%% \et
%% %
%% \prfs{
%% Immediate from Lemma~\ref{lemma:loc-implies-glob} and
%% Theorem~\ref{theorem:global:deadlock-free}. }
%% %
%% \prf{
%% Let $a$ be an arbitrary interaction in $\gamma$.
%% By $(\ex \l: \LLin(a,\l))$ and Lemma~\ref{lemma:locLinear-implies-globlinear}, we have $\GLin(a)$.
%% By Theorem~\ref{theorem:global.linear.deadlock-free},
%% for every reachable state $u$ of $(\B, Q_0)$:  $\wfg{\B}{u}$ is supercycle-free.
%% }




%% \paragraph{Complexity of evaluating $\LDFC(a, \l)$.}
%% Using explicit state enumeration, $\LDFC(a, \l)$ can be evaluated in
%% time $O(\SUM_{a \in \gamma} |\dsk{a}{\l}|)$, where $|\dsk{a}{\l}|$ denotes the
%% size of the transition system of $\dsk{a}{\l}$.








%%%%%%%%%%%%%%%%%%%%%%%%%%%%%%%%%%%%%%%%%%%%%%%%%%%%%%%%%
\subsection{Deadlock freedom using local abd global restrictions}


\bt[Deadlock-freedom via \LAO, \LLin] 
\label{theorem:local.deadlock-free}
Assume that
\bn
\item \label{theorem:local.deadlock-free.initial}
      for all $s_0 \in Q_0$, $\wfg{\B}{s_0}$ is supercycle-free, and
\item \label{theorem:local.deadlock-free.scfPres}
      for all interactions $\act$ of $\B$ (\ie $\act \in \gamma$), one of
      the following holds:
      \bn
      \item $\GAO(\B,Q_0,\act)$
      \item $\GLin(\B,Q_0,\act)$
      \item $\ex \l > 0: \LAO(\B,Q_0,\act,\l)$ 
      \item $\ex \l > 0: \LLin(\B,Q_0,\act,\l)$ 
      \en
\en
Then for every reachable state $u$ of $(\B, Q_0)$:  $\wfg{\B}{u}$ is supercycle-free, and so 
$(\B, Q_0)$ is free of local deadlock.
\et
\prf{Immediate from
Theorems~\ref{thm:GAO.SC-free-preserving}, \ref{thm:GLin.SC-free-preserving}, \ref{thm:LAO.SC-free-preserving}, \ref{thm:LLin.SC-free-preserving}
and Corollary~\ref{cor:SC-free-preserving.deadlock-free}.}



We now formulate a local version of $\GLin$. Observe that if
$\widepth{\B}{\B_i}{s} < \omega \lor \wodepth{\B}{\B_i}{s} < \omega$,
then there is some finite $\l$ such that 
$\widepth{\B}{\B_i}{s} = \l \lor \wodepth{\B}{\B_i}{s} = \l$.


\begin{definition}[$\LLin(\B, Q_0, \act, \l)$] \label{def:ldfc-k}
\label{def:locLinear} \label{defn:LLin}
%Let $(B, Q_0)$ be a BIP system, with $B =\gamma(\B_1,\dots,\B_n)$.
%Let $a$ be an interaction of $(B, Q_0)$, \ie $a \in \gamma$.
Let $\l \ge 1$, $\DS = \dsks{\act}{\l}$, $\QDS = Q_0 \pj \DS$.
Let $\tD \goesto[\act] \sD$ be an arbitrary reachable transition of the subsystem $(\DS, \QDS)$. 
Then, in $\sD$, the following holds. 
For every $\B_i \in \cmps{\act}$:  
either $\B_i$ has in-depth less than $2\l - 1$, or out-depth less than $2\l - 1$, in $\wfg{\dsk{\act}{\l}}{\sD}$. 
Formally,\\
\ind  $\fa \B_i \in \cmps{\act}: 
\widepth{\dsk{\act}{\l}}{\B_i}{\sD} < 2\l-1 \lor \wodepth{\dsk{\act}{\l}}{\B_i}{\sD} < 2\l-1$.
\end{definition}
%
To infer deadlock-freedom in $(\B, Q_0)$ by checking $\LLin(\B, Q_0, a, \l)$, we use \prop{edge-projection}: since wait-for edges project up and down,
it follows that wait-for paths project up and down, provided that the subsystem contains the entire wait-for path.

\begin{proposition}[In-projection, Out-projection] \label{prop:in-out-projection}
Let $\l \ge 1$, let $\B_i$ be an atomic component of $\B$, and let 
$(\B', Q'_0)$ be a subsystem of $(\B, Q_0)$ which is based on a superset of $\ssg{\act}{2\l}$.
Let $s$ be a state of $(B, Q_0)$, and $s' = s \pj B'$. Then
(1) $\widepth{\B}{\B_i}{s} < 2\l-1$ iff $\widepth{\B'}{\B_i}{s'} < 2\l-1$, and
(2) $\wodepth{\B}{\B_i}{s} < 2\l-1$ iff $\wodepth{\B'}{\B_i}{s'} < 2\l-1$.
\end{proposition}
%
\begin{proof}
We establish clause (1). The proof of clause (2) is analogous, except we replace paths ending in
$\B_i$ by paths starting from $\B_i$.
The proof of clause (1) is by double implication.

\vspace{1.0ex}
%%%%%%%%%%%%%%%%%%%%%%%%
\ul{$\widepth{\B}{\B_i}{s} < 2\l-1$ implies $\widepth{\B'}{\B_i}{s'} < 2\l-1$}:
%
Assume that $\widepth{\B}{\B_i}{s} < 2\l-1$.
Let $\pi$ be an arbitrary wait-for path in $\wfg{\B'}{s'}$ that ends in $\B_i$. 
%
Since $(\B', Q'_0)$ is a subsystem of $(\B, Q_0)$, by Definition~\ref{def:static:wait-for-graph} and $s' = s \pj \B'$,
$\wfg{\B'}{s'}$ is a subgraph of $\wfg{\B}{s}$, \ie $\wfg{\B'}{s'} \subg \wfg{\B}{s}$.
%
Hence $\pi$ is a wait-for path in $\wfg{\B}{s}$.
By $\widepth{\B}{\B_i}{s} < 2\l-1$, we have $|\pi| < 2\l-1$. 
Hence $\widepth{\B'}{\B_i}{s'} < 2\l-1$ since $\pi$ was arbitrarily chosen.



\vspace{1.0ex}
%%%%%%%%%%%%%%%%%%%%%%%%%%%%%%%%%%%%%%%%
\ul{$\widepth{\B'}{\B_i}{s'} < 2\l-1$ implies $\widepth{\B}{\B_i}{s} < 2\l-1$}:
%
Assume that $\widepth{\B}{\B_i}{s} \ge 2\l-1$. Then there exists a wait-for path $\pi$ in $\wfg{\B}{s}$ such that 
$|\pi| \ge 2\l-1$ and $\pi$ ends in $\B_i$. Let $\rho$ be the suffix of $\pi$ with length $2\l-1$. 
%
Since $(\B', Q'_0)$ is based on a superset of $\ssg{a}{2\l}$, and since the distance from $\B_i$ to the border of 
$\ssg{\act}{2\l}$ is $2\l-1$, we conclude that $\rho$ is a wait-for path
that is wholly contained in $\wfg{\B'}{s'}$. Hence we have $\widepth{\B'}{\B_i}{s'} \ge 2\l-1$.
%
We have thus established 
$\widepth{\B}{\B_i}{s} \ge 2\l-1$ implies $\widepth{\B'}{\B_i}{s'} \ge 2\l-1$.
The contrapositive is the desired result.
\end{proof}


%%%%%%%%%%%%%%%%%%%%%%%%%%%%%%%%%%%%%%%%%%%%%%%%%%%%%%



\vspace{0.5ex}

\noindent
We now show that $\LLin(\B, Q_0, \act, \l)$ implies $\GLin(\B, Q_0, \act)$, which in turn implies deadlock-freedom.  

\begin{lemma}
\label{lemma:loc-implies-glob}
\label{lem:locLinear-implies-globlinear}
\label{lemma:locLinear-implies-globlinear}
\label{LLinGLin}
Let $\act$ be an interaction of BIP-system $(\B, Q_0)$, \ie $\B = \gamma(\B_1,\dots,\B_n)$ and $\act \in \gamma$.
If $\LLin(\B, Q_0, \act, \l)$ holds for some finite $\l \ge 1$, then $\GLin(\B, Q_0, \act)$ holds.
\end{lemma}
%
\begin{proof}
Let $t \la{\act} s$ be a reachable transition of $(\B, Q_0)$ and let $\B_i \in \cmps{\act}$.
Let $\DS = \dsks{\act}{\l}$ and $\QDS = Q_0 \pj \DS$.
Let $\sD = s \pj \DS$, $\tD = t \pj \DS$.
Then $\tD \la{\act} \sD$ is a reachable transition of $(\DS, \QDS)$ by \cor{bip.reachability}.
By $\LLin(\B, Q_0, \act, \l)$, 
$\widepth{\dsk{\act}{\l}}{\B_i}{\sD} < 2\l-1 \lor \wodepth{\dsk{\act}{\l}}{\B_i}{\sD} < 2\l-1$.
Hence by Proposition~\ref{prop:in-out-projection},
$\widepth{\B}{\B_i}{s} < 2\l-1 \lor \wodepth{\B}{\B_i}{s} < 2\l-1$.
So
$\widepth{\B}{\B_i}{s} < \omega \lor \wodepth{\B}{\B_i}{s} < \omega$.
Hence $\GLin(\B, Q_0, \act)$.
\end{proof}






%% \paragraph{Complexity of evaluating $\LDFC(a, \l)$.}
%% Using explicit state enumeration, $\LDFC(a, \l)$ can be evaluated in
%% time $O(\SUM_{a \in \gamma} |\dsk{a}{\l}|)$, where $|\dsk{a}{\l}|$ denotes the
%% size of the transition system of $\dsk{a}{\l}$.


\begin{proposition} %[Finite out-depth implies local supercycle-violation]
\label{prop:finOutDepth-Implies-locScViol}
Let $d < \l$ and assume that node $v$ of $\lwfg{\B}{\sD}{\DS}$ 
has finite out-depth of $d \ge 1$ in $\lwfg{\B}{\sD}{\DS}$, \ie $\wodepth{\dsk{\act}{\l}}{v}{\sD} = d$.
Then $\locScViol{v}{d+1}{\sD}{\act}{\l}$.
\end{proposition}
%
\begin{proof}
Proof is by induction on $d$. 

\vspace{1.0ex}
\ul{Base case, $d=0$.} 
Hence by $\wodepth{\dsk{\act}{\l}}{v}{\sD} = 0$, and  Definitions~\ref{def:path} and \ref{def:depth},  
$v$ has no outgoing wait-for edges in $\lwfg{\B}{\sD}{\DS}$.
By \defn{blocksLoc}, $\neg \lblocks{\sD}{v}{\lwfg{\B}{\sD}{\DS}}$. 
By \defn{violFixLoc}, $v \in \lVFs{\sD}{\ewfg}$.
Hence $\locScViol{v}{1}{\sD}{\act}{\l}$ by \defn{supercycle.violation.local}.

\vspace{1.0ex}
\ul{Induction step, $d > 0$.}
Assume $(\wodepth{\dsk{\act}{\l}}{v}{\sD} = d)$. 
Let $w$ be an arbitrary successor of $v$, \ie a node $w$ such that $v \ar w \in \lwfg{\B}{\sD}{\DS}$.
By Definitions~\ref{def:path} and \ref{def:depth}, $w$ has an out-depth $d'$ that is less than $d$. 

By the induction hypothesis applied to $d'$, we obtain $\locScViol{w}{d'+1}{\sD}{\act}{\l}$, and so $w \in \lVFsi{\sD}{\ewfg}{d'+1}$ by \defn{supercycle.violation.local}.
Hence $w \in \lVFsi{\sD}{\ewfg}{d}$, since, by monotonicity of $\lVFsymb_{\sD}$, we have 
$ \lVFsi{\sD}{\ewfg}{n'} \subg  \lVFsi{\sD}{\ewfg}{n}$ when $n' \le n$.
Since  $w$ is an arbitrary successor of $v$, it follows that $v$ is only blocked by nodes in $\lVFsi{\sD}{\ewfg}{d}$.
Hence $\neg \lblocks{\sD}{v}{\complLoc{\lVFsi{\sD}{\ewfg}{d}}} $.
%
By \defn{violFixLoc}, $v \in \lVFs{\sD}{\lVFsi{\sD}{\ewfg}{d}}$, \ie $v \in  \lVFsi{\sD}{\ewfg}{d+1}$.
By \defn{supercycle.violation.local}, $\locScViol{v}{d+1}{\sD}{\act}{\l}$.
\end{proof}







\begin{lemma}
%\label{lemma:loc-implies-glob}
\label{lem:locLinear-implies-locANDOR}
\label{lemma:locLinear-implies-locANDOR}
\label{LLinLAO}
For all interactions $\act$ of $\B$, \ie $\act \in \gamma$, \\
\ind $\LLin(\B, Q_0, \act, \l)\ \impliess\ \LAO(\B, Q_0, \act, \l)$.
\end{lemma}
%
\begin{proof}
Assume $\LLin(\B, Q_0, \act, \l)$. Let $\tD \goesto[\act] \sD$ be an arbitrary reachable transition of $\dsk{\act}{\l}$,
and let $\B_i$ be an arbitrary component of $\cmps{\act}$.
Then, from \defn{LLin}, we have:
  $$\widepth{\dsk{\act}{\l}}{\B_i}{\sD} < 2\l-1 \lor \wodepth{\dsk{\act}{\l}}{\B_i}{\sD} < 2\l-1.$$
The proof proceeds by two cases. 

\vspace{1.0ex}
\ul{ $\widepth{\dsk{\act}{\l}}{\B_i}{\sD} < 2\l-1$}:
Hence $\B_i$ cannot be in a strongly connected supercycle, because $\B_i$ would then lie on at least one wait-for cycle, and so would have infinite
in-depth. Hence $\locConnViol{\B_i}{\sD}{\act}{\l}$ by Definition~\ref{def:sConn.violation.loc}, Clause~\ref{def:sConn.violation.loc:scc}.
Hence by \defn{locFormation.violation}, $\locFormViol{\B_i}{\sD}{\act}{\l}$.

\vspace{1.0ex}
\ul{$\wodepth{\dsk{\act}{\l}}{\B_i}{\sD} < 2\l-1$}:
Hence $\wodepth{\dsk{\act}{\l}}{\B_i}{\sD} = d$ for some $d < 2\l-1$.
By Proposition~\ref{prop:finOutDepth-Implies-locScViol}, $\locScViol{\B_i}{d+1}{\sD}{\act}{\l}$.
Hence by \defn{locFormation.violation}, $\locFormViol{\B_i}{\sD}{\act}{\l}$.

\vspace{1.0ex}
\noindent
In both cases, we have $\locFormViol{\B_i}{\sD}{\act}{\l}$.
Since $\B_i$ is an arbitrarily chosen  component of $\cmps{\act}$, we have $\fa \B_i \in \cmps{\act} : \locFormViol{\B_i}{\sD}{\act}{\l}$.
Hence, by Definition~\ref{def:lao}, we conclude $\LAO(\B, Q_0, \act, \l)$.
\end{proof}



\begin{theorem} \label{thm:LLin.SC-free-preserving}
$\LLin$ is supercycle-freedom preserving
\end{theorem}
%
\begin{proof}
Follows immediately from \thm{GLin.SC-free-preserving} and \lem{locLinear-implies-globlinear}.
%
Also follows immediately from \thm{GAO.SC-free-preserving} and \lem{locLinear-implies-locANDOR}.
\end{proof}
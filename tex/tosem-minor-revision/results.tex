

Figure~\ref{fig:implications} gives the implication relations between our four deadlock-freedom conditions.
Each implication arrow is labeled by the Lemma that provides the corresponding result.
\begin{figure}[ht]
\begin{center}
\scalebox{0.8}{\input{figs/implications.pdf_t}}
\caption{Implication relations between deadlock-freedom conditions}
\label{fig:implications}
\end{center}
\end{figure}


We can use the four conditions together: if, for each interaction, we verify one of the conditions, then we can infer deadlock-freedom, \ie combining the conditions in this manner is still sound \wrt deadlock-freedom.

\begin{theorem}[Deadlock-freedom via \GAO, \GLin, \LAO, \LLin]
\label{theorem:local.deadlock-free}
\label{thm:local.deadlock-free}
Assume that
\bn
\item \label{theorem:local.deadlock-free.initial}
      for all $s_0 \in Q_0$, $\wfg{\B}{s_0}$ is supercycle-free, and
\item \label{theorem:local.deadlock-free.scfPres}
      for all interactions $\act$ of $\B$ (\ie $\act \in \gamma$), one of
      the following holds:
      \bn
      \item $\GAO(\B,Q_0,\act)$
      \item $\GLin(\B,Q_0,\act)$
      \item $\ex \l > 0: \LAO(\B,Q_0,\act,\l)$ 
      \item $\ex \l > 0: \LLin(\B,Q_0,\act,\l)$ 
      \en
\en
Then for every reachable state $u$ of $(\B, Q_0)$:  $\wfg{\B}{u}$ is supercycle-free, and so 
$(\B, Q_0)$ is free of local and global deadlock.
\end{theorem}
%
\begin{proof}
Immediate from
Theorems~\ref{thm:GAO.SC-free-preserving}, \ref{thm:GLin.SC-free-preserving}, \ref{thm:LAO.SC-free-preserving}, \ref{thm:LLin.SC-free-preserving}
and Corollary~\ref{cor:SC-free-preserving.deadlock-free}.
\end{proof}




Finally, we establish that \GAO is complete \wrt deadlock-freedom: any system that is free of local and global deadlock will satisfy \GAO.

\begin{theorem}[Completeness of \GAO \wrt Deadlock-freedom]
\label{theorem:gao-is-complete}
Assume that $(\B,Q_0)$ is free from local and global deadlock. 
Then,  for all interactions $\act$ of $\B$ (\ie $\act \in \gamma$), 
$\GAO(\B,Q_0,\act)$ holds.
\end{theorem}
%
\begin{proof}
Let $\act$ be an arbitrary interaction in $\gamma$, and let 
$s \goesto[\act] t$ be a reachable transition of $(\B, Q_0)$.
Hence $t$ is a reachable state of $(\B,Q_0)$.
Suppose that $\wfg{\B}{t}$ contains a supercycle $\SC$. Then, by 
Proposition~\ref{prop:static:supercycle-is-sufficient}, the
subcomponent $B'$ consisting of all the atomic components $B_i \in \SC$
cannot execute a transition from any state reachable from $t$, and so is deadlocked.
Hence $(\B,Q_0)$ has a local deadlock in reachable state $t$, contrary to assumption.
Hence $\wfg{\B}{t}$ is supercycle-free.

Let $v$ be an arbitrary node in $\wfg{\B}{t}$.
By Definition~\ref{defn:supercycle.membership}, $\neg \scyc{\B}{s}{v}$ holds.
Hence by \prop{scViol-iff-notInSC}, $(\ex d \ge 1: \viol{v}{d}{t})$ holds.
By Definition~\ref{defn:formation.violation},  $\genViol{v}{t}$ holds.
Since $v$ is an arbitrary node in $\wfg{\B}{t}$, and all $\B_i \in \comps{\act}$ are nodes in 
$\wfg{\B}{t}$, we have $(\fas \B_i \in \comps{\act}, \formViol{\B_i}{t})$.
By Definition~\ref{defn:global.ANDOR-cond}, $\GAO(\B, Q_0, \act)$ holds.
Since $\act$ is an arbitrary interaction in $\gamma$, we have
$(\fas \act \in \gamma: \GAO(\B,Q_0,\act))$, and the theorem is established.
\end{proof}
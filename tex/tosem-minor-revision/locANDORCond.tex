
We now seek a local condition, which we evaluate in $\dsk{\act}{\l}$, and which implies \GAO.
We define local versions of both $\scViol{v}{d}{t}$ and $\connViol{v}{t}$.

To achieve a local and conservative approximation of $\scViol{v}{d}{t}$, we make the ``pessimistic'' assumption that the violation status of border
nodes of $\dsk{\act}{\l}$ cannot be known, since it depends on nodes outside of $\dsk{\act}{\l}$.
Now, if an internal node $v$ of $\dsk{\act}{\l}$ can be marked with a level $d$ sc-violation, by applying 
Definition~\ref{def:supercycle-violation} only within
$\dsk{\act}{\l}$, and with the border nodes marked as non-violating,
then it is also the case, as we show below, that $v$ has a level $d$ sc-violation overall.

To achieve a local and conservative approximation of
$\connViol{v}{t}$, we project onto a subsystem.




%\subsubsection{Local supercycle violation condition}





\subsubsection{Local strong connectedness condition}

We now present the local version of the strong connectedness violation condition, given above in Definition~\ref{def:sConn.violation}.

\begin{definition}[Local strong connectedness violation, $\locConnViol{v}{t_\act}{\act}{\l}$]
\label{def:sConn.violation.loc}

Let $L$ be the nodes of $\wfg{\dsk{a}{\l}}{t_\act}$ that have no local
supercycle violation, \ie $L = \set{v \stt v \in \dsk{a}{\l} \land \neg (\ex d \ge 1: \locScViol{v}{d}{t_\act}{\act}{\l}) }$.
Let $v$ be an arbitrary node in $L$. 
Let $WL = \wfg{\dsk{a}{\l}}{t_\act} \pj L$, \ie $WL$ is the subgraph of $\wfg{\dsk{a}{\l}}{t_\act}$ consisting of the
nodes in $L$, and the edges between those nodes that are also edges in $\wfg{\dsk{a}{\l}}{t_\act}$.

Then, $\locConnViol{v}{t_\act}{\act}{\l}$ holds iff:
\bn

\item \label{def:sConn.violation.loc:scc}
there does not exist a nontrivial strongly connected supercycle $\SSC$ such that $v \in \SSC$ and $\SSC \subg WL$, and

%\item \label{def:sConn.violation.loc:wait-for-out} every wait-for path $\pi$ from $v$ to a border node
%  of $\dsk{a}{\l}$ contains at least one node with a local supercycle violation

\item \label{def:sConn.violation.border}
either
    \bn

    \item \label{def:sConn.violation.loc:wait-for-out} every wait-for path $\pi$ from $v$ to a border node
      of $\dsk{a}{\l}$ contains at least one node with a local supercycle violation, \ie $\scVL{w}{t_\act}{\act}{\l}$

     or

    \item \label{def:sConn.violation.loc:wait-for-in} every wait-for path $\pi'$ from a border node
      of $\dsk{a}{\l}$ to $v$ contains at least one node $w$ with a
      local supercycle violation,  \ie $\scVL{w}{t_\act}{\act}{\l}$

    \en

\en
\end{definition}


We show that the local strong connectedness condition implies the global strong connectedness condition.

\begin{proposition}
\label{prop:locConnViol-implies-ConnViol}
 \label{prop:locConnViol-implies-connViol}
Let $t$ be an arbitrary reachable state of BIP-system $(\B, Q_0)$.
For all interactions $\act \in \gamma$, and $\l > 0$, let $t_\act = t \pj \dsk{\act}{\l}$.
Then\\
\ind $\locConnViol{v}{t_\act}{\act}{\l} \imp \connViol{v}{t}$.
\end{proposition}
%
\begin{proof}
By contradiction. Assume there exists a node $v$ in $\dsk{a}{\l}$ such that $\locConnViol{v}{t_\act}{\act}{\l} \land \neg \connViol{v}{t}$.
By $\neg \connViol{v}{t}$ and Definition~\ref{def:sConn.violation}, there exists a strongly connected
supercycle $\SSC$ such that $v \in \SSC$ and $\SSC \subg \wfg{B}{t}$. Then, there are two cases:
%
\bn
\item $\SSC \subg \wfg{\dsk{a}{\l}}{t_\act}$: let $x$ be any node in $\SSC$. Since $x$ is a node in a supercycle, we have by
%OLD  Proposition~\ref{prop:scViol-implies-notInSC}, that $\neg (\ex d \ge  1: \scViol{x}{d}{t})$. Hence 
  \defn{supercycle.membership} and \prop{scViol-iff-notInSC}, that $\neg (\ex d \ge 1: \scViol{x}{d}{t})$. Hence 
  $(\fa d \ge 1: \neg \scViol{x}{d}{t})$. Hence by Proposition~\ref{prop:locScViol-implies-scViol}, 
  we have $(\fa d \ge 1: \neg \locScViol{x}{d}{t_\act}{\act}{\l})$. Let $L, WL$ be as given in Definition~\ref{def:sConn.violation.loc}.
  Then $x \in L$, and since $x$ is an arbitrary node of $\SSC$, we have $SSC \sub WL$. 
  Thus Clause~\ref{def:sConn.violation.loc:scc} of Definition~\ref{def:sConn.violation.loc} is violated.

\item $\SSC \not\subg \wfg{\dsk{a}{\l}}{t_\act}$: then there exists a node $x \in \SSC -
  \dsk{a}{\l}$. Since $v \in \SSC$, there must exist a wait-for path $\pi$
  from $v$ to $x$ and a wait-for path $\pi'$ from $x$ to
  $v$. Since $v \in \dsk{a}{\l}$ and $x \not\in \dsk{a}{\l}$, it
  follows that both $\pi$, $\pi'$  cross a border node of
  $\dsk{a}{\l}$. Furthermore, since $\pi$, $\pi'$ are part of $\SSC$, every node
  along $\pi$, $\pi'$ is in a supercycle, and so cannot have a supercycle violation.
  By Proposition~\ref{prop:locScViol-implies-scViol}, the nodes on
  $\pi$, $\pi'$  cannot have a local supercycle violation.
  Hence Clauses~\ref{def:sConn.violation.loc:wait-for-out} and
  \ref{def:sConn.violation.loc:wait-for-in} of Definition~\ref{def:sConn.violation.loc} are violated,
  since they require that at least one node along $\pi$, $\pi'$ respectively, have a local supercycle violation.
  
\en
In both cases,  Definition~\ref{def:sConn.violation.loc} is violated. 
But  Definition~\ref{def:sConn.violation.loc} must hold, since we have $\locConnViol{v}{t_\act}{\act}{\l}$. 
Hence the desired contradiction.
\end{proof}


% Let $v$ be a node that is in a supercycle of $\wfg{\dsk{a}{\l}}{t_\act}$, and $t_\act$ a state of $\dsk{\act}{\l}$.
% Let $W$ be the result of removing from $\wfg{\dsk{a}{\l}}{t}$ every node $u$ such that 
% $(\ex d \ge 1: \locScViol{u}{d}{t_\act}{\act}{\l})$. Let $V$ be the maximal strongly connected component of $W$ that
% contains $v$. Then $\locConnViol{v}{t_\act}{\act}{\l}$ holds iff $V$ (by itself) is not a supercycle.
% For technical convenience, we also define $\locConnViol{v}{t_\act}{\act}{\l}$ to be false when $(\ex d \ge 1:
% \locScViol{v}{d}{t_\act}{\act}{\l})$, \ie when $v$ is not in a supercycle.
% Hence $\locConnViol{v}{t_\act}{\act}{\l}$ is always well-defined.




\subsubsection{General local violation condition}


We showed above that local supercycle violation implies global supercycle violation, and local
strong connectedness violation implies global string connectedness violation.  The general global
supercycle violation condition is the disjunction of global supercycle violation and global strong
connectedness violation.  Hence we formulate the general local supercycle violation condition as the
disjunction of local supercycle violation and local strong connectedness violation.  It follows that
the local supercycle formation condition implies the global supercycle formation condition.


\begin{definition}[General local supercycle violation, $\locFormViol{v}{t_\act}{\act}{\l}$]
\label{def:locFormation.violation}
Let $v$ be a node of $\dsk{\act}{\l}$.
Then $\locFormViol{v}{t_\act}{\act}{\l}  \df \scVL{v}{t_\act}{\act}{\l}) \lor \locConnViol{v}{t_\act}{\act}{\l}$.
%Then $\locFormViol{v}{t_\act}{\act}{\l}  \df  (\exs d \ge 1: \locScViol{v}{d}{t_\act}{\act}{\l}) \lor \locConnViol{v}{t_\act}{\act}{\l}$.
\end{definition}
%Let $s \goesto[\act] t$ be a reachable transition. If, for every $\B_i \in \cmps{\act}$, 
%$\formViol{v}{t}$ holds, then $s \goesto[\act] t$ does not introduce a supercycle, \ie if $s$ is
%supercycle-free, then so is $t$. We establish this in the sequel.
%

\begin{proposition} \label{prop:locFromViol-implies-formViol}
\label{prop:locformviol-implies-formviol}
Let $t$ be an arbitrary reachable state of BIP-system $(\B, Q_0)$.
For all interactions $\act \in \gamma$, and $\l > 0$, let $t_\act = t \pj \dsk{\act}{\l}$.
Then\\
%\ind $\fa d \ge 1: $\locFormViol{v}{t_\act}{\act}{\l}\locFormViol{v}{d}{t_\act}{\act}{\l} \imp \formViol{v}{d}{t}$.
\ind $ \locFormViol{v}{t_\act}{\act}{\l} \imp \formViol{v}{t}$.
\end{proposition}
%
\begin{proof}
Assume that $\locFormViol{v}{t_\act}{\act}{\l}$ holds. Then, by Definition~\ref{def:formation.violation}, 
$(\exs d \ge 1: \locScViol{v}{d}{t_\act}{\act}{\l}) \lor \locConnViol{v}{t_\act}{\act}{\l}$.
We proceed by cases:
\bn
\item $(\exs d \ge 1: \locScViol{v}{d}{t_\act}{\act}{\l})$: hence $(\exs d \ge 1: \scViol{v}{d}{t})$ by Proposition~\ref{prop:locScViol-implies-scViol}.
\item $\locConnViol{v}{t_\act}{\act}{\l}$: hence $\connViol{v}{t}$ by Proposition~\ref{prop:locConnViol-implies-connViol}.
\en
By Definition~\ref{def:formation.violation},  $\formViol{v}{t}  \df (\exs d \ge 1: \scViol{u}{d}{t}) \lor \connViol{v}{t}$.
Hence we conclude that $\formViol{v}{t}$ holds.
\end{proof}




\subsubsection{Local AND-OR Condition}

The actual local condition, \LAO, is given by applying the local supercycle formation condition to every reachable transition 
of the subsystem $\dsk{\act}{\l}$ being considered, and to every component $B_i \in \cmps{\act}$.

\begin{definition}[$\LAO(\B, Q_0, \act, \l)$] \label{def:lao}
Let $\l > 0$, and let $s_\act \goesto[\act] t_\act$ be an arbitrary reachable transition of $\dsk{\act}{\l}$.
Then, in $t_\act$, the following holds. 
For every component $\B_i$ of $\cmps{\act}$:  
$\B_i$ has a supercycle formation violation that can be confirmed within $\dsk{\act}{\l}$.
Formally,\\
\ind  $\fa \B_i \in \cmps{\act} : \locFormViol{\B_i}{t_\act}{\act}{\l}$.
\end{definition}
%
We showed previously that $\GAO$ implies deadlock-freedom, and so it remains to establish that $\LAO$ implies $\GAO$. 


\begin{lemma}
\label{lemma:loc.ANDOR.implies.glob.AND-OR}
\label{LAOGAO}
Let $\act \in \gamma$ be an interaction of BIP-system $(\B, Q_0)$. Then\\
\ind $(\ex \l > 0: \LAO(\B, Q_0, \act, \l))$ implies $\GAO(\B, Q_0, \act)$
\end{lemma}
%
%\prf{Immediate from Proposition~\ref{prop:locformviol-implies-formviol} and Definitions~\ref{def:global.ANDOR-cond}, \ref{def:lao}.}

\begin{proof}
Assume $\LAO(\B, Q_0, \act, \l)$ for some $\l > 0$. 
%
Let $s \goesto[\act] t$ be an arbitrary reachable transition of BIP-system $(\B, Q_0)$, and let 
$s_\act \goesto[\act] t_\act$ be the projection of $s \goesto[\act] t$ onto $\dsk{\act}{\l}$.
By Corollary~\ref{cor:bip.reachability}, $s_\act \goesto[\act] t_\act$ is a reachable transition of $\dsk{\act}{\l}$.

\noindent
By Definition~\ref{def:lao}, we have for some $\l > 0$:\\
\ind for every reachable transition $s_\act \goesto[\act] t_\act$ of $\dsk{\act}{\l}$:\\
\ind \ind $\fa \B_i \in \cmps{\act} : \locFormViol{\B_i}{t_\act}{\act}{\l}$.
%\ind \ind  $\fa \B_i \in \cmps{\act}, \exs d \ge 1: \lviol{\B_i}{d}{t_\act}{\act}{\l}$. 

\noindent
From this and Proposition~\ref{prop:locFromViol-implies-formViol},\\
\ind for every reachable transition $s \goesto[\act] t$ of  $(\B, Q_0)$:\\ 
\ind \ind $\fa \B_i \in \cmps{\act} : \formViol{B_i}{t}$

\noindent
Hence, by Definition~\ref{def:global.ANDOR-cond}, $\GAO(\B, Q_0, \act)$ holds.
\end{proof}



\begin{theorem} \label{thm:LAO.SC-free-preserving}
$\LAO$ is supercycle-freedom preserving
\end{theorem}
%
\begin{proof}
Follows immediately from Lemma~\ref{lemma:loc.ANDOR.implies.glob.AND-OR} and Theorem~\ref{thm:GAO.SC-free-preserving}.
\end{proof}




%%%%%%%%%%%%%%%%%%%%%%%%%%%%%%%%%%%%%%%%%%%%%%%%%%%%%%%%%%%%%%%%%%%%
\begin{figure}[t]

\begin{tabular}{|l|l|}
\hline
$\scViol{v}{d}{t}$  & $v$ confirmed at depth $d$ to not be in supercycle\\ 
              %supercycle violation condition
$\locScViol{v}{d}{t_\act}{\act}{\l}$ & $v$ locally determined to not be in a supercycle\\
              % local supercycle violation condition:

$\connViol{v}{t}$ & $v$ not in a strongly connected supercycle \\
              %strongly connected supercycle violation: 

$\locConnViol{v}{t_\act}{\act}{\l}$ & $v$ locally determined to not be in a strongly connected supercycle \\
               %local strongly connected supercycle violation 

$\formViol{v}{t}$ & $v$ does not contribute to a supercycle\\
               %supercycle formation violation: 

$\locFormViol{v}{t_\act}{\act}{\l}$ & $v$ locally determined to not contribute to a supercycle\\
                %local supercycle formation violation condition

\hline
\end{tabular}

\caption{Summary of predicates}
\label{fig:summaryPredicates}
\end{figure}



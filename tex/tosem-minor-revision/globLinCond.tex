In some cases, a simpler condition suffices to guarantee deadlock-freedom. This simpler condition is ``linear'', \ie it lacks the AND-OR alternation
aspect of $\GAO$. After execution of a reachable transition $s \la{\act} t$ of $(\B, Q_0)$, 
we consider the in-depth and out-depth of the components $\B_i \in \cmps{\act}$. There are three cases:
%
\begin{itemize}

\item \emph{Case 1} \label{case:finite-in} $\B_i$ has finite in-depth in $\wfg{\B}{t}$: then, if $\B_i \in \SC$, it can be removed and still leave a
  supercycle $\SC'$, by Proposition~\ref{prop:supercycle:essential-subgraph-of}. Hence $\SC'$ exists in $\wfg{\B}{s}$, and so $\B_i$ is not essential
  to the creation of a supercycle.

\item \emph{Case 2} \label{case:finite-out} $\B_i$ has finite out-depth in $\wfg{\B}{t}$: by \cor{supercycle:no-finite-outdepth}, $\B_i$ cannot be
  part of a supercycle, and so $\SC \subg \wfg{\B}{s}$.

\item \emph{Case 3} \label{case:infinite-both} $\B_i$ has infinite in-depth and infinite out-depth in $\wfg{\B}{t}$: in this case, $\B_i$ is possibly
  an essential part of $\SC$, \ie $\SC$ was created in going from $s$ to $t$.

\end{itemize}
We thus impose a condition which guarantees that only 
Case~1 %\ref{case:finite-in}   %labels give section numbers, not list item numbers
or Case~2 %\ref{case:finite-out} 
occur. 



\begin{definition}[$\GLin(B,Q_0,\act)$] \label{def:global:dfc}
%Let $(B, Q_0)$ be a BIP system, with $B =\gamma(\B_1,\dots,\B_n)$, and 
%$a$ an interaction of $(B, Q_0)$, \ie $a \in \gamma$.
$\GLin(B,Q_0,\act)$ holds iff, for every reachable transition $s \goesto[\act] t$ of BIP-system $(\B, Q_0)$, 
the following holds in state $t$:\\
\ind  $\fa \B_i \in \cmps{a}: \widepth{\B}{\B_i}{t} < \omega \lor \wodepth{\B}{\B_i}{t} < \omega.$\\
That is, for every component $\B_i$ of $\cmps{\act}$:  either $\B_i$ has finite in-depth, or finite out-depth, in $\wfg{\B}{t}$.
\end{definition}



\begin{proposition} \label{prop:indepth-finite-implies-scViol}
Assume that node $v$ of $\wfg{\B}{t}$ has a finite in-depth of $d$ in $\wfg{\B}{t}$, \ie 
$\widepth{\B}{v}{t} = d$. Then $\connViol{v}{t}$.
\end{proposition}
%
\begin{proof}
A node with finite in-depth cannot be in a wait-for-cycle, and
therefore cannot be in a strongly connected supercycle.
\end{proof}


% \prf{
% Proof is by induction on $d$.

% \noindent
% \ul{Base case, $d=1$.} Hence by Definitions~\ref{def:path} and \ref{def:depth},  
% $v$ has no incoming wait-for-edges in $\wfg{\B}{t}$. Hence by Definition~\ref{def:supercycle.violation},
% $\viol{v}{1}{t}$.

% \noindent
% \ul{Inductive step, $d > 1$.}
% By Definition~\ref{def:path} and \ref{def:depth}, all predecessors (\ie nodes $u$ such that $u \ar v \in \wfg{\B}{t}$)
% of $v$ have an in-depth of at most $d-1$. Hence, by the induction hypothesis applied to $d-1$, we obtain 
% $\viol{u}{d-1}{t}$ for all predecessors $u$ of $v$. Hence by Definition~\ref{def:supercycle.violation}, 
% Clauses~\ref{def:supercycle.violation.component.in} and \ref{def:supercycle.violation.interaction.in},
% $\viol{v}{1}{t}$.
% }




\begin{lemma} \label{lemma:glob-lin-implies-globANDOR} \label{GLinGAO}
$\fa \act \in \gamma: \GLin(\B, Q_0, \act) \imp \GAO(\B, Q_0, \act)$.  
\end{lemma}
%
\begin{proof}
Assume, for arbitrary $\act \in \gamma$, that $\GLin(\B, Q_0, \act)$ holds. That is, 
 
\ind For every reachable transition $s \goesto[\act] t$ of $(\B, Q_0)$,\\
\ind \ind  $\fas \B_i \in \cmps{\act}: \widepth{\B}{\B_i}{t} < \omega \lor \wodepth{\B}{\B_i}{t} < \omega$.

\noindent
By Propositions~\ref{prop:indepth-finite-implies-scViol} and \ref{prop:outdepth-finite-implies-scViol}, 

\ind For every reachable transition $s \goesto[\act] t$ of $(\B, Q_0)$,\\
\ind \ind  $\fas \B_i \in \cmps{\act}:  \connViol{\B_i}{t} \lor (\ex d \ge 1: \viol{\B_i}{d}{t})$.

\noindent
Hence by Definition~\ref{def:formation.violation}, 

\ind For every reachable transition $s \goesto[\act] t$ of $(\B, Q_0)$,\\
\ind \ind  $\fas \B_i \in \cmps{\act} : \formViol{\B_i}{t}$

\noindent
Hence $\GAO(\B, Q_0, \act)$ holds.
\end{proof}


\begin{theorem} \label{thm:GLin.SC-free-preserving}
$\GLin$ is supercycle-freedom preserving
\end{theorem}
%
\begin{proof}
Follows immediately from Lemma~\ref{lemma:glob-lin-implies-globANDOR} and Theorem~\ref{thm:GAO.SC-free-preserving}.
\end{proof}


%Since $\GLin(\act)$ implies $\GAO(\act)$, and $(\fa \act \in \gamma: \GAO(\act))$ implies deadlock-freedom (assuming also that initial
%state are deadlock-free), we obtain:



%%%%%%%%%%%%%%%%%%%%%%%%%%%%%%%%%%%%%%%%%%%%%%%%%%%%%%%%%
\subsection{Deadlock freedom using global restrictions}


\begin{corollary} [Deadlock-freedom via $\GAO, \GLin$] 
\label{theorem:global.deadlock-free}
Assume that
\bn
\item \label{theorem:global.deadlock-free.initial}
      for all $s_0 \in Q_0$, $\wfg{\B}{s_0}$ is supercycle-free, and
\item \label{theorem:global.deadlock-free.scfPres}
      for all interactions $\act$ of $\B$ (\ie $\act \in \gamma$):  $\GAO(\B, Q_0, \act) \lor \GLin(\B, Q_0, \act)$ holds.
\en
Then for every reachable state $u$ of $(\B, Q_0)$:  $\wfg{\B}{u}$ is supercycle-free, and so 
$(\B, Q_0)$ is free of local deadlock.
\end{corollary}
%
\begin{proof}
Immediate from Theorems~\ref{thm:GAO.SC-free-preserving}, \ref{thm:GLin.SC-free-preserving}, and Corollary~\ref{cor:SC-free-preserving.deadlock-free}.
\end{proof}



%% \bt[Deadlock-freedom via \GLin] 
%% \label{theorem:global:deadlock-free}
%% \label{theorem:global.linear.deadlock-free}
%% %Let $(B, Q_0)$ be a BIP system, with $B = \gamma(\B_1,\dots,\B_n)$.
%% Assume that
%% \bn
%% \item \label{theorem:global.linear.deadlock-free.initial} 
%%       for all $s_0 \in Q_0$, $\wfg{\B}{s_0}$ is supercycle-free, and
%% \item \label{theorem:global.linear.deadlock-free.globalLinear}
%%       for all interactions $\act$ of $B$ (\ie $\act \in \gamma$), $\GLin(\act)$ holds.
%% \en
%% Then for every reachable state $u$ of $(\B, Q_0)$:  $\wfg{\B}{u}$ is supercycle-free.
%% \et
%% %
%% \prf{
%% By Assumption~(\ref{theorem:global.linear.deadlock-free.globalLinear}) and 
%% Proposition~\ref{prop:glob-lin-implies-globANDOR}, we have 
%% $\fas \act \in \gamma: \GAO(\act)$. By this and
%% Assumption~(\ref{theorem:global.linear.deadlock-free.initial}), we obtain that the 
%% assumptions of Theorem~\ref{theorem:global.ANDOR.deadlock-free} hold. Hence, by applying 
%% Theorem~\ref{theorem:global.ANDOR.deadlock-free}, we obtain the conclusion

%% for every reachable state $u$ of $(\B, Q_0)$:  $\wfg{\B}{u}$ is supercycle-free.

%% and we are done.
%% }




%\setlength{\parindent}{1.5em}
%\setlength{\parskip}{1ex}
\renewcommand{\floatpagefraction}{0.8}

%\newlength{\parboxwidth}
%\setlength{\parboxwidth}{6.0in}


\sloppy

%%%%%%%%%%%%%%%%%%%%%%%%% EQUATION AND PROOF ENVIRONMENTS %%%%%%%%%%%%%%%%%%%%%%%%%%%%%%%%

%% Equation Environments
\newcommand{\leqn}[2]{\bs ~~~~~\parbox[t]{4.5in}{#2} \hfill{#1}\bs}

\newcommand{\prf}[1]{\vspace{-1ex} \noindent \textit{Proof.} #1 \hfill{$\Box$}}
%\newcommand{\prf}[1]{}

\newcommand{\prfs}[1]{}            % proof sketches
%\newcommand{\prfs}[1]{\vspace{-1ex} \noindent \textit{Proof sketch.} #1 \hfill{$\Box$}}

%\newenvironment{proof}{\vspace{-1.0ex}\textit{Proof.} }	% Proof environment
 %                     {\hfill{$\Box$}}



%%%%%%%%%%%%%%%%%%%%%%%%%% CONDITIONAL TEXT INCLUSION %%%%%%%%%%%%%%%%%%%%%%%%%%%%5

\newcommand{\full}[1]{#1}






%%%%%%%%%%%%%%% PACKAGES %%%%%%%%%%%%%%%%%%%%%%%%%%%%%%%

\usepackage{amsmath}
\usepackage{amssymb}
\usepackage{latexsym}
\usepackage{xspace}
 
%\usepackage{epic}
%\usepackage{eepic}
%\usepackage{epsfig}
\usepackage{url}

\usepackage{graphicx,color,xspace}
\usepackage{mathpartir}

\graphicspath{{figs/}} % figure path
\usepackage{float}


%%%%%%%%%%%%%%%%%%%%%%%%% REFERENCES

\newcommand{\defn}[1]{Def.~\ref{defn:#1}}
\newcommand{\fig}[2][]{Fig.~\ref{fig:#2}\ensuremath{#1}}
%\newcommand{\tab}[1]{Tab.~\ref{tab:#1}}
\newcommand{\eq}[1]{(\ref{eq:#1})}
\newcommand{\res}[1]{(\ref{res:#1})}
%\newcommand{\ex}[1]{Ex.~\ref{ex:#1}}
\newcommand{\secn}[1]{Sect.~\ref{secn:#1}}
\newcommand{\rem}[1]{Rem.~\ref{rem:#1}}
\newcommand{\ntn}[1]{Not.~\ref{ntn:#1}}
\newcommand{\lem}[1]{Lem.~\ref{lem:#1}}
\newcommand{\cor}[1]{Cor.~\ref{cor:#1}}
\newcommand{\thm}[1]{Th.~\ref{thm:#1}}
\newcommand{\app}[1]{App.~\ref{app:#1}}
\newcommand{\axs}[1]{Ax.~\ref{ax:#1}}
\newcommand{\axss}[2]{Ax.~\ref{ax:#1}, \ref{ax:#2}}
\newcommand{\ax}[1]{Ax.~\ref{ax:#1}}
\newcommand{\prop}[1]{Prop.~\ref{prop:#1}}
\newcommand{\clause}[1]{Clause~\ref{clause:#1}}



%%%%%%%%%%%%%%%%%%%%%


%\usepackage{algorithm}
%\usepackage{algorithmic}

\usepackage{subfigure}


%%%%%%%%%%%%%%%%%%%%%%%%%%%%%%%%%% META COMMANDS

\newcommand{\MATH}[1]{\ensuremath{#1}\xspace}

\newcommand{\MATHID}[1]{\ensuremath{\mathit{#1}}}
\newcommand{\MATHIDN}[1]{\ensuremath{\mathit{#1}}}
\newcommand{\MATHIDSP}[1]{\ensuremath{\mathit{#1}}\xspace}

\newcommand{\CMATHID}[1]{\ensuremath{\mathcal{#1}}\xspace}
\newcommand{\CMATHIDN}[1]{\ensuremath{\mathcal{#1}}}

\newcommand{\SMATHID}[1]{\ensuremath{\mathsf{#1}}}







%%%%%%%%%%%%%%%%%%%%%%%%%%%%%% ABBREVIATIONS %%%%%%%%%%%%%%%%%%%%%%%%%%%%%%%
\newcommand{\ie}{i.e.,\xspace}
\newcommand{\eg}{e.g.,\xspace}
\newcommand{\wrt}{w.r.t.\xspace}
\newcommand{\etal}{et.\ al.\xspace}

%% Lists
\newcommand{\be}{\begin{itemize}}
\newcommand{\ee}{\end{itemize}}
\newcommand{\bn}{\begin{enumerate}}
\newcommand{\en}{\end{enumerate}}
\newcommand{\bdn}{\begin{description}}
\newcommand{\edn}{\end{description}}
%\renewcommand{\i}{\item}

%% Theorem-like Environments
%\newcommand{\bp}{\begin{proposition}}
%\newcommand{\ep}{\end{proposition}}
%\newcommand{\bl}{\begin{lemma}}
%\newcommand{\el}{\end{lemma}}
%\newcommand{\bco}{\begin{corollary}}
%\newcommand{\eco}{\end{corollary}}
%\newcommand{\bt}{\begin{theorem}}
%\newcommand{\et}{\end{theorem}}
%\newcommand{\bpr}{\begin{proof}}
%\newcommand{\epr}{\end{proof}}
%\newcommand{\bd}{\begin{definition}}
%\newcommand{\ed}{\end{definition}}


%% Misc
\newcommand{\bc}{\begin{center}}
\newcommand{\ec}{\end{center}}
\newcommand{\ul}{\underline}
\newcommand{\ms}{\medskip}
\newcommand{\bs}{\bigskip}
%\newcommand{\ms}{\medskip}
\renewcommand{\ss}{\smallskip}

\newcommand{\bfg}{\begin{figure}}
\newcommand{\efg}{\end{figure}}





%%%%%%%%%%%%%%%%%%%%%%%%%%%%%%%%% BIP SPECIFIC MACROS

\newcommand{\act}{\ensuremath{\mathsf{a}}}       % general action
\newcommand{\actp}{\mathsf{aa}}   % second action name, used in violation defs
\newcommand{\actb}{\mathsf{b}}     % used in projection def
%\newcommand{\a}{\mathsf{a}}
\newcommand{\comp}{\mathsf{B}}
%\newcommand{\B}{\mathsf{B}}
\newcommand{\B}{\mathsf{B}}

\newcommand{\gd}[2]{\ensuremath{\MATHIDN{enb}_{#1}^{#2}}}

\newcommand{\enb}[2]{\mbox{$\, \stackrel{#1}{\longrightarrow}_{#2} \,$}}

\newcommand{\enbf}[2]{\ensuremath{\mathit{enabled}(#1, #2)}}   
\newcommand{\nenb}[2]{\mbox{$\, \stackrel{#1}{\not\longrightarrow}_{#2} \,$}}

\newcommand{\comps}[1]{\ensuremath{\MATHIDN{components}(#1)}}    % components of an interaction
\newcommand{\cmps}[1]{\ensuremath{\MATHIDN{components}(#1)}}    % components of an interaction

\newcommand{\rstates}[1]{\ensuremath{\MATHIDN{rstates}(#1)}}  % reachable states


%%%%%%%%%%%%%%%%%%%%%%%%%%%%%%%%%%%%%% DEADLOCK SPECIFIC MACROS

%\newcommand{\scf}[2]{\MATHID{sc\_free}_{#1}(#2)}

\newcommand{\power}[1]{\ensuremath{\CMATHIDN{P}(#1)}}   % powerset notation

\newcommand{\wfg}[2]{\ensuremath{\MATHIDN{W}_{#1}({#2})}}
%\newcommand{\wfgS}[2]{\ensuremath{\MATHIDN{\widehat{W}}_{#1}({#2})}}  %set of subgraphs of wiat for graph
\newcommand{\wfgS}[2]{\ensuremath{\power{\wfg{#1}{#2}}}}
\newcommand{\wfgPow}[2]{\ensuremath{\power{\wfg{#1}{#2}}}}



%\newcommand{\lwfg}[3]{\ensuremath{\MATHIDN{W}_{#1}^{#3}({#2})}}
\newcommand{\lwfg}[3]{\ensuremath{\MATHIDN{W}_{#3}({#2})}}
\newcommand{\lwfgS}[3]{\ensuremath{\power{\lwfg{#1}{#2}{#3}}}}

\newcommand{\UU}{\MATHID{U}} %wait for subgraphs
\newcommand{\US}{\MATHID{U}} % subgraph
\newcommand{\VS}{\MATHID{V}} % subgraph
\newcommand{\XS}{\MATHID{X}} % subgraph
\newcommand{\YS}{\MATHID{Y}} % subgraph
\newcommand{\ndv}{\MATHID{v}} % node in a wait for graph
\newcommand{\ndw}{\MATHID{w}} % node in a wait for graph
\newcommand{\ndx}{\MATHID{x}} % node in a wait for graph
\newcommand{\ndy}{\MATHID{y}} % node in a wait for graph


\newcommand{\compl}[1]{\ensuremath{\overline{\MATHIDN{#1}}}}
\newcommand{\complL}[1]{\ensuremath{\widehat{\MATHIDN{#1}}}} % local
                                % complement, widehat is placeholder, come up with something better


\newcommand{\mscc}[1]{\ensuremath{\MATHIDN{mscc(#1)}}}         % mscc component graph



\newcommand{\wfgr}[2]{\ensuremath{\MATHIDN{W}^R_{#1}({#2})}}
\newcommand{\sg}[1]{\ensuremath{\MATHIDN{G}_{#1}}}
\newcommand{\ssg}[2]{\ensuremath{\MATHIDN{G}_{#1}^{#2}}}

\newcommand{\scyc}[3]{\ensuremath{\MATHIDN{scyc}_{#1}({#3, #2})}}
\newcommand{\scscyc}[3]{\ensuremath{\MATHIDN{scscyc}_{#1}^{#2}({#3})}}


\newcommand{\preds}[3]{\ensuremath{\MATHID{preds}_{#1}({#2, #3})}}
\newcommand{\succs}[3]{\ensuremath{\MATHID{succs}_{#1}({#2, #3})}}


\newcommand{\SC}{\ensuremath{\MATHIDN{SC}}}
\newcommand{\CC}{\ensuremath{\MATHIDN{CC}}}

\newcommand{\In}[4]{\MATHID{In}_{#1}(#1, #2, #3)}
\newcommand{\Out}[4]{\MATHID{Out}_{#1}(#1, #2, #3)}

\newcommand{\InK}[3]{\MATHIDN{In}_{#2}({#1, #3})}



\newcommand{\PIn}[2]{\MATHIDN{PIn}_{#2}({#1})}
\newcommand{\PInK}[3]{\MATHIDN{PIn}_{#2}({#1, #3})}
\newcommand{\POut}[3]{\MATHIDN{POut}_{#2}({#1, #3})}

\newcommand{\DS}{\ensuremath{D}\xspace}     % deadlock subsystem
\newcommand{\ds}[1]{\ensuremath{\MATHIDN{D}_{#1}}}     % deadlock subsystem
\newcommand{\dsk}[2]{\ensuremath{\MATHIDN{D}_{#1}^{#2}}}     % generalized deadlock subsystem
\newcommand{\pdsk}[2]{\ensuremath{\MATHIDN{PD}_{#1}^{#2}}}     % generalized pessimistic deadlock subsystem



\newcommand{\Mdsk}[2]{\ensuremath{\MATHIDN{M}_{#1}^{#2}}}     % generalized deadlock subsystem



\newcommand{\lla}[2]{\mbox{$\, \stackrel{#1}{\longrightarrow}_{#2} \,$}}



\newcommand{\goesto}[1][]{\stackrel{#1}{\rightarrow}}
\newcommand{\goestog}[1][]{\stackrel{#1}{\rightarrow}}

\newcommand{\la}[1]{\ensuremath{\stackrel{#1}{\rightarrow}}}


\newcommand{\Bp}{B_\varphi}
\newcommand{\vph}{\varphi}


\newcommand{\pj}{\raisebox{0.2ex}{$\upharpoonright$}}
\newcommand{\al}{\alpha}

\newcommand{\Init}{\MATHID{Init}}
\newcommand{\CInv}{\MATH{\Phi}}
\newcommand{\IInv}{\MATH{\Psi}}
\newcommand{\GInv}{\MATH{\Phi}}




\newcommand{\DIS}{\MATHID{DIS}}
\newcommand{\intr}{\mathit{int}}
\newcommand{\BD}{\ensuremath{D}\xspace}
\newcommand{\mpi}{\gamma}


%%% supercycle violations

%% global supercycle violation
\newcommand{\scV}[2]{\ensuremath{\SMATHID{viol}_{\B}(#1,#2)}} % overall supercycle violation 
\newcommand{\scVd}[3]{\ensuremath{\SMATHID{viol}_{\B}(#1,#3,#2)}} % supercycle violation at depth $d$
\newcommand{\viol}[3]{\ensuremath{\SMATHID{viol}_{\B}(#1,#3,#2)}} % supercycle violation at depth $d$
\newcommand{\scViol}[3]{\ensuremath{\SMATHID{viol}_{\B}(#1,#3,#2)}} % supercycle violation at depth $d$

%% local supercycle violation
\newcommand{\scVL}[4]{\ensuremath{\SMATHID{violLoc}(#1,#2,\dsk{#3}{#4})}} % local supercycle violation condition

\newcommand{\lviol}[5]{\ensuremath{\SMATHID{violLoc}(#1,#3,#2,\dsk{#4}{#5})}} % local supercycle violation condition at depth $d$
\newcommand{\scVLd}[5]{\ensuremath{\SMATHID{violLoc}(#1,#3,#2,\dsk{#4}{#5})}}  % local supercycle violation condition at depth $d$
\newcommand{\locScViol}[5]{\ensuremath{\SMATHID{violLoc}(#1,#3,#2,\dsk{#4}{#5})}}  % local supercycle violation condition at depth $d$



% strong connectedness violation condition
\newcommand{\scviol}[2]{\ensuremath{\SMATHID{sConnViol}_{\B}(#1,#2)}}
\newcommand{\connViol}[2]{\ensuremath{\SMATHID{sConnViol}_{\B}(#1,#2)}}
\newcommand{\lconnViol}[4]{\ensuremath{\SMATHID{sConnViolLoc}(#1,#2,\dsk{#3}{#4})}}
\newcommand{\locConnViol}[4]{\ensuremath{\SMATHID{sConnViolLoc}(#1,#2,\dsk{#3}{#4})}}


% formation violation condition
\newcommand{\formViol}[2]{\ensuremath{\SMATHID{genViol}_{\B}(#1,#2)}}
\newcommand{\genViol}[2]{\ensuremath{\SMATHID{genViol}_{\B}(#1,#2)}}
\newcommand{\locFormViol}[4]{\ensuremath{\SMATHID{genViolLoc}(#1,#2,\dsk{#3}{#4})}}
\newcommand{\genLocViol}[4]{\ensuremath{\SMATHID{genViolLoc}(#1,#2,\dsk{#3}{#4})}}


% pseudocode
\newcommand{\lviolName}{\ensuremath{\SMATHID{violLoc}}}



\newcommand{\VA}[4]{\ensuremath{\MATHIDN{V_{#1,#2}[#3,#4]}}}    %violations array

\newcommand{\VV}[5]{\ensuremath{\MATHIDN{V_{\dsk{#1}{#2},#3} [#4 #5]}}}    %sc violations array, #1 is a, #2 is \l, #3 is t_a, #4 is v, #5 is d
                                                                        % i.e., \VV{a}{\l}{t_a}{v}{d}

\newcommand{\LF}[5]{\ensuremath{\MATHIDN{F_{\dsk{#1}{#2},#3}  [#4,#5]}}}    %loc form violations array, #1 is a, #2 is \l, #3 is t_a, #4 is v, #5 is d

\newcommand{\VN}[2]{\ensuremath{\MATHIDN{V_{#1,#2}}}}    %violations array name

\newcommand{\V}{\ensuremath{\MATHIDN{V}}}

\newcommand{\bU}{\overline{U}}
\newcommand{\bV}{\overline{V}}


% fixpoint
\newcommand{\lat}[2]{\ensuremath{\CMATHIDN{L}_{#1}({#2})}}           % wait for lattice
\newcommand{\llat}[3]{\ensuremath{\CMATHIDN{L}_{#3}({#2})}}           % wait for lattice
%\newcommand{\llat}[3]{\ensuremath{\CMATHIDN{L}_{#1}^{#3}({#2})}}           % wait for lattice
\newcommand{\ewfg}{\ensuremath{\emptyset}\xspace}                         % empty wait for graph
\newcommand{\ord}{\sqsubseteq}
\newcommand{\subg}{\sqsubseteq}  % subgraph
\newcommand{\meet}{\sqcap}
\newcommand{\join}{\sqcup}
\newcommand{\MEET}{\bigsqcap}
\newcommand{\JOIN}{\bigsqcup}

\newcommand{\SF}[1]{\ensuremath{\CMATHIDN{S}(#1)}}   % fixpoint mapping for supercycles
\newcommand{\SFs}{\ensuremath{\CMATHIDN{S}}}           % SF symbol with no args
\newcommand{\VF}[1]{\ensuremath{\CMATHIDN{V}(#1)}}   % fixpoint mapping for supercycle violation
\newcommand{\VFs}{\ensuremath{\CMATHIDN{V}}}  % VF symbol with no args
\newcommand{\blocks}[2]{\ensuremath{\MATHIDN{blocks}(#1, #2)}}
\newcommand{\lfp}[1]{\ensuremath{\MATHIDN{lfp}(#1)}}
\newcommand{\gfp}[1]{\ensuremath{\MATHIDN{gfp}(#1)}}


\newcommand{\lSF}[1]{\ensuremath{\CMATHIDN{SL}(#1)}}   % fixpoint mapping for supercycles
\newcommand{\lSFs}{\ensuremath{\CMATHIDN{SL}}}           % SF symbol with no args
\newcommand{\lVF}[1]{\ensuremath{\CMATHIDN{VL}(#1)}}   % fixpoint mapping for supercycle violation
\newcommand{\lVFs}{\ensuremath{\CMATHIDN{VL}}}  % VF symbol with no args
\newcommand{\lblocks}[2]{\ensuremath{\MATHIDN{lblocks}(#1, #2)}}



%%%%% REVISED CONDITION NAMES
\newcommand{\BC}{\CMATHID{BC}}

\newcommand{\GAO}{\CMATHID{GALT}}      %\newcommand{\GAO}{\CMATHID{GLOBAL\!-\!AND\!-\!OR}}
\newcommand{\GLin}{\CMATHID{GLIN}}     %{\CMATHID{GLOBAL\!-\!LINEAR}}

\newcommand{\LAO}{\CMATHID{LALT}}      %{\CMATHID{LOCAL\!-\!AND\!-\!OR}}
\newcommand{\LLin}{\CMATHID{LLIN}}     %{\CMATHID{LOCAL\!-\!LINEAR}}

\newcommand{\DFC}{\GLin}   %\CMATHID{DFC}}
\newcommand{\LDFC}{\LLin} %\CMATHID{LDFC}}

\newcommand{\ADFC}{\CMATHID{ADFC}}
\newcommand{\GDFC}{\CMATHID{GDFC}}

\newcommand{\LCDFC}{\CMATHID{LCDFC}}




%%%%%%%%%%%%%%%%%%%%% PROCEUDRE NAMES  %%%%%%%%%%%%%%%%%%%%%%%%%%%%%%%

\newcommand{\PNAME}[1]{\mbox{\textsc{#1}}}                    % Defines formatting of all algorithm names

\newcommand{\checkLAO}[1]{\PNAME{Lalt}(#1)}                %{\PNAME{checkLocalANDOR}(#1)}
\newcommand{\checkLAOInt}[1]{\PNAME{LaltInt}(#1)}          %{\PNAME{checkLocalANDORInt}(#1)}
\newcommand{\checkLAOIntDist}[1]{\PNAME{LaltIntDist}(#1)}  %{\PNAME{checkLocalANDORIntDist}(#1)}

\newcommand{\checkLin}[1]{\PNAME{LLin}(#1)}                %{\PNAME{checkLocalLinear}(#1)}
%\newcommand{\checkLin}[1]{\PNAME{checkLin}(#1)}                %{\PNAME{checkLocalLinear}(#1)}
\newcommand{\checkLinInt}[1]{\PNAME{LLinInt}(#1)}          %{\PNAME{checkLocalLinearInt}(#1)}
%\newcommand{\checkLinInt}[1]{\PNAME{checkLinInt}(#1)}          %{\PNAME{checkLocalLinearInt}(#1)}
\newcommand{\checkLinIntDist}[1]{\PNAME{LLinIntDist}(#1)}  %{\PNAME{checkLocalLinearIntDist}(#1)}
%\newcommand{\checkLinIntDist}[1]{\PNAME{checkLinIntDist}(#1)}  %{\PNAME{checkLocalLinearIntDist}(#1)}

\newcommand{\cViol}[1]{\PNAME{compute-local-SC-Violations}(#1)}
\newcommand{\cLScV}[1]{\PNAME{LocScViol}(#1)}

\newcommand{\cViolD}[1]{\PNAME{compute-local-SC-Violations-Dist}(#1)}
\newcommand{\cLScVD}[1]{\PNAME{LocScViolDist}(#1)}

\newcommand{\cViolDN}[1]{\PNAME{compute-local-SC-Violations-Dist-Node}(#1)}
\newcommand{\cViolDIN}[1]{\PNAME{violationsDistInteriorNode}({#1})}
\newcommand{\cViolDBN}[1]{\PNAME{violationsDistBorderNode}(#1)}

\newcommand{\cLScVDN}[1]{\PNAME{LocScViolDistNode}(#1)}



\newcommand{\cLFV}[1]{\PNAME{locFormViol}(#1)}
\newcommand{\cLconnScV}[1]{\PNAME{locSconnScViol}(#1)}


%,;lk;op-

\newcommand{\cInitSCFree}[1]{\PNAME{checkInitSupercycleFree}(#1)}

%%%%%%%%%%%%%%%%%%%%%%%%%%%%%%%%%%%%%%%%%%%%%%%%%%%%%%%%%%%%%%%%%%%%%%%%


\newcommand{\MAIN}{\PNAME{Main}}

\newcommand{\deadlocktool}{\PNAME{LALT-BIP}\xspace}
\newcommand{\ldfctool}{\PNAME{LDFC-BIP}\xspace}
\newcommand{\checkDF}[1]{\PNAME{checkDF}(#1)}
\newcommand{\globDFC}[1]{\PNAME{globDFC}(#1)}
\newcommand{\locLDFC}[1]{\PNAME{locLDFC}(#1)}



\newcommand{\locLCDFC}[1]{\PNAME{locLCDFC}(#1)}

\newcommand{\checkInt}[1]{\PNAME{checkInt}(#1)}
\newcommand{\checkLCDFC}[1]{\PNAME{checkLCDFC}(#1)}

\newcommand{\checkPath}[1]{\PNAME{checkPath}(#1)}
\newcommand{\checkSC}[1]{\PNAME{checkSC}(#1)}










%%%%%%%%%%%%%%%%%%%% DPHILS EXAMPLE %%%%%%%%%%%%%%%%%

\newcommand{\get}{\MATHID{get}}
\newcommand{\release}{\MATHID{release}}
\newcommand{\use}{\MATHID{use}}
\newcommand{\free}{\MATHID{free}}

\newcommand{\Get}{\MATHID{Get}}
\newcommand{\Rel}{\MATHID{Rel}}



%%%%%%%%%%%%%%%%%%%% FORMATTING COMMANDS %%%%%%%%%%%%%%%%%%%%%%%%%%%

\newcommand{\halfind}{\hspace*{1.5em}}
\newcommand{\ind}{\hspace*{3.0em}}
\newcommand{\pind}{\hspace*{3.0em}}


\newcommand{\horline}{\rule{\textwidth}{1pt}}

\newcommand{\remove}[1]{}               % remove blocks of text

\newcommand{\intrdef}[1]{\emph{#1\/}}   % emphasize first occurence of technical terms in definitions
\newcommand{\intrit}[1]{\emph{#1\/}}    % emphasize first occurence of technical terms that occur
                                        % in-line in the text

\newcommand{\intrb}[1]{\emph{\textbf{{#1\/}}}}

\newcommand{\lbr}{\linebreak}

\newcommand{\emp}{\emph}	
\newcommand{\empi}[1]{\textit{#1\/}}
\newcommand{\empb}[1]{\textbf{#1\/}}
\newcommand{\empbi}[1]{\textbf{\textit{#1\/}}}

\newcommand{\smpage}{\noindent \parbox{\textwidth}}

\newcommand{\intro}{\emph}	
\newcommand{\introdf}{\emph}	


%%%%%%%%%%%%%%%%% GRAPH THEORY %%%%%%%%%%%%%%%%%%%%%%%%%%%%%

\newcommand{\first}[1]{\MATHID{first}(#1)}
\newcommand{\last}[1]{\MATHID{last}(#1)}

\newcommand{\depth}[3]{\MATHIDN{depth}_{#2}({#1, #3})}

\newcommand{\idepth}[2]{\MATHIDN{in\_depth}_{#1}({#2})}        %indepth(G,v)
\newcommand{\odepth}[2]{\MATHIDN{out\_depth}_{#1}({#2})}     %outdepth(G,v)

\newcommand{\widepth}[3]{\MATHIDN{in\_depth}_{#1}({#2, #3})} % indepth(B,s,v) = indepth_{W_B(s)}(v)
\newcommand{\wodepth}[3]{\MATHIDN{out\_depth}_{#1}({#2, #3})} % outdepth(B,s,v) = outdepth_{W_B(s)}(v)

\newcommand{\noIn}{\ensuremath{\mathit{noIn}}}
\newcommand{\noOut}{\ensuremath{\mathit{noOut}}}

\newcommand{\pth}[2]{\MATHID{path}_{#1}(#2)}



%%%%%%%%%%%%%%%%%%%%%%% GENERAL MATH SYMBOLS %%%%%%%%%%%%%%%%%%%%%%%%%%%%%%

\newcommand{\ex}{\exists\,}
\newcommand{\fa}{\forall\,}
\newcommand{\exs}{\exists\,}
\newcommand{\fas}{\forall\,}
\newcommand{\exn}{\exists}    % no space after
\newcommand{\fan}{\forall}    % no space after

\newcommand{\stt}{~|~}





\newcommand{\MAX}{\mathrm{MAX}}

\newcommand{\df}{\triangleq}
%\newcommand{\df}{\stackrel{\mathit{def}}{=}}

%\newcommand{\AND}{\bigwedge}
\newcommand{\INT}{\bigcap}
%\newcommand{\OR}{\bigvee}

\newcommand{\UN}{\bigcup}
\newcommand{\PA}{/\!/}		% parallel assignment

\newcommand{\ar}{\rightarrow}	% abbreviated rightarrow
\newcommand{\choice}{\mbox{$[\hspace*{-1.0pt}]$}}
\renewcommand{\d}{\, : \,}	% separator in quantified formulae
%\newcommand{\df}{\mbox{$\:\stackrel{\rm df}{=\!\!=}\:$}}
\newcommand{\ev}{\equiv}
\newcommand{\ifof}{\Leftrightarrow}	% logical equivalence
\newcommand{\imp}{\Rightarrow}		% logical implication
\newcommand{\folf}{\Leftarrow}                 % follows from 
\newcommand{\ints}{\cap}
\newcommand{\lra}{\longrightarrow}
\newcommand{\oneton}{\set{1..n}}
\newcommand{\pl}{\!\parallel\!}
%\newcommand{\rng}[2]{[#1\!:\!#2]}	% integer range
\newcommand{\rng}[2]{\set{#1,\ldots,#2}}	% integer range
\newcommand{\s}{\mbox{$\hspace{-1pt}-\hspace{-2pt}$}}
\newcommand{\sat}{\models}
\newcommand{\spc}{\mbox{\vspace{-0.25in}}}
\newcommand{\st}{~|~}
\newcommand{\sub}{\subseteq}
\newcommand{\tl}[1]{\mbox{$\tilde{#1}$}}% abbreviated tilde
\newcommand{\un}{\cup}
%\newcommand{\up}{\mbox{$\hspace{-0.1em}\uparrow\hspace{-0.1em}$}}
\newcommand{\up}{\pj}

\newcommand{\struct}[2]{\raisebox{-0.1in}{$\stackrel { \displaystyle #1} {\scriptstyle #2}\,$}}

\newcommand{\set}[1]{\{ #1 \}}
\newcommand{\tpl}[1]{\ensuremath{\lpb #1 \rpb}}
\newcommand{\lpb}{\langle \hspace{-0.34em} \langle \hspace{-0.34em} \langle}	
						% left pair bracket
\newcommand{\rpb}{\rangle \hspace{-0.34em} \rangle \hspace{-0.34em} \rangle}
						% right pair bracket

\newcommand{\false}{\MATHID{false}}
\newcommand{\true}{\MATHID{true}}



\newcommand{\SUM}{\Sigma}


%%%%%%%%%%%%%%%%%%%%%%%%%%% LIST ENVIRONMENTS %%%%%%%%%%%%%%%%%%%%%%%%%%%%%%%

\newenvironment{lst}{\begin{list}	
                       {}
                       {\setlength{\topsep}{0em}
                        \setlength{\itemsep}{0em}
			\setlength{\leftmargin}{0.3in}
                        \setlength{\rightmargin}{0em}
		     }}
                    {\end{list}}

\newenvironment{blst}{\begin{list}		% bullet list
                       {$\bullet$}
                       {\setlength{\topsep}{0em}
                        \setlength{\itemsep}{0em}
			\setlength{\leftmargin}{0.25in}
		     }}
                    {\end{list}}

\newcounter{levelone}
\newenvironment{nlst1}{\begin{list}	% 1'st level enumerated list
                       {\arabic{levelone}.}
                       {\usecounter{levelone}
			\setlength{\topsep}{0.2ex}
                        \setlength{\itemsep}{0.1ex}
			\setlength{\leftmargin}{0.25in}
		     }}
                    {\end{list}}


\newcounter{leveltwo}
\newenvironment{nlst2}{\begin{list}	% 2'nd level enumerated list
                       {(\alph{leveltwo})}
                       {\usecounter{leveltwo}
			\setlength{\topsep}{0em}
                        \setlength{\itemsep}{0em}
			\setlength{\leftmargin}{0.4in}
		     }}
                    {\end{list}}








%%%%%%%%%%%%%%%%%%%%%%%%%%%%%%%%% MISC  %%%%%%%%%%%%%%%%%%%%%%%%%%%%%%%

\newcommand{\topcase}[2]{\vspace{1.5ex} \noindent \textit{Case} #1: #2.}   
\newcommand{\scase}[2]{\vspace{1.5ex} \noindent \textit{Subcase} #1: #2.}
\newcommand{\sscase}[2]{\vspace{1.0ex} \noindent \textit{Subsubcase} #1: #2.}

\newcommand{\cmnt}{\ensuremath{\rhd}}



%%%%%%%%%%%%%%%%%%%% PSEUDOCODE %%%%%%%%%%%%%%%%%%
\newcommand{\gts}{\leftarrow}                % assignment

\newcommand{\ttt}{\ensuremath{\mathsf{true}}}
\newcommand{\fff}{\ensuremath{\mathsf{false}}}
%\newcommand{\ttt}{\textup{\texttt{tt}}}
%\newcommand{\fff}{\textup{\texttt{ff}}}




\newcommand{\gt}{\ensuremath{:=}}   % assignment


\newcommand{\pseudocode}[1]{\ensuremath{\mathbf{#1}}}
\newcommand{\pseudocodensp}[1]{\ensuremath{\mathbf{#1}}}

\newcommand{\pseudocodesp}[1]{\ensuremath{\mathbf{#1}\ }}

%\newcommand{\pseudocode}[1]{\mbox{${\bf #1}$}\xspace}

\newcommand{\IFC}[1]{\pseudocode{if}\ (\ensuremath{#1})}
\newcommand{\ELSFC}[1]{\pseudocode{else\ if}\ (\ensuremath{#1})}
\newcommand{\WHILEC}[1]{\pseudocode{while}\ (\ensuremath{#1})}
%\newcommand{\FORALLC}[1]{\pseudocode{forall}\ (\ensuremath{#1})}
\newcommand{\FORALLC}[1]{\pseudocode{forall}\ #1}
\newcommand{\RETURNE}[1]{\pseudocodensp{return}(\ensuremath{#1})}
\newcommand{\CONTINUE}{\pseudocodensp{continue}}
\newcommand{\BREAK}{\pseudocodensp{break}}



\newcommand{\IF}{\pseudocode{if}}
\newcommand{\WHILE}{\pseudocode{while}}
\newcommand{\REPEAT}{\pseudocode{repeat}}
\newcommand{\FOR}{\pseudocode{for}}
\newcommand{\FI}{\pseudocode{fi}}
\newcommand{\THEN}{\pseudocode{then}}
\newcommand{\ELSE}{\pseudocode{else}}
\newcommand{\ELSF}{\pseudocode{else\ if}}
\newcommand{\ENDIF}{\pseudocode{endif}}
\newcommand{\DO}{\pseudocode{do}}
\newcommand{\OD}{\pseudocode{od}}

\newcommand{\ENDWHILE}{\pseudocode{endwhile}}
\newcommand{\ENDREPEAT}{\pseudocode{endrepeat}}
\newcommand{\ENDFOR}{\pseudocode{endfor}}



\newcommand{\BEGIN}{\pseudocode{begin}}
\newcommand{\END}{\pseudocode{end}}
\newcommand{\PROC}{\pseudocode{procedure}}
\newcommand{\CALL}{\pseudocode{call}}
\newcommand{\VAL}{\pseudocode{value}}
\newcommand{\VALRES}{\pseudocode{value\!-\!result}}
\newcommand{\RES}{\pseudocode{result}}
\newcommand{\RETURN}{\pseudocode{return}}
\newcommand{\HALT}{\pseudocode{halt}}
\newcommand{\DOWNTO}{\pseudocode{downto}}
\newcommand{\TO}{\pseudocode{to}}


%%%%%%%%%%%%%% MACROS FOR NUMBERED LINES IN CODE TABBING ENVIRONMENT %%%%%%%%%%%%%%%%

\newcounter{lctr}
\newcommand{\as}[1]{\{#1\}}

%% line number on left, body in math mode
\newcommand{\li}{\addtocounter{lctr}{1}\arabic{lctr}.}
\newcommand{\lio}[1]{\addtocounter{lctr}{1}\arabic{lctr}.\>\ensuremath{#1}\\}
\newcommand{\lit}[1]{\addtocounter{lctr}{1}\arabic{lctr}.\>\>\ensuremath{#1}\\}
\newcommand{\lih}[1]{\addtocounter{lctr}{1}\arabic{lctr}.\>\>\>\ensuremath{#1}\\}
\newcommand{\lif}[1]{\addtocounter{lctr}{1}\arabic{lctr}.\>\>\>\>\ensuremath{#1}\\}

%%%% 2'nd argument is for a comment
\newcommand{\lioc}[2]{\addtocounter{lctr}{1}\arabic{lctr}.\>\ensuremath{#1}\`{#2}\\}
\newcommand{\litc}[2]{\addtocounter{lctr}{1}\arabic{lctr}.\>\>\ensuremath{#1}\`{#2}\\}
\newcommand{\lihc}[2]{\addtocounter{lctr}{1}\arabic{lctr}.\>\>\>\ensuremath{#1}\`{#2}\\}
\newcommand{\lifc}[2]{\addtocounter{lctr}{1}\arabic{lctr}.\>\>\>\>\ensuremath{#1}\`{#2}\\}
\newcommand{\livc}[2]{\addtocounter{lctr}{1}\arabic{lctr}.\>\>\>\>\>\ensuremath{#1}\`{#2}\\}

%%%% 2'nd argument is for a comment, no newline at end (for last line
%%%% in listing)
\newcommand{\lion}[1]{\addtocounter{lctr}{1}\arabic{lctr}.\>\ensuremath{#1}}
\newcommand{\liocn}[2]{\addtocounter{lctr}{1}\arabic{lctr}.\>\ensuremath{#1}\`{#2}}




\usepackage{hyperref}
\usepackage{xcolor}
\definecolor{darkblue}{rgb}{0.0,0.0,0.6}
\definecolor{darkgreen}{rgb}{0.0,0.6,0.0}
\definecolor{darkred}{rgb}{0.6,0.0,0.0}
\definecolor{background}{HTML}{EEEEEE}

\usepackage{listings}
\lstset{ %
language=C,                % choose the language of the code
basicstyle=\tiny,       % the size of the fonts that are used for the code
numberstyle=\tiny,      % the size of the fonts that are used for the line-numbers
frame=Ltbr,
rulesep=.4pt,
mathescape, % Allows escaping to (La)TeX mode within $$,
stringstyle=\color{redb},
keywords=[1]{elseif, return, else, int, wire, bool, if, Boolean, do-together, foreach, endfor, endif, while, to, Let},
keywords=[2]{decl-list, init-list,next-list, wiredef-list, var-st, loc-st, append, assignment},
showstringspaces = false,
escapeinside={/*@}{@*/},
basicstyle=\scriptsize\ttfamily,
keywordstyle=[2]\color{darkred}\bf,
keywordstyle=[1]\color{darkblue}\bf,
commentstyle=\color{darkgreen},
alsoletter={-},
captionpos=b, %
breaklines=true,
backgroundcolor=\color{background},
}

\renewcommand{\l}{\ell}

% add break line for theorems and propostions
\makeatletter
\renewcommand*{\@opargbegintheorem}[3]{\trivlist
      \item[\hskip \labelsep{\bfseries #1\ #2}] \textbf{(#3)}\ \itshape}
\makeatother


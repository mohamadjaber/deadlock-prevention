We now formulate a local version of $\GLin$.
Observe that if
$\widepth{\B}{\B_i}{t} < \omega \lor \wodepth{\B}{\B_i}{t} < \omega$,
then there is some finite $\l$ such that 
$\widepth{\B}{\B_i}{t} = \l \lor \wodepth{\B}{\B_i}{t} = \l$.




\begin{definition}[$\LLin(\B, Q_0, \act, \l)$] \label{def:ldfc-k}
\label{def:locLinear}
%Let $(B, Q_0)$ be a BIP system, with $B =\gamma(\B_1,\dots,\B_n)$.
%Let $a$ be an interaction of $(B, Q_0)$, \ie $a \in \gamma$.
Let $\l > 0$ and $s_\act \goesto[\act] t_\act$ be an arbitrary reachable transition of $\dsk{\act}{\l}$.
Then, in $t_\act$, the following holds. 
For every component $\B_i$ of $\cmps{\act}$:  
either $\B_i$ has in-depth less than $2\l - 1$, or out-depth less than $2\l - 1$, in $\wfg{\dsk{\act}{\l}}{t_\act}$. 
Formally,\\
\ind  $\fa \B_i \in \cmps{\act}: 
\widepth{\dsk{\act}{\l}}{\B_i}{t_\act} < 2\l-1 \lor \wodepth{\dsk{\act}{\l}}{\B_i}{t_\act} < 2\l-1$.
\end{definition}


To infer deadlock-freedom in $(\B, Q_0)$ by checking $\LLin(\B, Q_0, a, \l)$,
% which is a condition on the behavior of the subsystem $\ds{a}$,
 we show that wait-for behavior in $\B$ ``projects down''
to any subcomponent $\B'$, and that wait-for behavior in $\B'$ ``projects up'' to $\B$.

% Let $e$ be a wait-for-edge with both endpoints in $B'$. Then if $e$ is
% present in the wait-for-graph of $B$ for state $s$, then $e$ is also present in the wait-for-graph
% $B'$ for $s \pj B'$ ($e$ projects down), and vice-versa ($e$ projects up).

%Thus $\wfg{B'}{s \pj B'}$ is a ``projection'' of $\wfg{\B}{s}$ onto $B'$.


%\vspace{0.5ex}

Since wait-for-edges project up and down, it follows that wait-for-paths
project up and down, provided that the subsystem contains the entire wait-for-path.

\begin{proposition}[In-projection, Out-projection] \label{prop:in-out-projection}
%Let $(B, Q_0)$ be a BIP system, 
Let $\l > 0$, let $\B_i$ be an atomic component of $\B$, and let 
$(\B', Q'_0)$ be a subsystem of $(\B, Q_0)$ which is based on a superset of $\ssg{a}{2\l}$.
Let $s$ be a state of $(B, Q_0)$, and $s' = s \pj B'$. Then
(1) $\widepth{\B}{\B_i}{s} < 2\l-1$ iff $\widepth{\B'}{\B_i}{s'} < 2\l-1$, and
(2) $\wodepth{\B}{\B_i}{s} < 2\l-1$ iff $\wodepth{\B'}{\B_i}{s'} < 2\l-1$.
\end{proposition}
%
% \prfs{Follows from Definition~\ref{def:depth},
% Proposition~\ref{prop:edge-projection}, and the observation that $\wfg{\B'}{s'}$
% contains all wait-for-paths of length $\le \l$ that start or end in $\B_i$.}
%
\begin{proof}
We establish clause (1). The proof of clause (2) is analogous, except we replace paths ending in
$\B_i$ by paths starting from $\B_i$.
The proof of clause (1) is by double implication.



%%%%%%%%%%%%%%%%%%%%%%%%
\ul{$\widepth{\B}{\B_i}{s} < 2\l-1$ implies $\widepth{\B'}{\B_i}{s'} < 2\l-1$}:
%
Assume that $\widepth{\B}{\B_i}{s} < 2\l-1$.
Let $\pi$ be an arbitrary wait-for path in $\wfg{\B'}{s'}$ that ends in $\B_i$. 
%
Since $(\B', Q'_0)$ is a subsystem of $(\B, Q_0)$, by Definition~\ref{def:static:wait-for-graph} and $s' = s \pj B'$,
$\wfg{\B'}{s'}$ is a subgraph of $\wfg{\B}{s}$.
%
Hence $\pi$ is a wait-for-path in $\wfg{\B}{s}$.
By $\widepth{\B}{\B_i}{s} < 2\l-1$, we have $|\pi| < 2\l-1$. 
Hence $\widepth{\B'}{\B_i}{s'} < 2\l-1$ since $\pi$ was arbitrarily chosen.




%%%%%%%%%%%%%%%%%%%%%%%%%%%%%%%%%%%%%%%%
\ul{$\widepth{\B'}{\B_i}{s'} < 2\l-1$ implies $\widepth{\B}{\B_i}{s} < 2\l-1$}:
%
Assume that $\widepth{\B}{\B_i}{s} \ge 2\l-1$. Then there exists a wait-for path $\pi$ in $\wfg{\B}{s}$ such that 
$|\pi| \ge 2\l-1$. Let $\rho$ be the prefix of $\pi$ with length $2\l-1$. 
%
Since $(\B', Q'_0)$ is based on a superset of $\ssg{a}{2\l}$, and since the distance from $\B_i$ to the border of 
$\ssg{a}{2\l}$ is $2\l-1$, we conclude that $\rho$ is a wait-for path
that is wholly contained in $\wfg{\B'}{s'}$. Hence we have $\widepth{\B'}{\B_i}{s'} \ge 2\l-1$.
%
We have thus established 
$\widepth{\B}{\B_i}{s} \ge 2\l-1$ implies $\widepth{\B'}{\B_i}{s'} \ge 2\l-1$.
The contrapositive is the desired result.
\end{proof}


%%%%%%%%%%%%%%%%%%%%%%%%%%%%%%%%%%%%%%%%%%%%%%%%%%%%%%



\vspace{0.5ex}

\noindent
We now show that $\LLin(\B, Q_0, \act, \l)$ implies $\GLin(\B, Q_0, \act)$, which in turn implies deadlock-freedom.  

\begin{lemma}
\label{lemma:loc-implies-glob}
\label{lemma:locLinear-implies-globlinear}
\label{LLinGLin}
%Let $(B, Q_0)$ be a BIP system, with $B = \gamma(\B_1,\dots,\B_n)$, and 
Let $\act$ be an interaction of $\B$, \ie $\act \in \gamma$.
If $\LLin(\B, Q_0, \act, \l)$ holds for some finite $\l > 0$, then $\GLin(\B, Q_0, \act)$ holds.
\end{lemma}
%
%% \prfs{
%% Let $s \la{\act} t$ be a reachable transition of $(B, Q_0)$ and let 
%% $s_\act = s \pj \dsk{\act}{\l}$, $t_\act = t \pj \dsk{\act}{\l}$.
%% Then $s_\act \la{\act} t_\act$ is a reachable transition of $\dsk{\act}{\l}$ by 
%% Corollary~\ref{cor.bip.reachability}.
%% By $\LLin(\B, Q_0, \act, \l)$,
%% $\widepth{\dsk{\act}{\l}}{\B_i}{t_\act} \le \l \lor \wodepth{\dsk{\act}{\l}}{\B_i}{t_\act} \le \l$.
%% Hence by Proposition~\ref{prop:in-out-projection},
%% $\widepth{\B}{\B_i}{t} \le \l \lor \wodepth{\B}{\B_i}{t} \le \l$.
%% So
%% $\widepth{\B}{\B_i}{t} < \omega \lor \wodepth{\B}{\B_i}{t} < \omega$.
%% Hence $\GLin(\B, Q_0, \act)$ holds.}
%
\begin{proof}
Let $s \la{\act} t$ be a reachable transition of $(\B, Q_0)$ and let 
$\B_i \in \cmps{\act}$, $s_\act = s \pj \dsk{\act}{\l}$, $t_\act = t \pj \dsk{\act}{\l}$.
Then $s_\act \la{\act} t_\act$ is a reachable transition of $\dsk{\act}{\l}$ by Corollary~\ref{cor.bip.reachability}.
By $\LLin(\B, Q_0, \act, \l)$, 
$\widepth{\dsk{\act}{\l}}{\B_i}{t_\act} < 2\l-1 \lor \wodepth{\dsk{\act}{\l}}{\B_i}{t_\act} < 2\l-1$.
Hence by Proposition~\ref{prop:in-out-projection},
$\widepth{\B}{\B_i}{t} < 2\l-1 \lor \wodepth{\B}{\B_i}{t} < 2\l-1$.
So
$\widepth{\B}{\B_i}{t} < \omega \lor \wodepth{\B}{\B_i}{t} < \omega$.
Hence $\GLin(\B, Q_0, \act)$.
\end{proof}






%% \paragraph{Complexity of evaluating $\LDFC(a, \l)$.}
%% Using explicit state enumeration, $\LDFC(a, \l)$ can be evaluated in
%% time $O(\SUM_{a \in \gamma} |\dsk{a}{\l}|)$, where $|\dsk{a}{\l}|$ denotes the
%% size of the transition system of $\dsk{a}{\l}$.

\begin{proposition}[Finite out-depth implies local supercycle-violation]
\label{prop:finOutDepth-Implies-locScViol}
For $d < \l$: $(\wodepth{\dsk{\act}{\l}}{v}{t_\act} = d)  \imp \locScViol{v}{d+1}{t_\act}{\act}{\l}$.
\end{proposition}
%
\begin{proof}
Proof is by induction on $d$. 

\ul{Base case, $d=0$.}
Then $v$ has no outgoing wait-for edges. Hence $\locScViol{v}{1}{t_\act}{\act}{\l}$ by Definition~\ref{def:supercycle.violation.local}.

\ul{Induction step, $d > 0$.}
Assume $(\wodepth{\dsk{\act}{\l}}{v}{t_\act} = d)$. Then, every outgoing wait-for edge of $v$ is to some $v'$ such that 
$(\wodepth{\dsk{\act}{\l}}{v'}{t_\act} = d' < d)$. By the induction hypothesis, $\locScViol{v'}{d'+1}{t_\act}{\act}{\l}$.
Hence, by Definition~\ref{def:supercycle.violation.local},  $\locScViol{v}{d+1}{t_\act}{\act}{\l}$.
\end{proof}


\begin{lemma}
%\label{lemma:loc-implies-glob}
\label{lemma:locLinear-implies-locANDOR}
\label{LLinLAO}
Let $\act$ be an interaction of $\B$, \ie $\act \in \gamma$.
Then $\LLin(\B, Q_0, \act, \l)$ implies $\LAO(\B, Q_0, \act, \l)$.
\end{lemma}
%
\begin{proof}
Assume $\LLin(\B, Q_0, \act, \l)$. Let $s_\act \goesto[\act] t_\act$ be an arbitrary reachable transition of $\dsk{\act}{\l}$,
and let $\B_i$ be an arbitrary component of $\cmps{\act}$.
Then, from Definition~\ref{def:locLinear}, we have:
  $$\widepth{\dsk{\act}{\l}}{\B_i}{t_\act} < 2\l-1 \lor \wodepth{\dsk{\act}{\l}}{\B_i}{t_\act} < 2\l-1.$$
The proof proceeds by two cases. 

\ul{ $\widepth{\dsk{\act}{\l}}{\B_i}{t_\act} < 2\l-1$}:
Hence $\B_i$ cannot be in a strongly connected supercycle, because $\B_i$ would then lie on at least one wait-for cycle, and so would have infinite
in-depth. Hence $\locConnViol{\B_i}{t_\act}{\act}{\l}$ by Definition~\ref{def:sConn.violation.loc}, Clause~\ref{def:sConn.violation.loc:scc}.
Hence by Definition~\ref{def:locFormation.violation}, $\locFormViol{\B_i}{t_\act}{\act}{\l}$.

\ul{$\wodepth{\dsk{\act}{\l}}{\B_i}{t_\act} < 2\l-1$}:
Hence $\wodepth{\dsk{\act}{\l}}{\B_i}{t_\act} = d$ for some $d < 2\l-1$.
By Proposition~\ref{prop:finOutDepth-Implies-locScViol}, $\locScViol{\B_i}{d+1}{t_\act}{\act}{\l}$.
Hence by Definition~\ref{def:locFormation.violation}, $\locFormViol{\B_i}{t_\act}{\act}{\l}$.

\noindent
In both cases, we have $\locFormViol{\B_i}{t_\act}{\act}{\l}$.
Since $\B_i$ is an arbitrarily chosen  component of $\cmps{\act}$, we have $\fa \B_i \in \cmps{\act} : \locFormViol{\B_i}{t_\act}{\act}{\l}$.
Hence, by Definition~\ref{def:lao}, we conclude $\LAO(\B, Q_0, \act, \l)$.
\end{proof}



\begin{theorem} \label{thm:LLin.SC-free-preserving}
$\LLin$ is supercycle-freedom preserving
\end{theorem}
%
\begin{proof}
Follows immediately from
Lemma~\ref{lemma:locLinear-implies-globlinear} and
Theorem~\ref{thm:GLin.SC-free-preserving}.
%
Also follows immediately from Lemma~\ref{lemma:locLinear-implies-locANDOR} and Theorem~\ref{thm:GAO.SC-free-preserving}
\end{proof}
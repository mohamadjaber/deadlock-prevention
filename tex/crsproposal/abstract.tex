Distributed computational systems have become ubiquitous, and today pervade all
aspects of life: workflow, banking, trading, social media, medical informatics,
etc.  A distributed system functions by means of the exchange of information
between various components, which requires that one component occasionally
\emph{waits} for another component to send it the required information.  Due to
faulty design and/or software errors (``bugs''), it is possible for patterns of
waiting to arise which cause a {\em deadlock}, \ie a subset of its components
are locked forever in a waiting pattern and cannot make any further progress.
One example of a waiting pattern that leads to deadlock is a \emph{cycle}, \eg
component 1 waits for component 2 to send it data, component 2 waits for
component 3 to send it data, and finally component 3 waits for component 1 to
send it data.

With multicore systems becoming commodity hardware, and with cloud programming
taking prominence, the challenge of assuring deadlock freedom and other
correctness properties becomes even greater, due to the increasing complexity of
both hardware and software, and in the increase in complexity of the interaction
between the hardware and the software. Such complexity leads to
counter-intuitive behavior, \eg in some cases, providing more resources actually
\emph{slows down} the system.

It is therefore of great importance to be able to design systems so that they do
not end up in a deadlock to begin with.  Exhaustive checking is impractical, as
large system are usually too complex. Thus all published work relies on checks
that are relatively easy to compute, and which imply freedom from deadlock. If
the check succeeds, the system is guaranteed to be deadlock free, (no false
positives), which if the check fails, then the system may or may not be
deadlock-free (possible false negative).  Existing work is plagued by this
phenomenon of false negatives, which limits its usefulness.

The proposed work develops a check for deadlock which is more flexible than
existing work: by doing more computation, our check becomes more accurate. Thus,
if an initial ``quick'' version of the check fails, we can invest more time in
a ``slower but more accurate'' check.  Our check has this flexibility because it
is ''theoretically complete'', unlike existing methods, which are not.



%Deciding deadlock freedom of finite-state concurrent programs is PSPACE-complete; a very expensive computational category in terms of required memory
%and runtime resources.
%

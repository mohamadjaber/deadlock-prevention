A computational system is said to be at {\em deadlock} 
when a subset of its components are constrained 
from making progress with no intervention from outside 
the subset. 
%
Deadlock freedom is a crucial property of concurrent and 
distributed hardware and software computational systems. 
%
With multicore systems becoming commodity hardware, 
increasing software system complexity, 
and with cloud programming taking prominence, 
the challenge of assuring deadlock freedom and other 
correctness properties becomes even greater.  
%
Deciding deadlock freedom of finite-state concurrent 
programs is PSPACE-complete; a very expensive computational 
category in terms of required memory and runtime resources. 
%
Researchers proposed methods to establish deadlock freedom
with proposed restrictions to the specified systems. 
%
In published work, 
we established a result that characterizes deadlock freedom
with a structural property of each system state. 
The work enables checking for global deadlock freedom
by model checking a set of subsystems 
of the overall large system. 
The work constructs a {\em wait-for-graph} for each state
in the considered subsytems
and checks whether it has a {\em supercycle}. 
%
When the property being checked is satisfied, 
it implies deadlock-freedom of the overall system. 
If not satisfied, then we re-evaluate over larger subsystems,
which improves the accuracy of the check.  
When the subsystem being checked becomes the entire system, 
our criterion becomes complete for
deadlock-freedom.  
%
The results of the published work shows that our method
addresses deadlock freedom for systems that are not possible 
to address with existing methods and in reasonable time. 
%
This proposal aims to extend the global deadlock freedom 
criteria in several directions. 
First, we would like to consider criteria for {\em local deadlock}
freedom where
a subsystem is deadlocked while the rest of the system executes. 
%
Second, 
we would like to exploit the {\em AND-OR} structural nature of the 
wait-for-graph that we ignored in the published work. 
%
Third, 
we would like to incorporate {\em counterexample refinement}
to improve our method by growing the subsystem we consider in a
guided manner rather than growing it in all directions as we do now. 
%
Fourth, 
the current work analyses the connections and interactions
between the components of the system, and does not exploit the 
semantics of the conditions governing these connections. 
We would like to benefit from these conditions to tighten 
our criteria by leveraging the emerging {\em satisfiability 
solver} and binary decision diagram technologies.
%
%Dynamic BIP
%Design rules for prevention of deadlock
%
Finally, 
we would like to explore whether our method applies to 
(1) deadlock freedom of parametrized systems and 
(2) properties other than deadlock that can be characterized 
with structural checks. 

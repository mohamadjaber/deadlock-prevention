\section{Preliminary Studies}

The core of the proposed inter-disciplinary project relies on the following four fields: 1) media studies; 2) applied natural language processing; 3) distributed systems and big data storage and (4) data analytics. All of those combined contribute to the data science pipeline. In the following we present some preliminary results by the project investigators to several aspects of the required fields.

\subsection{Media Studies}

The third PI's research and teaching interests center on visual culture studies, Arab and global media studies, critical theory, media history and technology studies, and urban studies. His current research is on the history of media technologies and infrastructures, and urban space in the Middle East, and is currently working on a project that examines the aesthetics and cultural politics of Arab satellite broadcasting in light of the history of the visualization of Beirut.

\subsection{Applied natural language processing}
In collaboration between the English and the Computer Science Departments, two of the PI's have introduced \texttt{TopoText} \cite{topotext1,topotext2}, an interactive tool for digital mapping of literary text. \texttt{TopoText} takes as input a literary piece of text such as a novel or a biography article and automatically extracts all place names in the text. The identified places are then geoparsed and displayed on an interactive map. \texttt{TopoText} calculates the number of times a place was mentioned in the text, which is then reflected on the map allowing the end-user to grasp the importance of the different places within the text. It also displays the most frequent words mentioned within a specified proximity of a place name in context or across the entire text. This can also be faceted according to part of speech tags.

\subsection{Distributed systems, big data storage, and analytics}
The above named PIs have already proposed a framework~\cite{bip-distributed} to generate efficient and correct distributed implementations from high-level component-based models. Additionally, in the context of big data storage analytics, they have submitted two papers related to big data storage~\cite{hbase-dynamique} and analytics~\cite{spark-cbs}. 
%

In~\cite{hbase-dynamique}, we introduce an algorithm that enhances client operation latency by monitoring and dynamically balancing (by splitting or moving regions) region servers of HBase system (non-relational database) based on the most requested keys and the average response time of clients' requests.
%

In~\cite{spark-cbs}, we introduce a new approach for linking Spark applications to build a complex one based on user-defined configuration. Given a set of Spark applications a configuration file defines a directed graph between the applications, where
each edge is a dependency between two applications. We have implemented a Domain Specific
Language (DSL) to easily define interfaces as well as connections. 

Although the lead PI has not published directly in the field of analytics, her research interests at the intersection of mathematics and theoretical computer science. Particularly, she has worked on developing efficient algorithms and data structures for use in computer algebra at large scale. Trends in theoretical computer science she has also experimented with include distributed algorithm design, multithreaded algorithm design, hybrid algorithm design, as well as cache-oblivious algorithms and data structures. The focal point of the lead PI's research has been to overcome constraints posed by the underlying computer architecture such as heterogenous clusters as well as the increase in the size of the data to be handled and the complexity of the underlying mathematical structures. We refer to \cite{S12,S14,S15} for selected work in this direction.
 



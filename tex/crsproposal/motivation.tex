\section{Motivation and Background}

\subsection{Motivation and example}

Large distributed computational systems are an indispensable part of the
critical infrastructure of modern economics. The internet itself can be regarded
as a large distributed system.  A distributed system functions by means of the
exchange of information between various components, and also by the sharing of
resources (\eg memory, communication links, etc). This requires that one
component occasionally \emph{waits} for another component to send it the required
information, or to relinquish a needed resource.  Due to faulty design and/or
software errors (``bugs''), it is possible for patterns of waiting to arise
which cause a {\em deadlock}, \ie a subset of its components are locked forever
in a waiting pattern and cannot make any further progress.
%
One example of a waiting pattern that leads to deadlock is a \emph{cycle}, \eg component 1 waits for component 2 to send it data, 
component 2 waits for component 3 to send it data, and finally 
component 3 waits for component 1 to send it data.
%
Existing work relies on preventing a cyclic waiting pattern, as illustrated above. However, consider the above example, but add to it component 4,
and suppose that component 1 can take data from either component 2 or from component 4. In this case, once component 4 sends data to component 1,
component 1 can proceed with its computation, and case send data to component 3, which then sends data to component 2. Hence the cycle is broken and
there is not deadlock.
%
This simple example illustrates the fact that, in a system in which components have a \emph{local choice} of action, waiting cycles are an
\emph{incomplete} characterization of deadlock: existence of a deadlock implies the existence of at least one waiting cycle, but existence of a
waiting cycle does not necessarily imply the existence of a deadlock.
%
The proposed work uses a better characterization of a deadlock, which we call a \emph{supercycle} \cite{FORTE13}. 
A key contribution is that we will show that a supercycle is a 
\emph{complete} characterization of deadlock: existence of a deadlock implies the existence of at least one supercycle, \emph{and} existence of a
supercycle implies the existence of a deadlock.

As noted in the abstract, a crucial design task is to ensure that a large distributed system does not enter a configuration in which some or all of
its components are deadlocked. 
A key problem is the conceptual complexity of large distributed computer systems: the number of possible configurations of a large system is 
too large to be checked exhaustively. Consider for example, a toy system with just one hundred components, each of which can be in
one of only two states: (1) sending date to another component, or (2) receiving date from another component. Thus there are overall $2^{100}$ possible
configurations. Even if each can be checked in a nanosecond, it would still take more than $2^{60}$ years to check all configurations explicitly.
%
Thus all published work relies on checks that are 
relatively easy to compute, and which imply freedom from deadlock. If the check succeeds, the system is guaranteed to be deadlock free, (no false
positives), which if the check fails, then the system may or may not be deadlock-free (possible false negative).  Existing work is plagued by this
phenomenon of false negatives, which limits its usefulness.
%
Existing work relies on incomplete characterizations of deadlock, based on the occurrence of a waiting cycle. 
As stated above, our work uses the complete, and more accurate, notion of a supercycle. Thus, our methods do not incur false negatives due to 
theoretical limitations, but only due to lack of computational resources. Experimental results to date show that our check performs well in practice.



%Researchers proposed methods to establish deadlock freedom with proposed restrictions to the specified systems.
%





% for each state in the considered subsytems and checks whether it has a {\em supercycle}.
% %
% When the property being checked is satisfied, it implies deadlock-freedom of the overall system.  If not satisfied, then we re-evaluate over larger
% subsystems, which improves the accuracy of the check.  When the subsystem being checked becomes the entire system, our criterion becomes complete for
% deadlock-freedom.
% %
% The results of the published work shows that our method addresses deadlock freedom for systems that are not possible to address with existing methods
% and in reasonable time.
% %



% This proposal aims to extend the global deadlock freedom 
% criteria in several directions. 
% First, we would like to consider criteria for {\em local deadlock}
% freedom where
% a subsystem is deadlocked while the rest of the system executes. 
% %
% Second, 
% we would like to exploit the {\em AND-OR} structural nature of the 
% wait-for-graph that we ignored in the published work. 
% %
% Third, 
% we would like to incorporate {\em counterexample refinement}
% to improve our method by growing the subsystem we consider in a
% guided manner rather than growing it in all directions as we do now. 
% %
% Fourth, 
% the current work analyses the connections and interactions
% between the components of the system, and does not exploit the 
% semantics of the conditions governing these connections. 
% We would like to benefit from these conditions to tighten 
% our criteria by leveraging the emerging {\em satisfiability 
% solver} and binary decision diagram technologies.
%
%Dynamic BIP
%Design rules for prevention of deadlock
%
% Finally, 
% we would like to explore whether our method applies to 
% (1) deadlock freedom of parametrized systems and 
% (2) properties other than deadlock that can be characterized 
% with structural checks. 


To motivate our work, consider a variant of 
the committee coordination problem described by Chandy and Misra: %~\cite{MisraCommittee}:
\begin{quote}
  Professors in a certain university have organized themselves into committees.  Each committee has an unchanging membership roster of one or more
  professors, and hence each professor belongs to a fixed set of committees.
At any given time, a professor is ready to attend some subset of it committees, \eg Professor $\Prof_1$ may be in committees $\Cm_1, \Cm_2, \Cm_4, \Cm_7$, but may
currently be willing to attend only committees $\Cm_2$ or $\Cm_5$.
%From time to time, a professor may decide to attend a committee meeting; it starts waiting and remains waiting until a meeting of a
%  committee of which it is a member is started.  
The restrictions on meetings are as follows: (1) a committee meeting may be started only if all
  members of that committee are willing to attend and (2) no two committees may meet simultaneously if they have a common member.  Given that all meetings
  terminate in finite time, the problem is to devise a protocol, satisfying these restrictions, that also guarantees that if all members of a
  committee are waiting then at least one of them will attend some meeting.
\end{quote}
A deadlock may arise when professors select committees in such a manner that no committee has all of its members willing to attend.  
%
For example, suppose that 
committee $\Cm_1$ consists of professors $\Prof_1, \Prof_2, \Prof_4$, 
committee $\Cm_2$ consists of professors $\Prof_1, \Prof_3, \Prof_4$, and 
committee $\Cm_3$ consists of professors $\Prof_2, \Prof_3, \Prof_4$.
%
Suppose also that 
Professor $\Prof_1$ is willing to attend committees $\Cm_1, \Cm_2$, 
Professor $\Prof_2$ is willing to attend committee $\Cm_3$, %so C_2 disabled
Professor $\Prof_3$ is willing to attend committee $\Cm_2$, and %so C_3 disabled
Professor $\Prof_4$ is willing to attend committees $\Cm_2, \Cm_3$. % so C_1 disabled
%
Then no committee has all of its members available, and so no committee is able to convene. 
Even though each professor is willing to attend \emph{some} committees, their individually decisions about what to attend are 
``inconsistent''. 
Note that the obvious solution, of having every professor being always willing to attend every committee is not practical, since in practice,
willingness to attend a committee (or engage in such similar tasks) requires resources and effort, \eg preparation for the meeting. 
Thus, we require solutions that are efficient, that that achieve the system objectives, with minimal expenditure of resources. 
Brute force solutions, which always provide maximal resources, render the deadlock problem trivial, but are not acceptable from a cost perspective.


%Our proposed work aims to address all of the mentioned gaps. 


\subsection{Technical background}

We express distributed system using the BIP \cite{bip06,BliudzeS08} framework. In BIP, an \emph{atomic component} $\B_i$ consists of a set of states and a set of
transitions. A state represents a particular configuration of a component, \eg sending data, receiving data, performing internal computation, etc..
A transition represents a change of configuration, \eg a component $\B_i$ executes a transition which moves $\B_i$ from a 
state of receiving data to that of performing internal computation (\eg when all data has been received). Subsequently, when the internal computation
is finished, $\B_i$ may execute another transition which moves from a state of internal computation to a state of sending data.
Each transition of $\B_i$ is labeled with a port name $p_i$. 
A large distributed system is represented by a \emph{composite component} $\B = \B_1 \pl \B_2 \pl \cdots \pl \B_n$. That is, $\B$ consists of $n$
atomic components $\B_1, \B_2,  \ldots,  \B_n$ which execute ``at the same time'', \ie concurrently. 
An \emph{interaction} $\act$ in BIP is a set $\{ p_j, p_k, p_{\l}, \ldots \}$ of ports, such that no two ports are taken from the same components.
(Different components have different ports). Every component which contributes a port to an interaction is a \emph{participant} of the interaction.
For example, if  $\act = \{ p_j, p_k, p_{\l}, p_m \}$ then the participants of $\act$ are the components $\B_j, \B_k, \B_{\l}, \B_m$.
The interaction $\act$ is executed when every participant executes a transition that is labeled with a port which occurs in $\act$. 
The participants execute their respective transitions synchronously,
\ie ``at the same time.'' 
%
Hence, a component $\B_i$ is ready to execute an interaction $\act$
if and only if $\B_i$ is ready to execute a transition that is labeled
with a port $p_i$ which is in $\act$.
An interaction $'act$ can be executed if and only if all of its
participants are ready to execute it.
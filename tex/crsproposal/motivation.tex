\section{Motivation and Background}

Large distributed computational systems are an indispensable part of the crticial infrastruture of modern economics. The internet itself can be
regarded as a large distributed system.
%
A distributed system functions by means of the exchange of information between various components, and also by the sharing of resources (\eg memory,
communication links, etc). This requires that one component ocassionaly \emph{waits} for another component to send it the required information, or to
relinquicsh a needed resource.  Due to faulty design and/or soiftware errors (``bugs''), it is possible for patterns of waiting to arise which cause a
{\em deadlock}, \ie a subset of its components are locked forever in a waiting pattern and cannot make any further progress.
%
One example of a waiting pattern that leads to deadlock is a \emph{cycle}, \eg component 1 waits for component 2 to send it data, 
component 2 waits for component 3 to send it data, and finally 
component 3 waits for component 1 to send it data.
%
Existing work relies on preventing a cyclic waiting pattern, as illustrated above. However, consider rthe above example, but add to it component 4,
and suppose that component 1 can take data from either component 2 or from componet 4. In this case, once compoent 4 sends data to component 1,
component 1 can proceed with its computation, and case send data to component 3, which then sends data to component 2. Hence the cycle is broken and
there is not deadlock.
%
This simple example illustrates the fact that, in a system in which components have a \emph{local choice} of action, waiting cycles are an
\emph{incomplete} characterization of deadlock: existence of a deadlock implies the existence of at least one waiting cycle, but existence of a
waiting cycle does not necessarily imply the existence of a deadlock.
%
The proposed work presents a better characterization of a deadlock, which we call a \emph{supercycle}. We will show that a supercycle is a 
\emph{complete} characterization of deadlock: existence of a deadlock implies the existence of at least one supercycle, \emph{and} existence of a
supercycle implies the existence of a deadlock.

As noted in the abstract, a crucial design task is to ensure that a large dfistributed system does not enter a configuration in which some or all of
its components are deadlocked. 
A key problem is the conceptual complexity of large distributed computer systems: the number of possible configurations of a large system is 
too large to be checked exhaustively. COnsider for example, a toy system with just one hundred components, each of which can be in
one of only two states: (1) sending date to another compoennt, or (2) receiving date from another component. Thus there are overall $2^{100}$ possible
configurations. 



Thus all published work relies on checks that are 
relatively easy to compute, and which imply freedom from deadlock. If the check succeeds, the system is guaranteed to be deadlock free, (no false
positives), which if the check fails, then the system may or may not be deadlock-free (possible faise negative).  Existing work isplagues by this
phonomenon of false negatives, which lilmits its usefullness.












Researchers proposed methods to establish deadlock freedom
with proposed restrictions to the specified systems. 
%
In published work, 
we established a result that characterizes deadlock freedom
with a structural property of each system state. 
The work enables checking for global deadlock freedom
by model checking a set of subsystems 
of the overall large system. 
The work constructs a {\em wait-for-graph} for each state
in the considered subsytems
and checks whether it has a {\em supercycle}. 
%
When the property being checked is satisfied, 
it implies deadlock-freedom of the overall system. 
If not satisfied, then we re-evaluate over larger subsystems,
which improves the accuracy of the check.  
When the subsystem being checked becomes the entire system, 
our criterion becomes complete for
deadlock-freedom.  
%
The results of the published work shows that our method
addresses deadlock freedom for systems that are not possible 
to address with existing methods and in reasonable time. 
%
This proposal aims to extend the global deadlock freedom 
criteria in several directions. 
First, we would like to consider criteria for {\em local deadlock}
freedom where
a subsystem is deadlocked while the rest of the system executes. 
%
Second, 
we would like to exploit the {\em AND-OR} structural nature of the 
wait-for-graph that we ignored in the published work. 
%
Third, 
we would like to incorporate {\em counterexample refinement}
to improve our method by growing the subsystem we consider in a
guided manner rather than growing it in all directions as we do now. 
%
Fourth, 
the current work analyses the connections and interactions
between the components of the system, and does not exploit the 
semantics of the conditions governing these connections. 
We would like to benefit from these conditions to tighten 
our criteria by leveraging the emerging {\em satisfiability 
solver} and binary decision diagram technologies.
%
%Dynamic BIP
%Design rules for prevention of deadlock
%
Finally, 
we would like to explore whether our method applies to 
(1) deadlock freedom of parametrized systems and 
(2) properties other than deadlock that can be characterized 
with structural checks. 


Example: 
Consider the committee coordination problem described by Chandy and Misra~\cite{MisraCommittee}. 

\begin{quote}
  Professors in a certain university
  have organized themselves into committees.
  Each committee has an unchanging membership
  roster of one or more professors. From
  time to time, a professor may decide to attend
  a committee meeting; it starts waiting
  and remains waiting until a meeting of a committee
  of which it is a member is started.
  The restrictions on meetings are as follows:
  (1) a committee meeting may be started only
  if all members of that committee are waiting
  and (2) no two committees may meet simultaneously
  if they have a common member. 
  Given that all meetings terminate in finite
  time, the problem is to devise a protocol,
  satisfying these restrictions, that also guarantees
  that if all members of a committee are
  waiting then at least one of them will attend
  some meeting.
\end{quote}

A deadlock may arise when professors 

Our proposed work aims to address all of the mentioned gaps. 


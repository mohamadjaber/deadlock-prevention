\section{Methods of Inquiry and Analysis}

\subsection{Wait-for graphs}

Local deadlock denotes the state where a subset of the system components are deadlocked while the rest of the system can execute.
%
We characterize the waiting relations in a system by means of a {\em wait-for-graph}. 
If a component $\B_i$ is ready to execute an interaction $\act$ then there is an edge $\B_i ar \act$ from $\B_i$ to $\act$.
If a component $\B_i$ is not ready to execute an interaction $\act$ then there is an edge $\act \ar \B_i$ from $\act$ to $\B_i$.
A key point is that the wait-for graph depends on the currewnt configuration (\ie the current ``global state'') of the system.
%
Components communicate and execute through interactions. 
An interaction is enabled to execute when all components that 
are parts of it are ready. 
So an edge from an interaction to a component in a wait-for-graph
is an AND edge. 
A component executes if any one of the interaction it is part of
executes. So an edge from a component to an interaction is an OR
edge. 
The wait-for graph therefore is an AND-OR graph. 
We will leverage this interesting structure to simplify the
check for deadlock freedom. 



\subsection{Supercyles}

We characterize a deadlock as a particular waiting pattern in a wait-for graph.
Roughly speaking, a supercycle $\SC$ is a subset of the compoenets and interactions such that every component and interaction in $\SC$ is blocked by
the other components and interactions in $\SC$. 
An intteraction $\act$ is blocked by $\SC$ iff there is some participant $\B_j$ of $\act$ such that $\B_j$ is in $\SC$ and $\B_j$ is not ready to
execute $\act$.
A component $\B_i$ is blocked by $\SC$ iff every interaction $\act$ that $\B_i$ is ready to execute happens to bein $\SC$.
A supercyle is the AND-OR generalization of a simple cycle.



\subsection{Preventing the formation of supercycles}

A supercyle implies a deadlock, and so is stable: once a supercycle forms, it persists forever.
Suppose there is no supercycel in global state $t$, and an interaction $\act$ is executed which leads to a new global state $s$ in which a supercyel $\SC$
exists. 
Then we show that some participant $\B_i$ of $'act$ must be in the newly formewd supercycle $\SC$. 
We exploit this fact to impose constraints on the effewcts of executing $\act$, which prevent the formation of the supercycle $\SC$.

\section{Methods of Inquiry and Analysis}

\begin{figure}[t]
  \begin{center}

    \resizebox{\textwidth}{!}{
     \input{fig/arch.pdf_t}
   }
    \caption{Methodology overview.}
    \label{fig:overview}
  \end{center}
\end{figure}

\subsection{Overview} 

The diagram in Figure~\ref{fig:overview} illustrates the 
methodology we will use to achieve the aims of the proposal. 
The front end will take a BIP specification and produce
a BOP system $\B$. 
The subsystem selection will produce a subsystem$B'$ of $\B$. 

Structural analysis considers the subsystem $B'$ and builds
a wait-for graph $W_s$ for each reachable state $s$ of $B'$. 
In case $B'$ is found to be deadlock free, we conclude. 
In case $B'$ is found to be deadlock prone, we compute 
a counterexample. 

Semantic analysis takes $B'$ and considers the predicates
annotating the ports and interactions. 
It constructs an enablement abstraction around from 
port enablement and interaction enablement. 
The abstraction is expressed in a first order logic (FOL)
formula that we pass to an SMT SAT solver. 
The result of the SMT SAT solver refines the set of reachable 
states. 

The subsystem selection makes use of the counterexamples 
and the reachable states to
guide the construction of the next subsystem. 
The structural analysis also leverages the reachable state
results from the semantic analysis to improve its runtime. 

In what follows we provide detailed description of the 
different parts of our methodology and illustrate how 
the methodology will help us achieve the specific aims
of Section~\ref{sec:aims}

In what follows, we provide necessary background 
that covers completed work and then we present our 
methods to address the specific aims one at a time. 

\subsection{Wait-for graphs}

Local deadlock denotes the state where a subset of the system components are deadlocked while the rest of the system can execute.
%
We characterize the waiting relations in a system by means of a {\em wait-for-graph}. 
If a component $\B_i$ is ready to execute an interaction $\act$ then there is an edge $\B_i \ar \act$ from $\B_i$ to $\act$.
If a component $\B_i$ is not ready to execute an interaction $\act$ then there is an edge $\act \ar \B_i$ from $\act$ to $\B_i$.
A key point is that the wait-for graph depends on the current configuration (\ie the current ``global state'') of the system.
%
Components communicate and execute through interactions. 
An interaction is enabled to execute when all components that 
are parts of it are ready. 
So an edge from an interaction to a component in a wait-for-graph
is an AND edge. 
A component executes if any one of the interactions it is part of
executes. So an edge from a component to an interaction is an OR
edge. 
The wait-for graph therefore is an AND-OR graph. 
We will leverage this interesting structure to simplify the
check for deadlock freedom. 



\subsection{Supercycles}

We characterize a deadlock as a particular waiting pattern in a wait-for graph.
Roughly speaking, a supercycle $\SC$ is a subset of the components and interactions such that every component and interaction in $\SC$ is blocked by
the other components and interactions in $\SC$. 
An interaction $\act$ is blocked by $\SC$ iff there is some participant $\B_j$ of $\act$ such that $\B_j$ is in $\SC$ and $\B_j$ is not ready to
execute $\act$.
A component $\B_i$ is blocked by $\SC$ iff every interaction $\act$ that $\B_i$ is ready to execute happens to be in $\SC$.
A supercycle is the AND-OR generalization of a simple cycle.



\subsection{Preventing the formation of supercycles}

A supercycle implies a deadlock, and so is stable: once a supercycle forms, it persists forever.
Suppose there is no supercycle in global state $t$, and an interaction $\act$ is executed which leads to a new global state $s$ in which a supercycle $\SC$
exists. 
Then we show that some participant $\B_i$ of $\act$ must be in the newly formed supercycle $\SC$. 
We exploit this fact to impose constraints on the effects of executing $\act$, which prevent the formation of the supercycle $\SC$.

\subsection{Aim 1: Local deadlock freedom} 

Our method visits a complete number of 
subsystems of varying sizes to check whether
the whole system is deadlock free. 
We will extend the method to traverse these subsystems
in a careful manner and to ensure a stronger structural 
condition such that we ensure the property 
``There is no subsystem of $B$ that has a deadlock''. 
With that our method will cover local deadlock freedom. 

\subsection{Aim 2: Counterexample based refinement} 

When the method detects a supercycle, it issues an alarm of a 
deadlock prone system. 
The supercycle characterizes the deadlock as an interdependent 
set of components and interactions
at a specific state. 
We will extend the method to compute a counterexample
that illustrates the deadlock. 
First, we will compute an execution path that leads 
to the reachable state from the initial state. 
Then we will translate the set of components and interactions
into a sequence of interactions that will lead to the deadlock. 

The counterexample can also be used to devise a method to
select the next subsystem. 
We will inspect the variables that are assigned
in the counterexample and those that are not assigned. 
The assigned variables force a deadlock and the values of the
other variables do not matter. 
We will explore extending the subsystem in the direction of 
the assigned 
variables instead of extending it in all directions uniformly 
as the current method does. 

\subsection{Aim 3: Enablement abstraction } 

The method currently iterates over all reachable states of
a subsystem, constructs a wait-for graph structure and 
performs a supercycle check. 
The reachable states in a subsystem are an over-approximation
of the reachable states in the whole system. 

We will extend the method to tighten this over-approximation. 
Interactions connect components through ports. 
%
We will extend the formalization to include predicates 
that annotate the ports and the interactions. 
These predicates are Boolean expressions that range over the
variables of the system and all fall within 
the Presburger arithmetic theory. 
%
We will encode each state of a component with a Boolean 
variable.
Then, we perform a strongest postcondition computation to 
obtain a system of equations that constrains the reachable 
states.
%
The system of equations can now eliminate some of the states
we used to visit in our supercycle checks. 

The system of equations can also act as an enablement abstraction.
The wait-for graph currently contains component and interaction nodes. 
We can extend it to contain ports. 
The AND-OR structure becomes now an AND-$\phi$-OR structure
where $\phi$ is a first order logic formula projected from
the system of equations over a specific port. 
This allows the composition of formula that indicates
a tighter supercycle check. 
In case the formula has a satisfying assignment, then
a supercycle exists. 

We will use SMT solvers to check whether a state 
satisfies the system of equations for reachable state 
enumeration. 
We will also use SMT solvers to perform a symbolic
supercycle check.

\subsection{Aim 4: Parameterized systems} 

A parameterized system $B=\B_1 \pl B_1 \ldots \pl B_1$ is a 
composite system of $N>1$ $B_1$ components. 
Our method naturally extends to address parameterized systems 
of a specific size $N=n_0$. 

An interesting question is that given a parameterized system $B$, 
can we devise a method that answers the deadlock freedom
question for any $N$. 

We will explore ways to extend our method to answer the 
parameterized deadlock freedom question. 
We can start by checking the symmetry of the connections between
the $B_1$ components.
The symmetry can allow us to generalize the method. 
For example, if all $B^3=B_1\pl B_1 \pl B_1$ 
($3-$component subsystems) of $B$ are equivalent,
then deadlock freedom of $B^3$ implies deadlock freedom 
of $B$. 


\subsection{Aim 5: Structural checks for other safety properties} 

Researches addressed safety properties of distributed and 
component based system using a variety of model checking
approaches that are mostly state based. 
These approaches are inherently expensive due to the state
space explosion problem. 
Typically they are at least exponential in the size of the 
state space. 
%
We characterized deadlock freedom with a structural property 
of a subsystem. 
This allowed for an over approximation of the reachable state
space. 
This also allowed us to practically model check 
large systems,
since these checks are typically polynomial in the size 
of the structure. 

We are interested in selecting a set of other general safety
properties and providing structural characterizations 
for them.
We will explore methods to characterize {\em race conditions}
and {\em order violations} with structural properties
similar to the supercycle check.

A race condition occurs when two components are granted access
to the same resource. Both components can write to the resource
which renders the value of the resource non-deterministic while
and after accessing it. 

An order violation might occur when events that are governed 
with contracts happen out of order.
They are temporal in nature and their detection requires
inspecting multiple states in the system at once. 
For example, a file open should precede a file close. 

We will devise {\em access} graphs that we infer from 
accessing shared variables in the different components 
and interaction.
We will explore properties of the access graph that 
imply race conditions. 

We will also devise {\em order} graphs. 
We will infer the order from contracts that 
govern operations executed in the components
and interactions. 
We will explore properties of the order graph that 
imply order violations.  

\subsection{Aim 6: Benchmarking and validation } 
 
 We will devise a general resource allocation system
 in BIP  that we can parameterized.
 The system will be a multi-token system 
 with a set of $N$ arbiters, a set of $M$ components,
 and $k$ tokens. 
 The arbiters pass the tokens to the components or to other
 arbiters. 
 Each component requires a fixed number of tokens to 
 execute. 

 This system (with all its possible configurations)
 subsumes a large set of resource allocation
 systems that are deadlock prone. 

 We will use the general resource allocation system 
 with increasing $N, M,$ and $k$ 
 along with other industrial and textbook systems
 to validate and benchmark our implementation
 against other existing methods. 


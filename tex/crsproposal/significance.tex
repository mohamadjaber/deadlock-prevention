\section{Significance of the Project}

Below we elaborate on several aspects behind the significance of the proposed work:

\subsection{Novelty of the proposal}

Although several initiatives have attempted to provide interactive maps displaying interpretations of news from the Syrian conflict, to the best of our knowledge, no existing work involves a filtering component that excludes news that fail to be credible enough. A significant amount of literature addresses media bias, but not to the extent that excludes news based on credibility. We are aware of several United Nations initiatives harnessing big data for development and humanitarian action, but again, we are not aware that such schemes invest in filtering news. The authors are also aware of certain trends in data journalism that reveal faulty news after they have been published, but this type of work is certainly not of a predictive nature. Also no existing work maps sufficiently in real-time on big data. 


\subsection{Importance of the project's interdisciplinary aspects}

Generally speaking, Data Science is a highly interdisciplinary field that is currently bringing together expertise from very different domains in ways that could not have been thought possible. Particularly, this proposal cannot be realised without the contribution of three main lines of expertise. Expertise in linguistics will be required in the initial phases as we build our corpus incorporating documents from English and Arabic. Carefully tuned topic classification or clustering will be needed to fine-tune the overwhelming amount of data into sub-clusters that share common, major features. Particularly, natural language processing techniques need to be applied to identify ideologies in a variety of sources. Expertise in media studies will be needed to help survey and assess the latest literature on media bias subject to cultural or political constraints. As most, if at all, of the existing word on media bias do not address the situation in the middle east, we will need to adapt the existing studies to the intricacies of the middle east. Expertise in Computer Science is needed to perform the analytics phase incorporating ideas from machine and statistical learning. It is also needed in the big data, real-time processing and visualisation. Big data storage, processing and high-frequency real-time processing require scalable distributed clusters as well as theoretical and practical expertise in distributed systems and high-level operators in order to easily and efficiently build parallel and distributed applications.


\subsection{Expected impact of the project's outcomes}

The proposed project presents a real case study in the emerging field of Data Science. As such, it has a great academic impact per se. It prompts collaboration between the arts and sciences in a very challenging manner to tell a powerful story. In addition to the academic merits associated with the proposal, we believe our work will have impact beyond the academic circles. Particularly, we believe it can be put to use by NGOs and other agencies involved in assessing humanitarian conditions and delivering humanitarian aid in Syria. That our model needs to be generic enough implies that it can also be put to use in other conflict zones. 



\subsection{Dissemination plan of results}

Firstly, the obvious channels of dissemination are academic in nature -- for example, that the delivered work needs to be published in existing Big Data or Analytics venues, and also a number of communication and media studies journals/conferences. Additionally, we want to ensure that our work is put to good use by the relevant NGOs or stakeholders in the conflict. The authors are aware that the United Nations Global Pulse is an initiative that attempts to harness existing data towards relief operations of the UN in Syria, but once again, no subtle analytics or big data manipulations are present.


\subsection{Project's relevance at the local, regional, and international levels}

AUB has expertise in many sub-disciplines that make data science achievable: this is one step forward. Locally and regionally, the region suffers from poor data documentation, exploration, and analysis, despite the wealth of data emerging from it, and the many challenges the Arab world suffers from, for which data science will undoubtedly be crucial. Internationally, we believe our work will add to the existing yet ever growing trend in open data science for the social good. 

\section{Specific Aims}

This project is intended for two years.
The main aims of this project are as follows.

\begin{enumerate}

\item 
Local deadlock denotes the state where a subset of the 
system components are deadlocked while the rest of the 
system can execute. 

\item
We will characterize the waiting relations in a system by means of a
{\em wait-for-graph}. We will then characterize local deadlock by
means of a pattern of waiting that we call a \emph{supercycle}.
%local deadlock freedom with a {structural property and build algorithms for checking local deadlock freedom.

\item 
Components communicate and execute through interactions. 
An interaction is enabled to execute when all components that 
are parts of it are ready. 
So an edge from an interaction to a component in a wait-for-graph
is an AND edge. 
A component executes if any one of the interaction it is part of
executes. So an edge from a component to an interaction is an OR
edge. 
The wait-for graph therefore is an AND-OR graph. 
We will leverage this interesting structure to simplify the
check for deadlock freedom. 
A supercyle is the AND-OR generalization of a simple cycle.


\item 
When we detect a deadlock-prone system, we issue an alarm. 
We will improve the check to also compute a witness of the deadlock scenario 
and return that witness in the form of a counterexample 
to the developers of the system. 
The developers can use to counterexample to refine the system and solve
the deadlock problem. 

\item 
System components are connected through ports and interactions
These ports and interactions are typically annotated with predicates
that constrain the semantics of the connectivity. 
We will use these predicates with emerging satisfiability modulo theory (SMT) 
solvers to improve the quality of the deadlock freedom check.

		
\item 
A system is paramterized with parameter $N>1$ 
in case it connects $N$ interaction components of the 
same type and functionality in a specified connectivity topology. 
Existing work study model checking parametrized systems for safty properties
such as freedom of race conditions. 
We will explore the applicability of our method to deadlock freedom 
of parametrized systems
		. 

\item
Other system safety properties can also be characterized with structural 
properties of smaller subsystems. 
We will explore applying our methodology to characterize safety properties 
othe than deadlock freedom with structural checks. 

\item 
We will build benchmarks for our work from real life examples.
We will implement our algorithms and provide them to the research and 
applied community. 
We will start by an implementation that works for systems specified with 
the Behavior Interaction Priority (BIP) ~\cite{bip06} framework.

\end{enumerate}


